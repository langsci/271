\label{chap:vd}
\section{Introduction}
In Chapter \ref{chap:vz} we explored the workings of {\vz}, a non-active head responsible for processes of anticausativization. The literature on transitivity alternations has, to a large extent, focused on this ``direction'' of derivation (direction intended here intuitively, since both causatives and anticausatives are derived from the same base). It is perhaps no accident that this literature has also been based for the most part on European languages. In this chapter, we will probe the function of {\vd}, a novel theoretical proposal. I will first describe the general properties of {\vd} and how the template it gives rise to, {\thif}, participates in alternations. We will then look more closely at the relationship between a verb in {\tkal} and its alternants in {\tnif} and {\thif}, in Section @@. This discussion will be followed by a brief examination of cases in which anticausative verbs appear in {\thif}, in Section @@. Section @@ considers some alternative approaches.

\section{Causative marking}
Recall the core observation motivating the trivalent system: Hebrew is morphological marking of both causative and anticausative forms, in addition to a ``simple'' verbal form, (\nextx).
\ex\label{vd:ex:alternations-heb}
	\begin{tabular}{ll|ll|ll}
	\multicolumn{2}{P{4.2cm}|}{causative} &	\multicolumn{2}{P{4cm}|}{underspecified}	& \multicolumn{2}{P{4.2cm}}{anticausative}\\\hline
	\multicolumn{2}{c|}{\thif}	&	\multicolumn{2}{c|}{\tkal}	& \multicolumn{2}{c}{\tnif}\\
	\emph{heexil}	& `fed' &	\emph{axal}	& `ate'	&	\emph{neexal}	& `was eaten' \\
	\emph{hextiv}	& `dictated' &	\emph{katav}	& `wrote'	&	\emph{nixtav}	& `was written' \\
	\end{tabular}
\xe

	\subsection{An active Voice head}
The template {\thif} is traditionally called the ``causative'' one, since verbs in it are often causative versions of a verb in {\tkal} (or {\tnif}), as in~(\nextx a). Many verbs are also causative without alternating, as in~(\nextx b), and others are unergative,~(\nextx c). This is the usual case in {\thif}: over 500 of the 550--600 verbs in this template would fit in this list.
\ex\label{vd:ex:alternations-heb}
	\begin{tabular}{lll|ll|ll}
	& \multicolumn{4}{c|}{anticausative} & \multicolumn{2}{c}{causative}\\
	& \multicolumn{2}{c|}{\tnif}	&	\multicolumn{2}{c|}{\tkal}	& \multicolumn{2}{c}{\thif}\\\hline
	a.& \emph{nixnas} & `entered' & && \emph{hexnis} & `inserted'\\
	 & \emph{notar} & `remained' & && \emph{hotir} & `left behind'\\
	 & \emph{nikxad} & `went extinct' & && \emph{hekxid} & `eradicated'\\
	 & \emph{ni{\texttslig}al} & `was saved' & && \emph{he{\texttslig}il} & `saved'\\
	 & \emph{nee{\texttslig}av} & `was saddened' & && \emph{hee{\texttslig}iv} & `saddened'\\
	 & \emph{nexlaʃ} & `grew weak' & && \emph{hexliʃ} & `weakened'\\
	 & && \emph{kafa} & `froze' & \emph{hekpi} & `froze'\\
	 & && \emph{baar} & `burned' & \emph{hevir} & `lit up'\\
	 & && \emph{xazar} & `returned' & \emph{hexzir} & `returned'\\
	 & && \emph{tava} & `drowned' & \emph{hetbia} & `drowned'\\
	 & && \emph{jaʃav} & `sat down' & \emph{hoʃiv} & `sat down'\\
	 & && \emph{paxad} & `was afraid' & \emph{hefxid} & `scared'\\
	 & && \emph{rakad} & `danced' & \emph{herkid} & `caused to dance'\\
	 \hline
	b.& &&&& \emph{heʃmid} & `destroyed' \\
	& &&&& \emph{heir} & `illuminated'\\
	& &&&& \emph{hevis} & `defeated'\\
	& &&&& \emph{hegdir} & `defined'\\
	& &&&& \emph{hezmin} & `invited'\\
	& &&&& \emph{heka} & `struck'\\
	& &&&& \emph{hesnif} & `sniffed'\\
	& &&&& \emph{heflil} & `incriminated'\\
	\hline
	c.	& &&&&  \emph{hedrim} & `went south' \\
		& &&&&  \emph{hegzim} & `exaggerated' \\
		& &&&&  \emph{heflig} & `set sail' \\
		& &&&&  \emph{heria} & `cheered' \\
		& &&&& \emph{heezin} & `listened'\\
		& &&&& \emph{hemtin} & `waited'\\
		& &&&& \emph{heskim} & `agreed'\\
	\end{tabular}
\xe

Many of the unergatives are most natural with an indirect object, e.g.~\emph{hemtin le-} `waited for'. The purpose of the list in~(\lastx) is not to exhaust the possible syntactic configurations of verbs in {\thif}, simply to show why this template has traditionally been viewed as causative. But it is more accurate to say that the template is predominantly \emph{active}, i.e.~it has an agentive, external argument.

If it is true that there is an external argument regardless of the root, the presence of this argument should be encoded in the syntax. This goal is achieved using {\vd}: it requires that a DP be merged in its specifier, guaranteeing that an external argument appear. Remember that I am abstracting away from the difference between Agents and Causers (Chapter @@).
\pex 
	\a \textbf{\vd:}\\
	{\vd} is a Voice head with a [\!+\!D] feature, requiring that some element with a [D] feature merge in its specifier.
	\a \denote{\vd} = $\lambda$x$\lambda$e.Cause(x,e)
\xe
This semantics of {\vd} in~(\lastx) will be slightly modified in Section @. The structure is straightforwardly as in~(\nextx), where {\vd} obligatorily introduces an external argument.
\ex\label{vd:tree:thif}
\Tree
        [.VoiceP
            [.DP ]
            [
                [.{\vd}\\\emph{he-} ]
                [.vP
                    [.v
                        [.v ]
                        [.\root{\gsc{ROOT}} ]
                    ]
                    [.(DP) ]
                ]
            ]
        ]
    \xe

This proposal is, on its own, enough to describe most of the empirical landscape. In Section @@ I return to a handful of cases in which inchoative verbs appear in {\thif}. But first, there are a number of consequences of causation that are worth exploring.

	\subsection{The causative-causative alternation}
Recall the basics of the anticausative alternation which we have been assuming. Both (marked) anticausatives and (unmarked) causatives share a common base, formally the vP. This phrase is a predicate over eventualities, to which Voice can add an external argument \citep{schaefer08,layering15}, (\nextx).
\pex 
	\a \emph{John} \textbf{Voice} [_{\text{vP}} \emph{broke the glass}].
	\a \textbf{\vz} [_{\text{vP}} \emph{The glass broke}].
\xe
But if this is the case, how and why does {\vd} differ from underspecified Voice? If Voice allows the grammar to add an external argument, what's left for an additional device (\vd) to do, the hypothetical (\nextx)?
\ex \emph{John} \textbf{\vd} [_{\text{vP}} \emph{broke the glass}].
\xe

Since the syntactic behavior is identical as far as licensing a specifier is concerned, in this section I will discuss the semantic difference between the resulting causative verbs. Concretely, I will argue that Voice-causatives are more transparent than {\vd}-causatives, whereby the morphological markedness of the latter mirrors some semantic markedness or opacity.

		\subsubsection{Markedness}
Very little contemporary work has analyzed causative alternations in depth within a general theory of argument structure alternations; such work normally draws on languages that are typologically distant from Indo-European, like Japanese \citep{jacobsen92}. \citet[62f3]{layering15} speculate that a marked causative should entail thematic Voice (semantically if not syntactically), but as far as I know no proposal has explored the implications of marked anticausatives and marked causatives existing side by side.

For explicitness, let us give the two alternations the following names:
\ex
\begin{tabular}{l|ccc}
	&	Anticausative & Causative & Causative\\\hline
Basic alternation	& {\vz} & Voice &   \\
Marked alternation		&		&  Voice & {\vd}\\
\end{tabular}
\xe

The first question to ask is how productive these two alternations are. As already noted above, there are many roots showing both {\tkal} (Voice) and {\thif} (\vd) verbal forms. A rough count is as follows:

{\tnif} total @@
	{\tnif} anticausatives @@
		Basic alternation with {\tkal} @@
{\thit} total @@
	{\thit} anticausatives @@
		Basic alternation with {\tpie} @@
{\thif} total 550@
	{\thif} actives 500@
		Marked alternation with {\tkal} @@

It is fairly clear that both types of alternation are well attested.

The second question is whether there is a difference between the semantics of the alternations, and here I believe the basic alternation is more transparent. The question is one of predictability: given the anticausative variant, can we predict the meaning of the causative variant. In the basic alternation, the answer is usually affirmative, just like with the prototypical \emph{break}-\emph{break} and \emph{open}-\emph{open} examples in English; the Hebrew cases were discussed at depth in Chapter @@. A few examples were seen again above, one is elaborated on in~(\nextx):
\pex
	\a \emph{nixtav} `was written'\\
		Writing event of a DP, no cause specified.
	\a \emph{katab} `wrote'\\
		Writing event of a DP, external argument specified in the syntax and understood as agent.
\xe
I suggest that the causative variant of a basic alternation introduces a direct causer \citep{bittner99,kratzer05} but that the external argument in the marked alternation is less restricted.\footnote{A similar intuition was expressed in \citealt{doron03}, where the strongest claims about a template's meaning were limited to cases in which a root alternates between two templates.} For the marked causative, the informal phrasing in~(\nextx) will do for now (I return below to the question of whether a writing event even holds in these cases):
\ex \emph{hextiv} `dictated'\\
	Writing event of a DP, external argument specified in the syntax and understood as an (indirect) causer.
\xe

In terms of syntax, there is no difference between the constructions. 
\pex 
	\a \begingl
		\gla ha-talmidim \textbf{katvu} et ha-nosim//
		\glb the-students wrote \gsc{ACC} topics//
		\glft `The students wrote down the list of topics.'//
		\endgl\\
		\Tree [. [.students ] [. [.Voice ] [. [.\root{\gsc{WROTE}} ] [.topics ] ] ] ]		
	\a \begingl
		\gla ha-more \textbf{hextiv} et ha-nosim (la-talmidim)//
		\glb the-teacher dictated \gsc{ACC} topics to.the-students//
		\glft `The teacher dictated the list of topics (to the students).//
	\endgl\\
		\Tree [. [.teacher ] [. [.{\vd} ] [. [.\root{\gsc{WROTE}} ] [.topics ] ] ] ]
\xe

In the basic alternation, adding a causer to writing immediately identifies the writer. But adding the marked causer changes the event slightly: the teacher does not cause writing to occur, strictly speaking. Rather, he is the causer of a dictating event, which itself brings about the writing down of the topics.

A similar pattern can be seen with \emph{neexal} `was eaten', where the basic variant \emph{axal} means `ate' and the marked variant \emph{heexil (et)} means `fed (s.o.~s.th)'. Why should this be the meaning, and not `made someone eat'? The different kind of event (feeding versus causing to eat) also implies a different position for the eater: subject in the basic variant, object in the marked variant.
\pex
	\a \begingl
		\gla beki axl-a uxmanjot//
		\glb Becky ate-\gsc{F} blueberries//
		\glft `Becky ate blueberries.'//
		\endgl\\
		\Tree [. [.\textbf{Becky} ] [. [.Voice ] [. [.\root{\gsc{ATE}} ] [.blueberries ] ] ] ]
	\a \begingl
		\gla aba heexil et beki (uxmanjot)//
		\glb dad fed \gsc{ACC} Becky blueberries//
		\glft `Dad fed Becky (blueberries).'//
		\endgl\\
		\Tree [. [.Mary ] [. [.{\vd} ] [. [.\root{\gsc{ATE}} ] [.\textbf{Becky} ] ] ] ]
\xe

This contrast illustrates the limits of syntax within the current system: it can rigidly dictate which elements go where, but the structure itself is not driven by semantic or lexical-semantic considerations.

A few more examples of the marked alternation are given in~(\nextx), showing that the exact type of ``causative'' relation for the marked variant is not uniform.
\pex\label{vd:ex:triplets-caus}
		\begin{tabular}{l|ll|ll|llcc}
		\multicolumn{7}{c}{}		& Make O V	& Make O be V-ed\\\hline
		 a.& \emph{neexal}	& `was eaten'	& \emph{axal} & `ate'		& \emph{heexil} & `fed'			& \cmark	& \xmark\\
		b.& \emph{nikra}	& `was read'	& \emph{kara} & `read'		& \emph{hekri}	& `read out'	& \cmark	& \xmark \\\hdashline
		c.&	\emph{nidxak}	& `was pushed aside'	& \emph{daxak}	& `shoved'	& \emph{hedxik}	& `suppressed'\footnotemark	& \xmark	& \cmark\\
		d.& \emph{nilxa\texttslig}	& `was pressed' &  \emph{laxa\texttslig} & `pressed'	& \emph{helxi\texttslig} & `pressured'	& \xmark	& \cmark \\\hdashline
		e. & \emph{nidgam} & `was sampled'	& \emph{dagam} & `sampled'	& \emph{hedgim}		& `demonstrated'	& \xmark	& \xmark\\
		f.& \emph{nisgar}	& `closed'	& \emph{sagar} & `closed'		& \emph{hesgir} & `extradited'	& \xmark	& \cmark?\\
		\end{tabular}
\xe
%jalad nolad holid
\footnotetext{E.g.~memories or emotions.}

The last examples,~(\lastx e--f), are particularly revealing: to extradite someone is in no way an obvious semantic extension of ``closing'' them. More examples like this can be found, in which the basic alternant has predictable semantics but the marked one does not.

Before we go on to discuss the consequences of these differences, it is important to note a possible objection. The fact that marked causatives can vary so widely in their interpretation from the basic variants could be taken as an argument against treating these verbs as sharing the same abstract root. That is to say, why should we even think that closing and extraditing share the same root? Would that not be stretching its assumed shared semantics too thin? I will not defend the current position here, as I believe the overall picture emerging from this book and from work treating roots more directly is that we do want to discuss abstract roots, but be more specific in what their shared meaning is and under which circumstances it can vary.

More concretely, however, we can make an argument from the lack of doublets. There are no \emph{additional} verbs in {\tnif} that alternate with {\thif}. That is to say, suppose that \emph{nikra} `was read' and \emph{kara} `read' are derived from \root{krj1} and that \emph{hekri} `read out' was derived from a homophonous root \root{krj2}. Assume similarly that \emph{nisgar} `closed' and \emph{sagar} `closed' were derived from \root{sgr1} but that \emph{hesgir} `extradited' was derived from \root{sgr2}. And so on for all cases of non-predictable causative variants. If this were the case, we would suppose that \root{krj2} and \root{sgr2} could also be instantiated with other functional heads, for instance with {\vz} to create an anticausative in {\tnif}. But this is not the case: there is no \#\emph{nikra} `was read out' to alternate with \emph{hekri} `read out', and no \#\emph{nisgar} `got extradited' to alternate with \emph{hesgir} `extradited'. In other words, there are no \emph{doublets} (compare the discussion of root suppletion in \citealt{harley14thlia,harley14thlib,harley15roots} and \citealt{borer14thli}).

		\subsubsection{Markedness in Voice heads}
The observations made so far bring us to the proposed generalization for causitivity marking:
\pex \textbf{The causative generalization for transitivity alternations}\\
	If a language has both anticausative and causative marking:
	\a The anticausative alternation is transparent.
	\a The causative alternation is not (root-specific).
\xe

This discussion of causative marking will conclude by examining the extent to which~(\lastx) can be derived directly from our theoretical assumptions. I believe that it can. Concretely, this generalizations follows directly from the general layering approach to transitivity alternations. If the base vP already has causation, then it is clear what \emph{not} adding an external argument would mean: that is the anticausative alternant. Adding an external argument, as noted above, amounts to introducing the most direct agent. This much derives~(\lastx a).

We will need to assume that structures derived with different Voice heads will have different meanings, perhaps by some principle of economy. The question is now how to differentiate between the two meanings, and the answer is that adding an external argument necessitates a change in meaning as well.

Say we have an event of causation, (\nextx):
\ex
\Tree
	[.vP
		[.v ]
		[.DP ]
	]
\xe


Not adding a causer is easy, (\nextx):
\ex
\Tree
[.VoiceP
	[.{\vz} ]
	[.vP
		[.v ]
		[.DP ]
	]
]
\xe

Then, various kinds of external arguments can go with different causation events:
\ex a.
\Tree
[.VoiceP
	[.DP_1 ]
	[.
		[.Voice ]
		[.vP
			[.v ]
			[.DP ]
		]
	]
]
b.
\Tree
[.VoiceP
	[.DP_2 ]
	[.
		[.{\vd} ]
		[.vP
			[.v ]
			[.DP ]
		]
	]
]
\xe

This result makes sense if {\vd} is a marked head which only appears once the language already has underspecified Voice. In other words, the two heads stand in an implicational relationship and we do not expect to find a language with {\vd} (and {\vz}) but without Voice.

On the other hand, it is not possible to have various kinds of \emph{lack} of external arguments. This point brings us to a novel prediction, namely that a specific kind of argument structure triplet should be highly rare, if not impossible.
\pex \textbf{The triplet prediction for argument structure alternations}\\
	If a language has both anticausative and causative marking:
	\a Triplets of the form [marked unaccusative $\sim$ unmarked \textbf{causative} $\sim$ marked causative] may be possible.
	\a Triplets of the form [marked unaccusative $\sim$ unmarked \textbf{uanccusative} $\sim$ marked causative] will not be possible.
\xe	

We have already seen examples of triplets such as those predicted by~(\lastx a) to exist in~(\ref{vd:ex:triplets-caus}). Those like~(\lastx b) are much more difficult to find. The following two triplets can be argued to exist in Hebrew:
\pex 
	\a \root{xrv} `\root{\gsc{DEMOLISH}}': \emph{nexrav} $\sim$ ??\emph{xarav} $\sim$ \emph{hexriv}
	\a \root{ʃlm} `\root{\gsc{BE.WHOLE}}': \emph{niʃlam} $\sim$ ??\emph{ʃalam} $\sim$ \emph{heʃlim}
\xe
In both cases, the {\tkal} form is archaic and exists in contemporary speech only in a few set idioms, if at all. Speakers seem to prefer the {\tnif} form for the anticausative.

A final points about the semantic flexibility of {\vd} concerns its productivity. The template {\thif} is a productive causative template \citep{lev16,kastner18tlr} in which speakers may innovate new forms on the fly. The verb \emph{taka} `stuck' is an ordinary transitive verb in Hebrew, but the online comment in~(\nextx) innovates \emph{hetkia} in {\thif}, presumably for effect. The article concerns a roller coaster which became stuck in mid-ride on a Saturday, stranding those riding it for the better part of an hour.
\ex `Why don't you understand that the roller coaster also wanted to observe the Sabbath and rested 40 minutes {\dots}' \\
	\begingl
	\gla {...} elokim hevi la-xem siman ʃe-lo taalu al ha-mitkan be-ʃabat ve-hine hu \textbf{hetkia} et-xem le-40 dakot be-jom ʃabat kodeʃ//
	\glb {} G-d brought to-you sign that-\gsc{NEG} you.will.rise on the-device in-Saturday and-here he stuck \gsc{ACC-2PL} for-40 minutes in-day Saturday holy//
	\glft `{...} G-d gave you a sign no to go on the ride on Saturday, and there you go, he made you be stuck for 40 minutes on the holy Sabbath.'\trailingcitation{\url{https://www.ynet.co.il/Ext/App/TalkBack/CdaViewOpenTalkBack/0,11382,L-3441716-7,00.html}}//
	\endgl	
\xe
Additional examples can be found in \cite{lev16}.

In such cases, there is no strong prediction with regards to the kind of causation event; our expectation would be that different kinds of causation would be possible, as was the case for the examples in~(\ref{vd:ex:triplets-caus}). This much also seems to be correct. The verb \emph{hexʃid}, from \emph{xaʃad} `suspected', is attested in both readings: `be suspected', `be made into a suspect' in~(\nextx) and `make X suspect s.th' in~(\anextx).

\pex Make O V-ed (turn into a súspect)
	\a \begingl
		\gla be-tviat-am toanim ha-ʃnaim ki gilboa \textbf{hexʃid} et deri be-re{\texttslig}ixat-a ʃel ester verderber//
		\glb in-lawsuit-theirs claim the-two that Gilboa suspect.\gsc{CAUS} \gsc{ACC} Deri in-murder-hers of Esther Verderber//
		\glft `The two claim in their lawsuit that Gilboa turned Deri into a suspect in the murder of Esther Verderber.'\trailingcitation{\url{https://www.ynet.co.il/articles/0,7340,L-2443354,00.html}}//
	\endgl
	\a \begingl
		\gla ha-seruv \textbf{hexʃid} et netanjahu ve-sar-av ha-krovim ki re{\texttslig}on-am litol le-a{\texttslig}mam samxut-al jexudit, ve-lo linhog be-ʃkifut//
		\glb the-refusal suspect.\gsc{CAUS} \gsc{ACC} Netanyahu and-ministers-his the-close that will-theirs to.take to-themselves authority-superior unique and-\gsc{NEG} to.behave in-transparency//
		\glft This refusal makes one suspect that Netanyahu and his closest ministers wish to avail themselves of unique authority, rather than conduct themselves transparently.\trailingcitation{\url{https://www.israelhayom.co.il/opinion/294269}}//
	\endgl
\xe
	
\pex Make X verb (make someone suspéct s.th, make someone suspicious)
	\a \begingl
		\gla ʃalom lifnej beerex 5 elef hexlafti galgalʃ kolel ʃarʃeret (z750 2010) ve-ha-mexir ʃe-kibalti k{\texttslig}at \textbf{hexʃid} ot-i//
		\glb hello before about 5 thousand I.changed sprocket including chain (z750 2010) and-the-price that-I.got a.little suspect.\gsc{CAUS} \gsc{ACC}-me//
		\glft `Hello, I changed my gear and chain (z750 2010) about five years ago and the price I got made me a little suspicious.'//\trailingcitation{\url{http://fullgaz.co.il/forums/archive/index.php/t-793.html}}//
	\endgl
	\a \begingl
		\gla galaj ha-mataxot lo hetria u-{vexol zot} ha-falestini \textbf{hexʃid} et loxamej {miʃmar ha-gvul}//
		\glb detector the-metal \gsc{NEG} warn and-nevertheless the-Palestinian suspect.\gsc{CAUS} \gsc{ACC} warriors.of {the Border Patrol}//
		\glft `The metal detector did not give any warning but nevertheless, the Palestinian aroused the suspicion of the Border Patrol soldiers.'\trailingcitation{\url{http://www.93fm.co.il/radio/445111/}}//
	\endgl
\xe

%Remember: ``[T]here is actually no coherent lexical semantic or conceptual reasoning available as to why an \emph{individual} verb (or verbal) concept in an \emph{individual} language shows up in one or the other class.'' \citep[65]{layering15}


%Entailments?
%	John opened the door but it didn't open
%	John fed the baby but it didn't eat
%	The teacher dictated the list of essay topics but the students didn't write them down
%	
%	Hans öffnete die Tür, aber sie hat sich nicht geöffnet
%	Hans fütterte das Kind, aber es hat nicht(s) gegessen
%	
%	
%	dani patax et ha-delet aval hi lo niftexa
%	tereza heexila et ha-tinok aval hu lo axal
%	ha-more hextiv et reSimat ha-nosim la-xibur aval ha-talmidim lo katvu otam
%
%\cite{fodor70}


	
	\subsection{Alternative: Added structure}



\section{Inchoative alternations}



\begin{table}[ht] \small \centering \singlespacing
	\begin{tabular}{|p{3cm}||p{4.5cm}|p{4.5cm}||p{3cm}|}\hline
		&	Unaccusative	&  Unergative & Transitive \\\hline\hline
	Change of color & \emph{he'edim} `reddened', \emph{helbin} `whitened', \emph{hekxil} `became blue', \emph{hetshiv} `yellowed', \emph{heʃxir} `blackened', \emph{hezhiv} `goldened', 
		& --- & --- \\\hline
	
	Change of bodily function, shape or appearance & \emph{heʃmin} `fattened', \emph{herza} `thinned', \emph{hezkin} `grew old', \emph{hekriax} `became bald', \emph{hevri} `became healthy', \emph{hertsin} `became serious', \emph{hexvir} `grew pale' &
		\emph{he'emik} `deepened', \emph{he'erix} `lengthened', \emph{hetser} `narrowed', \emph{hesmik} `blushed' & \emph{hefʃit} `undressed', \emph{henmix} `lowered', \emph{hextim} `stained', \dots \\\hline

	Change of consistency, taste or smell & \emph{hekʃiax} `stiffened', \emph{hefʃir} `thawed', \emph{hevʃil} `ripened', \emph{hekrim} `crusted'
		& \emph{hexmits} `soured', \emph{herkiv} `rotted'
		& \emph{hetsis} `fermented', \emph{heriax} `smelled', \emph{hetpil} `desalinated', \emph{heflir} `flouridated', \dots \\\hline
	
	Emission & --- & \emph{heki} `threw up', \emph{hesriax} `stank', \emph{hezia} `sweat', \emph{heflits} `farted', \emph{hev'iʃ} `became putrid', \emph{hetsxin} `smelled pungent' & --- \\\hline
	
	Change of speed or direction & --- & \emph{he'its} `sped up', \emph{he'et} `slowed down', \emph{hesmil} `went left' & \emph{heziz} `moved', \emph{hotsi} `removed', \dots \\\hline
	
	Change of sound & --- & \emph{her'iʃ} `made loud noise', \emph{hexriʃ} `quieted down' & \emph{heʃtik} `shut up' \\\hline
	
	Other & \emph{hetsliax} `succeeded', \emph{hexmir} `deteriorated' & \emph{hektsin} `escalated' &  \\\hline
	\end{tabular}
\caption{Lexical semantic classes for alternating verbs in \thif~and transitive foils.\label{table:thif-roots}}
\end{table}


The verbal morphology of Modern Hebrew famously consists of seven ``templates'', in which consonants (here X, Y and Z) slot into vocalic and affixal patterns \citep{doron03,arad05,borer13oup,kastner16phd}. Our focus is on the template {\thif}, often called the ``causative'' form; verbs in it are generally active, as in~(\nextx).\footnote{This template usually appears in the literature as \emph{h\textbf{i}XYiZ}, with an /i/-/i/ vocalic pattern. Yet contemporary speakers use /\textipa{E}/ \citep{trachtman16}, and so I transcribe ``e'' throughout. Conversely, the initial /h/ is usually dropped in speech but I retain it for two reasons. First, /h/ is still pronounced by some older speakers and certain sociolinguistic groups, often marginalized ones \citep[cf.~][]{schwarzwald81biu,gafter14phd}. And second, the initial \emph{h}- should help non-Semitist readers to distinguish this template from other ones.}
\pex\label{ex:thif-active}
	\a \textbf{Causative:} \emph{he\textipa{S}mid} `destroyed', \emph{hexnis} `inserted', \emph{hekpi} `froze', \dots
	\a \textbf{Unergative:} \emph{he{\texttslig}pin} `went north', \emph{hedrim} `went south', \emph{hejmin} `went to the right', \emph{heemin} `believed', \emph{hegzim} `exaggerated', \emph{hemtin} `awaited', \emph{heflig} `set sail', \emph{hebit} `looked', \emph{heria} `cheered', \dots
\xe

Hebrew does not generally have a labile alternation (zero-derivation as in English \emph{break}$\sim$\emph{break}), with the exception of certain verbs in {\thif}.\footnote{A handful of examples in other templates includes \emph{a{\texttslig}ar} `stopped' (often dispreferred to \emph{nee{\texttslig}ar} as an inchoative), \emph{miher} `hurried' and \emph{ixer} `delayed', with the latter two suggested by a reviewer and attested in use though not part of my own causative vocabulary.} The list in~(\lastx)---which is not meant to be comprehensive---presents a number of verbs that do not participate in the alternation. This is the unmarked case in \thif: over 500 of the 550--600 verbs in this template would fit in this list.

I will use \emph{inchoative} as a descriptive term: an inchoative verb in {\thif} is one in which the sole argument has undergone the change of state (or changed on a scale). \emph{Causative} is likewise a descriptive term here, identical in use to \emph{transitive}: a structure with an external argument and an internal argument (complement to the verb). The two kinds will receive different analyses in Section~\ref{sec:template}.

Some examples of verbs that do undergo the alternation are given in~(\nextx). Even in those cases where the inchoative is frequent, a causative context can be set up fairly easily. Full lists are given later on, in Section~\ref{sec:roots}.
\pex\label{ex:thif-alt}Alternating unergatives in \thif:
	\a \textbf{Full alternation:} \emph{hei{\texttslig}} `sped up', \emph{heemik} `deepened', \emph{heerix} `lengthened', \emph{hek\textipa{S}iax} `stiffened', \emph{hef\textipa{S}ir} `thawed', \emph{he\textipa{S}min} `fattened', \emph{herza} `grew thin', \dots
	\a \textbf{Unergative preferred but causative innovation attested:} \emph{hesriax} `stank', \emph{hesmil} `went to the left',\footnote{Attested example for causative ``leften'':
	\ex
		\begingl
		\gla kol ha-kavod le-barak. \textbf{hesmil} et netanjahu//
		\glb all the-respect to-Barak. made.left \gsc{ACC} Netanyahu//
		\glft `Well done to [Ehud] Barak. He made [Benjamin] Netanyahu look like a leftist.'\trailingcitation{	\url{http://www.ynet.co.il/Ext/App/TalkBack/CdaViewOpenTalkBack/0,11382,L-4010352,00.html}}//
		\endgl
	\xe
	} \emph{hetsxin} `smelled pungent', \emph{herkiv} `rotted', \dots
	\a \textbf{Unaccusative preferred but causative innovation attested:} \emph{he'edim} `reddened', \emph{helbin} `whitened', \emph{he\textipa{S}xir} `blackened', \emph{hevri} `got healthy',
		\emph{hexvir} `grew pale',\footnote{Attested example for causative ``palen'':
		\ex ``The girl looked as though someone wrapped her up in massive metallic toilet paper. \dots
			\begingl
			\gla afilu ha-tseva ha-meanjen \emph{[}\dots\emph{]} \textbf{hexvir} et hofa'a-ta \textipa{S}el danst//
			\glb even the-color the-interesting {} paled \gsc{ACC} appearence-hers of Dunst//
			\glft `Even the interesting color \dots~made Dunst's appearance pale.'\trailingcitation{\url{http://www.mako.co.il/women-fashion/whats_in/Article-174f70ed642f121004.htm}}//
			\endgl
		\xe
		} \emph{hertsin} `became serious', \dots
\xe

The alternation is further exemplified by \emph{hef\textipa{S}ir} `thawed' in~(\nextx).
\pex\label{ex:thif-hefSir}
	\a \begingl
		\gla ha-jaxasim ben \textipa{S}tej ha-medinot \textbf{hef\textipa{S}ir-u} axarej bikur ro\textipa{S} ha-mem\textipa{S}ala//
		\glb the-relations between both the-states thawed.\gsc{CAUS}-\gsc{3PL} after visit head.\gsc{CS} the-government//
		\glft `The relations between the two countries thawed after the PM's visit.'//
		\endgl
	
	\a \begingl
		\gla bikur ro\textipa{S} ha-mem\textipa{S}ala \textbf{hef\textipa{S}ir} et ha-jaxasim ben \textipa{S}tej ha-medinot//
		\glb visit head.\gsc{CS} the-government thawed.\gsc{CAUS} \gsc{ACC} the-relations between both the-states//
		\glft `The PM's visit thawed the relations between the two countries.'//
		\endgl
\xe

This paper attempts to understand, on the one hand, what is special about the roots in~(\ref{ex:thif-alt}) that allows their verbs to alternate, and on the other hand, what is special about the morphological template that allows some verbs to alternate. A satisfying analysis of these patterns must address two questions: why these roots and why this template. I take these questions up in turn.

%Unaccusativity judgments are fickle in Hebrew: the ``possessive dative'' of \cite{borergrodzinsky86} has been critiqued by \cite{gafter14li} and \cite{linzen14pd}, while verb preposing (Verb-Subject order in an otherwise SVO language, \citealt{shlonsky87}) is not always reliable. See \cite{kastner17gjgl} for discussion. Nevertheless, it is possible to find unaccusative verbs in \thif~which perform satisfactorily on the Verb-Subject order diagnostic, as the following examples show. \citet[149]{borer91} likewise argues that inchoatives in \thif~can be either unergative or unaccusative.
%\pex VS order with unaccusative inchoatives in \thif. No \emph{by}-phrase possible.
%	\a \begingl
%		\gla \underline{hef\textipa{S}ir-a} \emph{(}l-i\emph{)} kol ha-glida \emph{(}*{al jedej} ha-xom\emph{)}//
%		\glb thawed-\gsc{F} to-me all the-ice.cream \phantom{*(}by the-heat//
%		\glft `All (my) ice cream defrosted completely (*by the heat).'//
%	\endgl
%	
%	\a \begingl
%		\gla \underline{hev\textipa{S}il-u} ha-tna'im le-haf\textipa{S}ara ba-jaxasim \emph{(}*{al jedej} ha-bikur\emph{)}//
%		\glb ripened-\gsc{3PL} the-conditions to-thawing in.the-relations \phantom{*(}by the-visit//
%		\glft `The conditions matured for the relations to thaw (*by the visit).'//
%	\endgl
%\xe
%Accordingly, I will assume that all three constructions (transitive, unergative and unaccusative) are possible in this template in principle.
%
%I have classified the alternating verbs by the alternations they participate in. Barring a judgment survey, and given that I know of no comparable lists, the following lists reflect my own intuitions. Some verbs are equally acceptable in both alternations, as with \emph{hef\textipa{S}ir} `thawed' in~(\ref{ex:thif-hefSir}). Others are predominantly used as inchoatives, though causative uses have been attested. This procedure is repeated separately for unaccusatives in~(\ref{ex:thif-unacc-alt}) and for unergatives in~(\ref{ex:thif-act-alt}).

%I have not yet found, or at least not yet noticed, any alternations in which the causative is preferred and the inchoative is a recent innovation; or inchoatives in \thif~which have no causative counterpart. I take these findings to be emblematic of the causative meaning inherent in \thif: even if inchoative verbs have arisen, contemporary usage overwhelmingly tends to coin causatives in this template rather than another kind of verb \citep{laks14}.


\section{Roots} \label{sec:roots}
Based on the corpus of~\cite{ehrenfeld12}, approximately 550--600 verbs exist in Modern Hebrew {\thif}. Of these, 33 show the causative-inchoative alternation by my own estimate.\footnote{\cite{arad05} counted 11 such verbs in her corpus whereas \cite{laks11} found 34. \cite{lev16} counted 81 in a survey taking into account many naturally attested, but perhaps spurious, forms.} The 33 alternating verbs are broken down as follows: 15 alternating unergatives and 18 alternating unaccusatives. 

The analytical question is whether we can identify which roots form verbs that participate in the labile alternation. In this section I classify these verbs according to the broad lexical semantics of the root, building towards the claim that they are all degree achievements. The resulting classification is necessarily based on the existing verbs in {\thif}, rather than on categories such as those in \cite{levin93}.

	\subsection{Classification}
The classification also notes whether these verbs are unaccusative or unergative. Inchoatives demonstrably lack a Causer: they are compatible with `by itself' \citep{unaccusativity95,alexiadouanagnostopoulou04,koontzgarboden09,alexiadoudoron12,kastner17gjgl}, for example, (\nextx).
\ex
	\begingl
	\gla ha-\textipa{S}eleg \textbf{hef\textipa{S}ir} me-a{\texttslig}mo//
	\glb the-snow thawed.\gsc{CAUS} of-itself//
	\glft `The snow thawed by itself.'//
	\endgl
\xe

They also pass unaccusativity diagnostics, although these are are fickle in Hebrew: the ``possessive dative'' of \cite{borergrodzinsky86} has been critiqued by \cite{gafter14li}, \cite{linzen14pd} and \cite{barashersiegalboneh15,barashersiegalboneh16}, while verb preposing (Verb-Subject order in an otherwise SVO language, \citealt{shlonsky87}) is not always reliable. See \cite{kastner17gjgl} for discussion. Nevertheless, it is possible to find unaccusative verbs in {\thif} which perform satisfactorily on the Verb-Subject order diagnostic. Barring a judgment survey, and given that I know of no comparable lists, the following lists reflect my own intuitions and will not be motivated as such.

First off, there are classes that contain only unergative (alternating) inchoatives. These include \textbf{verbs of emission}: \emph{hesriax} `stank', \emph{hevi\textipa{S}} `became putrid', \emph{he{\texttslig}xin} `smelled pungent';\footnote{Other verbs of emission do not entail change of state: \emph{heki} `threw up', \emph{hezia} `sweat', \emph{hefli{\texttslig}} `farted'.} %No non-alternating transitive verbs of emission exist, as far as I could find.
%It is often unclear whether verbs of emission are unaccusative or unergative, with the answer often varying by language \citep{rosen84}. In Hebrew I treat them as unergative, given that they do not allow VS order (not shown here) and following \cite{potashnik12}.
\textbf{verbs of change of speed or direction}: \emph{hei{\texttslig}} `accelerated', \emph{heet} `slowed down', \emph{hesmil} `went left'; %Transitive verbs of this semantic class do exist, e.g.~\emph{heziz} `moved', \emph{ho{\texttslig}i} `removed'.
and \textbf{verbs of change of sound}: \emph{heri\textipa{S}} `made loud noise', \emph{hexri\textipa{S}} `quieted down'. %One transitive verb exists: \emph{he\textipa{S}tik} `shut up'.
%The last two clearly show change on a scale. Their transitive variant means ``caused NP to be Aer''. The first can either have a cognate-like object (what was thrown up), or also a causative meaning (e.g.~make the room pungent).
One more unergative verb which does not fit into any of these categories is \emph{hek{\texttslig}in} `escalated'.

Next we have classes whose inchoatives can be either unergative or unaccusative. For
\textbf{verbs of change of consistency, taste or smell}, the unergatives are \emph{hexmi{\texttslig}} `soured' and \emph{herkiv} `rotted'. The unaccusatives are \emph{hek\textipa{S}iax} `stiffened', \emph{hef\textipa{S}ir} `thawed', \emph{hev\textipa{S}il} `ripened' and \emph{hekrim} `crusted'. %Transitive-only verbs exist too, e.g.~\emph{he{\texttslig}is} `fermented', \emph{heriax} `smelled', \emph{hetpil} `desalinated', \emph{heflir} `flouridated'.
\textbf{For change of physical function, shape or appearance}, the unergatives are \emph{heemik} `deepened', \emph{heerix} `lengthened', \emph{he{\texttslig}er} `narrowed' and \emph{hesmik} `blushed'. The unaccusatives are \emph{he\textipa{S}min} `fattened', \emph{herza} `thinned', \emph{hezkin} `grew old', \emph{hekriax} `became bald', \emph{hevri} `became healthy', \emph{her{\texttslig}in} `became serious' and \emph{hexvir} `grew pale'. %Transitives include \emph{hef\textipa{S}it} `undressed', \emph{henmix} `lowered', \emph{hextim} `stained' and others.

And lastly, there are classes whose inchoatives are solely unaccusative. These are the \textbf{verbs of change of color}: \emph{heedim} `reddened', \emph{helbin} `whitened', \emph{hekxil} `became blue', \emph{he{\texttslig}hiv} `yellowed', \emph{he\textipa{S}xir} `blackened' and \emph{hezhiv} `goldened'. %No transitive-only verbs exist in this category.

Additionally, \emph{hexmir} `deteriorated' appears to be an unaccusative verb that does not fall under any of the categories above.

\subsection{Discussion}
A number of tentative generalizations can be drawn from the lists above: for instance, it seems clear that change of color allows for inchoative verbs (unaccusative ones). Yet a large degree of arbitrariness exists, as when we might also have expected the forms in~(\nextx) to exist, contrary to fact. The semantic criteria alone are not enough to predict how all roots in the language will behave.
\pex
	\a Change of speed:
		*\emph{hemhir} ($\nless$ \emph{mahir} `quick').
	\a Change of color:
		*\emph{hesgil} ($\nless$ \emph{sagol} `purple'), *\emph{hektim}/*\emph{hextim} ($\nless$ \emph{katom} `orange').
\xe

It is also not the case that any root in the categories above necessarily derives an inchoative in {\thif}: \emph{heziz} `moved' is a change of direction, \emph{he\textipa{S}tik} `shut up' is a change of sound and \emph{henmix} `lowered' is a change of physical shape, but these three verbs (and many others) are only causative, never inchoative.

One insightful claim, made recently by \cite{lev16}, it that inchoatives in {\thif} are degree achievements: change of state verbs which are derived from gradable adjectives (for related discussion see \citealt{dowty91,hayetal99,rotsteinwinter04,kennedylevin08,bobaljik12}, a.m.o). This hypothesis covers a fair bit of empirical landscape and I will adopt it in what follows, although it is important to note that this it does not account for a few datapoints which \cite{lev16} subsumes under a different part of his analysis. 

\cite{lev16} additionally sketches a theory in which labile verbs are less agentive in general, a claim that could explain why \emph{heet} `slowed down' is possible as opposed to *\emph{hemhir} (from `quick'), but that cannot be extended to explain the existence of minimally different \emph{hei{\texttslig}} `accelerated'. I leave the classification above as is, assuming the relevance of degree achievements and turning to examine the syntactic context.

%Lev: hesmik et (caused X to Aer) vs hezia et (cognate object) b/c blushing is externally controlled and sweating internally brought about

\section{Template} \label{sec:template}
Two questions must now be asked about {\thif}: first, what is the appropriate formal analysis for verbs with this form \citep{schwarzwald81,bolozky82}? And second, why do labile alternations exist in this template alone?
This section concerns itself with the first question of the two. I argue next that causatives have different structure than inchoatives, echoing claims made by \cite{borer91}. Causatives will be argued to be derived from the root, whereas inchoatives will be argued to be derived from an existing adjective or noun.%\footnote{From a cross-Semitic perspective, Arabic ``Form 9'' \emph{iXYaZZ} verbs show some parallels with \thif, though the Arabic forms are exclusively nonactive.} 

	\subsection{Causatives}
As noted in the introduction, verbs in {\thif} are almost always active (500+ out of the 550 or so). Furthermore, every inchoative has a causative alternation. If it is true that there is always an external argument, the presence of this argument should be encoded in the syntax. This goal is achieved using the functional head \textbf{\vd} \citep{schaefer08,wood15springer,nie17,osekikastner17,kastner18nllt}, the active counterpart of Voice \citep{kratzer96,pylkkanen08} which I implement within Distributed Morphology \citep{dm}: it requires that a DP be merged in its specifier, guaranteeing that an external argument appear, (\nextx)--(\anextx).\\
\begin{minipage}[t]{0.5\textwidth}
\pex 
	\a \textbf{\vd:} Voice with a [+D] EPP feature, requiring that some element with a [D] feature merge in its specifier.
	\a \denote{\vd} = $\lambda$x$\lambda$e.Cause(x,e) \hfill (or Agent)\\
	With additional allosemes to be introduced in~(\ref{ex:thif-sem}).
\xe
\end{minipage} \hfill
%This definition of \vd~will be amended in \S\ref{syn:templates:thif:analysis}, to a feature valuation operation (Agree) that probes first upwards and then downwards, so that we can account for a number of exceptions.
\begin{minipage}[t]{0.45\textwidth}
{
    \ex\label{tree:thif}
        \scalebox{0.8}{
        \Tree
        [.VoiceP
            [.DP ]
            [
                [.{{\vd}\\\emph{he-}} ]
                [.vP
                    [.v
                        [.v ]
                        [.\root{\gsc{ROOT}} ]
                    ]
                    [.(DP) ]
                ]
            ]
        ]
        }
    \xe
}
\end{minipage}

This head must be distinct from underspecified Voice, which I assume underlies the ``simple'' template {\tkal}. Given that the same root can be used to derive transitive verbs in both {\tkal} and {\thif}, (\nextx)---that is, verbs with identical syntactic structures---the difference between the forms must be due not to the structure but to the identity of Voice itself. For further motivation see \cite{kastner16phd,kastner17gjgl,kastner18nllt}, \cite{nie17} and \cite{osekikastner17}.\footnote{The representation of roots in~(\ref{table:kal-thif}) is meant as shorthand, obscuring many issues with their morphophonological analysis. See \cite{faust12,faust16} and \cite{kastner18nllt}.}
\ex\label{table:kal-thif}	\raisebox{-3em}{ \begin{tabular}{l|l|ll|ll}
	 & Root			& \multicolumn{2}{c|}{\tkal} & \multicolumn{2}{c}{\thif}\\\hline
	a.& \root{axl} & \emph{axal} & `ate'  & \emph{heexil (be-)} & `fed (with)'\\
	b.& \root{raj} & \emph{raa} & `saw'  & \emph{hera (le-)} & 		 `showed (to)'\\
	c.& \root{\textipa{S}ma} & \emph{\textipa{S}ama} & `heard'  & \emph{he\textipa{S}mia (le-)} & `played (to)'\\
	d.& \root{n\textipa{S}m} & \emph{na\textipa{S}am} & `breathed'  & \emph{hen\textipa{S}im} & `resuscitated'\\
	\end{tabular}
	}
\xe
		
%\ex\label{tree:thif2}
%    a. \begin{forest} qtree
%    [Causative:\\VoiceP
%        [DP_{\text{EA}} ]
%        [
%            [{\vd}\\\emph{he-} ]
%            [vP
%                [v
%                    [\textbf{v} ]
%                    [\root{\gsc{ROOT}} ]
%                ]
%                [DP_{\text{IA}} ]
%            ]
%        ]
%    ]
%    \end{forest} \phantom{xxxxxxx}
%    b. \begin{forest} qtree
%    [Inchoative:\\VoiceP
%        [DP_i ]
%        [
%            [{\vd}\\\emph{he-} ]
%            [vP
%                [v
%	                [\phantom{xx}v\phantom{xx} ]
%	                [a/n
%	                    [\textbf{a/n} ]
%	                    [\root{\gsc{ROOT}} ]
%	                ]
%                ]
%                [(DP)_i ]
%            ]
%        ]
%    ]
%    \end{forest}
%\xe
If we assume that the [D] feature on {\vd} is an EPP feature, we can capture the simple (causative) cases: both an external argument and an internal argument are merged in the structure. The external argument satisfies [D] on {\vd} and the derivation converges. Similarly for an unergative construction without the internal argument. But what of the inchoatives?

	\subsection{Inchoatives}
		\subsubsection{Structure}
As a first step, I will assume that inchoatives in {\thif} are never derived directly from the root, but from an underlying adjective or noun. A similar claim was already made by \cite{borer91}, who argued that causatives are derived directly from the root while these inchoatives are derived from an underlying adjective. As I point out here, inchoatives can also be derived from an underlying noun:
\pex
	\a Underlying adjective: \emph{he'edim} $<$ \emph{adom} `red', \emph{he\textipa{S}min} $<$ \emph{\textipa{S}amen} `fat'.
	\a Underlying noun: \emph{heki} $<$ \emph{ki} `vomit', \emph{he{\texttslig}xin} $<$ \emph{{\texttslig}axana} `stench'.
\xe

The structure is as in~(\nextx).
\ex
	\scalebox{0.9}{
	\Tree
 [.VoiceP
     [.DP$_i$ ]
     [
         [.{\vd}\\\emph{he-} ]
         [.vP
             [.v
              [.\phantom{xx}v\phantom{xx} ]
              [.a/n
                  [.a/n ]
                  [.\root{\gsc{ROOT}} ]
              ]
             ]
             [.(DP)$_i$ ]
         ]
     ]
 ]	
 }
\xe

This assumption is admittedly a bit of a morphophonological stretch in certain cases (the verb \emph{hei{\texttslig}} `accelerated' is arguably not derived from the noun \emph{teu{\texttslig}a} `acceleration', whose initial /t/ is not preserved), indicating that perhaps the claim should be weakened such that some inchoatives are derived from adjectives/nouns and other from the root. Nevertheless, the strong view carries a few benefits. First, it allows us to talk about different constructions in terms of explicit structures. Second, it allows for the degree semantics of the underlying adjective to transfer to the verb straightforwardly. And third, it makes a correct prediction. 

My theory of morphosemantics assumes the so-called Arad/Marantz hypothesis, according to which the first categorizing head selects the alloseme of the root \citep{arad03,marantz13,elenasamioti14}. If~(\lastx) is the right structure for inchoatives, it is predicted that for roots which participate in the alternation, the causative might have a meaning that the inchoative does not share. This is because in causatives {\vd} is local enough to the root to select a special meaning, whereas in inchoatives little a or little n will have already chosen an alloseme. This prediction is borne out by idioms involving \emph{helbin} `whitened', as in~(\nextx), and \emph{he\textipa{S}xir} `blackened' (with the metaphorical meaning `tarnished'), as in \citet[79]{kastner16phd}.
\pex
	\a Causative, literal meaning:\\
		\begingl
			\gla ha-sid \textbf{helbin} et ha-kir//
			\glb the-lime.plaster whitened \gsc{ACC} the-wall//
			\glft `The lime plaster made the wall white.'//
		\endgl
	
	\a Causative, non-transparent meaning:\\
		\begingl
			\gla sar ha-xuts \textbf{helbin} ksafim//
			\glb minister the-exterior whitened moneys//
			\glft `The Minister of Foreign Affairs took part in money laundering.'//
		\endgl
	
	\a Passive of causative, non-transparent meaning retained:\\
		\begingl
			\gla nitan \textipa{S}e-ha-ksafim \textbf{hulben-u} {al jedej} sar ha-xuts//
			\glb was.claimed \gsc{COMP}-the-moneys whitened.\gsc{PASS}-\gsc{3PL} by minister the-exterior//
			\glft `It was claimed that the money was laundered by the Minister of Foreign Affairs.'//
		\endgl
	
	\a Inchoative, only literal meaning:\\
		\begingl
			\gla ha-\textipa{S}tarot \textbf{helbin-u}//
			\glb the-bills whitened-\gsc{3PL}//
			\glft `The bills became white.'\\
				(not: `The bills got laundered.')//
		\endgl
\xe


%\pex
%	\a Causative, literal meaning:\\
%		\begingl
%			\gla ha-piax \underline{he\textipa{S}xir} et ha-avir//
%			\glb the-soot blackened \gsc{ACC} the-air//
%			\glft `The air grew black with soot.'//
%		\endgl
%	
%	\a Causative, non-transparent meaning:\\
%		\begingl
%			\gla son'e-j israel menas-im \underline{leha\textipa{S}xir} et pane-ha \textipa{S}el medina-t israel ba-zira ha-benleumit//
%			\glb haters-\gsc{CS} Israel try.\gsc{PTCP}-\gsc{M.PL} to.blacken \gsc{ACC} faces-\gsc{3F} of state-\gsc{CS} Israel in.the-arena the-international//
%			\glft `Israel's haters are trying to make the State of Israel look bad on the international stage.'\trailingcitation{\url{http://www.ynet.co.il/articles/0,7340,L-4781034,00.html}}//
%		\endgl
%	\a Inchoative, only literal meaning:\\
%		\begingl
%			\gla\ljudge{??}pane-ha \textipa{S}el ha-medina \underline{he\textipa{S}xir-u} axarej ha-\textipa{S}aarurija ha-axrona//
%			\glb faces-\gsc{3F} of the-state blackened-\gsc{3PL} after the-scandal the-last//
%			\glft (int. `The country was made to look bad after the latest scandal')//
%		\endgl
%\xe

%	panav helbinu (*ba-rabim) != he was humiliated (in public).
%		ha-malbin pnej xaver-o ba-rabim, keilu Safax dam-o

%The small list of roots that derive only inchoatives in \thif~trigger a different alloseme as in~(\ref{ex:thif-sem}a). The elsewhere case is repeated from~(\lastx).\footnote{This situation is reminiscent of deponents in some European languages. In our case, active syntax allows for a few inchoative exceptions. With deponents, non-active syntax allows for a few active exceptions. See \cite{kastnerzu15li} for one overview of the issues.}
%\pex\label{ex:thif-sem}
%	\a \denote{\vd} = $\lambda$P.P / \trace~\{\root{p\textipa{S}r} \gsc{THAW}, \root{\textipa{S}mn} \gsc{FAT}, \root{'dm} \gsc{RED}, \root{lbn} \gsc{WHITE}, \dots \}
%	\a \denote{\vd} = $\lambda$P$\lambda$x.P \& Cause(x,P)
%\xe

\cite{borer91} provides additional arguments for deriving the inchoative from the adjective, though these have been subjected to some scrutiny in \citet[83]{kastner16phd}.

The full semantics for \vd~looks as in~(\nextx), without introducing a causer for inchoative events in~(\nextx a--b):
\pex\label{ex:thif-sem}
	\a \denote{\vd} = $\lambda$e.e / \trace~(v) a \phantom{xxxxx} (v does not select an alloseme)
	\a \denote{\vd} = $\lambda$e.e / \trace~(v) n \phantom{xxxxx} (v does not select an alloseme)
	\a \denote{\vd} = $\lambda$x$\lambda$e.Cause(x,e) \hfill (or Agent)
\xe


		\subsubsection{Derivation}
The derivation proceeds straightforwardly for most cases, except that we must allow for unaccusative inchoatives. Our definition of {\vd}, however, states that its EPP feature requires its specifier to be filled; this definition is not compatible with an unaccusative argument remaining low as in VS order.

To account for these cases, assume instead that the [D] feature on {\vd} requires valuation of phi-features \citep{nie17,schaefer17oup}. This valuation proceeds straightforwardly under Spec-Head Agreement but something else needs to be said if the sole argument in the phase is the internal argument. In this case, I propose that the feature [D] can be checked by the internal argument \emph{in situ}: {\vd} probes into its specifier upwards, finds no target, and so it probes downwards and is valued by the internal argument. For more in-depth discussion of the direction of Agree, see works such as \cite{bejarrezac09}, \cite{zeijlstra12}, \cite{preminger13tlr} and \cite{deal15nels}.

Here is what this means for an inchoative example like~(\ref{ex:thif-inch}) with the structure in~(\ref{tree:thif-inch}). {\vd} has nothing in its specifier, so it probes downward and checks its unvalued phi-features with the internal argument \emph{ha-xatul} `the cat'. The derivation converges in the syntax. The interpretation is as in~(\ref{ex:thif-sem}a): no Cause is introduced.

\ex\label{ex:thif-inch} \begingl
	\gla ha-xatul \textbf{he\textipa{S}min}//
	\glb the-cat fattened//
	\glft `The cat grew fat.'//
	\endgl
\xe
%\begin{wrapfigure}[8]{r}{0.6\textwidth}
%	\vspace{-6em}
	
	\ex\label{tree:thif-inch}
		\scalebox{0.9}{
	    \Tree
	    [.VoiceP
	        [.\tikz{\node (spec) {};} ]
	        [
	            [.\tikz{\node (vd) {\vd};} ]
	            [.vP
	                [.v
		                [.\phantom{xx}v\phantom{xx} ]
		                [.a
		                    [.a ]
		                    [.\root{\textipa{S}mn} ]
		                ]
	                ]
	                [.\tikz{\node (IA) {\emph{ha-xatul}};} ]
	            ]
	        ]
	    ]
	    \begin{tikzpicture}[overlay]
		    \draw[dashed,->] (vd) .. controls +(south:1) and +(south west:1) .. node{\LARGE $\times$} node[below]{\ding{172}}(spec);
	 	    \draw[dashed,->] (vd) .. controls +(south:4) and +(south:3) .. node[below]{\ding{173}}(IA);    
	    \end{tikzpicture}
	    }
	\xe
%\end{wrapfigure}


As a consequence, ungrammatical cases like~(\nextx) must now be ruled out.
\pex\label{ex:counterex}
	\a \ljudge{*}
		\begingl
		\gla ha-xatul \textbf{hexnis}//
		\glb the-cat inserted//
		\glft (int. `The cat got inserted')//
	\endgl
	
	\a \ljudge{*}
		\begingl
		\gla ha-oto \textbf{hemhir}//
		\glb the-car \gsc{FAST}.\gsc{CAUS}//
		\glft (int. `The car grew fast')//
	\endgl

	\a \ljudge{*}
		\begingl
		\gla ha-xatul \textbf{hekpi}//
		\glb the-cat froze//
		\glft (int. `The cat froze')//
	\endgl
\xe

For~(\lastx a) there is no adjective `inserted' that could be verbalized and no inchoative can be generated. In~(\lastx b) an adjective \emph{mahir} `quick' does exist, but it cannot be instantiated in {\thif} in general due to some arbitrary gap, as already mentioned in Section~\ref{sec:roots} (or at least, I assume that this is an arbitrary gap, in lieu of a more principled explanation).

Finally, (\ref{ex:counterex}c) is not a possible inchoative even though there exists an underlying adjective, namely \emph{kafu} `frozen'. There are a number of possible explanations which can be pursued. One is that \emph{freeze} is not a degree achievement in Hebrew, and so that adjective is not a possible input to the structure. Another kind of explanation falls along the lines of extra-grammatical paradigmatic pressure, in that an inchoative (non-alternating) freezing verb already exists in another template: \emph{kafa} `froze' in {\tkal}. In this regard, I should note that speakers do steer clear of {\thif} for certain inchoatives, instantiating them in other, more canonically non-active templates: \emph{hitarex} `grew long' in {\thit} rather than \emph{heerix}, \emph{hizdaken} `grew old' in {\thit} rather than \emph{hezkin}, \emph{raza} `thinned' in {\tkal} instead of \emph{herza}, and \emph{hitadem} `reddened' in {\thit} rather than \emph{heedim} (but see \citealt[22]{doron03} for a grammatical difference between the two).

With the formal analysis in place, we can now turn to the final point of the discussion: why is the labile alternation formed in {\thif} specifically?


\section{The labile alternation}
The main characteristic of {\vd} is that it guarantees the availability of an external argument; in other words, a transitive construction is possible if the even has change-of-state semantics, i.e.~an internal argument. Let us assume that the process of inchoative formation in {\thif} is productive, as argued for by \cite{lev16}, and not a short list of exceptions, as assumed in most of the literature. Then, when the speaker is faced with the choice of a construction for their de-adjectival or denominal verb, they would choose {\vd} because this structure guarantees that a causer can be added.

One consequence of this analysis is that it allows us to state in formal terms the difference in argument distribution between causatives and inchoatives. As seen in~(\ref{ex:thif-active}) and~(\ref{table:kal-thif}) above, active verbs in {\thif} might be unergative, transitive or ditransitive. These possibilities are wiped out for inchoatives, which are uniformly intransitive. Let us assume that whatever requirements a verb has for its complements emerge upon combination of v and the root. I speculate that once the structure contains a more deeply embedded a/n node as in~(\ref{tree:thif-inch}), v is too far away from the root for particular selectional requirements to be stated. This idea receives potential corroboration from the behavior of -\emph{en} in English. As noted by \cite{harley09n}, English verbalizers such as -\emph{ify}, \emph{-ize} and \emph{-ate} can derive verbs that are uniformly unaccusative (e.g.~\emph{oscillate}), uniformly unergative (e.g.~\emph{detoriorate}) or labile (e.g.~\emph{activate}), but -\emph{en} verbs are always labile. An examination of the list in \citet[245]{levin93} confirms this claim. If we assume that these latter verbs contain additional structure, for instance [v [CMPR [a \root{Root}~\!]]] \citep{bobaljik12}, we arrive at a similar analysis to that of {\thif} inchoatives: they cannot impose selectional restrictions and are ``stuck'' with the argument structure imposed by the syntax. But I will not push this point further due to space limitations.

Finally, the strong claim about separate derivational strategies for causatives and inchoatives awaits a more articulated semantic analysis. As a reviewer points out, in~(\ref{ex:thif-hefSir}) the verb \emph{hef\textipa{S}ir} means `thawed', i.e.~became warmer, while the underlying adjective \emph{po\textipa{S}er} means `lukewarm', i.e.~not warm. Another incongruity between verb and adjective can be seen with \emph{he\textipa{S}min}, `grew fatter', which does not entail that its argument becomes \emph{\textipa{S}amen} `fat'. Important discussion of the relevant scales and entailments is given by~\cite{borer91}.


Discussion of alternative analyses is likewise not possible in the current paper. Three possibilities are existential closure over a Cause in Spec,{\vd} \citep{doron03}, treating {\thif} like a standard verbalizing affix on a par with -\emph{en} \citep{borer91}; and a solution in terms of contextual allomorphy of {\vz}, the non-active counterpart of {\vd}. These are all discussed in \citet[81]{kastner16phd}.


\section{Conclusion}

The template \thif~predominantly instantiates active verbs, usually causatives. It is also reasonably productive. Yet a number of roots derive inchoative verbs in this template. The analysis proposed here showed how the influence of a certain class of roots can be accommodated in the grammar, while keeping constant the overall behavior of the functional head which derives this template morphophonologically.

The two factors conspiring to create a labile alternation in a language that otherwise does not allow such an alternation are the root and the syntactic structure. The roots fall under various lexical semantic classes but all appear to derive degree achievements from underlying nouns or adjectives, as suggested by \cite{lev16}. The syntax which facilitates this derivation is one in which a noun or adjective is first formed before it is verbalized, and then combined with a specific causative head {\vd}. This theoretical approach allows us to ask more specific questions about how the idiosyncratic information associated with roots interacts with the syntactic structure in which they are embedded.


\section{Alternatives}
	\subsection{Alternative: Existential closure}
One alternative analysis would posit a silent, generic Cause in Spec,\vd. The analysis in \citet[61]{doron03}---which in many ways is a precursor to the theory presented in this work---assumes that a Causative head $\gamma$ gives rise to \thif. The problem for the system in \cite{doron03} is that if these verbs are derived using a Causative head rather than a Middle head, we have no explanation for their unaccusativity.

As a result, \citeauthor{doron03} must conclude that ``\emph{x reddened} is equivalent to \emph{Something caused x to redden}'' \citep[62]{doron03}, with the Causative head $\gamma$ introducing a Causer that is existentially quantified over. This kind of account is more in line with a passive analysis than a causative one.

Assume for the sake of the argument that a silent element fills Spec,\vd~in inchoatives. One would need to specify the exact featural makeup of this element, for example a null subject \emph{pro}. The result would be a transitive structure where \emph{pro} should be assigned Nominative case and the IA should be assigned Accusative case. Definite accusative objects in Hebrew take the direct object marker \emph{et}, so we would predict that \emph{et} appears before inchoatives in \thif. But this is incorrect: the generic Cause cannot be a silent pronoun in a transitive relationship with the internal argument.
\pex
	\a \ljudge{*} \begingl
		\gla heʃmin \textbf{et} ha-xatul//
		\glb fattened \gsc{ACC} the-cat//
	\endgl
	
	\a \ljudge{*} \begingl
		\gla \textbf{et} ha-xatul heʃmin//
		\glb \gsc{ACC} the-cat fattened//
		\glft (int. `The cat grew fat')//
	\endgl
\xe

Another tack would be to say that instead of \emph{pro}, the silent external argument is a Weak Implicit Argument in the sense of \cite{landau10}: a bundle of ɸ-features with no [D] feature, distinguishing it from a Strong Implicit Argument such as \emph{pro}. If there is no [D] feature on the weak EA, it does not participate in the calculus of case and the IA will receive unmarked case, i.e.~Nominative.

But note that this analysis ends up being very similar to ours: the EA is not taking part in any relevant syntactic process, and whatever requirements \vd~has still need to be satisfied. Furthermore, the exact characterization of Weak Implicit Arguments is not fully fleshed out; \citet[380]{landau10} concludes that they may not be direct objects, only oblique objects, and that they may not be subjects of predication, but his system makes no claims as to whether they can function as the generic kind of EA we would need here. In the absence of a convincing account for implied causers, I reject this analysis.

	\subsection{Alternative: Contextual allomorphy}
Another possible analysis is strictly morphological in nature. Under this account, unaccusative inchoatives are true unaccusatives derived with the \vz~morpheme of \S\ref{syn:middle:nonactive}, except that the allomorphic rule in~(\ref{ex:hifil-unacc}) causes \vz~to be pronounced like \gsc{CAUS} (which is a placeholder for of the morphology of \thif) rather than \gsc{MID} (which is a placeholder for the morphology of \tnif~and \thit).
\pex
  \a \vz~\lra~\gsc{CAUS} / \trace~\{\root{lbn}, \root{'dm}, \root{xl\dgs{k}}, \root{xvr}, \root{ʃmn}, \dots \}\label{ex:hifil-unacc}
  \a \vz~\lra~\gsc{MID}
\xe
Since the number of unaccusative verbs in \thif~is fairly small, or at the very least non-productive, it is plausible that an arbitrary list of roots conditioning this allomorphy can be learned. Still, the mystery remains why it is specifically \thif~that houses inchoatives: why doesn't the rule in~(\lastx a) insert the form of any other template, such as \tkal, \tnif~or \tpie? This solution is technically possible but conceptually unenlightening.

%The rule in~(\ref{ex:hifil-unacc}) is stated in order to account for a number of exceptions which presumably must be learned. Yet it also makes a prediction: that the roots to which this rule applies cannot surface in the ``middle'' template \tnif, only in ``causative'' \thif. The reason is that the ``middle'' template is generated using \vz, but \vz~is pronounced \gsc{CAUS} for these roots. This prediction seems to be correct:
%\ex\label{ex:hifil-nonif}
%\begin{tabular}[t]{lllp{8.5em}llll}
%	a. & \emph{helbin} & $\sim$ & *\emph{nilban} & c. & \emph{hexlik} & $\sim$ & \emph{*nixlax}\\
%	b. & \emph{he'edim} & $\sim$ & *\emph{ni'dam} & d. & \emph{heʃmin} & $\sim$ & \emph{*niʃman}\\
%\end{tabular}
%\xe

	\subsection{Alternative: Verbalizing affix} \label{syn:templates:thif:borer}
\cite{borer91} presents an analysis of \thif~alternations couched in Parallel Morphology, which I will {}translate into {comparable} terms{ in the current theory}. Her account consists of two main parts. In the first, she argues that inchoative forms are derived from adjectives while causative forms are derived from a root/verb. In the second, she presents an analysis showing why it must be the case---given certain assumptions---that causatives are formed in the lexicon and inchoatives in the syntax. Our analysis is similar to hers in adopting separate structures for causatives and inchoatives, albeit using different argumentation. The content of the analysis is different, though, since for \cite{borer91} \thif~is a single verbalizing morpheme which subcategorizes for an adjectival element.

In her analysis, \citet[136]{borer91} takes Hebrew \thif~and English -\emph{en} to be verbalizers subcategorizing for an adjectival stem, be it a property root or an adjective. %Both the adjective and the verbalizer are able to---but need not---assign an external theta-role \citep[142]{borer91}.
When this is done in the ``lexicon'' by verbalizing a root, the result is a causative verb:
\ex\label{ex:thif-borer-caus}{[}_{\text{v}} \root{\gsc{wide}} -\emph{en}]
\xe
When this is done in the syntax by verbalizing an adjective, the result is an inchoative verb:
\ex\label{ex:thif-borer-inch}{[}_{\text{v}} [_{\text{a}} \root{\gsc{wide}} a] -\emph{en}]
\xe

Crucially for us, the analysis does not answer the questions posed at the beginning of the discussion: why this template and why these roots. \thif~is assumed to be a de-adjectival verbalizer, just like -\emph{en}, without discussion of this template's role in the overall morphosyntax of the language. While it is stipulated that \thif~as a verbalizer subcategorizes for an adjective, this is not always the case: as alluded to above, the run-of-the-mill causatives \emph{hexnis} `inserted', \emph{he'exil} `fed' and \emph{helbiʃ} `dressed' are not derived from underlying adjectives.

\textbf{\emph{hexnis}} `inserted' is derived from \root{kns}, but without a simple adjective *[_{a} \root{kns} a]. One could posit an abstract adjective that is never lexicalized, but it is unclear what this non-existent adjective would be like or what its phonological form would have been (*\emph{kanus}?).
\ex \begingl
	\gla ha-nasix \underline{hexnis} et ha-sefer la-tik//
	\glb the-prince inserted.\gsc{CAUS} \gsc{ACC} the-book to.the-bag//
	\glft `The prince put the book in the bag.'//
	\endgl
\xe

\textbf{\emph{he'exil}} `fed' is derived from \root{'kl}, but probably not from \emph{axul} `consumed', a rare adjectival passive of \emph{axal} `ate'.
\pex
	\a \begingl
		\gla ha-nasix \underline{he'exil} et ha-kivsa//
		\glb the-prince fed.\gsc{CAUS} \gsc{ACC} the-sheep//
		\glft `The prince fed the sheep'//
		\endgl
		
	\a \ljudge{$\ne$} \begingl
		\gla ha-nasix garam la-kivsa lihiot \textbf{axula}//
		\glb the-prince caused to.the-sheep to.be consumed//
		\glft `The prince caused the sheep to be consumed (e.g.~by worms)'//
		\endgl
\xe

\textbf{\emph{helbiʃ}} `dressed' is derived from \root{lbʃ}, but probably not from \emph{lavuʃ} `dressed up', the adjectival passive of \emph{lavaʃ} `wore', which seems to be reserved for descriptions of a full costume.
\pex
	\a \begingl
		\gla ha-ima \underline{helbiʃ-a} et ha-jeled (be-)xalifa jafa//
		\glb the-mom dressed.\gsc{CAUS}-\gsc{F.SG} \gsc{ACC} the-boy in-suit pretty//
		\glft `The mother put the boy's pretty suit on (him).//
		\endgl
	
	\a ``On making his discovery, the astronomer had presented it to the International Astronomical Congress, in a great demonstration, \dots\\
	\begingl
%		\gla le'axar ʃe-gila et taglit-o, hetsig ota ha-astronom ha-turki bifne-j ha-kongres ha-astronomi ha-benleumi, be-tetsuga marʃima,
		\gla aval iʃ lo he'ezin le-dvara-v, miʃum ʃe-haja \underline{lavuʃ} be-tilobʃet turkit. ka'ele hem ha-mevugarim//
%		\glb after \gsc{COMP}-discovered \gsc{ACC} discovery-his presented it the-astronomer the-Turkish in.front.of the-congress the-astronominal the-international in-display impressive
		\glb but nobody \gsc{NEG} listened to-words-his, since \gsc{COMP}-was dressed.up in-outfit Turkish such \gsc{3PL} the-grown.ups//
		\glft But he was in Turkish costume, and so nobody would believe what he said. Grown-ups are like that.''\hfill {(Antoine de Saint-Exup\'ery, \emph{The Little Prince}, Chapter 4. Hebrew by Jude Shva\footnotemark)}//
	\endgl
\footnotetext{\url{http://www.oocities.org/sant\_exupery/c4.htm}}
\xe

\cite{borer91} did not claim that the sole function of \thif~is to verbalize adjectives. But even if this is one of its functions, we have seen that not all verbs in \thif~show the alternation. As her system stands, it is not clear how it could allow for a certain root to be instantiated only in a ``syntactic'' (inchoative) derivation but not in a ``lexical'' (causative) one. Similarly, and as shown above, not all inchoatives in this template are de-adjectival: \emph{heki} `threw up' comes from the noun \emph{ki} `vomit', \emph{hekrim} `clotted' from the noun \emph{krem} `cream', and \emph{hetsxin} `smelled pungent' from the noun \emph{tsaxana} `pungent smell'.
%Another question is whether -\emph{en} and \thif~are productive enough as de-adjectival verbal forms to merit a derivational rule as in the Parallel Morphology system, rather than what is in effect a list of exceptional forms as in mine.

The analysis in \cite{borer91} does not aim to find an underlying reason for why \thif~is used for both causatives and inchoatives, as well as for general causativization in the rest of the system. Nevertheless, it remains the only in-depth study of this alternation that I know of. Recall, for the last part of this discussion, that this analysis also postulates a structural difference between \thif~causatives, (\ref{ex:thif-borer-caus}), and inchoatives, (\ref{ex:thif-borer-inch}). I review this distinction next.

The logic works as follows: if an adjective passes certain diagnostics, and the inchoative does but the causative does not, then the adjective must be embedded in the inchoative \citep[130]{borer91}. Starting with an English example, the adjective \emph{wide} is said to license comparisons with \emph{as}/\emph{like} and comparative forms, whereas the inchoative \emph{widen} does not. \citeauthor{borer91}'s claim is that comparison adverbials and the comparative must be licensed by an adjective (judgments hers).
\pex
  \a The canal is \{as \underline{wide} as a river / \underline{wider} than a river.\}
  \a The canal \underline{widened} \{like a river / more than a river\}.\\
	  (int. `The canal became as wide as a river is wide / became more wide than a river is wide')
  \a \ljudge{*} The flood \underline{widened} the canal \{like a river / more than a river\}.\\
	  (int. `The flood made the canal as wide as a river is wide / made the canal wider than a river is wide')
\xe
I suspect that there is much more variation in acceptability for the utterances in~(\lastx), and that an adverbial reading normally overpowers the scalar one (`The flood widened the canal like a river widens it'){. Three native speaker linguists I have consulted do not share these contrasts} but I leave a judgment survey for future work. Let us return to the novel claims about Hebrew instead.

Taking the adjective \emph{ʃmen-a} `fat-\gsc{F.SG}', it is claimed to license comparatives,~(\nextx a). Inchoatives license comparatives too,~(\nextx b), but causatives do not,~(\nextx c). Judgments are as in \cite{borer91}{; example} (\nextx c) {does not sound as degraded to me, but it does to another speaker whom I consulted informally.}
\pex
  \a Adjective:\\ \begingl
    \gla ha-xatula \textbf{ʃmena} \emph{\{}kmo xazir / joter mi-xazir\emph{\}}//
    \glb the-cat fat like pig {} more than-pig//
    \glft `The cat is fat as a pig / fatter than a pig.'//
  \endgl
  
  \a Inchoative:\\ \begingl
    \gla \textbf{ha-xatula} \underline{heʃmina} \emph{\{}kmo xazir / joter mi-xazir\emph{\}}//
    \glb the-cat fattened like pig {} more than-pig//
    \glft `The cat grew as fast as a pig / fatter than a pig.'//
  \endgl

  \a Causative:\\ \begingl
    \gla \ljudge{*}\textbf{ha-zrika} \underline{heʃmina} et ha-xatula \emph{\{}*kmo xazir / *joter me-xazir\emph{\}}//
    \glb the-injection fattened \gsc{OM} the-cat.\gsc{F} like pig {} more than-pig//
    \glft (int. `The injection made the cat fat as a pig / more than a pig is fat.')//
  \endgl
\xe

Similarly, some adverbs (\emph{haxi ʃe-efʃar} `as much as possible') must be licensed by an adjective and accordingly only appear with inchoatives, not causatives.

It seems to me that the success of this diagnostic depends to a large extent on the lexical items chosen. For example, using the antonym \emph{herza} `grew thin', my judgments are slightly different:
\pex
  \a Adjective:\\ \begingl
    \gla ha-xatula \underline{raza} \emph{\{}kmo makel / ?joter mi-makel\emph{\}}//
    \glb the-cat thin like stick {} more than-stick//
    \glft `The cat is as thin as a rail / skinnier than a rail.'//
  \endgl
  
  \a Inchoative:\\ \begingl
    \gla\ljudge{?}\textbf{ha-xatula} \underline{herzeta} \emph{\{}kmo makel / ??joter mi-makel\emph{\}}//
    \glb the-cat thinned like stick {} more than-stick//
    \glft (int. `The cat became as thin as a rail / skinnier than a rail.')//
  \endgl

  \a Causative:\\ \begingl
    \gla \ljudge{??}\textbf{ha-zrika} \underline{herzeta} et ha-xatula \emph{\{}kmo makel / joter me-makel\emph{\}}//
    \glb the-injection thinned \gsc{OM} the-cat like stick {} more than-stick//
    \glft (int. `The injection made the cat as thin as a rail / skinnier than a rail.')//
  \endgl
\xe

With \emph{he'et} `slowed down' I judge inchoatives unacceptable and causatives slightly better though still degraded. These judgments are meant to highlight the variance, not to be taken as categorical for all alternations or all speakers.
\pex
	\a Adjective:\\ \begingl
		\gla ha-mexonit ha-zo \textbf{itit} \emph{\{}kmo tsav / joter mi-tsav\emph{\}}//
		\glb the-car the-this slow like turtle {} more than-turtle//
		\glft `This car is as slow as a turtle / slower than a turtle.'//
	\endgl

	\a Inchoative:\\ \begingl
		\gla\ljudge{*}ha-mexonit ha-zo \underline{he'eta} \emph{\{}kmo tsav / joter mi-tsav\emph{\}}//
		\glb the-car the-this slowed like turtle {} more than-turtle//
		\glft (int. `This car slowed down to turtle speed / to sub-turtle speed.')\\
			(More acceptable on a reading of `The car slowed down like a turtle slowed down'.)//
	\endgl

	\a Causative:\\ \begingl
		\gla\ljudge{??}\textbf{ha-ba'aja} \textbf{ba-hiluxim} \underline{he'eta} et ha-mexonit \emph{\{}kmo tsav / joter mi-tsav\emph{\}}//
		\glb the-problem in.the-gears slowed \gsc{ACC} the-car like turtle {} more than-turtle//
		\glft (int. `The problem with the gear box slowed the car down to turtle speed / to sub-turtle speed.')//
	\endgl
\xe

It is also left vague what precisely this diagnostic is probing. In~(\nextx), for instance, there is no underlying adjective `beloved' but the utterance is completely acceptable:\footnote{Thanks to Idan Landau for pointing this out to me.}
\ex
	\begingl
	\gla ani \underline{ohev} otxa \textbf{kmo} \textbf{ax}//
	\glb I love.\gsc{SMPL}.\gsc{PTCP} you.\gsc{M} like brother//
	\glft `I love you like a brother.'//
	\endgl
\xe

The remainder of \citeauthor{borer91}'s article is devoted to working through the different possible structures that her framework generates and discussing whether they are licit or not. The final few pages \citep[150]{borer91} raise the issue of whether these verbs give comparative (change on a scale) or absolute (result) readings, concluding that both are in principle possible but are constrained by the root. That is to say, both English \emph{reddened} and Hebrew \emph{he'edim} `reddened' are as compatible with a `became redder' meaning as with a `became red' meaning, but some verbs like \emph{quicken} are only compatible with a `became quicker' meaning (see \citealt[ch.~5]{bobaljik12} for some relevant recent discussion).

Since I am not sure that the argument from comparatives generalizes, and given that no explicit syntax or semantics for this modification was put forward, I do not endorse at this point the arguments for distinct structures put forward in \cite{borer91}. Nevertheless, I have recast the original intuition in contemporary terms and supported it using different arguments. In any case, the original proposal should serve as a stepping stone in the next stage of investigation into these kinds of alternations. Future experimental studies could test the predictions that these theories of {\thif} make for speakers' usage of nonce verbs in this template, as sketched in \S\ref{acq:prod:template}.




\section{Psych-verbs}


\section{Summary}
The template \thif~predominantly instantiates active verbs, usually causatives. It is also reasonably productive. Yet a number of roots derive inchoative verbs in this template. Our analysis showed how the limited influence of a small class of verbs can be accommodated in the grammar, while keeping constant the overall behavior of the head \vd~which derives this template morphophonologically. In so doing, I have motivated a certain view of feature valuation (Agree) that may probe downward after probing its specifier.

I have also provided anecdotal evidence that this template is productive, able to take existing inchoatives and causativize them via zero-derivation (\emph{hezi'a} `sweat', \emph{hexmits} `soured'). The structure allows the template to be causative by default with an allowance for exceptions. These exceptions, once they are better understood, stand to reveal the kind of lexical-semantic generalizations that might be relevant at the interface with LF. 

Before proceeding to the next template, I should note that speakers tend to veer away from this template for inchoatives, instantiating verbs in other, more canonically non-active templates: \emph{hitarex} `grew long' in \thit~rather than \emph{he'erix}, \emph{hizdaken} `grew old' in \thit~rather than \emph{hizkin}, \emph{hit'adem} `reddened' in \thit~rather than \emph{he'edim} (but see \citealt[22]{doron03} for a grammatical difference between the two), and \emph{raza} `thinned' in \tkal~instead of \emph{herza}. In \S\ref{acq:prod:template} I outline an experiment to test the productivity of this template as a causativizer and an inchoativizer.





\section{Alternatives: root-based approaches}\label{syn:other-root}
This section contrasts my approach to Hebrew verbal morphology with a number of prominent proposals that share the two major assumptions: morphological structure is built in the syntax, and an abstract root (consonantal in Semitic) lies at the base of the derivation. Lexicalist analyses are reviewed in \S\ref{syn:other-stem}. 

Returning to the overarching themes guiding this dissertation, a contrast was brought up between the ``emergent'' approach to verbal templates and the morpheme-based approach. My theory treats templates as epiphenomenal, as does that of \cite{doron03} in \S\ref{syn:other-root:ed}. The analyses of \cite{arad05} in \S\ref{syn:other-root:arad} and \cite{borer13oup} in \S\ref{syn:other-root:borer} treat each template as its own morpheme.

This difference relates to a broader question: how syntax feeds the interfaces. If a template is a morpheme, then there is no strong architectural claim to be made. But if different syntactic structures can give rise to identical morphophonology, as I claim, we are forced into a strong view of the syntax as the sole generative engine. That is to say, if similar interpretations can arise from different syntactic structures---for example reflexive readings with entirely different underlying structures---we are led to a strong view of the autonomy of syntax; the existence of figure reflexives and regular reflexives does not entail one syntax for both constructions. Instead, independent structures may be interpreted similarly. The accounts surveyed below all differ from mine on these points to different extents.

	\subsection{Distributed morphosemantics \citep{doron03}} \label{syn:other-root:ed}
Like in my theory, the seminal analysis of Hebrew verbs in \cite{doron03} employs a number of functional heads to derive the different templates. \cite{doron03} was the first to identify basic non-templatic elements that combine compositionally in order to form Hebrew verbs. For example, a causative head $\gamma$ is used to derive the ``causative'' {\thif} template, where I make use of a \vd~head.

This dissertation is influenced directly by \citeauthor{doron03}'s work and to a large extent the heads discussed above mirror hers. Table~\ref{table:doronme} provides an overview.
\begin{table}[h!] \centering \small
\begin{tabular}{|l|lll|} \hline
This chapter & \multicolumn{3}{c|}{\cite{doron03,doron14adj,alexiadoudoron12}}\\\hline
v & V & & Event \\
Voice & v &  & Causative (external argument) \\
\va & $\iota$ & (INTNS) & Agentive/Action \\
\vz, \pz & $\mu$ & (MIDDLE) & Non-active \\
\vd & $\gamma$ & (CAUSE) & Causative \\
Pass & $\pi$ & (PASS) & Passive\\\hline
\end{tabular}
\caption{\label{table:doronme}Functional heads in two related theories of Hebrew.}
\end{table}

The important conceptual difference is that the elements in the left-hand column are syntactic whereas those on the right can be characterized as morphosemantic: each one has a distinct semantic role. As I hope to have shown, however, it is not the case that a single head derives a single template. The morphology of \tnif, for instance, masks a syntactic and semantic distinction between nonactives and figure reflexives. One morpheme such as MIDDLE cannot generate both. Instead, I have argued for a more fine-grained breakdown of verb types in the middle templates, explaining why some are unaccusative and others take obligatory arguments.

The system proposed in this chapter uses contemporary syntactic machinery rather than a new system with Hebrew-specific heads (though \citealt{alexiadoudoron12} do extend their system to English and Greek). A \citeauthor{doron03}-style system takes the semantics as its starting point, attempting to reach the templates from syntactic-semantic primitives signified by the functional heads. Such a system runs into the basic problem of Semitic morphology: one cannot map the phonology directly onto the semantics. There is no way in which a causative verb has a unique morphophonological exponent (which would look like an affix in non-Semitic languages).

To appreciate the difficulty of the task, consider the proposed structure for a  ``causative'' verb like \emph{hefgiʃ} `caused to meet' \citep[61]{doron03}. This structure is assumed to contain a MIDDLE head $\mu$ as well as a CAUSE head $\gamma$:
\ex \label{ex:doron}[External Argument [$\gamma$ [Internal Argument [$\mu$ [\root{pgʃ} ]]]]] 
\xe
While the CAUSE head $\gamma$ can be argued to reflect a causative morpheme/prefix, the existence of a MIDDLE head $\mu$ shows that additional assumptions are required if the semantics is taken as the starting point; this latter head also appears in the ``middle'' templates \tnif~and \thit, though the semantics and phonology of these templates are different. Furthermore, it is unclear how to derive either the syntax or the phonology from the semantic structure: it is difficult to treat the elements in~(\ref{ex:doron}) as morphemes and arrange them in a structure that can be spelled out cyclically; MIDDLE and CAUSE in~(\ref{ex:doron}) cannot both be spelled out overtly to derive a single prefix over the internal argument. In contrast, my account explains how the morphophonology can be derived directly from the syntax.

Let us return to reflexive verbs to see a false prediction made by this system. \citet[60]{doron03} derives reflexives in \thit~by assuming that MIDDLE (\vz) assigns the Agent role for this root. This explains why \emph{histager} `secluded himself' is agentive, hence reflexive. However, if the only relevant elements are \vz~and the root, then a verb in the same root in \tnif~(where I have \vz~and \citealt{doron03} has MIDDLE) is also predicted to be agentive. This expectation is incorrect: \emph{nisgar} `closed' is unaccusative. The previous analysis is almost a mirror image of the one presented here: while I let \va~add agentivity to a structure with \vz, thereby deriving reflexives, the morphosemantic account invokes added agentivity for certain roots, bypassing the syntax in ways that lead to false predictions.

I should take a moment to emphasize the most important gains of the morphosemantic theory. Treating templates as emergent from heads that do separate syntactic and semantic work gave us a new way to analyze argument structure alternations across templates, based on a wealth of empirical data. The theory also made a compelling case for the root as an atomic element participating in the derivation, making a number of novel observations along the way. Where we have made progress in this dissertation it is by flipping one of the assumptions on its head: that the primitives have strict syntactic content and flexible semantic content (allosemes), rather than strict semantic content and unclear syntactic content.

	\subsection{Templates as morphemes \citep{arad05}} \label{syn:other-root:arad}
In juxtaposition to an emergent view, \cite{arad05} treated verbal templates as distinct spell-outs of Voice. Together with \cite{arad03}, both works mounted a strong defense of the root as the base of the derivation beyond Hebrew as well.

For \citeauthor{arad05}, templates are different ``flavors'' of v, each a separate CV-skeleton, and the vocalic melody is inserted at Voice. Each template might have a feature such as [--transitive], which is shared by \tnif~and \thit. The general problem with morphemic approaches to templates has been raised a number of times so far. A given template simply does not have a deterministic syntax or semantics (save for the two passive templates). \citet[198]{arad05} actually speculates that a configurational approach (like in our theory) might be more viable than a feature-based approach. 

Regarding argument structure alternations, \cite{arad05} claims that all templates can derive roots from verbs but that two of the templates---\tnif~and \thit---can also derive verbs from other verbs. In so doing, she hopes to capture alternations such as those between \tkal~and \tnif~or between \tpie~and \thit.  When \citeauthor{arad05} tries to predict which templates can detransitivize or causativize verbs in other templates, she is forced to stipulate these relations. The basic observation is entirely accurate, in my view: in a given template, some verbs are root-derived and some are derived from another verb in a different template. In \tnif, inchoatives are root-derived and anticausatives are derived from underlying active verbs in \tkal. I submit that the theory as a whole is more constrained if functional heads have specific syntactic and semantic properties; idiosyncrasy can then be left to the root, as proposed by \citeauthor{arad05} herself.
	
	\subsection{The Exo-Skeletal Model \citep{borer13oup,borer15roots}} \label{syn:other-root:borer}
Another contemporary morphemic approach to templates is sketched in the Exo-Skeletal Model of \citet[Ch.~11]{borer13oup}. The bulk of \cite{borer13oup} is devoted to developing the model based on English, with the Hebrew chapter included for cross-linguistic support and as the beginning of an exploration in its own right. Given that the Exo-Skeletal analysis is still preliminary, I do not expect it to address as wide a range of phenomena as I attempt to cover in this dissertation. Nevertheless, it is worthwhile pointing out a number of potential problems as this type of analysis is developed in the future.

On the Exo-Skeletal approach, templates are once again derivational morphemes with no hierarchical structure between them. All objections I have leveled at the morphemic approach to templates are valid here too. It is not clear what the syntax and semantics of a given template should be (other than with the two passive templates) given the range of verb types that can appear in each templatic form. \cite{borer15roots} speculates that additional features like the heads from \cite{doron03} can condition the selection of certain templates, but the question then arises of how the relation between these features and these templates should be seen.

In order to derive argument structure alternations, \citet[564]{borer13oup} raises the possibility that some templates might stand in a hierarchical relationship to others, for example \tpie~and its detransitivized version \thit. The promissory discussion seems to single out these two templates but leaves open the possibility of additional hierarchical relations, rightly in my view. 

The empirical core of \citeauthor{borer13oup}'s chapter lies in patterns of nominalization and with the underspecification of \tkal. Her view of nominalization is similar to mine in \S\ref{syn:templates:nmlz} and her analysis of \tkal~is similar to mine in \S\ref{syn:templates:tkal}; see also \S\ref{sec:others-same:borer}. To a large extent the hypotheses converge, though I maintain that an analysis of Hebrew in which templates are individual morphemes cannot have the same empirical coverage or even conceptual consistency as the kind of account provided in this dissertation.


\section{Alternatives: stem-based approaches}\label{syn:other-stem}
Syntactic derivations with functional heads are necessary if we are to predict argument structure alternations correctly. As the next chapter shows, these heads also allow us to account for the morphophonology. But before proceeding we must discuss a different kind of alternative analysis. The other notable attempts to derive the Hebrew patterns are lexicalist, stem-based ones, which I consider next.

	\subsection{Morphomes and conjugation classes \citep{aronoff94,aronoff07}}
\cite{aronoff94,aronoff07} puts forward a view of morphology as an independent component of the grammar. His view is committed to treating individual stems (lexemes, ``morphomes'') as the basic lexical unit that feeds additional derivation and inflection. On such a view, each template is in essence a different conjugation class. Roots do not exist as contentful elements, only as collections of consonants over which paradigmatic, phonological generalizations can be made (though \citealt[827]{aronoff07} does suggest to treat them ``much like Latin roots'').

Theoretical differences aside, \citeauthor{aronoff94}'s framework makes no attempt to explain the argument structure alternations in the language. There is no attempt to account for how reflexive verbs only appear in \thit, or why verbs in \tpie~are agentive, or indeed any of the phenomena discussed thus far. Divorcing roots from meanings also does little to explain why meanings do persist across templates for a given root, or to delineate in what configurations meaning must be conserved (transparent argument structure alternations) and where special meaning can be tapped (alloseme of the root).
	
	\subsection{The Theta System \citep{reinhartsiloni05,laks11}}\label{syn:other-stem:reinhart}
\cite{reinhartsiloni05}, and following them \cite{laks11,laks13ws,laks13morpho,laks14}, present a lexicalist account of the causative alternation in Hebrew. These works argue that a process of decausativization applies in the language. Under such a view, causative verbs are the basic verbal forms in the language, from which the speaker derives reflexives, anticausatives and reciprocals in other templates. Each template thus has a prototypical role in the lexicon. For example, \tpie~houses causative verbs, from which anticausatives in \thit~are derived.

The question then arises of how to account for nonactive verbs that have no active alternation in another template. If there is no active base, how can a decausativized alternation arise? This is exactly the case of inchoative middle verbs, described in \S\ref{syn:middle:nonactive:null}. The proposed solution is that the causative verb exists in the lexicon as a \emph{frozen} entry, a verb that cannot be used in the syntactic derivation but can be used in the lexicon to derive other forms.

Two objections now arise. First, this kind of theory does not explain why a given template has whatever morphosyntactic behavior it has, e.g.~transitive or intransitive, reflexive or not. The generalizations are stipulated on a template-by-template basis: \tnif~and \thit~are anticausative, for example, because they do not receive a [cause change] feature in the lexical derivation. Yet we have seen that these ``middle'' templates are not intransitive across the board; they do house active verbs, specifically figure reflexives and in some cases reciprocals. Under a decausativization analysis, middle templates are correctly predicted to be more marked than their active bases; but it is not explained why each template has the specific morphophonological characteristics it has, and to what extent these correlate with its morphosyntax.

The second problem has to do with the notion of a ``frozen'' lexical entry. According to \citet[116]{laks14}, ``[\emph{F}]\emph{rozen entries lack phonological matrix and morphological properties} [\dots] \emph{but they are assumed to be conceptually represented in the lexicon. The frozen entry, which is not accessible for syntactic derivations, can nonetheless serve as input for lexical operations.}'' It is unclear how this notion can be falsified if a ``frozen'' entry exists to rescue the theory whenever one is necessitated. In fact, what this approach is relying on is a concept very similar to the abstract root while denying it at the same time.

%\citeauthor{laks14} goes on to describe a process of ``morphological filling'' or ``defrosting'': as mentioned earlier, the inchoative \emph{hit'alef} `fainted' does not have a standard causative correspondent *\emph{'ilef} `made faint'. Be that as it may, young innovators do use the verb \emph{'ilef}. For \citeauthor{laks14}, this is an example of the ``frozen'' entry being defrosted and becoming an active part of the lexicon. Yet now another issue arises, having to do with idiomatic meanings in the causative alternation. In their study of idioms in Hebrew and their implications for word-based and root-based approaches to morphology, \cite{horvathsiloni09} argue that any lexical entry can be associated with idiomatic meaning. Thus, looking at unaccusative-causative pairs, it may be that only one of the two alternates has idiomatic meaning, that both share an idiomatic meaning, and of course that neither does. Given that in a root-based system the meaning of the verb is a result of the interaction of the verb with different functional heads (\citealt{arad03} and the system outlined above), the two theories make the same predictions (contra the claims in \citealt{horvathsiloni09}).\footnote{I use ``idiomatic'' in a general sense to indicate non-literal meaning, as used by \cite{horvathsiloni09}. See \cite{anagnostopoulou14thli} and \cite{harley14thlib} for finer-grained distinctions.}
%
%Let us return now to \%\emph{'ilef}. The meaning of this verb can be literal, `made someone physically faint', but the verb is also used figuratively to mean `was dazzling', `knocked you off your feet'. The idiomatic meaning is only part of the causative lexical entry. Yet if this entry was frozen, it means that the idiomatic meaning was added to the verb the moment it was defrosted. This is a perplexing state of affairs. If the verb was defrosted in order to create a vehicle for an idiomatic meaning, why is the literal meaning necessary? In other words, why fill in the defrosted causative verb \emph{'ilef} rather than create an arbitrary new verb *\emph{dizel}?

%This is unexpected if the relation between \emph{'ilef} and \emph{hit'alef} involves pure valency changing, because \emph{hit'alef} can only be used in a literal sense, never in a figurative sense of `be amazed'.

%			\gll ha-kise ʃel sar ha-ocar matxil le-hitnadned\\
%			the-chair of minister.\gsc{CS} the-treasury begins to-wobble.\gsc{INTNS.MID}\\
%			\glt `The Minister of Finance is on the verge of getting fired' (lit. `The finance minister's chair has started rocking')
%			
%			\ex \gll roʃ ha-memʃala matxil le-nadned le-sar ha-ocar et ha-kise\\
%			head.\gsc{CS} the-government starts to-swing.\gsc{INTNS} to-minister.\gsc{CS} the-treasury \gsc{ACC} the-chair\\
%			\glt `The Prime Minister is starting to rock the Minister of Finance's chair.' [literal]\\
%			\# `The Prime Minister is about to fire the Minister of Finance.'

%\begin{exe}
%	\ex
%		\gll mekgayver xatax et ha-xut ve-ha-pcaca hitpoceca\\
%		MacGyver cut \gsc{ACC} the-wire and-the-bomb exploded.\gsc{INTNS.MID}\\
%		\glt `MacGyver cut the wire and the bomb blew up (as a result)
%	\ex \gll mekgayver pocec et ha-pcaca\\
%		MacGyver blew.up.\gsc{INTNS} \gsc{ACC} the-bomb\\
%		\glt `MacGyver blew up the bomb', `MacGyver detonated the bomb'
%\end{exe}

%On the root-based view taken in this paper, the problem does not arise: \emph{'ilef} would be derived by merging \va~with the root, resulting in whichever meaning the root chooses to have~\citep{doron03,arad03,arad05}. This is exactly what the locality-based approach to argument structure alternations predicts: when a verb is derived from the root, its meaning might be unpredictable, but when derived from another verb it is easier to characterize.

One recent attempt to demonstrate that the ``frozen'' idea is falsifiable is described by \cite{fadlon12}, who presents two experiments that are claimed to adjudicate between the lexicalist analysis of \cite{reinhartsiloni05} and the structural analysis of \cite{arad05} in favor of the former. However, on closer inspection the results do not provide support for one theory over the other.

		\subsubsection{Experimental arguments for ``frozen'' entries \citep{fadlon12}}  