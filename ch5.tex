\label{chap:passn}
\section{Passivization} \label{passn:pass}

It is generally accepted that verbal passives in Hebrew are derived from an active counterpart via some operation of passivization in the syntax, be the framework syntactic \citep{doron03,alexiadoudoron12,borer13oup} or lexicalist \citep{reinhartsiloni05,ussishkin05,laks11}. The meaning of a verb in the passive is compositional and transparent in a way that non-passive templates are not. For example, verbs in the ``passive intensive'' {\tpua} are the passivized version of an active verb in ``intensive'' {\tpie}, (\nextx a), and verbs in ``passive causative'' {\thuf} are the passivized version of an active verb in ``causative'' {\thif}, (\nextx b).
\ex Predictable alternations in the passive templates:\\
	\begin{tabular}{ll|ll|ll}
	& & \multicolumn{2}{c|}{Active} & \multicolumn{2}{c}{Passive} \\\hline
	a. & {\tpie} $\sim$ {\tpua} & \emph{bi\textipa{S}el} & `cooked' & \emph{bu\textipa{S}al} & `was cooked'\\
	b. & {\thif} $\sim$ {\thuf} & \emph{he\textipa{S}mid} & `destroyed' & \emph{hu\textipa{S}mad} & `was destroyed'\\
	\end{tabular}
\xe
This derivational analysis accounts for two important facts about passives in Hebrew: first, that there do not exist any passive verbs (that is, verbs in {\tpua} and {\thuf}) without an active base from which they are derived; and second, that passive verbs cannot mean anything other than passivization of the active form, where ``passivization'' means suppression of the external argument via existential closure.

In order to account for these patterns we adopt the proposal in \cite{doron03} and \cite{alexiadoudoron12}, working within Distributed Morphology, that passives are brought about by merger of a head Pass above VoiceP, the phrase in which the external argument would have otherwise been introduced.

In the syntax, Pass is incompatible with merger of a DP in Spec,VoiceP immediately below it. \cite{bruening13} implements this constraint as a selectional requirement on the size of the VoiceP combining with the passive head. In the semantics, Pass likewise brings about existential closure over an implicit external argument, (\nextx):
\ex \denote{Pass} = $\lambda$e$\exists$x.e \& Agent(x,e) \hfill (or see the proposal in \citealt{bruening13})
\xe

\cite{doron03} observes that in Hebrew, the external argument in a passive clause (whether implied or introduced using a \emph{by}-phrase) must be agentive and cannot be a cause. For example, \textit{he\textipa{S}mid} `destroyed' can have a Cause subject (e.g., the environment, the fire, the scandal, etc.) but the implied external argument or overt \emph{by}-phrase subject for its passive counterpart \textit{hu\textipa{S}mad} `was destroyed' may only be a volitional or self-propelled actor (see also \citealt{folliharley08}). The agentivity requirement is likely brought about by the passive morpheme in Hebrew which is otherwise similar to the Pass head of \cite{bruening13}.

Along with the literature cited above, we take passivization to happen after an active verb was derived. In our terms, this means that an existing VoiceP needs to be passivized (by Pass). Support for this claim has been recognized based on meaning as well as phonology: {\tpua} is always the passive of {\tpie}, and {\thuf} is always the passive of {\thif} (see \citealt{kastner16nllt} for a phonological derivation along the current lines).

To make things precise, take~(\blastx b). \cite{doron03}, \cite{arad05} and \cite{kastner16phd} all analyze the template as arising through a dedicated Voice head, call it {\vd} as in~(\nextx):
\ex
	\Tree
	[
		[.{\vd}\\\emph{he-} ]
		[
			[.v
				[.\root{\textipa{S}md} ]
				[.v ]
			]
			[.DP ]
		]
	]
\xe

Here is why it is not possible to posit an additional, passive variant of Voice for Hebrew (cf.~\citealt{embick00}). If {\thif} is derived using \vd, as assumed, then a transparent passivization cannot be accomplished by changing the Voice head: we would end up with an entirely different construction, one that loses all connection (semantic and phonological) to \vd/{\thif}. So passive verbs are derived by use of a Pass head above Voice and below T.


\section{Adjectival passives} \label{passn:adjpass}

The distinction between verbal passives and adjectival passives is well-known in the literature, although accounts differ on where the line should be drawn \citep{wasow77,levinrappaport86,borerwexler87,embick04li,bruening14nllt}. However diagnosed and analyzed, verbal passives are taken to be part of the verbal system and adjectival passives to pattern distributionally with adjectives. We return to the exact analysis in Sect.~\ref{sec:hebrew:analysis}; for now, what matters is that in Hebrew, passive verbs can be \textbf{distinguished structurally} from adjectival passives. \cite{embick04li} demonstrated that if the door is \emph{closed}, it could have been built closed (adjectival passive, stative) or been closed from an open state (verbal passive, eventive). The same logic holds for a verb like \emph{cover} in Hebrew. The implied present tense in~(\nextx a) is ambiguous between a verbal (progressive) reading and an adjectival (stative) reading. However, in Hebrew the future copula diagnoses an adjectival passive form \citep{doron00}. Accordingly, the future tense in~(\nextx b) is unambiguously adjectival \citep{doron00,horvathsiloni08,meltzerasscher11}.\footnote{Abbreviations used: \gsc{ACC} accusative, \gsc{CAUS} ``causative'' template, \gsc{CS} Construct State, \gsc{F} feminine, \gsc{FIN} finite, \gsc{IMP} imperative, Imperf imperfect, \gsc{INF} infinitive, \gsc{INTNS} ``intensive'' template, \gsc{MID} ``middle'' template, \gsc{NACT} nonactive, \gsc{NOM} nominative, \gsc{PASS} passive, \gsc{PASSPTCP} passive/perfect participle, Perf perfect, Pres present, \gsc{PL} plural, \gsc{REFL} reflexive, \gsc{SBJ} subjunctive, \gsc{SG} singular, \gsc{TH} theme vowel. Hebrew transcriptions are given in IPA, with ``e'' standing in for /\textipa{E}/ and ``r'' standing in for /\textipa{K}/. Present forms utilize the present participle, which can be used as a present-tense verb, a nominal or an adjective \citep{boneh13tense,doron13ehll}.}
\pex \label{ex:pres-ambig}
    \a \begingl
        \gla\rightcomment{(synthetic)}ha-sir mexuse//
        \glb the-pot cover.\gsc{INTNS.PASS}.Pres//
        \glft `Someone is covering the pot.' (verbal)\\`The pot is covered.' (adjectival)//
    \endgl
        
    \a \begingl
        \gla\rightcomment{(analytic)}ha-sir \textbf{jihie} mexuse//
        \glb the-pot will.be cover.\gsc{INTNS.PASS}.Pres//
        \glft `The pot will be covered.' (adjectival only)//
    \endgl
\xe

Another pair of examples comes from work by \cite{doron00}:
\pex \label{ex:pres-ambig2}
    \a \begingl
        \gla\rightcomment{(synthetic)}ha-kontsert muklat//
        \glb the-concert record.\gsc{CAUS.PASS}.Pres//
        \glft `Someone is recording the concert.' (verbal)\\`The concert is on tape.' (adjectival)//
    \endgl
        
    \a \begingl
        \gla\rightcomment{(analytic)}ha-kontsert \textbf{jihie} muklat//
        \glb the-concert will.be record.\gsc{CAUS.PASS}.Pres//
        \glft `The concert will be on tape.' (adjectival only)//
    \endgl
\xe

In infinitival and imperative constructions, the passive form is comprised of a copula and the present participle. As is to be expected from an analytic, stative adjective, it does not entail that an event took place, only that the state holds \citep{doron00,kratzer00bls,embick04li}:
\pex \label{ex:analytic-pass}
    \a \begingl
        \gla\rightcomment{(analytic)}ratsiti \textbf{lihiot} mesorak//
        \glb I.wanted to.be comb.\gsc{INTNS}.\gsc{PASS}.Pres//
        \glft `I wanted to be combed' (adjectival only)//
    \endgl
    
    \a \begingl
        \gla\rightcomment{(analytic)}\textbf{\emph{(}ti\emph{)}hie} mesorak!//
        \glb be.\gsc{IMP/FUT} comb.\gsc{INTNS}.\gsc{PASS}.Pres//
        \glft `Be combed!' (\#`Undergo a combing event!')//
    \endgl
\xe

\cite{horvathsiloni08} and \cite{meltzerasscher11} provide similar arguments for the difference between adjectival and verbal passives in the language. 

Next, whereas the analytic forms may have an \textbf{idiomatic reading}~(\nextx a), synthetic passives~(\nextx b) are always compositional.
\pex \label{ex:idiom}
    \a 
        \begingl
        \gla \rightcomment{(analytic)}ze \textbf{jihie} muvan me-elav//
        \glb this will.be understand.\gsc{CAUS}.\gsc{PASS}.Pres from-to.him//
        \glft `It will be self-evident.' (idiomatic)//
        \endgl

    \a \ljudge{\#}
        \begingl
        \gla \rightcomment{(synthetic)}ze juvan me-elav//
        \glb this understand.\gsc{CAUS}.\gsc{PASS}.Fut from-to.him//
        \glft (no immediate clear meaning)//
        \endgl
\xe
Idiomatic readings of analytic passives arise in infinitives as well, which will be relevant for the following discussion.
\ex
	\begingl
	\gla ha-bal\textipa{S}anit garma la-hesber [ \textbf{lihiot} muvan me-elav ]//
	\glb the-linguist caused to.the-explanation to.be understand.\gsc{CAUS}.\gsc{PASS}.Pres from-to.him//
	\glft `The linguist made the explanation be self-evident.'//
	\endgl
\xe

This observation was already made by \cite{horvathsiloni08,horvathsiloni09}, who discuss additional cases of semantic drift.

Lastly, synthetic passives force disjoint readings in which the external argument and the internal argument cannot refer to the same entity \citep{bakeretal89}. The analytic form~(\nextx a), with the participle, allows \textbf{coreference} whereas the synthetic form~(\nextx b) does not \citep[720]{sichel09}:
\pex \label{ex:disjoint}
    \a \begingl
        \gla \rightcomment{(analytic)}ha-jalda \textbf{hajta} mesorek-et//
        \glb the-girl was comb.\gsc{INTNS.PASS}.Pres-\gsc{F}//
        \glft `The girl was combed.' (agent = / $\neq$ theme)//
        \endgl
    \a
        \begingl
        \gla \rightcomment{(synthetic)}ha-jalda sork-a//
        \glb the-girl comb.\gsc{INTNS}.\gsc{PASS}.Past-\gsc{F}//
        \glft `The girl got combed.' (agent $\neq$ theme)//
        \endgl
\xe
In infinitives the analytic passive form exhibits the same ambiguity as (\lastx a). 
\pex
	\a 	\begingl
	    \gla ha-sapar garam la-jalda [ \textbf{lihiot} mesorek-et ]//
	    \glb the-stylist caused to.the-girl to.be comb.\gsc{INTNS.PASS}.Pres-\gsc{F}//
	    \glft `The hair stylist caused the girl to be combed.' (agent $\neq$ theme)//
	    \endgl

	\a 	\begingl
	    \gla ha-jalda garma le-atsma [ \textbf{lihiot} mesorek-et ]//
	    \glb the-girl caused to-herself to.be comb.\gsc{INTNS.PASS}.Pres-\gsc{F}//
	    \glft `The girl caused herself to be combed.' (agent = theme)//
	    \endgl
\xe

The three diagnostics indicate that the synthetic and analytic forms are not equivalent, as summarized in Table~\ref{table:verbal-adjectival}. This result is not surprising from our perspective, since the two forms are generated independently of one another: one is a verbal passive and one is an adjectival passive.

\begin{table}[ht]
\centering
\begin{tabular}{l|ll}
 & Synthetic (verbal) & Analytic (adjectival) \\\hline
 Eventive or stative? & Eventive/stative  & Stative  \\
 Idiomatic? & Compositional  & Idioms possible  \\
 Disjoint reference? & Only disjoint & Co-reference possible  \\
\end{tabular}
\caption{The synthetic-analytic distinction of Hebrew passives parallels the verbal-adjectival distinction. \label{table:verbal-adjectival}}
\end{table}


      \subsubsection{Adjectival passives: The state of the art}
A number of recent works have greatly advanced our understanding of adjectival passives. It has now become commonplace to assume that adjectival passives which entail prior events are compatible with at least some agents of these events. The main insights are as follows.

Adjectives can be distinguished according to whether they describe a stative characteristic of an entity or a state that has come about as the result of some previous event; this is the stative/resultative distinction from \cite{embick04li}, who presented the following diagnostics to distinguish between stative \emph{open} and resultative \emph{opened} by way of example.
\pex Event-oriented adverbs: resultatives only for agent-oriented adverbs as in (a), disambiguated readings for other adverbs as in (b).
	\a \emph{The package remained \textbf{carefully} \xmark open/\cmark opened}.
	\a \emph{The \textbf{recently} open door} (it was open recently).\\
		\emph{The \textbf{recently} opened door} (ambiguous: door was open recently or door was being opened recently).
\xe

\ex Verbs of creation (statives only).\\
	\emph{The door was \{\textbf{built}/\textbf{created}/\textbf{made}\} \cmark open/\xmark opened}.
\xe

\ex Secondary predicates (statives only).\\
	\emph{John \textbf{kicked} the door \cmark open/ \xmark opened}.
\xe

\ex Prefixation of -\emph{un} (mostly resultatives).\\
	\xmark \emph{\textbf{un}open} / \cmark \emph{\textbf{un}opened}
\xe

In some cases the morphology indicates whether a certain form is stative or resultative: \emph{open} and \emph{molten} are stative, whereas \emph{opened} and \emph{melted} are resultative. In many cases, however, the form is ambiguous: \emph{closed}, \emph{fractured} and so on. The tests above distinguish ``simple'' adjectives from adjectives embedding an event. In \citeauthor{embick04li}'s analysis, the former are derived by adjectivizing a root, and the latter by adjectivizing an event (vP/VoiceP). \citeauthor{embick04li}'s resultatives thus fold in both ``target state'' and ``result state'' adjectival passives, a semantic distinction which depends on whether the adjectival passive can be modified by `still' \citep{kratzer00bls,alexiadouetal14}.

Work since has investigated the kind of modifiers that can be attached to an adjectival passive \citep{meltzerasscher11,mcintyre13,alexiadouetal14,bruening14nllt,gehrkemarco14}. At least the following constructions are available for (resultative) adjectival passives in English, German, Hebrew and Spanish{. Agent implication is possible:}
\pex
	\a \ljudge{*} \emph{The door is \underline{opened}, but no one has opened it}.
	\a \ljudge{*} \begingl
		\gla Die M\"unze ist schon lange \underline{versunken} aber keiner hat sie je versenkt//
		\glb the coin is already long sunk.Adj but nobody has she ever sunk.\gsc{PASSPTCP}//
		\glft `The coin has been sunk for a while, but nobody has sunk it.' \trailingcitation{(German, \citealt[124]{alexiadouetal14})}//
	\endgl
\xe
{\emph{By}-phrases are possible only if their modification of the agent, and therefore of the event, is discernible by examining the end state. One can tell that an editor did good work but not that the editor was bored:}
\ex
	\begingl
		\gla ha-sefer \underline{arux} {al jedej} orex {\cmark}metsujan / {\xmark}meʃoamam//
		\glb the-book edited.Adj by editor \phantom{\cmark}excellent {} \phantom{\xmark}bored//
		\glft `The book was edited by an excellent/*bored editor.' \trailingcitation{(Hebrew, \citealt[823]{meltzerasscher11})}//
	\endgl
\xe
{Similarly, instrumentals are possible only if their modification of the event can be discerned by examining the end state. The writing of a blue pencil is distinguished from that of other pencils but the writing of a pretty pencil is not} (though cf.~\citealt{mcintyre13} and \citealt{bruening14nllt}):
\ex
	\begingl
		\gla ha-mixtav \underline{katuv} be-iparon {\cmark}kaxol / {\xmark}jafe//
		\glb the-letter written.Adj in-pencil \phantom{\cmark}blue {} \phantom{\xmark}pretty//
		\glft `The letter was written with a blue/*pretty pencil.' \trailingcitation{(\citealt[825]{meltzerasscher11}, attributed to Julia Horvath)}//
	\endgl
\xe

\cite{alexiadouetal14} discuss cases of disjoint reference effects and control into purpose clauses in adjectival passives, further making the point for the agent to be active in the structure in one way or another. In terms of analysis, they synthesize the existing literature by proposing that VoiceP can always be embedded under an adjectivizing head. Put otherwise, Voice is present in adjectival passives that are derived from transitive verbs. This means that resultative participles are derived by merging VoiceP with \emph{a} (or some similar head). The question of why only certain modifiers are possible receives a semantic rather than syntactic explanation: an implicit EA is available even when not represented syntactically, but the kinds of modifiers available for it are restricted semantically \citep{bhattpancheva06}. For languages such as Greek which allow all possible modifiers with adjectival passives, a structural explanation is given according to which an aspectual head is embedded below the adjectivizer \citep{anagnostopoulou03,alexiadouetal14}.

The next analytic question, then, is whether there are any cases in which a vP (but not VoiceP) merges with the adjectivizing head. While the authors present evidence for overt v in certain participles, they do not show that another Voice layer is impossible in such constructions. I will thus assume that the two possible structures are as follows.
\pex
	\a Adjective (stative): {[}\root{Root} a]
	\a Adjectival passive (resultative): {[}[[\root{Root} v] Voice] a]
\xe
\citet[391]{bruening14nllt} makes note of constructions like \emph{suddenly fallen leaves}, speculating that unaccusative adjectival passives are the result of merging \vz~with a vP, such that the event-modifying adverb attaches to vP rather than VoiceP. Another possibility is that his example is a case of the adjectivizer attaching to vP, although such an analysis cannot explain why these underlying verbs would be unaccusative in the first place. I will stick to the structures in~(\lastx).

I also adopt the following semantics for a resultant state adjective from \cite{kratzer00bls}:
\ex \denote{Adj} = $\lambda$R$\lambda$t$\exists$e,y.R(e)(y) \& $\tau$(e) $\le$ t
\xe

We are now in a position to examine the Hebrew facts.



Returning to adjectival passives, there is reason to think that the split between synthetic and analytic forms is the result of a difference between verbs and adjectives. The Hebrew direct object marker \emph{et} is licensed by verbs, (\nextx a), but it never appears in analytic forms in Hebrew when they have a stative reading, (\nextx b), shown here with active forms (which license Accusative). \cite{horvathsiloni08} give additional reasons for assigning the two forms to these two lexical categories. Accordingly, we analyze the synthetic forms as verbal and the analytic forms as adjectival.
\pex
	\a \begingl
		\gla ha-arafel texef jexase \textbf{et} kol ha-rexov//
		\glb the-fog soon will.cover \gsc{ACC} all the-street//
		\glft `The fog is about to cover the entire street.'//
		\endgl
	
	\a \ljudge{*}
		\begingl
		\gla ha-arafel ha-kaved \textbf{jihie} mexase \textbf{et} kol ha-rexov//
		\glb the-fog the-heavy will.be cover.\gsc{INTNS}.Pres \gsc{ACC} all the-street//
		\glft (int.~`The heave fog will be covering the entire street)//
		\endgl
\xe

Recent syntactic analyses of adjectival passives have motivated an embedded VoiceP layer \citep{mcintyre13,alexiadouetal14,bruening14nllt,doron14adj}.
The main difference between verbal passives and adjectival passives is that VoiceP is embedded by a verbal passive head in the former and by an adjectival/aspectual head in the latter. Full structures are given next. For an account of how the nonconcatenative morphology itself is instantiated, i.e.~how the functional heads get mapped to vowels, see \cite{tucker15roots} or \cite{kastner16nllt}.
\pex\label{ex:heb-synth}Synthetic, verbal passive:
    \a \begingl
        \gla dani jesorak//
        \glb Danny combed.\gsc{INTNS.PASS}.Fut//
        \glft `Danny will get combed.'//
    \endgl
    
	\Tree
        [.TP
            [.DP\\\emph{dani} ]
            [
                [.T$_{\textrm{[Fut]}}$\\\emph{je-} ]
                [.PassP
                    [.Pass\\\emph{-o-a-} ]
                    [.VoiceP
                        [.Voice ]
                        [.vP
                            [.v$_{intns}$
                                [.v$_{intns}$ ]
                                [.\root{srk} ]
                            ]
                            [.DP\\\sout{\emph{dani}} ]
                        ]
                    ]
                ]
            ]
        ]
\xe

The internal argument of adjectival passives has been proposed by \citet[385]{bruening14nllt} to be an Operator, coreferential with the noun interpreted as the argument. We adopt this treatment of adjectival passives in our analysis of Hebrew.

\pex\label{ex:heb-ana}Analytic, adjectival passive:
    \a \begingl
        \gla dani \textbf{jihie} mesorak//
        \glb Danny will.be combed.\gsc{INTNS.PASS}.Pres//
        \glft `Danny will be combed (already).'//
    \endgl
    \a \Tree
        [.TP
            [.{DP$_i$}\\\emph{dani} ]
            [
                [.T$_{\textrm{[Fut]}}$\\\emph{ji-} ]
                [.vP
                    [.v$_{be}$\\\emph{hie} ]
                    [.DP$_i$
	                    [.D ]
	                    [.NP
		                    [.N\\\sout{\emph{dani}} ]
	                        [.aP
	                            [.a\\\emph{me-o-a-} ]
	                            [.VoiceP
	                                [.Voice ]
	                                [.vP
	                                    [.v$_{intns}$
	                                        [.v$_{intns}$ ]
	                                        [.{\root{srk}} ]
	                                    ]
	                                    [.{Op$_i$} ]
	                                ]
	                            ]
	                        ]
	                    ]
                    ]
                ]
            ]
        ]
\xe

The derivation is similar for non-finite environments. The analytic passive form is not in violation of (\ref{selection}) as T$_{\textrm{[-Fin]}}$ selects a vP, not a PassP.
\pex
	\a \begingl
		\gla {\dots} garam le-dani \textbf{lihiot} mesorak//
		\glb {} caused to-Dani to.be combed.\gsc{INTNS.PASS}.Pres//
		\glft `... caused Dani to be combed.'//
	\endgl

    \a \Tree
        [.TP
            [.{DP$_i$}\\\emph{dani} ]
            [
                [.T$_{\textrm{[-Fin]}}$\\\emph{li-} ]
                [.vP
                    [.v$_{be}$\\\emph{hiot} ]
                    [.DP$_i$
	                    [.D ]
	                    [.NP
		                    [.N\\\sout{\emph{dani}} ]
	                        [.aP
	                            [.a\\\emph{me-o-a-} ]
	                            [.VoiceP
	                                [.Voice ]
	                                [.vP
	                                    [.v$_{intns}$
	                                        [.v$_{intns}$ ]
	                                        [.{\root{srk}} ]
	                                    ]
	                                    [.{Op$_i$} ]
	                                ]
	                            ]
	                        ]
	                    ]
                    ]
                ]
            ]
        ]
\xe



      \subsubsection{Diagnostics in Hebrew}
Adjectival passives appear in one of the two passive participial forms \mpua~and \mhuf~(participles of \tpua~and \thuf~respectively), or in the \emph{XaYuZ} form associated with \tkal. Hebrew participles serve as present tense verbal forms and as Romance-style participles, by which I mean a mixed nominal-adjectival category. The Hebrew participle is, in general, ambiguous in form between a verb and an adjective or noun \citep{boneh13tense,doron13ehll}. In \tkal~the active participle can be either a verb or a noun. In other templates (and in the \emph{XaYuZ}~passive participle) an adjectival reading is also available, as with \emph{metsujan} `excellent' in~(\nextx b).
\pex
	\a \begingl
		\gla ha-ʃelet \underline{more} al ha-derex la-park//
		\glb the-sign indicates.\gsc{PTCP.SMPL} on the-road to.the-park//
		\glft `The sign is indicating the way to the park.'//
	\endgl
	
	\a \begingl
		\gla josi \underline{more} \textbf{metsujan}//
		\glb Yossi teacher.\gsc{PTCP.SMPL} excellent.\gsc{INTNS.PASS.Pres}//
		\glft `Yossi is an excellent teacher.'//
	\endgl
\xe
What this implies for us currently is that \mpua~and \mhuf~are ambiguous between a verbal form and an adjectival form, just like English \emph{closed}.

\cite{doron00} establishes ten diagnostics distinguishing verbal passives from adjectival passives (in fact, many of them distinguish verbs from adjectives in general). Here I give a few examples of what these differences look like. Importantly, only bounded events (change-of-state and inchoatives) can serve as input to adjectival passives (which are resultative).

In active forms, the finite verb often contrasts with an analytic combination of copula and participle. Consider synthetic future verbs~(\nextx a) and analytic future participles~(\nextx b).
\pex \label{ex:pres-act}
 \a Future verb:\\
 \begingl
     \gla maxar ani \{\underline{oxal} \emph{/} \underline{aklit}\}//
     \glb tomorrow I will.eat.\gsc{SMPL} {} will.record.\gsc{CAUS}//
     \glft `Tomorrow I'll eat/record something.'//
 \endgl

 \a Future copula with a participle:\\
 \begingl
     \gla \ljudge{*}maxar ani \underline{eheje} \{\textbf{ox\'el} \emph{/} \textbf{maklit}\}//
     \glb tomorrow I will.be eat.\gsc{SMPL.PRES} {} record.\gsc{SMPL.CAUS}//
     \glft (int. `Tomorrow I will be eating/recording.')//
 \endgl
\xe
\cite{doron00} shows that verbs are not allowed after a future tense copula, so the forms in (\lastx b) must be adjectives or nominals. They can be used when the participle is used in a generic context as a noun, as in ``\underline{eater} of vermin'' (\nextx a) or ``\underline{recorder} of things'' (\nextx b). This is to be expected if the complement of the copula in~(\nextx) is a participle.
\pex \label{ex:pres-act2}
 \a Analytic use of the ``simple'' participle:\\
 \begingl
     \gla az tagidi, ʃe-rak ani \underline{eheje} \textbf{ox\'el} ʃratsim ve-ʃ'ar mini basar ha-'asurin al jehudim? ;-)//
     \glb so say.\gsc{2SG.F.FUT}, \gsc{COMP}-only I will.be eat.\gsc{SMPL.PRES} vermin and-rest kinds.\gsc{CS} meat the-proscribed on Jews//
     \glft `So say so! What, you want me to be the only one here who eats vermin and other kinds of meat that are proscribed for Jews? ;-)'\footnotemark//
 \endgl
\footnotetext{\url{http://www.tapuz.co.il/forums2008/archive.aspx?ForumId=1277&MessageId=96791273} (retrieved November 2014). The example appears in a forum conversation in which participants discuss their experiences eating shrimp in Norway. \emph{ʃratsim} `vermin' is the traditional term for non-Kosher foods such as seafood. The adjective \emph{asurin} `proscribed' is written in an intentionally jocular/archaic way, with a final -\emph{n} that has changed to -\emph{m} in the modern language.}

 \a Analytic use of the ``causative'' participle:\\
 \begingl
     \gla kanir'e ʃe-ani \underline{eheje} \textbf{maklit} kavua ʃel ze//
     \glb probably \gsc{COMP}-I will.be record.\gsc{CAUS}.Pres constant of this//
     \glft `Looks like I'll be the one recording this', `Looks like I'll be a constant recorder of this'\trailingcitation{\url{http://www.forumtvnetil.com/index.php?showtopic=18312}}//
 \endgl
\xe

It is thus possible to tell apart verbal passives from adjectival passives in Hebrew, and to tease apart different readings of the participle. For example, the implied present tense in~(\nextx a) is ambiguous between a verbal (progressive) reading and an adjectival (resultative) reading, whereas the future tense in~(\nextx b) is unambiguously adjectival \citep{doron00,horvathsiloni08,meltzerasscher11}.
\pex \label{ex:pres-ambig}
    \a \begingl
        \gla ha-kontsert \underline{muklat}//
        \glb the-concert record.\gsc{CAUS.PASS}.Pres//
        \glft `The concert is being recorded.'\\`The concert has been recorded.'//
    \endgl
        
    \a \begingl
        \gla ha-kontsert \underline{jihie} \textbf{muklat}//
        \glb the-concert will.be record.\gsc{CAUS.PASS}.Pres//
        \glft `The concert will have (already) been recorded.'//
    \endgl
\xe

\cite{kastnerzu15li} emphasize two differences between verbal and adjectival passives which were mentioned in passing in the literature (see especially \citealt{meltzerasscher11}). First, whereas the analytic forms may have an \textbf{idiomatic reading}~(\nextx a), synthetic passives~(\nextx b) are always compositional (cf.~\citealt{horvathsiloni09}).
\pex \label{ex:idiom}
    \a 
        \begingl
        \gla \rightcomment{(idiomatic)}ze \underline{jihie} \textbf{muvan} \textbf{me-elav}//
        \glb this will.be understand.\gsc{CAUS}.\gsc{PASS}.Pres from-to.him//
        \glft `It will be self-evident.'//
        \endgl

    \a \ljudge{\#}
        \begingl
        \gla \rightcomment{(literal)}ze \underline{juvan} me-elav//
        \glb this understand.\gsc{CAUS}.\gsc{PASS}.Fut from-to.him//
        \glft (no immediate clear meaning)//
        \endgl
\xe

Passive participles, being adjectival passives, can take on idiomatic readings regardless of their template. The passive participle of ``simple'' \emph{matsats} `sucked' can have an idiomatic reading~(\nextx a), but mediopassive ``middle'' \emph{nimtsats} is understood literally~(\nextx b).
\pex
    \a
        \begingl
        \gla \rightcomment{(idiomatic)}ze \underline{haja} \textbf{matsuts} \textbf{me-ha-etsba}//
        \glb this was sucked.\gsc{SMPL} from-the-finger//
        \glft `It was entirely made up.'//
        \endgl
    
    \a
        \begingl
        \gla \rightcomment{(literal)}ze \underline{nimtsats} me-ha-etsba//
        \glb this suck.\gsc{MID}.Past from-the-finger//
        \glft `This was sucked from the finger.' (no idiomatic reading)//
        \endgl
\xe

Second, synthetic passives force disjoint readings in which the external argument and the internal argument cannot refer to the same entity \citep{bakeretal89}. The adjectival form~(\nextx a), with the participle, allows coreference whereas the verbal form~(\nextx b) does not \citep[720]{sichel09}:
\pex \label{ex:disjoint}
    \a \begingl
        \gla \rightcomment{(agent =/$\neq$ theme)}ha-jalda \underline{hajta} \textbf{mesorek-et}//
        \glb the-girl was comb.\gsc{INTNS.PASS}.Pres-\gsc{F}//
        \glft `The girl was combed.'//
        \endgl
    \a
        \begingl
        \gla \rightcomment{(agent $\neq$ theme)}ha-jalda \underline{sork-a}//
        \glb the-girl comb.\gsc{INTNS}.\gsc{PASS}.Past-\gsc{F}//
        \glft `The girl got combed.'//
        \endgl
\xe

The picture for Hebrew is thus fairly similar to that in the Romance and Germanic languages discussed in the literature. Where Hebrew differs is in the differences between templates.

All three templatic forms are compatible with both stative adjectives and adjectival passives, as already mentioned. And while all three adjectival passive forms are compatible with external arguments (under the restrictions noted above), adjectival passives in \mhuf~\emph{require} an implied EA to be interpreted.

\citet[170]{doron14adj} shows that \textbf{stative adjectives} are incompatible with event modifications or event readings. Some of them even have no corresponding underlying verb:
\pex
	\a \begingl
		\gla ti'un \underline{barur} (*bekfida)//
		\glb argument clear \phantom{(*}carefully//
		\glft `A clear argument'//
	\endgl

	\a \begingl
		\gla beged \underline{mexoar} (*beriʃul)//
		\glb garment ugly_{\text{\gsc{INTNS}}} \phantom{(*}carelessly//
		\glft `An ugly garment'//
	\endgl

	\a \begingl
		\gla pirxax \underline{mufra} (*bexipazon)//
		\glb brat deranged_{\text{\gsc{CAUS}}} \phantom{(*}hastily//
		\glft `A deranged brat'//
	\endgl
\xe

\citet[175]{doron14adj} also observes that (resultative) \textbf{adjectival passives} in ``causative'' {\mhuf} obligatorily entail an EA, even if it is implicit and not overtly represented. While an adjectival passive in \mpua~does not entail the existence of EA, (\nextx a), every adjectival passive in \mhuf~does, (\nextx b). In a telling near-minimal pair, the athletes in~(\nextx a) might have trained on their own, but the athletes in~(\nextx b) must have been trained through some kind of organized program.
\pex\label{ex:sportaim}
	\a \begingl
		\gla\rightcomment{(\mpua)}sportaim \underline{meuman-im} bekfida//
		\glb athletes trained.\gsc{INTNS.PASS}-\gsc{PL} carefully//
		\glft `Carefully trained athletes'//
	\endgl
	
	\a \begingl
		\gla\rightcomment{(\mhuf)}sportaim \underline{muxʃar-im} bekfida//
		\glb athletes prepared.\gsc{CAUS.PASS}-\gsc{PL} carefully//
		\glft `Carefully trained athletes'//
	\endgl
\xe
\cite{doron14adj} attributes this difference to the behavior of the causative head $\gamma$ which underlies \thif. My analysis, using \vd~(as in \S\ref{syn:templates:thif}), follows in the same vein. Note that the implied EA is not syntactically represented; it cannot, for example, create a new discourse referent.
\ex\label{ex:sportait} \ljudge{*}
	\begingl
		\gla nadia komanetʃi hajta sportait \emph{(EA_i)} \underline{muxʃer-et} bekfida. \textbf{hu_i} asa avoda tova aval safag harbe bikoret//
		\glb Nadia Com\u{a}neci was athlete.\gsc{F} {} prepared.\gsc{CAUS.PASS}-\gsc{F} carefully he did job good but absorbed much criticism//
		\glft (int. `Nadia Com\u{a}neci was a carefully trained athlete. He (=B\'ela K\'arolyi) did a good job but was heavily criticized.')//
	\endgl
\xe
Let us derive the different forms.

      \subsubsection{Adjectival passives in Hebrew}
Simple (stative) adjectives are derived by merging an adjectivizing \emph{a} head with the root, though we will need to postulate different \emph{a} heads. Resultatives are derived by merging a general \emph{a} head with a full VoiceP.

\textbf{Stative adjectives} have no event implications or internal structure. In the languages discussed above they are derived by merging an adjectivizing head with the root \citep{embick04li}:
\ex\label{ex:adj-en}
	\begin{minipage}[t]{0.3\textwidth}
		a. \emph{open}\\
%			[a
%				[{\root{\gsc{open}}} ]
%				[a ]
%			]
	\end{minipage}
	\begin{minipage}[t]{0.3\textwidth}
		b. \emph{closed} (stative reading)\\
%			[a
%				[{\root{\gsc{close}}} ]
%				[a\\\emph{-ed} ]
%			]
	\end{minipage}
\xe

In Hebrew, adjectives can appear in various morphophonological patterns.\footnote{As noted in Chapter 1, I use the term \emph{pattern} when referring to one of the morphophonological forms in the adjectival or nominal domains. There are, in principle, an unlimited number of distinct patterns, but only seven verbal \emph{templates}.} Two of these patterns are homopohonous with the present-tense (participle) verbal passives \mpua~and \mhuf. Therefore, I am forced to assume the existence of two separate adjectival heads, namely a_{\text{\gsc{INTNS}}} and a_{\text{\gsc{CAUS}}}, alongside any other possible patterns, just like English shows evidence of adjectival -\emph{ed} (\emph{wingéd}, \emph{learnéd}) alongside verbal -\emph{ed}. A given root typically has one basic adjectival form,~(\nextx).
\ex
	\begin{minipage}[t]{0.3\textwidth}
		a. \emph{barur} `clear' (\emph{XaYuZ})\\
%			[a
%				[{\root{brr}} ]
%				[a_{\text{XaYuZ}} ]
%			]
	\end{minipage}
	\begin{minipage}[t]{0.3\textwidth}
		b. \emph{katan} `small' (\emph{XaYaZ})\\
%			[a
%				[{\root{\dgs{k}tn}} ]
%				[a_{\text{XaYaZ}} ]
%			]
	\end{minipage}
	\begin{minipage}[t]{0.3\textwidth}
		c. \emph{ʃamen} `fat' (\emph{XaYeZ})\\
%			[a
%				[{\root{ʃmn}} ]
%				[a_{\text{XaYeZ}} ]
%			]
	\end{minipage}
\xe
An adjective might appear in this form or in the participial-like forms, with either subtle~(\ref{ex:adj-tpie}a--b) or substantial~(\ref{ex:adj-thif}a--b) differences in meaning.
\ex\label{ex:adj-tpie}
	\begin{minipage}[t]{0.3\textwidth}
		a. \emph{kaur} `ugly' (\emph{XaYuZ})\\
%			[a
%				[{\root{k'r}} ]
%				[a_{\text{XaYuZ}} ]
%			]
	\end{minipage}
	\begin{minipage}[t]{0.3\textwidth}
		b. \emph{mexoar} `ugly'\\
%			[a_{\text{\gsc{INTNS}}}
%				[{\root{k'r}} ]
%				[a_{\text{\gsc{INTNS}}} ]
%			]
	\end{minipage}
\xe

\ex\label{ex:adj-thif}
	\begin{minipage}[t]{0.3\textwidth}
		a. \emph{parua} `wild' (\emph{XaYuZ})\\
%			[a
%				[{\root{pr'}} ]
%				[a_{\text{XaYuZ}} ]
%			]
	\end{minipage}
	\begin{minipage}[t]{0.3\textwidth}
		b. \emph{mufra} `deranged'\\
%			[a_{\text{\gsc{CAUS}}}
%				[{\root{pr'}} ]
%				[a_{\text{\gsc{CAUS}}} ]
%			]
	\end{minipage}
\xe

{An alternative would be to assume that even these stative adjectives have underlying verbal structure, except that this structure is not interpreted. This approach is reminiscent of the Greek facts mentioned in \S\ref{syn:middle:nonactive:elena}. To recap, certain adjectives in Greek can only be derived if a verbalizing suffix is first added to the root. No verbal, eventive semantics is entailed: there is no weaving in~(\ref{ex:elena1b}) or planting for~(\ref{ex:elena2b}). Adjectival -\emph{tos} is argued to need an eventive vP as its base, something which is not possible with nominal roots like `weave' and `plant'.}
\ex \label{ex:elena1b} \emph{if-an-tos} weave-\gsc{VBLZ}-\gsc{ADJ} `woven'
\xe
\ex \label{ex:elena2b} \emph{fit-ef-tos} plant-\gsc{VBLZ}-\gsc{ADJ} `planted' \hfill \citep[97]{elenasamioti14}
\xe
{Perhaps in the Hebrew cases above there is only one adjectivizing head \emph{a}, which takes a verbal structure that is not interpreted. I do not have particular reason to support one view or the other, and so I stick to the analyses in~(\ref{ex:adj-tpie})--(\ref{ex:adj-thif}) simply because they involve less structure. The same point can be made for R-nominals in the next section, \S\ref{syn:templates:nmlz}. Note, however, that this alternative should then extend to English cases such as (\ref{ex:adj-en}b): what is to stop us from assuming underlying verbal structure in \emph{closed} which is simply not interpreted before being adjectivized by -\emph{ed}?}

Moving on to \textbf{resultatives (adjectival passives)}, the main difference between them and stative adjectives is that the former embed VoiceP. The internal argument of adjectival passives has been argued by \citet[386]{bruening14nllt} to be an Operator, bound by the noun interpreted as the argument.
\pex\label{ex:heb-ana}
    \a \begingl
        \gla ha-sefer \underline{jihie} \textbf{katuv} be-et kaxol//
        \glb the-book will.be written in-pen blue//
        \glft `The book will be (will have been) written in blue ink.'//
    \endgl
    \a 
%        [TP
%            [{DP_i}\\\emph{ha-sefer}\\\emph{the book} ]
%            [
%                [T_{\textrm{[Fut]}}\\\emph{ji-} ]
%                [vP
%                    [v_{be}\\\emph{-hie} ]
%                    [\textbf{aP}
%                        [\phantom{xxxx}{Op_i}\phantom{xxxx} ]
%                        [\textbf{aP}
%                            [a\\\emph{-a-u-} ]
%                            [VoiceP
%	                            [VoiceP
%	                                [Voice ]
%	                                [vP
%		                                [v
%		                                    [v ]
%		                                    [{\root{ktv}}\\\root{\gsc{WRITE}} ]
%										]
%	                                    [\sout{Op_i} ]
%	                                ]
%	                            ]
%	                            [pP
%		                            [\emph{be-et kaxol}\\\emph{in blue ink}, triangle ]
%								]
%	                        ]
%                        ]
%                    ]
%                ]
%            ]
%        ]
\xe

Let us make sure that the combinatorial possibilities of our syntactic inventory predict the correct adjectival passives:

\begin{itemize}
\item \textbf{[a [Voice [v \root{root}~\!]]]} -- attested, as in~(\lastx).

\item \textbf{[a [[Voice {\va}~\!] [v \root{root}~\!]]]} -- attested, adjectival passive in \mpua. Our account of {\va} predicts agentive entailments.

\item \textbf{[a [{\vd} [v \root{root}~\!]]]} -- attested, adjectival passive in \mhuf. Our account of {\vd} requires a DP in Spec,VoiceP. On the other hand, external arguments are not represented in adjectival passives. Intuitively, the result should be an external argument which is not represented syntactically. This seems to be correct, as noted in connection with examples~(\ref{ex:sportaim}--\ref{ex:sportait}), where an implicit external argument is entailed.
\ex
%		[aP
%			[DP_i\\\textsc{Internal Argument}]
%			[aP
%				[a\\\emph{m-} ]
%				[VoiceP
%					[{\vd}\\\emph{u-a} ]
%					[vP
%						[v
%							[v ]
%							[\root{root} ]
%						]
%						[Op_i ]
%					]
%				]
%			]
%		]
\xe

\item \textbf{\vz} -- incompatible with adjectival passives. Informally, adjectival passives denote the result of an event without explicitly naming the cause, though one is assumed; in this sense they are similar to verbal passives. \cite{alexiadouetal14} and \cite{bruening14nllt} implement this by allowing Adj (and Pass) to only select for Voice that needs to fill its specifier. \vz~is not such a Voice head (although \citealt{embick04li} does allow his \vz~to derive unaccusative adjectival passives in English): since there is no expectation of an external argument, there is no adjectival passive.
%\citet[188]{doron14adj}: ``Resultative participles are derived from the root by the minimal non-active structure''. I guess it depends on the semantics of \vz. If it says ``no EA'' in the semantics as well as the syntax then it's ok.
\end{itemize}

These derivations are similar to the ones in \cite{doron14adj}, though I depart from her specific implementation for a number of reasons. First---and as I return to in \S\ref{syn:other-root:ed}---the functional heads used by \citeauthor{doron14adj} are syntactico-semantic primitives which drive the semantics but do not translate straightforwardly into the morphophonology as syntactic heads usually do. Additionally, and more specifically to adjectival passives, \citeauthor{doron14adj} utilizes an active Voice head introducing the EA-related head v, which in turn introduces the external argument. In order to produce a verb in active voice, then, her system needs a lower head that requires Active Voice -- this is CAUSE/$\gamma$ -- so that CAUSE introduces Active Voice, Active Voice introduces v, and v introduces the external argument. Some of these heads split up the semantic work that can be done by one head (Voice and v in particular), and not all of them have overt spell-out. There are consequently more syntactic elements than seems necessary. Finally, the agentive head INTNS/$\iota$ is diffused in certain cases, for example in stative \mpua~adjectives, but it is not explained how this head loses its agentive semantics in this context.


%\cite{doron14adj}
%So you can either have a/Asp over vP or a_{INTNS}/Asp_{INTNS} over vP, but if you include Voice it's VoiceD? Or it's a_D?
%11: So what about Meltzer's examples, by-phrases etc? That's because you see the effect on the result state.
%16: So far so good. [+Voice] really means +EA.
%21: SMPL and INTNS resultatives have just v, no Voice.
%  For her then, agentivity is a propert of Voice in the environment of iota/INTNS. It makes sense.
%22: So I can say that either Voice gets pronounced as CAUS in the environment of a/Asp, or that you really do need \vd, but in that case, why not the others?
%  In ED's system, the question is why you need gamma and can't get away with just having v. She has v introduced by the active Voice though, and then in turn introducing the EA.
%  So for Active Voice she needs something the requires Active, and that's Gamma, but Gamma introduces (active) Voice which in turn introduces the external argument, and that's just one element too many.


    \subsubsection{Breakdown by template}
I will now summarize the main generalizations noted by previous works. The overview proceeds by template and relies heavily on \cite{doron00,doron14adj} and \cite{meltzerasscher11}. Both authors attempted to predict whether a certain root can be instantiated as an adjective, an adjectival passive, or both, in each of the three relevant templates. In general, \cite{doron00} notes that change-of-state roots are better inputs to adjectival passives than atelic events.

% doron00: Some roots have only adjpass, some have only verbpass. COS, telos, inchoative aspect (beginning of event, raxuv vs *dahur, state that can be referred to). Some other factors as well: roots of motion (musa, muval) are only verbal. Roots of relation (sanu, ahuv, bazuy) or only adjectival and don't all have a corresponding verb.

% Consistently, \tpua~and \thuf~allow an EA but \tkal~not always. Depends on the lexical semantics of the root.

% For each template:
%   1. Adjpass and/or verbpass? Depends on the template.
%   2. EA in adjpass possible (Meltzer's ``true adjectival passive'' vs ``adjectival decausative'')? Depends on the root in \tpua, possible in \tpie~and \thif, in fact obligatory in \thif.

\paragraph*{\tkal~(adjectival form \emph{XaYuZ})}
No verbal passive exists for \tkal, but stative and resultative adjectives are both possible.

Only change of state roots are possible input to adjectives in this template \citep{doron00}. For example, the form *\emph{karu} (int. `read') does not exist as a stative adjective or as an adjectival passive:
\ex
  \begingl
    \gla ha-mixtav \underline{katuv} \emph{/} *\underline{karu}//
    \glb the-letter written {} read//
    \glft `The letter is written (*is read).'//
  \endgl
\xe

For those roots that can form adjectives, the main difference is between roots that derive intransitive verbs in \tkal~and those that derive transitive verbs. The former lead to stative adjectives and the latter to adjectival passives {(see }\citealt{meltzerasscher11}{ for a lexicalist account)}.
\pex 
  \a Stative verbs from intransitives: \emph{kafu} `frozen' $<$ \emph{kafa} `froze'; \emph{davuk} `glued' $<$ \emph{davak} 'stuck to'. %ED has more on p56
  \a Adjectival passives are possible with change of state roots: \emph{ʃavur} `broken' $<$ \emph{ʃavar} `broke'; \emph{sagur} `closed' $<$ \emph{sagar} `closed'; \emph{saruf} `burnt' $<$ \emph{saraf} `burned'. %ED has more on p53
  \a No corresponding verb in \tkal, no telos: \emph{paʃut} `simple', \emph{savux} `complex', \emph{pazur} `scatterd', \emph{ʃaluv} `intertwined', \emph{akum} `crooked', \emph{tarud} `preoccupied'.
  \xe
The roots underlying~(\lastx c) do not appear as verbs in \tkal, meaning that they cannot combine with v and Voice. If this is the case, they cannot form the underlying VoiceP necessary for an adjectival passive and are only possible as input to stative adjectives. For the roots in~(\lastx a), their corresponding \tkal~verbs are intransitive. This means that the interpretation of [Voice [v \root{db\dgs{k}}~\!]], for example, is unaccusative. If this is the case, then an implicit EA cannot be licensed, since unaccusatives have no EA.


\paragraph*{\tpie~(adjectival form \mpua)}
Both verbal and adjectival passives are possible in this template. \cite{lakscohen16} provide evidence that the middle stem vowel might be pronounced slightly differently for verbs and adjectives, further supporting the split between the two.

Among the adjectives, there are two kinds of stative adjectives: those that do not have a corresponding verb, (\nextx a), and those that are homophonous with an adjectival passive like English \emph{closed} is, as it can be stative or resultative, (\nextx b).
\pex
  \a No corresponding verb: \emph{meguʃam} `clumsy' ($\nless$ *\emph{giʃem}), \emph{meunax} `vertical' ($\nless$ *\emph{inex}), \emph{memuʃma} `disciplined' ($\nless$ *\emph{miʃmea}), \emph{metupaʃ} `silly' ($\nless$ *\emph{tipeʃ}).
  \a Ambiguous between resultative and stative: \emph{megune} `obscene', \emph{mekubal} `accepted', \emph{mefuzar} `scattered', \emph{meluxlax} `dirty', \emph{megulgal} `rolled up', \emph{mekulkal} `out of order'. %mesubax, meSulav, meukam,
\xe
% mefuSat, menupax - AMA:831 says these are necessarily EA but I don't think so

The verbs underlying~(\lastx b), and any which do not fall under~(\lastx a), can form adjectival passives. For the forms in~(\lastx b), the stative reading is more salient and is often different than the compositional adjectival passive reading. For instance, the adjectival passive \emph{megune} literally means `that which has been censured'.

\paragraph*{\thif~(adjectival form \mhuf)}
Both verbal and adjectival passives are possible in this template.

Stative adjectives are only possible from roots that do not have a corresponding verb in \thif, (\nextx a). A form ambiguous with a resultative might also exist, in which case its meaning is different, (\nextx b). For example, \emph{muʃlam} `perfect (stative adj.)'/`that which has been completed (adj. pass)'.
\pex
  \a No corresponding verb: \emph{muda} `aware', \emph{muʃlag} `snowy', \emph{mugaz} `carbonated'.
  \a Ambiguous between resultative and stative: \emph{muʃlam} `perfect', \emph{mufʃat} `abstract'.
\xe
As an innovation, a verb might be back-formed based on adjectives like those in~(\lastx a) or derived from the related noun. For example, the substandard verb \emph{heʃlig} `snowed' is attested in the poet Bialik's work and can be found in use online.
%mutslax?
%mukaf? doron14adj ff1 says yeah, not a problem.

Adjectival passives are available for all roots that have verbs in \thif. As discussed above, these constructions entail an implied EA.

	\subsubsection{Summary}
In closing, we have now accounted for the existing generalizations regarding what kind of passive (verbal or adjectival) and what kind of adjective (stative or resultative) can appear with what kind of root in each of the templates. Table \ref{table:adjpass-heb} is repeated to conclude this section. The analysis of Hebrew provides further evidence for an eventive layer in adjectival passives. Hebrew also supports the claim that the same morphophonological form can spell out both stative and adjectival passives. This subsection provided an explicit syntax of how the various readings arise.
\begin{table}[h!t] \centering \small
\begin{tabular}{|c|c|l|c|ll|} \hline
	& Interpretation & Heads/structure & EA? & Form & (template) \\\hline\hline
\multirow{3}{*}{Adjectives} & \multirow{3}{*}{stative} & \root{root} a_{\text{\gsc{SMPL}}} & \xmark & \emph{XaYuZ} & (\tkal)\\\cline{3-6}
& & \root{root} a_{\text{\gsc{INTNS}}} & \xmark & \mpua & (\tpie) \\\cline{3-6}
& & \root{root} a_{\text{\gsc{CAUS}}} & \xmark & \mhuf & (\thif) \\\hline\hline
%& & [\root{Root} \va] a & \xmark & \mpua & (\tpie) \\\hline\hline
\multirow{3}{*}{Adjectival passives} & \multirow{3}{*}{resultative} & [Voice [v \root{root}]] a & \cmark/\xmark & \emph{XaYuZ} & (\tkal)\\\cline{3-6}
& & [Voice {\va} [v \root{root}]] a & \cmark/\xmark & \mpua & (\tpie)\\\cline{3-6}
& & [{\vd} [v \root{root}]] a & \cmark & \mhuf & (\thif)\\\hline
\end{tabular}
\caption{Adjectival passives in Hebrew by template.}
\end{table}

Finally, it is worth pointing out that {the adjectival passive} is still productive, especially since passives have been characterized as no longer productive in Hebrew, a claim that seems too strong given novel forms such as{ the adjectival passive} \emph{meturgat} `targeted':
\ex ``For whatever reason, after years of complete openness with Google, and full access to all of the data and information that I produce, it looks like the only thing they know about [me] is that I'm a man. Enough already! I'm tired of ads for shaving, cars, insurance and cologne! \dots ''\\
	\begingl
		\gla ex ani jaxol ligrom le-gugel latet l-i pirsom-ot ʃe-beemet \underline{meturgat-ot} el-aj//
		\glb how I can to.cause to-Google to.give to-me ad-\gsc{F.PL} \gsc{COMP}-really targeted.\gsc{INTNS.PASS.Pres}-\gsc{F.PL} to-me//
		\glft `How can I get Google to give me ads that are really targeted to me?'\trailingcitation{\url{http://www.facebook.com/elad.lerner/posts/1207164259295353}}//
	\endgl
\xe
Here, as elsewhere in the language, only change of state roots can serve as input to adjectival passives.



\section{Nominalization} \label{passn:n}


This section addresses the deverbal nominalization, also known as gerund, gerundive, action noun and \emph{masdar}. The overall claim will be similar to that made for adjectival passives in \S\ref{syn:templates:adjpass}: nominal forms can arise in two ways. One is by nominalization of an existing verbal form, in which case the nominalizer is little n and the result is a nominal with internal verbal structure. The other is by nominalizing a root using a nominalizer with specific morphophonological form, which may or may not be similar to that of eventive nominalizations.

I operate under the working assumption that Hebrew has the same three kinds of nouns as English and other languages \citep{grimshaw90,alexiadou10b,borer14lingua}.
	\begin{itemize*}
	\item \textbf{``Simple'' nominals} appear monomorphemic.
	\item \textbf{``AS-nominals''} (argument structure nominals) are nominalizations of verbal forms. They have internal argument structure and are often homophonous with an R-nominal (see next).
	\item \textbf{``R-nominals''} (result nominals) are nominalizations without argument structure, though they appear polymorphemic. They are often homophonous with an AS-nominal.
	\end{itemize*}
%The following table exemplifies using Hebrew and English.
%\ex Nominalizations in Hebrew and English:\\
%	\begin{tabular}{l|lll|l}
%	 & & Hebrew & & English\\\hline
%	\multirow{2}{*}{Simple}     & a. & \emph{kelev} & `dog' & \emph{book}, \emph{dog}\\
%			&    b. & \emph{telefon} & `phone' & \emph{phone}, \emph{car}\\\hline
%	AS-nominal & c. & \emph{kibuts} & `a gathering' & \emph{destruction}\\\hline
%	R-nominal  & d. & \emph{kibuts} & `kibbutz'     & \emph{destruction}\\
%	\end{tabular}
%\xe

\textbf{Simple nominals} have no internal structure: there are no arguments to bookhood.
\ex \ljudge{*} The enemy's book of the city.
\xe
In Hebrew various nominal patterns can be used to derive nouns, all of which are variants of nominalizing little n, just like in English.
\ex
	\begin{minipage}[t]{0.3\textwidth}
		a. \emph{kelev} `dog'\\
%			[n
%				[{\root{klb}} ]
%				[n_{\text{XeYeZ}} ]
%			]
	\end{minipage}
	\begin{minipage}[t]{0.3\textwidth}
		b. \emph{telefon} `phone'\\
%			[n
%				[{\root{tlfn}} ]
%				[n_{\text{XeYeZoW}} ]
%			]
	\end{minipage}
\xe

\textbf{AS-nominals} have internal structure, as discussed at length by various authors \citep{chomsky70,grimshaw90,marantz97,harley09n,alexiadou10b,borer13oup}. In~(\nextx b), Hebrew \emph{haʃmada} `destruction' is derived from the \thif~verb \emph{heʃmid} `destroyed'.
\pex\label{ex:nom-destruct}
	\a The enemy's destruction_{\text{AS}} of the city (in less than a day).
	\a \begingl
		\gla haʃmada-t ha-ojev et ha-ir (tox jom)//
		\glb destruction_{\text{AS}}.\gsc{CAUS}-\gsc{CS} the-enemy \gsc{ACC} the-city within day//
		\glft `The enemy's destruction of the city (in a day).'//
	\endgl
\xe
To derive an AS-nominal, simply nominalize an existing verbal structure by adding n above an existing VoiceP structure \citep{hazout95,engelhardt00}. I abstract away from the question of where arguments are generated; we can assume that an operator is base-generated as the internal argument and the full DP is adjoined to the noun, as we did for the internal arguments of adjectival passives in~(\ref{ex:heb-ana}). See \citet[559]{borer13oup} for a similar conclusion on the similarity between nominalizations and adjectival passives.
\ex {\emph{haʃmadá} `destruction_{\text{AS}}'}\\
%	[n
%		[n\\\emph{-á} ]
%		[
%			[{\vd}\\\emph{ha-a} ]
%			[
%				[v
%					[v ]
%					[\root{ʃmd}\\\root{\gsc{DESTROY}} ]
%				]
%				[Op ]
%			]
%		]
%	]
\xe

\textbf{R-nominals} and AS-nominals can be homophonous, but they differ in their argument structure. The R-nominal equivalent of~(\ref{ex:nom-destruct}a) fails the various diagnostics for events:
\ex The destruction_{\text{R}} of the city (*in less than a day) was widespread.
\xe

Another difference between AS-nominals and R-nominals is that while the meaning of the AS-nominal is transparently related to that of the underlying verb, the R-nominal can have a special meaning. The pair in~(\nextx) exemplifies for Hebrew. The form \emph{kibuts} is ambiguous between an action nominalization (AS-nominal) of the verb \emph{kibets} `gathered' and an R-nominal derived directly from the root.
\pex
	\a {[}n [Voice {\va} [v \root{\dgs{k}bts}~\!]]]\\
	\begingl
		\gla medina-t israel tihie ptuxa le-alia jehudit ve-le-\textbf{kibuts} galujot//
		\glb state-\gsc{CS} Israel will.be open to-immigration Jewish and-to-gathering_{\text{AS}} diasporas//
		\glft `The State of Israel will be open for Jewish immigration and for the Ingathering of the Exiles.'\trailingcitation{(Israeli Declaration of Independence)}//
	\endgl
	\a {[}n_{\gsc{INTNS}} \root{\dgs{k}bts}~\!]\\
		``According to his testimony, in the early 60s, before he began his political career in the USA, \dots\\
	\begingl
		\gla ʃaha sanderz kama xodaʃim be-israel ve-hitnadev be-\textbf{kibuts}//
		\glb stayed Sanders a.few months in-Israel and-volunteered in-kibbutz_{\text{R}}//
		\glft \dots Sanders stayed in Israel for a few months and volunteered in a Kibbutz.''	\trailingcitation{\url{http://www.haaretz.co.il/news/world/america/us-election-2016/.premium-1.2842479}}//
	\endgl
\xe

An R-nominal might not even have any corresponding AS-nominal. The R-nominal \emph{kibuʃ} `occupation' is not derived from an underlying verb in \tpie. AS-nominals like \emph{kibuts} in the pattern \emph{Xi\dgs{Y}uZ} are derived from verbs like \emph{kibets} in \tpie, but the R-nominal \emph{kibuʃ} which has the pattern \emph{Xi\dgs{Y}uZ} is not derived from a verb in \tpie.
\pex
	\a \begingl
		\gla daj la-kibuʃ//
		\glb enough to.the-occupation_{\text{R}}//
		\glft `Down with the occupation!'//
	\endgl
	
	\a \ljudge{*} \emph{kibeʃ}
\xe
Separate nominal n heads are needed to derive ``R-nominals'', just like separate \emph{a} heads are needed to derive stative passives.
\ex
	\emph{kibuts} `kibbutz'\\
%		[n
%			[{\root{\dgs{k}bts}} ]
%			[n_{\gsc{INTNS}} ]
%		]
\xe

The following table summarizes.
\ex Nominalizations in Hebrew and English:\\
	\begin{tabular}{l|lll|l}
	 & & Hebrew & & English\\\hline
	\multirow{2}{*}{Simple}     & a. & \emph{kelev} & `dog' & \emph{book}, \emph{dog}\\
			&    b. & \emph{telefon} & `phone' & \emph{phone}, \emph{car}\\\hline
	AS-nominal & c. & \emph{kibuts} & `a gathering' & \emph{destruction}\\\hline
	R-nominal  & d. & \emph{kibuts} & `kibbutz'     & \emph{destruction}\\
	\end{tabular}
\xe

The architectural bottom line is that a VoiceP can serve as the input to further derivation. If Pass, little a or little n are merged above it, the result is entirely predictable: a passive verb, an adjectival passive or an AS-nominalization. Adjectives and nominalizations have forced us to make the theory slightly weaker in that there exist independent adjectivizers and nominalizers which look like the existing templates. We have just discussed a nominalizer n_{\gsc{INTNS}} which has the same output as nominalizing an existing verb, [n \tpie]. This result seems to be a necessary evil on the morphophonological side, leading to predictable results on the syntactic-semantic side. Nominalizations and adjectivizations of the root do not have internal structure and might carry different meaning than that of the homophonous complex form derived from the verb.

%\citet[559]{borer13oup} same for adjectives (adjectival passives). But we've already reached the conclusion for adjectival passives that we need independent templatic adjectivizers for idiosyncratic meaning. And really for meSufSaf there's little to choose from between calling it an adjective and a nominal. I think the overall picture is good for me: different templatic a/n heads are possible, and they give R-nomials with idiosyncratic meaning. But you can also attach Pass/a/n above a VoiceP to easily derive the compositional verbal passive / AS-nominal of that template. Hint at experimental differences between template priming and pattern priming?


%Do you need verbal structure for the morphology, and is the meaning the same as the embedded verb, and does the argument go anywhere it shouldn't as opposed to in English, for R-nominals and AS-nominals.
%
%dikui (dike)\\
%kibuS (*kibeS)
%
%haxtava (AS), from \vd P\\
%haxtava (R), from \root{ktv} with n_{\thif}. But then why does it have the same meaning as \emph{hixtiv}, HB asks.
%  Well does it? hafgana?
%  
%I'm not sure what breaks here. Maybe ask her again. Or work from her book.
%
%VoiceP (VP) in AS-nominalizations: hazout95,hazour98,engelhardt98,engelhardt02. AAS15?
%
%\citet[534ff13,555]{borer13oup} has template-specific nominalizers merging above templatic verbalizers. The forms are available in the system, which is what I'd want to say too. Now, their meanings are often similar but need not be. You can get different R-nominal forms that have different interpretation than the verb (Silum), and even R-nominals that have no verbs (kibuS). \citet[555]{borer13oup} has (31) where the R-nominal haftara in its liturgical sense, 3, has a nominalizer above the verbalizer and this nominalizer gives meaning, as opposed to the AS-nominal where it doesn't. So my question again is if she needs a \thif~nominalizer anyway, why not just attach it directly to the root to get the R-nominal. And I guess the answer is because you want the meaning to be similar to that of the verb... except we don't, because you do different en-searches and find different Contents, so that shouldn't be an argument.
%
%\citet[557]{borer13oup}, (33):
%  AS, compositional	R, idiosyncratic
%  Si'ur		Si'ur
%  kibuts	kibbuts
%  tsiun		tsiun
%  jiSuv		jiSub (ha-jeSuv)
%  tsivuj	tsivuj (zman)
%  SilSul (ha-xevel)	SilSul
%  hasbarat (ha-raajon) HASBARA
%  hanaxat (ha-garzen)	hanaxa (kesef, maxSava)
%  hitnaxlut	hitnaxlut

\section{Denominal verbs}
	Phonology: I might need a bit on word-derived words just to shut up the Bat-Els of the world once and for all (basically implementing Arad 2003 again because everyone else is too lazy to acknowledge it, e.g. Outi's 2017 Anderson festschrift chapter).


\section{Psych-verbs}
