\chapter{\vd}
\label{chap:vd}

\section{Introduction}
When looking at the verbal system of Hebrew, it has been our goal to understand which syntactic environments a template appears in and what alternations the templatic morphology tracks. The alternations examined in this book so far have been anticausative, descriptively speaking. Taking {\tkal} or {\tpie} to be an intuitive base form, we derived anticausative versions in {\tnif} and {\thit}. Of course, I have argued that what is actually happening is quite different, conceptually: the core vP is always there, but we either add Voice or add {\vz}. The former can give us causative verbs and the latter anticausative ones.

In this chapter we turn our attention to what can be seen as causative alternations. The following basic observation has been guiding us throughout the book: Hebrew shows morphological marking of both causative and anticausative forms, in addition to a ``simple'' verbal form, which itself can be either causative or anticausative. See in particular the middle column of~(\nextx), repeated from Chapter~\ref{chap:intro}. This three-way distinction leads to the trivalent theory of Voice defended in the monograph.
\ex \label{vd:ex:alternations-heb}
\raisebox{-2.5em}{
	\begin{tabular}{ll|ll|ll}
%	\multicolumn{2}{c|}{\textbf{\vd}}	&	\multicolumn{2}{c|}{\textbf{Voice}}	& \multicolumn{2}{c}{\textbf{\vz}}\\\cline{2-7}
	\multicolumn{2}{c|}{non-active} &	\multicolumn{2}{c|}{unspecified}	& \multicolumn{2}{c}{active}\\\hline
	\multicolumn{2}{c|}{\tnif}	&	\multicolumn{2}{c|}{\tkal}	& \multicolumn{2}{c}{\thif}\\
	\emph{neexal}	& `was eaten' & \emph{axal}	& `ate'	&	\emph{heexil}	& `fed' \\
	\emph{nixtav}	& `was written'  & \emph{katav}	& `wrote'	&	\emph{hextiv}	& `dictated' 		\\%\cdashline{3-4}
	\multicolumn{2}{c|}{--- (idiosyncratic gap)} & \emph{nafal}	& `fell' & \emph{hepil} & `dropped' \\
	\end{tabular}
}
\xe

The formal literature on transitivity alternations has, to a large extent, focused on comparing transitive verbs to their anticausative counterparts. It is perhaps no accident that this literature has also been based for the most part on European languages. In this chapter I examine {\thif} in depth and propose the addition of the head {\vd} to our toolbox, a functional head distinct from ordinary Voice (and as such a novel theoretical proposal). I will first describe the general properties of the template {\thif} in Section~\ref{vd:thif}.\footnote{For remarks on my notation see Chapter~\ref{sec:data:notation}.} An analysis using {\vd} follows in Section~\ref{vd:vd}. We will then look more closely at the relationship between a verb in {\tkal} and its alternants in {\tnif} and {\thif}, in Section~\ref{vd:caus}. This discussion will be followed in Section~\ref{vd:others} by a comparison with alternative approaches. Section~\ref{vd:sum} summarizes.


\section{\thif: Descriptive generalizations} \label{vd:thif}
The template {\thif} is traditionally called the ``causative'' one: verbs instantiated in it are often causative versions of a verb in {\tkal} (or {\tnif}; see \ref{vd:others:arad}). In practice, these verbs are active, i.e.~transitive or unergative. I use the term ``causative'' informally because there is no syntactic implementation of this term which is appropriate. This is, however, the traditional name, and in most transitive uses it makes intuitive sense.

The database of \cite{ehrenfeld12} and \cite{ahdout19phd} lists between 500--600 verbs in {\thif}. Of these, more than 500 are active. They are described in Section~\ref{vd:thif:caus}. There is also a small group of anticausative verbs, which obligatorily also form zero-alternations with causative readings in the same template. These verbs are presented in~\ref{vd:thif:inch}.

	\subsection{Causative verbs} \label{vd:thif:caus}
A few full examples are given for causatives in~(\nextx)--(\anextx) and for an unergative in~(\ref{ex:vd:unerg}).
\pex
	\a \begingl
		\gla ha-orxim \textbf{rakd-u} ba-mesiba//
		\glb the-guests danced-\gsc{PL} in.the-party//
		\glft `The guests danced at the party.'//
	\endgl
	\a \begingl
		\gla ha-zameret \textbf{herkid-a} et ha-orxim//
		\glb the-singer.\gsc{F} made.dance-\gsc{F} \gsc{ACC} the-guests//
		\glft `The singer made the guests dance.'//
	\endgl
\xe

\pex
	\a \begingl
		\gla ema \textbf{axl-a} uxmanjot//
		\glb Emma ate-\gsc{F} blueberries//
		\glft `Emma ate blueberries.'//
	\endgl
	\a \begingl
		\gla kevin \textbf{heexil} et ema (uxmanjot)//
		\glb Kevin fed \gsc{ACC} Emma blueberries//
		\glft `Kevin fed Emma (blueberries).'//
	\endgl %Emma is a dog
\xe

\ex \label{ex:vd:unerg} \begingl
	\gla ema \textbf{hemtin-a} ad ʃe-ha-oxel haja muxan//
	\glb Emma waited-\gsc{F} until \gsc{COMP}-the-food was ready//
	\glft `Emma waited until the food was ready.'//
	\endgl
\xe

I am not sure whether there are \emph{obligatory} ditransitives in this template (about 80 are at least non-obligatorily ditransitive in the database of \citealt{ahdout19phd}). Four candidates are \emph{heʃil} `lent out'~(\nextx), \emph{helva} `lent',  \emph{heskir} `rented out' and \emph{hezkir} `reminded'. Whether or not the goal argument is obligatory in contemporary speech is unclear, and in any case does not bear directly on the rest of this chapter.
\ex \begingl
		\gla ha-safranit \textbf{heʃil-a} (l-i) et ha-sefer//
		\glb the-librarian lent.\gsc{CAUS}-\gsc{F} to-me \gsc{ACC} the-book//
		\glft `The librarian lent me the book.'//
		\endgl
\xe
	
A few more alternations between an active verb in {\tkal} (transitive or non-core transitive) and a causative in {\thif} are presented in~(\nextx).
\ex\label{ex:vd:kal-thif}
\raisebox{-2.5em}{ \begin{tabular}{l|l|ll|ll}
	 & Root			& \multicolumn{2}{c|}{Active \tkal} & \multicolumn{2}{c}{Causative \thif}\\\hline
	a.& \root{'kl} & \emph{axal} & `ate'  & \emph{heexil (be-)} & `fed (with)'\\
	b.& \root{r'j} & \emph{raa} & `saw'  & \emph{hera (le-)} & 		 `showed (to)'\\
	c.& \root{ʃm'} & \emph{ʃama} & `heard'  & \emph{heʃmia (le-)} & `played (to)'\\
	d.& \root{nʃm} & \emph{naʃam} & `breathed'  & \emph{henʃim} & `resuscitated'\\
	\end{tabular}
	}
\xe

The most common way of characterizing these alternations informally is by saying that {\thif} is a causative version of {\tkal}. Yet what we see in actuality is that verbs in {\thif} are active, regardless of whether they alternate with an unaccusative verb in {\tkal}, a transitive verb in {\tkal}, or nothing in {\tkal}. A few examples of a the causative alternation in this template are given in~(\nextx a). Many verbs are also causative without alternating, as in~(\nextx b), and others are unergative,~(\nextx c).
\ex\label{vd:ex:alternations-heb-long}
	\begin{tabular}{l|ll|ll|ll}
	& \multicolumn{4}{c|}{anticausative/inchoative} & \multicolumn{2}{c}{active}\\
	& \multicolumn{2}{c|}{\tnif}	&	\multicolumn{2}{c|}{\tkal}	& \multicolumn{2}{c}{\thif}\\\hline
	a.& \emph{nixnas} & `entered' & && \emph{hexnis} & `inserted'\\
	 & \emph{notar} & `remained' & && \emph{hotir} & `left behind'\\
	 & \emph{nikxad} & `went extinct' & && \emph{hekxid} & `eradicated'\\
	 & \emph{ni{ts}al} & `was saved' & && \emph{he{ts}il} & `saved'\\
	 & \emph{nee{ts}av} & `was saddened' & && \emph{hee{ts}iv} & `saddened'\\
	 & \emph{nexlaʃ} & `grew weak' & && \emph{hexliʃ} & `weakened'\\\cdashline{2-7}
	 & && \emph{nafal} & `fell' & \emph{hepil} & `dropped'\\
	 & && \emph{kafa} & `froze' & \emph{hekpi} & `froze'\\
	 & && \emph{baar} & `burned' & \emph{hevir} & `lit up'\\
	 & && \emph{tava} & `drowned' & \emph{hetbia} & `drowned'\\\cdashline{2-7}
	 & && \emph{xazar} & `returned' & \emph{hexzir} & `returned'\\
	 & && \emph{jaʃav} & `sat down' & \emph{hoʃiv} & `sat down'\\
	 & && \emph{paxad} & `was afraid' & \emph{hefxid} & `scared'\\
	 & && \emph{rakad} & `danced' & \emph{herkid} & `caused to dance'\\
	 \hline
	b.& &&&& \emph{heʃmid} & `destroyed' \\
	& &&&& \emph{heir} & `illuminated'\\
	& &&&& \emph{hevis} & `defeated'\\
	& &&&& \emph{hegdir} & `defined'\\
	& &&&& \emph{hezmin} & `invited'\\
	& &&&& \emph{heka} & `struck'\\
	& &&&& \emph{hesnif} & `sniffed'\\
	& &&&& \emph{heflil} & `incriminated'\\
	\hline
	c.	& &&&&  \emph{hedrim} & `went south' \\
		& &&&&  \emph{hegzim} & `exaggerated' \\
		& &&&&  \emph{heflig} & `set sail' \\
		& &&&&  \emph{heria} & `cheered' \\
		& &&&& \emph{heezin} & `listened'\\
		& &&&& \emph{hemtin} & `waited'\\
		& &&&& \emph{heskim} & `agreed'\\
	\end{tabular}
\xe

The template is predominantly \emph{active}, i.e.~it has an agentive, external argument. The exact nature of what this ``causation'' is will be outlined (but not decisively defined) in Section~\ref{vd:caus:marked}.


	\subsection{The labile alternation} \label{vd:thif:inch}
		\subsubsection{The pattern}
Hebrew does not generally have the alternation referred to as ``labile'', ``zero-derivation'' or ``conversion'' (as with English transitive and intransitive \emph{break}$\sim$\emph{break}), with the exception of certain verbs in {\thif}. A handful of examples are attested in other templates, including \emph{a{ts}ar} `stopped' (often dispreferred to \emph{nee{ts}ar} as an inchoative), \emph{miher} `hurried' and \emph{ixer} `delayed', although the latter two are not part of my own causative vocabulary. Over 500 of the 550--600 verbs in {\thif} are active. In this section, I explore the 33 that are non-active and undergo the labile alternation.

I will once again use the label \textbf{inchoative} as a descriptive term: an inchoative verb in {\thif} is one in which the sole argument has undergone the change of state (or changed on a scale). \textbf{Causative} is likewise a descriptive term in this section, identical in use to ``transitive'': a structure with an external argument and an internal argument (complement to the verb). The two kinds will receive different analyses in Section~\ref{vd:vd}. The alternation is exemplified by \emph{hefʃir} `thawed' in~(\nextx). None of the inchoatives in this template have reflexive (agentive) readings.
\pex\label{ex:vd:thif-hefSir}
	\a \begingl
		\gla ha-jaxasim ben ʃtej ha-medinot \textbf{hefʃir-u} axarej bikur roʃ ha-memʃala//
		\glb the-relations between both the-states thawed.\gsc{CAUS}-\gsc{3PL} after visit head.of the-government//
		\glft `The relations between the two countries thawed after the PM's visit.'//
		\endgl
	
	\a \begingl
		\gla bikur roʃ ha-memʃala \textbf{hefʃir} et ha-jaxasim ben ʃtej ha-medinot//
		\glb visit head.of the-government thawed.\gsc{CAUS} \gsc{ACC} the-relations between both the-states//
		\glft `The PM's visit thawed the relations between the two countries.'//
		\endgl
\xe

Some examples of verbs that undergo the alternation are given in~(\nextx). Even in those cases where the inchoative is frequent, a causative context can be set up fairly easily. Full lists are given later on in this section.
\pex\label{ex:vd:thif-alt}Alternating unergatives in \thif:
	\a \textbf{Full alternation:} \emph{hei{ts}} `sped up', \emph{heemik} `deepened', \emph{heerix} `lengthened', \emph{hekʃiax} `stiffened', \emph{hefʃir} `thawed', \emph{heʃmin} `fattened', \emph{herza} `grew thin', \dots
	\a \textbf{Unergative preferred but causative innovation attested:} \emph{hesriax} `stank', \emph{hesmil} `went to the left',\footnote{Attested example for causative ``leften'':
	\ex
		\begingl
		\gla kol ha-kavod le-barak. \textbf{hesmil} et netanjahu//
		\glb all the-respect to-Barak. made.left \gsc{ACC} Netanyahu//
		\glft `Well done to [Ehud] Barak. He made [Benjamin] Netanyahu look like a leftist.'\trailingcitation{	\url{http://www.ynet.co.il/Ext/App/TalkBack/CdaViewOpenTalkBack/0,11382,L-4010352,00.html}}//
		\endgl
	\xe
	} \emph{he{ts}xin} `smelled pungent', \emph{herkiv} `rotted', \dots
	\a \textbf{Unaccusative preferred but causative innovation attested:} \emph{heedim} `reddened', \emph{helbin} `whitened', \emph{heʃxir} `blackened', \emph{hevri} `got healthy',
		\emph{hexvir} `grew pale',\footnote{Attested example for causative ``palen'':
		\ex ``The girl looked as though someone wrapped her up in massive metallic toilet paper. \dots
			\begingl
			\gla afilu ha-tseva ha-meanjen \emph{[}\dots\emph{]} \textbf{hexvir} et hofa'a-ta ʃel danst//
			\glb even the-color the-interesting {} paled \gsc{ACC} appearence-hers of Dunst//
			\glft `Even the interesting color \dots~made Dunst's appearance pale.'\trailingcitation{\url{http://www.mako.co.il/women-fashion/whats_in/Article-174f70ed642f121004.htm}}//
			\endgl
		\xe
		} \emph{her{ts}in} `became serious', \dots
\xe

As mentioned in Chapters~\ref{vz:tnif:nact:unacc} and~\ref{vz:va:vzva:refl}, unaccusativity judgments can be fickle in Hebrew: the ``possessive dative'' has been critiqued by \cite{gafter14li} and \cite{linzen14pd} as diagnosing saliency rather than internal argumenthood, while VS (Verb-Subject order in an otherwise SVO language) is not necessarily reliable as a diagnostic of deep unaccustivity. Nevertheless, it is possible to find unaccusative verbs in \thif~which perform satisfactorily on the `by itself' and VS diagnostics, as the examples in~(\nextx)--(\anextx) show. \citet[149]{borer91} likewise argues that inchoatives in {\thif} can be either unergative or unaccusative. Accordingly, I will assume that all three constructions (transitive, unergative and unaccusative) are possible in this template in principle.

\pex `By itself' with {\thif} inchoatives.
	\a \begingl
		\gla ha-glida \textbf{hefʃira} \underline{me-a{\ts}ma}//
		\glb the-ice.cream thawed.\gsc{CAUS}-\gsc{F} of-herself//
		\glft `The ice cream defrosted on its own.'//
	\endgl
	
	\a \begingl
		\gla ha-tnaim le-\textbf{hafʃara} ba-jaxasim hevʃilu-u \underline{me-a{\ts}mam}//
		\glb the-conditions to-thawing in.the-relations ripened.\gsc{CAUS}-\gsc{3PL} of-themselves//
		\glft `The conditions matured enough on their own for the relations to warm.'//
	\endgl
\xe

\pex\label{ex:vd:vs} VS order with {\thif} inchoatives in \thif. No \emph{by}-phrase possible.
	\a \begingl
		\gla \textbf{hefʃir-a} \emph{(}l-i\emph{)} kol ha-glida \emph{(}*{al jedej} ha-xom\emph{)}//
		\glb thawed-\gsc{F} to-me all the-ice.cream \phantom{*(}by the-heat//
		\glft `All (my) ice cream defrosted completely (*by the heat).'//
	\endgl
	
	\a \begingl
		\gla \textbf{hevʃil-u} ha-tnaim le-hafʃara ba-jaxasim \emph{(}*{al jedej} ha-bikur\emph{)}//
		\glb ripened-\gsc{3PL} the-conditions to-thawing in.the-relations \phantom{*(}by the-visit//
		\glft `The conditions matured enough for the relations to warm (*by the visit).'//
	\endgl
\xe

But what is special about the 33 roots such as those in~(\ref{ex:vd:thif-alt}) that allows their verbs to alternate, on the one hand, and what is special about the morphological template that allows these verbs to alternate, on the other hand? A satisfying analysis of these patterns must address two questions: why these roots and why this template. A generalization about the roots is suggested next and the analysis of the template is addressed in Section~\ref{vd:vd:syn}, summarizing claims made in \cite{kastner19tlr}.

		\subsubsection{Inchoatives as degree achievements} \label{vd:thif:inch:roots}
Not many verbs take part in the labile alternation in {\thit}. A number of estimates can be found in the recent literature: \cite{arad05} counted 11 such verbs in her corpus whereas \cite{laks11} found 34. \cite{lev16} counted 81 in a survey taking into account many naturally attested, but perhaps spurious, forms. My 33 alternating verbs are broken down as follows: 15 alternating unergatives and 18 alternating unaccusatives. I have classified the alternating verbs by the alternations they participate in. Barring a judgment survey, and given that I know of no comparable lists at this level of granularity, the lists below reflect my own intuitions.

I have attempted to identify, at an informal level, which roots form verbs that participate in the labile alternation. I propose a pretheoretical classification into verb classes which is based on broad lexical semantic categories. The verbs are classified according to these categories, building towards the claim that they are all degree achievements.

Table~\ref{tab:vd:thif-roots} lists the alternating verbs in~{\thif}. The first three rows show classes where the only inchoatives are unergative. The next row (change of color) shows a class in which the only inchoatives are unaccusative. Verbs in the other classes may be unergative or unaccusative, decided on a verb-by-verb basis. I also list whether there are transitive verbs in this template whose lexical semantics makes them eligible to be part of the verb class.

\begin{table}[htb] \small \centering \singlespacing
	\begin{tabular}{|p{3cm}||p{4.5cm}|p{4.5cm}||p{3cm}|}\hline
		&	\textbf{Unaccusative}	&  \textbf{Unergative} & \textbf{Transitive} \\\hline\hline
	Emission & --- & \emph{hesriax} `stank', \emph{heviʃ} `became putrid', \emph{hetsxin} `smelled pungent'\footnotemark & --- \\\hline
	
	Change of speed or direction & --- & \emph{heits} `accelerated', \emph{heet} `slowed down', \emph{hesmil} `went left' & \emph{heziz} `moved', \emph{hotsi} `removed', \dots \\\hline
	
	Change of sound & --- & \emph{heriʃ} `made loud noise', \emph{hexriʃ} `quieted down' & \emph{heʃtik} `shut up' \\\hline\hline
	
	Change of color & \emph{heedim} `reddened', \emph{helbin} `whitened', \emph{hekxil} `became blue', \emph{he{ts}hiv} `yellowed', \emph{heʃxir} `blackened', \emph{hezhiv} `goldened', 
				& --- & --- \\\hline\hline
		
	Change of physical function, shape or appearance & \emph{heʃmin} `fattened', \emph{herza} `thinned', \emph{hezkin} `grew old', \emph{hekriax} `became bald', \emph{hevri} `became healthy', \emph{her{ts}in} `became serious', \emph{hexvir} `grew pale' &
		\emph{heemik} `deepened', \emph{heerix} `lengthened', \emph{he{ts}er} `narrowed', \emph{hesmik} `blushed' & \emph{hefʃit} `undressed', \emph{henmix} `lowered', \emph{hextim} `stained', \dots \\\hline

	Change of consistency, taste or smell & \emph{hekʃiax} `stiffened', \emph{hefʃir} `thawed', \emph{hevʃil} `ripened', \emph{hekrim} `crusted'
		& \emph{hexmits} `soured', \emph{herkiv} `rotted'
		& \emph{hetsis} `fermented', \emph{heriax} `smelled', \emph{hetpil} `desalinated', \emph{heflir} `flouridated', \dots \\\hline\hline
		
	Other & \emph{hexmir} `deteriorated' & \emph{hek{ts}in} `escalated' &  \\\hline
	\end{tabular}
\caption{Lexical semantic classes for alternating verbs in \thif~and transitive foils.\label{tab:vd:thif-roots}}
\end{table}

\footnotetext{Other verbs of emission do not entail change of state: \emph{heki} `threw up', \emph{hezia} `sweat', \emph{heflits} `farted'.}


A number of tentative generalizations can be drawn from Table~\ref{tab:vd:thif-roots}. For instance, it seems clear that change of color allows for inchoative verbs (unaccusative ones). Yet a large degree of arbitrariness exists, as when we might also have expected the forms in~(\nextx) to exist, contrary to fact. The semantic criteria alone are not enough to predict how all roots in the language will behave.
\pex
	\a Change of speed:
		*\emph{hemhir} ($\nless$ \emph{mahir} `quick').
	\a Change of color:
		*\emph{hesgil} ($\nless$ \emph{sagol} `purple'), *\emph{hektim}/*\emph{hextim} ($\nless$ \emph{katom} `orange').
\xe

It is also not the case that any root in the categories above necessarily derives an inchoative in {\thif}: \emph{heziz} `moved' is a change of direction, \emph{heʃtik} `shut up' is a change of sound and \emph{henmix} `lowered' is a change of physical shape, but these three verbs (and many others) are only causative, never inchoative.

One insightful claim, made recently by \cite{lev16} and endorsed by \cite{kastner19tlr}, is that inchoatives in {\thif} are \textbf{degree achievements} (\citealt{dowty91,hayetal99,rotsteinwinter04,kennedylevin08,bobaljik12,mcnally17}, a.m.o). These are change of state verbs such as \emph{widen} and \emph{cool} which are derived from gradable adjectives. As such, they have scalar semantics leading to a possible endpoint. \citeauthor{lev16}'s claim is that this is exactly the unifying factor for the Hebrew inchoatives in {\thif}, although it is not a bidirectional implication (not all possible degree achievements are inchoatives in this templates), nor does this generalization drive his own analysis.

It does play a role in my own syntactic analysis insofar as inchoatives are derived from an underlying adjective (or noun). This hypothesis covers a fair bit of empirical ground and I follow \cite{lev16} in adopting it. 

	\subsection{Summary}
To summarize the empirical state of affairs, verbs in {\thif} are almost always active: either transitive or unergative. They often form causative versions of other verbs. And a few dozen verbs are degree achievements, intransitive change of state verbs derived from an underlying adjective or noun.

\hammer{
\pex \label{ex:gen-thif}\textbf{Generalizations about {\thif}}
	\a \textbf{Configurations:} Verbs appear in transitive and unergative configurations; a small class of verbs forms unaccusative degree achievements.
	\a \textbf{Alternations:} Some verbs are causative or active versions of verbs in other templates, especially {\tkal}. A small class of verbs creates a labile alternation within {\thif}.
\xe
}


\section{\vd: An active Voice head} \label{vd:vd}
To account for this set of data I propose {\vd}, a variant of Voice which requires that a DP be merged in its specifier, guaranteeing that an external argument appear. It introduces the Agent/Cause role, although unaccusatives are possible when deriving degree achievements.\footnote{Again abstracting away from the difference between Agents and Causers, regarding which see Chapter~\ref{intro:arch}.}
\pex  \label{ex:vd-basics} \textbf{\vd:}
	\a A Voice head with a [\!+\!D] feature, requiring that some element check the [D] feature in its specifier (usually via Merge).
	\a \label{ex:vd:sem}\denote{\vd} = $\begin{cases}
	\lambda P.P & / \text{\trace~(v) a} \\
	\lambda P.P & / \text{\trace~(v) n} \\
	\lambda x \lambda e.\text{Agent(x,e) or }\lambda x \lambda e.\text{Cause(x,e)}\\
	\end{cases}$
	\a {\vd} {\lra} {\thif}
\xe

	\subsection{Syntax and semantics} \label{vd:vd:syn}

%
%\pex\label{ex:vd:sem-full}
%	\a \denote{\vd} = $\lambda$e.e / \trace~(v) a \hfill (v does not select an alloseme)
%	\a \denote{\vd} = $\lambda$e.e / \trace~(v) n \hfill (v does not select an alloseme)
%	\a \denote{\vd} = $\lambda$x$\lambda$e.Cause(x,e) \hfill (or Agent)
%\xe

The syntax of {\vd} is as in~(\nextx), where this head obligatorily introduces an external argument. Merging that DP in Spec,{\vd} is enough to check the [D] feature, however the Spec-Head relationship is formalized. Note that I was careful to say that the feature must be checked, not that an element must be merged in the specifier; this is because of the analysis of inchoatives coming up.
\ex\label{vd:tree:thif}
\Tree
        [.VoiceP
            [.DP ]
            [
                [.{\vd}\\\emph{he-} ]
                [.vP
                    [.v
                        [.\root{\gsc{ROOT}} ]
                        [.v ]
                    ]
                    [.(DP) ]
                ]
            ]
        ]
    \xe

The relevant clause in the semantics is the Elsewhere case of~(\ref{ex:vd-basics}b). Since the spell-out of {\vd} is {\thif} (by hypothesis), we predict that all verbs in this template will have an external argument in the syntax and semantics.
\ex \denote{\vd} = 	$\lambda$x$\lambda$e.Agent(x,e) or $\lambda$x$\lambda$ e.\text{Cause(x,e)}
\xe

As we have seen, this proposal is enough to describe most of the empirical landscape. It also treats causatives in {\thif} as monoclausal, ``lexical'' causatives, as expected.

But it is not enough to explain the inchoatives, where two questions in fact arise. First, how must we change our definition of {\vd}? And second, why is it this head alone that leads to labile alternations in the language?

The remainder of this section concerns itself with the first question of the two. I propose next that causatives have different structure than inchoatives, echoing claims made by \cite{borer91}. Causatives are argued to be derived from the root, whereas inchoatives are argued to be derived from an existing adjective or noun.\footnote{From a cross-Semitic perspective, Arabic ``Form 9'' \emph{iXYaZZ} verbs show some parallels with {\thif}, though the Arabic forms are exclusively nonactive.} The more general question about labile alternations in {\thif} alone will wait until Section~\ref{vd:caus:labile}.

		\subsubsection{Inchoatives: Structure}
As a first step, I will assume that inchoatives in {\thif} are never derived directly from the root but from an underlying adjective or noun. A similar claim was already made by \cite{borer91}, who argued that causatives are derived directly from the root while these inchoatives are derived from an underlying adjective. As I point out here, inchoatives can also be derived from an underlying noun:
\pex
	\a Underlying adjective: \emph{heedim} $<$ \emph{adom} `red', \emph{heʃmin} $<$ \emph{ʃamen} `fat'.
	\a Underlying noun: \emph{heki} $<$ \emph{ki} `vomit', \emph{he{ts}xin} $<$ \emph{{ts}axana} `stench'.
\xe

The structure is as in~(\nextx), covering both unergatives and unaccusatives.
\ex
	\scalebox{0.9}{
	\Tree
 [.VoiceP
     [.DP$_i$ ]
     [
         [.{\vd}\\\emph{he-} ]
         [.vP
             [.v
              [.\phantom{xx}v\phantom{xx} ]
              [.a/n
                  [.\root{\gsc{ROOT}} ]
                  [.a/n ]
              ]
             ]
             [.(DP)$_i$ ]
         ]
     ]
 ]	
 }
\xe

This assumption is admittedly a bit of a morphophonological stretch in certain cases.\footnote{I thank the \emph{TLR} reviewers of \cite{kastner19tlr} for emphasizing this point. I have not made progress on this issue since the publication of that paper.} For example, the verb \emph{hei{ts}} `accelerated' is arguably not derived from the noun \emph{teu{ts}a} `acceleration', whose initial /t/ is not preserved. This much indicates that perhaps the claim should be weakened such that some inchoatives are derived from adjectives/nouns and others from the root. Nevertheless, the strong assumption of crosscategorial derivation carries a few benefits. First, it allows us to talk about different constructions in terms of explicit, uniform structures. Second, it allows for the degree semantics of the underlying adjective to transfer to the verb. And third, it makes a correct prediction regarding idiomatic meaning, as I show next.

My theory of morphosemantics assumes the so-called Arad/Marantz hypothesis, according to which the first categorizing head selects the meaning of the root (see Chapter~\ref{vz:vz:sem}). If~(\lastx) is the right structure for inchoatives, then we predict that for roots which participate in the alternation, the causative might have a meaning that the inchoative does not share. This is because in causatives {\vd} is local enough to the root to select a special meaning, whereas in inchoatives little a or little n will have already chosen the meaning of the root. This prediction is borne out by idioms involving \emph{helbin} `whitened' with the metaphorical meaning `laundered', as in~(\nextx), and \emph{heʃxir} `blackened' with the metaphorical meaning `tarnished', as in~(\anextx).
\pex
	\a Causative, literal meaning:\\
		\begingl
			\gla ha-sid \textbf{helbin} et ha-kir//
			\glb the-lime.plaster whitened.\gsc{CAUS} \gsc{ACC} the-wall//
			\glft `The lime plaster made the wall white.'//
		\endgl
	
	\a Causative, non-transparent meaning:\\
		\begingl
			\gla sar ha-xuts \textbf{helbin} ksafim//
			\glb minister the-exterior whitened\gsc{CAUS} moneys//
			\glft `The Minister of Foreign Affairs took part in money laundering.'//
		\endgl
	
	\a Passive of causative, non-transparent meaning retained:\\
		\begingl
			\gla nitan ʃe-ha-ksafim \textbf{hulben-u} {al jedej} sar ha-xuts//
			\glb was.claimed \gsc{COMP}-the-moneys whitened.\gsc{CAUS.PASS}-\gsc{3PL} by minister the-exterior//
			\glft `It was claimed that the money was laundered by the Minister of Foreign Affairs.'//
		\endgl
	
	\a Inchoative, only literal meaning:\\
		\begingl
			\gla ha-ʃtarot \textbf{helbin-u}//
			\glb the-bills whitened.\gsc{CAUS}-\gsc{3PL}//
			\glft `The bills became white.'\\
				(not: `The bills got laundered.')//
		\endgl
\xe

\pex
	\a Causative, literal meaning:\\
		\begingl
			\gla ha-piax \textbf{heʃxir} et ha-avir//
			\glb the-soot blackened.\gsc{CAUS} \gsc{ACC} the-air//
			\glft `The air grew black with soot.'//
		\endgl
	
	\a Causative, non-transparent meaning:\\
		\begingl
			\gla son'e-j israel menas-im \textbf{lehaʃxir} et pane-ha ʃel medina-t israel ba-zira ha-benleumit//
			\glb haters-\gsc{CS} Israel try.\gsc{PTCP}-\gsc{M.PL} to.blacken.\gsc{CAUS} \gsc{ACC} faces-\gsc{3F} of state-\gsc{CS} Israel in.the-arena the-international//
			\glft `Israel's haters are trying to make the State of Israel look bad on the international stage.'\trailingcitation{\url{http://www.ynet.co.il/articles/0,7340,L-4781034,00.html}}//
		\endgl
	\a Inchoative, only literal meaning:\\
		\begingl
			\gla\ljudge{??}pane-ha ʃel ha-medina \textbf{heʃxir-u} axarej ha-ʃaarurija ha-axrona//
			\glb faces-\gsc{3F} of the-state blackened.\gsc{CAUS}-\gsc{3PL} after the-scandal the-last//
			\glft (int. `The country was made to look bad after the latest scandal')//
		\endgl
\xe

%	panav helbinu (*ba-rabim) != he was humiliated (in public).
%		ha-malbin pnej xaver-o ba-rabim, keilu Safax dam-o

\cite{borer91} provides additional arguments for deriving the inchoative from the adjective, which I scrutinize in Section~\ref{vd:others:borer}.

The full semantics for {\vd} then looks as in~(\nextx), without introducing a causer for inchoative events in~(\nextx a--b).
\pex\label{ex:vd:sem-full}
	\a \denote{\vd} = $\lambda$P.P / \trace~(v) a \hfill (v does not select the meaning)
	\a \denote{\vd} = $\lambda$P.P / \trace~(v) n \hfill (v does not select the meaning)
	\a \denote{\vd} = $\lambda$x$\lambda$e.Cause(x,e) or $\lambda$x$\lambda$e.Agent(x,e)
\xe

		\subsubsection{Inchoatives: Derivation}
Merging a DP in Spec,{\vd} will not do for the inchoatives since they are unaccusative. Allowing the internal argument to raise to the specifier and check the [D] feature there must also be ruled out because of the results of the VS diagnostic: it shows us that at least in some cases the internal argument must be allowed to remain low, (\ref{ex:vd:vs}).

To account for these cases, I assume instead that the [D] feature on {\vd} requires valuation of phi-features under Agree \citep{nie17,schaefer17oup}. This valuation proceeds straightforwardly under Spec-Head Agreement, as we have seen, but something else needs to be said if the sole argument in the phase is the internal argument. In this case, I propose that [D] can be checked by the internal argument \emph{in situ}: {\vd} probes into its specifier upwards, finds no target, and so it probes downwards and is valued by the internal argument. For more in-depth discussion of the direction of Agree, see works such as \cite{bejarrezac09}, \cite{zeijlstra12}, \cite{preminger13tlr} and \cite{deal15nels}.

Here is what the current proposal means for an inchoative example like~(\ref{ex:vd:thif-inch}) with the structure in~(\ref{tree:vd:thif-inch}). {\vd} probes its specifier and finds nothing, (\ref{tree:vd:thif-inch}-\ding{172}), so it probes downward and checks its unvalued phi-features with the internal argument \emph{ha-xatul} `the cat' (\ref{tree:vd:thif-inch}-\ding{173}). The interpretation is as in~(\ref{ex:vd:sem-full}a): no Cause is introduced.
\ex\label{ex:vd:thif-inch} \begingl
	\gla ha-xatul \textbf{heʃmin}//
	\glb the-cat fattened.\gsc{CAUS}//
	\glft `The cat grew fat.'//
	\endgl
\xe
%\begin{wrapfigure}[8]{r}{0.6\textwidth}
%	\vspace{-6em}
	
	\ex\label{tree:vd:thif-inch}
		\scalebox{0.9}{
	    \Tree
	    [.VoiceP
	        [.\tikz{\node (spec) {};} ]
	        [
	            [.\tikz{\node (vd) {\vd};} ]
	            [.vP
	                [.v
		                [.\phantom{xx}v\phantom{xx} ]
		                [.a
		                    [.a ]
		                    [.\root{ʃmn} ]
		                ]
	                ]
	                [.\tikz{\node (IA) {\emph{ha-xatul}};} ]
	            ]
	        ]
	    ]
	    \begin{tikzpicture}[overlay]
		    \draw[dashed,->] (vd) .. controls +(south:1) and +(south west:1) .. node{\LARGE $\times$} node[below]{\ding{172}}(spec);
	 	    \draw[dashed,->] (vd) .. controls +(south:4) and +(south:3) .. node[below]{\ding{173}}(IA);    
	    \end{tikzpicture}
	    }
	\xe
%\end{wrapfigure}
\bigskip

As a consequence, ungrammatical cases like~(\nextx) must now be ruled out.
\pex\label{ex:counterex}
	\a \ljudge{*}
		\begingl
		\gla ha-xatul \textbf{hexnis}//
		\glb the-cat inserted.\gsc{CAUS}//
		\glft (int. `The cat got inserted')//
	\endgl
	
	\a \ljudge{*}
		\begingl
		\gla ha-oto \textbf{hemhir}//
		\glb the-car \gsc{FAST}.\gsc{CAUS}//
		\glft (int. `The car grew fast')//
	\endgl

	\a \ljudge{*}
		\begingl
		\gla ha-xatul \textbf{hekpi}//
		\glb the-cat froze.\gsc{CAUS}//
		\glft (int. `The cat froze')//
	\endgl
\xe

For~(\lastx a) there is no adjective `inserted' that could be verbalized and no inchoative can be generated. In~(\lastx b) an adjective \emph{mahir} `quick' does exist, but it cannot be instantiated in {\thif} in general due to some arbitrary gap, as already mentioned in Section~\ref{vd:thif:inch:roots} (or at least, I assume that this is an arbitrary gap, in lieu of a more principled explanation).

Finally, (\ref{ex:counterex}c) is not a possible inchoative even though there exists an underlying adjective, namely \emph{kafu} `frozen'. There are a number of possible explanations which can be pursued here. One is that \emph{freeze} is not a degree achievement in Hebrew, and so that adjective is not a possible input to the structure. Another kind of explanation falls along the lines of extra-grammatical paradigmatic pressure, in that an inchoative (non-alternating) freezing verb already exists in another template: \emph{kafa} `froze' in {\tkal}. In this regard, I should note that speakers do steer clear of {\thif} for certain inchoatives, instantiating them in other, more canonically non-active templates: \emph{hitarex} `grew long' in {\thit} rather than \emph{heerix}, \emph{hizdaken} `grew old' in {\thit} rather than \emph{hezkin}, \emph{raza} `thinned' in {\tkal} instead of \emph{herza}, and \emph{hitadem} `reddened' in {\thit} rather than \emph{heedim} (but see \citealt[22]{doron03} for a grammatical difference between the two).

It should go without saying that the strong claim about separate derivational strategies for causatives and inchoatives awaits a more articulated semantic analysis. As a reviewer for \cite{kastner19tlr} pointed out, in~(\ref{ex:vd:thif-hefSir}) the verb \emph{hefʃir} means `thawed', i.e.~became warmer, while the underlying adjective \emph{poʃer} means `lukewarm', i.e.~not warm. Yet the inchoative does not mean ``became lukewarm''. Another incongruity between verb and adjective can be seen with \emph{heʃmin}, `grew fatter', which does not entail that its argument becomes \emph{ʃamen} `fat'. Important discussion of the relevant scales and entailments is given by~\cite{borer91}, which I turn to in the discussion of alternatives.

With the formal analysis in place, I flesh out the morphophonlogical part of the chapter before turning to more general discussion, including the question of why the labile alternation is formed with {\vd} specifically.

	
	\subsection{Phonology} \label{vd:vd:phono}
The basic VI given in~(\ref{ex:vd-basics}) was as follows:
\ex {\vd} {\lra} {\thif}
\xe

Using {\thif} as the Vocabulary Item spelling out {\vd} is shorthand for a more detailed morphophonology. A sample derivation is adapted here from~\cite{kastner18nllt}. I contrast the \gsc{3SG.M.Past} form with the \gsc{1SG.Past} form, which has an affix and a different stem vowel.
\pex
	\a \emph{hevʃ\textbf{i}l} `he ripened'
	\a \emph{hevʃ\textbf{a}l-\textbf{ti}} `I ripened'
\xe
The relevant Vocabulary Items:
\pex
	\a \root{bʃl} \lra~\emph{bʃl}
	\a 	\vd~\lra~\emph{he}, $\begin{cases}
			\emph{i}\\
			\emph{a} & \text{/ T[1st] \trace}
			\end{cases}$
	\a 1\gsc{SG} \lra~\emph{ti} / \trace~Past
	\a \gsc{3SG} \lra~{\zero} / \trace~Past
\xe
The cyclic derivation:
\pex
	\a \emph{hevʃil} `he ripened': [T[Past,\gsc{3SG.M}] [{\vd} [v \root{bʃl}~\!]]]\\
	Cycle 1 (VoiceP): he-vʃil\\
	Cycle 2 (TP): {\zero}-hevʃil $\Rightarrow$ hevʃil
	\a \emph{hevʃ\textbf{a}l\textbf{ti}} `I ripened': [T[Past,\gsc{1SG}] [{\vd} [v \root{bʃl}~\!]]] \\
	Cycle 1 (VoiceP): he-vʃ\textbf{a}l\\
	Cycle 2 (TP): \textbf{ti}-hevʃal $\Rightarrow$ hevʃalti
\xe	

See the work cited for various additional cyclic and allomorphic predictions.

		
%\ex\label{tree:thif2}
%    a. \begin{forest} qtree
%    [Causative:\\VoiceP
%        [DP_{\text{EA}} ]
%        [
%            [{\vd}\\\emph{he-} ]
%            [vP
%                [v
%                    [\textbf{v} ]
%                    [\root{\gsc{ROOT}} ]
%                ]
%                [DP_{\text{IA}} ]
%            ]
%        ]
%    ]
%    \end{forest} \phantom{xxxxxxx}
%    b. \begin{forest} qtree
%    [Inchoative:\\VoiceP
%        [DP_i ]
%        [
%            [{\vd}\\\emph{he-} ]
%            [vP
%                [v
%	                [\phantom{xx}v\phantom{xx} ]
%	                [a/n
%	                    [\textbf{a/n} ]
%	                    [\root{\gsc{ROOT}} ]
%	                ]
%                ]
%                [(DP)_i ]
%            ]
%        ]
%    ]
%    \end{forest}
%\xe


\section{Causation and alternation} \label{vd:caus}
This section contains a number of general points about causative alternations which I would like to mention. Section~\ref{vd:caus:marked} discusses markedness in causation in general and in terms of Voice heads in particular. Section~\ref{vd:caus:product} notes how productive {\vd} is, and Section~\ref{vd:caus:labile} returns to the labile alternation. A possible way of generalizing this account is surveyed in Section~\ref{vd:caus:pred}. For recent ways of conceptualizing causation in Hebrew in particular, see \cite{barashersiegalboneh18wccfl}.

	\subsection{Markedness in causation} \label{vd:caus:marked}
Recall the basics of the anticausative alternation which we have been assuming. Both (marked) anticausatives and (unmarked) causatives share a common base, formally the vP. This phrase is a predicate over eventualities, to which Voice can add an external argument \citep{schaefer08,layering15}, (\nextx).
\pex 
	\a \emph{John} \textbf{Voice} [$_{\text{vP}}$ \emph{broke the glass}].
	\a {\zero} \textbf{\vz} [$_{\text{vP}}$ \emph{The glass broke}].
\xe

Considering the current proposal, how and why should {\vd} differ from Unspecified Voice? If Voice allows the grammar to add an external argument, what's left for an additional device (\vd) to do, the hypothetical (\nextx)?
\ex \emph{John} \textbf{\vd} [$_{\text{vP}}$ \emph{broke the glass}].
\xe

Since the syntactic behavior of Unspecified Voice and {\vd} is identical as far as licensing a specifier is concerned, in this section I will discuss the semantic difference between the resulting causative verbs. Concretely, I will suggest that Voice-causatives are more transparent than {\vd}-causatives, whereby the morphological markedness of the latter mirrors some semantic markedness or opacity. The {\vd} causatives are simply lexical causatives, in the sense of \cite{fodor70}, \cite{miyagawa98} and \cite{harley08}: a transitive verb which is not derived through causativization of an existing verb. Let us explore what this means when contrasting them with ``regular'' causative or transitive verbs. For Hebrew, this means contrasting causatives in {\tkal} with those in {\thif}, or rather causatives derived using Unspecified Voice with causatives derived using {\vd}.

		\subsubsection{Basic and marked alternations}
For concreteness, let us give the two alternations the names in~(\nextx). The claim will be that the marked alternation is marked not only morphologically but also semantically -- a lexical causative, i.e.~a non-transparent one.
\ex
\begin{tabular}{l|ccc}
	&	Anticausative & Causative & Causative\\\hline
Basic alternation	& {\vz} & Voice & ---  \\
Marked alternation		&	---	&  Voice & {\vd}\\
\end{tabular}
\xe

Very little contemporary work has analyzed causative alternations in depth within a general theory of argument structure alternations; such work normally draws on languages like Japanese \citep{jacobsen92} that are typologically distinct from Indo-European ones. \citet[62ff]{layering15} speculate that a marked causative should entail thematic/active Voice (semantically if not syntactically), but as far as I know no formal theory has explored the implications of marked anticausatives and marked causatives existing side by side.

The first question to ask is how prevalent these two alternations are. The basic alternation was discussed at length in Chapter~\ref{vz:vz}. For the marked alternation, various examples were already given in~(\ref{ex:vd:kal-thif}) and~(\ref{vd:ex:alternations-heb-long}b). Out of 300+ pairs of {\tkal}--{\thif} alternations in my database, 64 show the marked alternation.%\footnote{The other @ exhibit a causative alternation of the kind seen in the second set of rows in~(\ref{vd:ex:alternations-heb-long}a).}

%A rough count is as follows:
%
%{\tnif} total @@
%	{\tnif} anticausatives @@
%		Basic alternation with {\tkal} @@
%{\thit} total @@
%	{\thit} anticausatives @@
%		Basic alternation with {\tpie} @@
%{\thif} total 550@
%	{\thif} actives 500@
%		Marked alternation with {\tkal} @@
%
%It is fairly clear that both types of alternation are well attested.

The second question is whether there is a difference between the semantics of the alternations, and here I believe the basic alternation is more transparent. The question is one of predictability: given the anticausative variant, can we predict the meaning of the causative variant? In the basic alternation, the answer is usually affirmative, just like with the prototypical \emph{break}-\emph{break} and \emph{open}-\emph{open} examples in English. A few examples were given  above and one is elaborated on in~(\nextx):
\pex
	\a \emph{nixtav} `was written'\\
		Writing event of a DP, no cause specified; or a passive reading with an implicit Agent.
	\a \emph{katav} `wrote'\\
		Writing event of a DP, external argument specified in the syntax and interpreted as Agent.
\xe

I suggest that the causative variant of a basic alternation introduces a direct causer \citep{bittner99,kratzer05} but that the external argument in the marked alternation is less restricted.\footnote{A similar intuition was expressed by \cite{doron03}, where the strongest claims about a template's meaning were limited to cases in which a root alternates between two templates.} For the marked causative, the informal phrasing in~(\nextx) will do for now (I return below to the question of whether a writing event even holds in these cases):
\ex \emph{hextiv} `dictated'\\
	Writing event of a DP, external argument specified in the syntax and understood as an (indirect) causer.
\xe

In terms of syntax, there appears to be no difference between the constructions. The unmarked causative has a regular monoeventive reading, cannot be modified by conflicting temporal adverbs, and has two basic readings of `again' \citep{vonstechow96}.

\pex  \begingl
		\gla ha-talmidim \textbf{katvu} et ha-nosim (\#aval lo be-a{\ts}mam)//
		\glb the-students wrote \gsc{ACC} the-topics but not in-themselves//
		\glft `The students wrote down the list of topics (\# but not by themselves).'//
		\endgl
	\a Means: The students wrote the list themselves.
	\a Cannot mean: paid someone online to write the list for them.
\xe

\ex \ljudge{*} \begingl
		\gla ha-talmidim \textbf{katvu} etmol et ha-nosim maxar//
		\glb the-students wrote yesterday \gsc{ACC} the-topics tomorrow//
		\glft (int.~`The students wrote something down yesterday to get the list of topics tomorrow')//
		\endgl
\xe

\pex \begingl
		\gla ha-talmidim \textbf{katvu} et ha-nosim ʃuv//
		\glb the-students wrote \gsc{ACC} the-topics again//
		\glft `The students wrote down the list of topics again.'//
		\endgl
	\a Can mean: The students wrote down a list which already existed.
	\a Can mean: The students wrote down a list after having written it down once before.
	\a Cannot mean: Someone_{i} had written down the list of topics, and now the students made someone_{i/j} write the list of topics another time.
\xe

The marked causative patterns identically: monoeventive reading, no conflicting temporal adverbs, and the same readings of `again'.
\pex
	\a \begingl
		\gla ha-more \textbf{hextiv} et ha-nosim (la-talmidim) (\#bli lomar mila)//
		\glb the-teacher dictated \gsc{ACC} the-topics to.the-students without to.say word//
		\glft `The teacher dictated the list of topics (to the students) (\# without saying a word).//
	\endgl
	\a Means: The teacher read the list out and the students wrote it down.
	\a Cannot mean: He stood menacingly over the students until they started writing.
\xe

\ex \ljudge{*} \begingl
		\gla ha-more \textbf{hextiv} etmol et ha-nosim (la-talmidim) maxar//
		\glb the-teacher dictated yesteday \gsc{ACC} the-topics to.the-students tomorrow//
		\glft (int.~`The teacher read out the list yesterday for the students to write down tomorrow')//
	\endgl
\xe

\pex
	\begingl
		\gla ha-more \textbf{hextiv} et ha-nosim (la-talmidim) ʃuv//
		\glb the-teacher dictated \gsc{ACC} the-topics to.the-students again//
		\glft `The teacher dictated the list of topics (to the students) again.//
	\endgl
	\a Can mean: The teach dictated/wrote a list which (someone else) has already dictated/written.
	\a Can mean: The teacher dictated/wrote a list having dictated/written it once before.
	\a Cannot mean: Someone_{i} had dictated/written the list, and now the teacher made someone_{i/j} write/dictate it another time.
\xe

The structures are therefore virtually identical, with the only difference being the feature on Voice (other than the identity of the external argument).

\pex 
	\a \begingl
		\gla ha-talmidim \textbf{katvu} et ha-nosim//
		\glb the-students wrote \gsc{ACC} topics//
		\glft `The students wrote down the list of topics.'//
		\endgl\\
		\Tree [. [.students ] [. [.Voice ] [. [.\root{\gsc{WROTE}} ] [.topics ] ] ] ]		
	\a \begingl
		\gla ha-more \textbf{hextiv} et ha-nosim (la-talmidim)//
		\glb the-teacher dictated \gsc{ACC} topics to.the-students//
		\glft `The teacher dictated the list of topics (to the students).//
	\endgl\\
		\Tree [. [.teacher ] [. [.{\vd} ] [. [.\root{\gsc{WROTE}} ] [.topics ] ] ] ]
\xe

In the basic alternation, adding a causer to writing immediately identifies the writer. But adding the marked causer changes the event slightly: the teacher does not cause writing to occur, strictly speaking. Rather, he is the causer of a dictating event, which itself brings about the writing down of the topics.

A similar pattern can be seen with \emph{neexal} `was eaten', where the basic variant \emph{axal} means `ate' and the marked variant \emph{heexil (et)} means `fed (s.o.~s.th)'. Why should this be the meaning, and not `made someone eat'? The different kind of event (feeding versus causing to eat) also implies a different position for the eater: subject in the basic variant, object in the marked variant.
\pex
	\a \begingl
		\gla beki axl-a uxmanjot//
		\glb Becky ate-\gsc{F} blueberries//
		\glft `Becky ate blueberries.'//
		\endgl\\
		\Tree [. [.\textbf{Becky} ] [. [.Voice ] [. [.\root{\gsc{ATE}} ] [.blueberries ] ] ] ]
	\a \begingl
		\gla aba heexil et beki (uxmanjot)//
		\glb dad fed \gsc{ACC} Becky blueberries//
		\glft `Dad fed Becky (blueberries).'//
		\endgl\\
		\Tree [. [.dad ] [. [.{\vd} ] [. [.\root{\gsc{ATE}} ] [.\textbf{Becky} ] ] ] ]
\xe

This contrast illustrates the limits of syntax within the current system: it can rigidly dictate which elements go where, but the structure itself is not driven by semantic or lexical-semantic considerations.

A few more examples of the marked alternation are given in~(\nextx), showing that the exact type of ``causative'' relation for the marked variant is not uniform.\footnote{The lexical semantics of the root could have something to do with the type of causation in the marked alternation, a question I leave open. See \citet[44]{doron03} for a proposed explanation in terms of whether the {\tkal} form is a verb of consumption or a psych-verb, building on \cite{colesridhar77}.} We have already seen the different meanings of~(\nextx a) and~(\nextx c).
\ex\label{vd:ex:triplets-caus}
\xe
\begin{small}
\hspace{-4em}\begin{tabular}{l|ll|ll|llcc}
		\multicolumn{7}{c}{}		& Make O V	& Make O be V-ed\\\hline
		 a.& \emph{neexal}	& `was eaten'	& \emph{axal} & `ate'		& \emph{heexil} & `fed'			& \cmark	& \xmark\\
		 b.& \emph{nilbaʃ}	& `was worn'	& \emph{lavaʃ} & `wore' 	& \emph{helbiʃ}	&	`dressed up' 	& \cmark	& \xmark\\\hdashline
		 c.& \emph{nixtav} & `was written' & \emph{katav} & `wrote' & \emph{hextiv} & `dictated' & \xmark	& \cmark\\
		d.& \emph{nikra}	& `was read'	& \emph{kara} & `read'		& \emph{hekri}	& `read out'	& \xmark	& \cmark \\
		e.&	\emph{nidxak}	& `was pushed aside'	& \emph{daxak}	& `shoved'	& \emph{hedxik}	& `suppressed'\footnotemark	& \xmark	& \cmark\\
		f.& \emph{nilxats}	& `was pressed' &  \emph{laxats} & `pressed'	& \emph{helxits} & `pressured'	& \xmark	& \cmark \\\hdashline
		 g.& \emph{nixtam}	& `was signed'	& \emph{xatam} & `signed'	& \emph{hextim}	& `made sign'	& \cmark	& \cmark\\\hdashline
		h. & \emph{nidgam} & `was sampled'	& \emph{dagam} & `sampled'	& \emph{hedgim}		& `demonstrated'	& \xmark	& \xmark\\
		i. & \emph{niklat} & `was received' & \emph{kalat} & `received' & \emph{heklit} & `recorded' & \xmark & \cmark?\\
		j.& \emph{nisgar}	& `closed'	& \emph{sagar} & `closed'		& \emph{hesgir} & `extradited'	& \xmark	& \cmark?\\
		\end{tabular}
\end{small}
%jalad nolad holid
\footnotetext{E.g.~memories or emotions.}

The last examples,~(\lastx h--j), are particularly revealing: to extradite someone is in no way an obvious semantic extension of ``closing'' them. More examples like this can be found, in which the basic alternant has predictable semantics but the marked one does not.

Before we go on to discuss the consequences of these differences, it is important to note a possible objection. The fact that marked causatives can vary so widely in their interpretation from the basic variants could be taken as an argument against treating these verbs as sharing the same abstract root. That is to say, why should we even think that closing and extraditing share the same root? Would that not be stretching its assumed shared semantics too thin? I believe the overall picture emerging from this book and from work treating roots more directly is that we do want to assume abstract roots, but be more specific in what their shared meaning is and under which circumstances it can vary. See also \cite{kastnertucker19cup} for related discussion of root meaning.

More concretely, however, we can make an argument from the lack of doublets. There are no \emph{additional} verbs in {\tnif} that alternate with {\thif}. That is to say, suppose that \emph{nikra} `was read' and \emph{kara} `read' are derived from \root{krj1} and that \emph{hekri} `read out' was derived from a homophonous root \root{krj2}. Assume similarly that \emph{nisgar} `closed' and \emph{sagar} `closed' were derived from \root{sgr1} but that \emph{hesgir} `extradited' was derived from \root{sgr2}, and so on for all cases of non-predictable causative variants. If this were the case, we would suppose that \root{krj2} and \root{sgr2} could also be instantiated with other functional heads, for instance with {\vz} to create an anticausative in {\tnif}. But this is not the case: there is no \#\emph{nikra} `was read out' to alternate with \emph{hekri} `read out', and no \#\emph{nisgar} `got extradited' to alternate with \emph{hesgir} `extradited'. In other words, there are no \emph{doublets} (compare the discussion of root suppletion in \citealt{harley14thlia,harley14thlib,harley15roots} and \citealt{borer14thli}).

I conclude tentatively that {\vd} invokes some indirect notion of causation, in those cases where regular (direct) causation is already triggered by Voice. But why might this be the case?

			\subsubsection{Markedness in Voice heads} \label{vd:caus:markvoice}
The observations made so far bring us to the proposed generalization for causitivity marking in~(\nextx).
\pex\label{ex:vd:causgen}\textbf{The causative generalization for transitivity alternations}\\
	If a language has both anticausative and causative marking:
	\a The anticausative alternation is transparent.
	\a The causative alternation is not (indirect, root-specific).
\xe

I would not be surprised if closer examination of other languages reveals similar patterns. In French, for instance, the prefixes \emph{a-} and \emph{en-} are often described as having a general ``transitivizing'' or ``causative'' function \citep{junker87}. In some cases, especially denominal and de-adjectival ones, the causative alternation between an unprefixed anticausative verb and a prefixed causative verb is transparent (\nextx a). But this is not always the case: in (\nextx b) the unprefixed verb is an activity and the prefixed verb is transitive (it has the obligatory reflexive marker) but does not strictly speaking mean `make yourself fly' or `fly yourself'. Even more strikingly---and certainly reminiscent of the Hebrew datapoints---is~(\nextx c), where the prefixed version has different meaning than the unprefixed one.
\ex
\begin{tabular}{lll|ll}
	a.&	faiblir	& grow weak	& \textbf{a}ffaiblir	& weaken s.o\\
	b.& voler	& fly			&	s'\textbf{en}voler	& take off\\
	c.& fermer	& close		& \textbf{en}fermer & imprison, lock up\\
\end{tabular}
\xe
It is tempting to analyze these prefixes as Voice heads, perhaps even {\vd}, but that idea goes beyond the scope of the current work. What we see is that there is no transparent relationship of ``causation'', however defined, between a marked form and an unmarked one.

The discussion of causative marking will conclude by examining whether~(\ref{ex:vd:causgen}) can be derived directly from our theoretical assumptions. I believe that it can. Concretely, this generalization follows directly from the general layering approach to transitivity alternations. If the core vP already has a causative component, then it is clear what \emph{not} adding an external argument would mean: that is the anticausative alternant. Adding an external argument, as noted above, amounts to introducing the most direct causer. This much derives~(\ref{ex:vd:causgen}a).

Say we have an event of causation, (\nextx):
\ex
\Tree
	[.vP
		[.v ]
		[.DP ]
	]
\xe


Not adding a causer is easy, (\nextx):
\ex
\Tree
[.VoiceP
	[.{\vz} ]
	[.vP
		[.v ]
		[.DP ]
	]
]
\xe

For~(\ref{ex:vd:causgen}b) We will need to assume that structures derived with different Voice heads will have different meanings, perhaps by some principle of economy.
%The question is now how to differentiate between the two meanings, and the answer is that adding an external argument necessitates a change in meaning as well.
Then, various kinds of external arguments can go with different causation events:
\ex a.
\Tree
[.VoiceP
	[.DP_1 ]
	[.
		[.Voice ]
		[.vP
			[.v ]
			[.DP ]
		]
	]
]
b.
\Tree
[.VoiceP
	[.DP_2 ]
	[.
		[.{\vd} ]
		[.vP
			[.v ]
			[.DP ]
		]
	]
]
\xe

This result makes sense if {\vd} is a marked head which only appears in the inventory of a language once this language already has Unspecified Voice (see also Chapter~\ref{chap:aas}). In other words, the two heads stand in an implicational relationship and we do not expect to find a language with {\vd} (and {\vz}) but without Voice.

On the other hand, it is not possible to have various kinds of \emph{lack} of external arguments. This point brings us to a novel prediction, namely that a specific kind of argument structure triplet should be highly rare, if not impossible.
\pex \label{ex:vd:causpred}\textbf{The triplet prediction for argument structure alternations}\\
	If a language has both anticausative and causative marking:
	\a Triplets of the form\\
		{[}marked \underline{unaccusative} $\sim$ unmarked \textbf{causative} $\sim$ marked causative] \\
		may be possible.
	\a Triplets of the form\\
	 	{[}marked \underline{unaccusative} $\sim$ unmarked \textbf{\underline{unaccusative}} $\sim$ marked causative] \\
	 	will not be possible.
\xe	

We have already seen examples of triplets such as those predicted by~(\lastx a) to exist in~(\ref{vd:ex:triplets-caus}). Those like~(\lastx b) are much more difficult to find. The following two triplets could be argued to exist in Hebrew:
\ex
\xe 
	\begin{small}
	\begin{tabular}{lllllll}
	Root	& {\tnif} & & {\tkal} & & {\thif} & \\\hline
	\root{xrv}  & \emph{nexrav} & `turned to ruins' & ??\emph{xarav} & `turned to ruins' & \emph{hexriv} & `demolished, turned to ruins'\\
	\root{ʃlm} & \emph{niʃlam} & `reached conclusion'	& ??\emph{ʃalam} & `became whole' & \emph{heʃlim} & `made up with someone'\\
	\end{tabular}
	\end{small}

In both cases the {\tkal} form is archaic and exists in contemporary speech only in a few set idioms, if at all. Speakers seem to prefer the {\tnif} form for the anticausative, in accordance with~(\blastx).

Outside of Hebrew, Acehnese has been reported to have an unmarked anticausative and an additional marked anticausative \citep{ko09afla}. Interestingly, this latter marked anticausative looks like it is derived from the marked causative. I therefore submit the generalization in~(\ref{ex:vd:causgen}) and the prediction in~(\ref{ex:vd:causpred}) as claims to be tested in more careful crosslinguistic work.

	\subsection{Productivity} \label{vd:caus:product}
Another point about the semantic flexibility of {\vd} concerns its productivity. The template {\thif} is a productive causative template in which speakers may innovate new forms on the fly \citep{lev16}. The verb \emph{taka} `stuck', for instance, is an ordinary transitive verb in {\tkal}, but the online comment in~(\nextx) innovates \emph{hetkia} in {\thif} (presumably for literary or comic effect). The article concerns a roller coaster which became stuck mid-ride on a Saturday, the prescribed day of rest, stranding those riding it for the better part of an hour.
\ex `Why don't you understand that the roller coaster also wanted to observe the Sabbath and rested 40 minutes {\dots}' \\
	\begingl
	\gla {...} elokim hevi la-xem siman ʃe-lo taalu al ha-mitkan be-ʃabat ve-hine hu \textbf{hetkia} et-xem le-40 dakot be-jom ʃabat kodeʃ//
	\glb {} G-d brought to-you sign that-\gsc{NEG} you.will.rise on the-device in-Saturday and-here he stuck \gsc{ACC-2PL} for-40 minutes in-day Saturday holy//
	\glft `{...} G-d gave you a sign not to go on the ride on Saturday, and there you go, he made you get stuck for 40 minutes on the holy Sabbath.'\trailingcitation{\url{https://www.ynet.co.il/Ext/App/TalkBack/CdaViewOpenTalkBack/0,11382,L-3441716-7,00.html}}//
	\endgl	
\xe
Additional examples can be found in \cite{lev16}.

In such cases, there is no strong prediction with regards to the kind of causation event; my expectation would be that different kinds of causation would be possible, as was the case for the examples in~(\ref{vd:ex:triplets-caus}). This much seems to be correct. The verb \emph{hexʃid}, from \emph{xaʃad} `suspected', is attested in both readings: `be suspected', `be made into a suspect' in~(\nextx) and `make X suspect s.th' in~(\anextx).

\pex Make O V-ed (turn into a súspect)
	\a \begingl
		\gla be-tviat-am toanim ha-ʃnaim ki gilboa \textbf{hexʃid} et deri be-re{ts}ixat-a ʃel ester verderber//
		\glb in-lawsuit-theirs claim the-two that Gilboa suspect.\gsc{CAUS} \gsc{ACC} Deri in-murder-hers of Esther Verderber//
		\glft `The two claim in their lawsuit that Gilboa turned Deri into a suspect in the murder of Esther Verderber.'\trailingcitation{\url{https://www.ynet.co.il/articles/0,7340,L-2443354,00.html}}//
	\endgl
	\a \begingl
		\gla ha-seruv \textbf{hexʃid} et netanjahu ve-sar-av ha-krovim ki re{ts}on-am litol le-a{ts}mam samxut-al jexudit, ve-lo linhog be-ʃkifut//
		\glb the-refusal suspect.\gsc{CAUS} \gsc{ACC} Netanyahu and-ministers-his the-close that will-theirs to.take to-themselves authority-superior unique and-\gsc{NEG} to.behave in-transparency//
		\glft This refusal makes one suspect that Netanyahu and his closest ministers wish to avail themselves of unique authority, rather than conduct themselves transparently.\trailingcitation{\url{https://www.israelhayom.co.il/opinion/294269}}//
	\endgl
\xe
	
\pex Make X verb (make someone suspéct s.th, make someone suspicious)
	\a \begingl
		\gla ʃalom lifnej beerex 5 elef hexlafti galgalʃ kolel ʃarʃeret (z750 2010) ve-ha-mexir ʃe-kibalti k{ts}at \textbf{hexʃid} ot-i//
		\glb hello before about 5 thousand I.changed sprocket including chain (z750 2010) and-the-price that-I.got a.little suspect.\gsc{CAUS} \gsc{ACC}-me//
		\glft `Hello, I changed my gear and chain (z750 2010) about five years ago and the price I got made me a little suspicious.'//\trailingcitation{\url{http://fullgaz.co.il/forums/archive/index.php/t-793.html}}//
	\endgl
	\a \begingl
		\gla galaj ha-mataxot lo hetria u-{vexol zot} ha-falestini \textbf{hexʃid} et loxamej {miʃmar ha-gvul}//
		\glb detector the-metal \gsc{NEG} warn and-nevertheless the-Palestinian suspect.\gsc{CAUS} \gsc{ACC} warriors.of {the Border Patrol}//
		\glft `The metal detector did not give any warning but nevertheless, the Palestinian aroused the suspicion of the Border Patrol soldiers.'\trailingcitation{\url{http://www.93fm.co.il/radio/445111/}}//
	\endgl
\xe

These examples confirm that there are clear compositional differences with the marked causative alternation: forms built from Voice/{\vz} are transparent, while those built from {\vd} are marked.
%Remember: ``[T]here is actually no coherent lexical semantic or conceptual reasoning available as to why an \emph{individual} verb (or verbal) concept in an \emph{individual} language shows up in one or the other class.'' \citep[65]{layering15}


%Entailments?
%	John opened the door but it didn't open
%	John fed the baby but it didn't eat
%	The teacher dictated the list of essay topics but the students didn't write them down
%	
%	Hans öffnete die Tür, aber sie hat sich nicht geöffnet
%	Hans fütterte das Kind, aber es hat nicht(s) gegessen
%	
%	
%	dani patax et ha-delet aval hi lo niftexa
%	tereza heexila et ha-tinok aval hu lo axal
%	ha-more hextiv et reSimat ha-nosim la-xibur aval ha-talmidim lo katvu otam
%
%\cite{fodor70}


	\subsection{The labile alternation} \label{vd:caus:labile}
The main characteristic of {\vd} is that it is supposed to guarantee the availability of an external argument; in other words, a transitive construction is possible if the event has change-of-state semantics, i.e.~an internal argument. Looking back at the labile alternation, I have not yet found any alternations in which the causative is preferred and the inchoative is a recent innovation; or inchoatives in {\thif} which have no causative counterpart. I take these findings to be emblematic of the causative meaning inherent in {\vd}: even if inchoative verbs have arisen, contemporary usage overwhelmingly tends to coin causatives in this template rather than another kind of verb \citep{laks14}.

Let us continue to assume that the process of inchoative formation in {\thif} is productive, as argued for by \cite{lev16}, and is not merely a list of exceptions, as implicitly assumed in most of the literature (with the exception of \citealt{doron03}, to be discussed in Section~\ref{vd:others:ed}). Then, when the speaker is faced with the choice of a construction for their de-adjectival or denominal verb, they might choose {\vd} because this structure guarantees that an external causer can be added (we should keep in mind that {\thit} is the more productive template for novel derived forms, e.g.~\citealt{laks11}). That is perhaps why this is the only head which is compatible with labile alternations.

One consequence of the overall analysis is that it allows us to state in formal terms the difference in argument distribution between causatives and inchoatives. %As seen in Sections~\ref{vd:vd}--\ref{vd:caus}, active verbs in {\thif} might be unergative, transitive or ditransitive. These possibilities are wiped out for inchoatives, which are uniformly intransitive. Let us assume that whatever requirements a verb has for its complements emerge upon combination of v and the root.
I have proposed that once the structure contains a more deeply embedded a/n node, v is too far away from the root for particular selectional requirements to be stated. This idea receives potential corroboration from the behavior of -\emph{en} in English.\footnote{Thanks to Jim Wood for pointing out this phenomenon.} As noted by \cite{harley09n}, English verbalizers such as -\emph{ify}, \emph{-ize} and \emph{-ate} can derive verbs that are uniformly unaccusative (e.g.~\emph{oscillate}), uniformly unergative (e.g.~\emph{detoriorate}) or labile (e.g.~\emph{activate}), but -\emph{en} verbs are always labile. An examination of the list in \citet[245]{levin93} confirms this claim. If we assume that these latter verbs contain additional structure, for instance [v [CMPR [a \root{Root}~\!]]] \citep{bobaljik12}, we arrive at a similar analysis to that of {\thif} inchoatives: they cannot impose selectional restrictions and are ``stuck'' with the argument structure imposed by the syntax. But I will not develop this idea or the crosslinguistic ramifications any further.

	\subsection{Generalizing to Pred/\textit{i}*} \label{vd:caus:pred}
Before turning to alternative accounts of the patterns above, I would like to briefly consider a variant of the account given above (suggested by Jim Wood, p.c.~November 2019). The intuition here is that the head deriving inchoatives in {\thif} is not {\vd} itself but a variant of the predicative head Pred, which itself is another label for \textit{i*} (a generalized argument introducer I discuss later on, in Chapter~\ref{i:i}).

The formal analysis builds on the notion of a predicative head Pred, which has been invoked in various ways in the literature. Since the specifics are not important for this short discussion, I will simply refer to \cite{bowers93,bowers01} for one influential account and \cite{matushansky19} for a recent reply.

According to the Pred alternative, denominal and de-adjectival verbs in this template at least are derived using the head {\predd}. Causatives would then have the structure in~(\nextx) and unaccusatives the structure in~(\anextx): the internal argument is introduced in Spec,PredP, which is then embedded under Voice.
\ex
	\Tree 
	[.VoiceP
		[.DP ]
		[.
			[.{\vd} ]
			[.vP
				[.v ]
				[.PredP
					[.DP ]
					[.
						[.{\predd} ]
						\qroof{red}.\emph{aP}
					]
				]
			]
		]
	]
\xe

\ex
	\Tree
	[.VoiceP
		[.Voice ]
		[.vP
			[.v ]
			[.PredP 
				[.DP ]
				[.
					[.{\predd} ]
					\qroof{red}.\emph{aP}
				]
			]
		]
	] 
\xe

If we assume this analysis, we can then replace {\predd} with the generalized \emph{i*}$_{\text{[\!+\!D]}}$, which should have the same spell-out as {\vd}; again, see Chapter~\ref{i:i}.

I do not have any strong reason to adopt or reject this proposal, although a main cause for concern would be the increased combinatorics associated with an additional head, in this case Pred: what about Unspecified Pred and {\predz}? More concretely, I cannot think of cases where {\predz} would be motivated. The account of figure reflexives in Chapter~\ref{vz:pz} discusses {\pz} but no {\pd}. Nevertheless, we could assume a covert {\pd} in ditransitive verbs; evidence for {\predz} is harder to establish.\footnote{Jim Wood (p.c.) tentatively suggests that constructions such as \emph{The wizard turned invisible to avoid being detected} or \emph{The fish turned red to impress its mate} could be the adjectival counterparts of figure reflexives, at least in English.}

In addition, {\predd} would need to be morphologically conditioned by T over the intervening Voice head. But this issue does already arise for {\pz}, as discussed in Chapter~\ref{vz:pz:phono}.


\section{Alternative accounts} \label{vd:others}
This section focuses on a number of competing analyses aiming to explain the behavior of verbs in {\thif}, concentrating mostly on the inchoative alternation. Apart from these, \cite{lev16} sketched a theory in which labile verbs are less agentive than others, a claim that could explain why \emph{heet} `slowed down' is possible as opposed to *\emph{hemhir} (from `quick'). However, that idea cannot be extended to explain the existence of minimally different \emph{hei{ts}} `accelerated' so it will be set aside.

I start with a general question of how alternations should be treated, one which in a way ties together the threads of the last three chapters, before turning to more specific points about {\thif}.
%In addition, a possible objection to the treatment of marked causatives in cases of low semantic similarity between the two causative forms was already shot down at the end of Section~\ref{vd:caus:marked} based on the lack of doublets.
	
	\subsection{Where do alternations live?} \label{vd:others:arad}
As I have tried to make clear, under the current proposal there is no formal way in which an anticausative verb is derived from a causative verb, or a causative verb from an inchoative verb \citep{schaefer08}. There are only different Voice heads which can be merged with a core vP. In contrast to this approach, the traditional view of argument structure alternations for Semitic (and beyond) assumes that verbs in one template are derived, or might be derived, from verbs in another template.

\cite{arad05} is unique in making such a theory formally explicit and internally consistent. The precise formulation enables us to see exactly what the strengths and weaknesses of such an approach are. Some of these were already mentioned in Section~\ref{vz:others:morph}; for others see \cite{kastnertucker19cup}. I will invoke her analysis once more in order to further explain how alternations---or perceived alternations---work in different theories..

As noted in Section~\ref{vz:others:morph}, \cite{arad05} assumes that alternations hold between specific templates, as in the following conjugation classes \cite[226]{arad05}. The ones relevant to {\thif} are highlighted.
\pex
	\a Conjugation 1: {\tnif} $\rightarrow$ {\tkal}
	\a \textbf{Conjugation 2: {\tkal} $\rightarrow$ {\thif}}
	\a \textbf{Conjugation 3: {\tnif} $\rightarrow$ {\thif}}
	\a Conjugation 4: {\thit} $\rightarrow$ {\tpie}
	\a Conjugation 5: {\thit} $\rightarrow$ {\tpie}
	\a \textbf{Conjugation 6: {\thif} $\rightarrow$ {\thif}}
\xe
Three conjugation classes are needed because {\thif} ostensibly alternates with three other templates: {\tkal}, {\tnif} and {\thif} itself. For example, \emph{nirdam} `fell asleep' alternates with \emph{herdim} `put someone to sleep; applied anesthesia', instantiating~(\lastx c).

The spell-out rules relevant to {\thif} are highlighted in~(\nextx). In prose: If Class 2, then the inchoative is {\tkal} and the causative is {\thif}. If Class 3, then the inchoative is {\tnif} and the causative is {\thif}. If Class 6, then the inchoative is {\thif} and the causative is also {\thif}. If the root does not take part in an alternation, then a verb in {\thif} can spell out unmarked v, inchoative v or causative v (but not stative v).
\pex Distributed Conjugation Diacritics in \citet[230]{arad05}: \label{ex:arad-classes2}
\begin{multicols}{2}
	\a  v$_{unmarked}$:\\
			$\alpha$ $\rightarrow$ {\tkal}\\
			$\beta$ $\rightarrow$ {\tnif}\\
			$\gamma$ $\rightarrow$ {\tpie}\\
			\textbf{$\delta$ $\rightarrow$ {\thif}}\\
			$\epsilon$ $\rightarrow$ {\thit}
	\a v$_{inch}$:\\
			$\alpha$ $\rightarrow$ {\tkal}\\
			\textbf{$\beta$ $\rightarrow$ {\tnif}} \\
			\dots \\
			\textbf{$\delta$ $\rightarrow$ {\thif}}\\
			$\epsilon$ $\rightarrow$ {\thit}\\
			\dots \\
			\textbf{Conjugation 2 $\rightarrow$ {\tkal}}\\
			\textbf{Conjugation 3 $\rightarrow$ {\tnif}}\\
			Conjugation 4 $\rightarrow$ {\thit}\\
			\textbf{Conjugation 6 $\rightarrow$ {\thif}}\\
			\dots
		\columnbreak
	\a v$_{stat}$:\\
			$\alpha$ $\rightarrow$ {\tkal}\\
			Class 3 $\rightarrow$ {\tkal}\\
			Class 5 $\rightarrow$ {\tkal}
	\a v$_{caus}$:\\
			$\gamma$ $\rightarrow$ {\tpie}\\
			\textbf{$\epsilon$ $\rightarrow$ {\thif}}\\
			Conjugation 1 $\rightarrow$ {\tkal}\\
			\textbf{Conjugation 2 $\rightarrow$ {\thif}}\footnote{\citet[231]{arad05} has this as {\tnif}, which as far as I can tell is a typo.}\\
			\textbf{Conjugation 3 $\rightarrow$ {\thif}}\\
			Conjugation 4 $\rightarrow$ {\tpie}\\
			\textbf{Conjugation 5 $\rightarrow$ {\thif}}\\
			\textbf{Conjugation 6 $\rightarrow$ {\thif}}\\
			\dots
	\end{multicols}
\xe

In the current theory none of this machinery is required. Alternations are an intuitive way of describing what happens when a given core vP combines with Unspecified Voice compared to {\vz} and compared to {\vd}.

Importantly, I am not claiming that the current theory does away with all of the idiosyncratic listing that \citeauthor{arad05}'s had. As emphasized throughout this book, all theories need to list at some level which roots can combine with which functional heads/templates. But I hope it is clear why the current theory is to be preferred: beyond the empirical arguments I adduce, the overall view of the grammar is more streamlined, less stipulative, and much more in line with our theories of non-Semitic languages.

A theory using conjugation classes needs to make reference to silent flavors of v (whereas the elements I have proposed are all overt). It can also encode virtually any alternation. That may well be too powerful but it does absolves the theory of the need to worry about the combinatorics of individual heads, which is a potential advantage over the current theory insofar as the behavior of {\vd} is concerned. Concretely, my account does not contain a principled answer to the question why {\vd} cannot attach to a [{\va} vP] -- this is possible, but I am not claiming that any template instantiates this combination. One possible answer is that a structure such as [{\vd} [{\va} vP]] would just entail an agentive reading, something that {\vd} is already compatible with, or a specific causative reading from among the kinds in~(\ref{vd:ex:triplets-caus}). Furthermore, the phonology of {\va} could in principle be impoverished under {\vd}. Under a functionalist view, {\va} and {\vd} do similar work. This kind of issue, however, is much more pressing when pointed right back at the conjugation class and morphemic accounts: why are these specifically the conjugation classes that exist and the templates that exist?

	\subsection{Added structure} \label{vd:others:struct}
A different alternative view might counter that {\vd} does not exist. On this view, verbs in {\thif} are not derived using {\vd} but by additional structure atop of regular, active Voice.

Let me briefly outline what such an analysis would look like. This structure would presumably consist of some higher causative head, perhaps another layer of Voice.\footnote{For variants of the idea that an analytic causative can be deriving using Voice-over-Voice, see \cite{tubinoblanco11}, \cite{harley13lingua,harley17oup} and \cite{nie19}.} While this idea holds theoretical promise, there are two reasons I do not think it can account for the Hebrew data, having to do with incorrect predictions in the phonology and in the syntax-semantics.

First, {\vd} seems as integrated into the morphophonological system as the other Voice heads, namely Unspecified Voice and {\vz}. In particular, the analysis in \cite{kastner18nllt} shows that the spell-out of {\vd} is subject to the same locality constraints as that of Voice and {\vz}. Splitting {\vd} into two layers of Voice will disrupt these locality relations, thereby making the wrong predictions in the phonological component.

Second, this approach would treat {\thif} as a straightforward ``make''-causative (analytical causative). This much is clearly wrong for the unergative and unaccusative verbs in this template as it misses the fact that causatives in this template are ``lexical'', as discussed in Section~\ref{vd:caus:marked}. 

Third, assuming that another Voice head can be added to VoiceP makes the false prediction that a transitive in {\tkal} and a causative in {\thif} will have the same internal argument. This is once again incorrect, as we saw in Section~\ref{vd:caus:marked}.

And fourth, in Chapter~\ref{passn:pass} I discuss how passives are derived by use of an additional head Pass. As explained there, Pass can only attach to {\vd} or to Voice+{\va}, but not to Voice on its own. If we were to assume that {\vd} is actually Voice+Voice, we would need to make an additional stipulation regarding configurations that can be passivized. This last problem is not insurmountable but it would complicate the theory.

	
	\subsection{CAUS and existential closure} \label{vd:others:ed}
A different kind of alternative analysis concerns itself mostly with the inchoatives of Section~\ref{vd:thif:inch}. This analysis would posit a silent, generic Cause in Spec,{\vd}. The analysis in \citet[61]{doron03}---which in many ways is a precursor to the theory presented in this work---assumes that a causative head \gsc{CAUS} gives rise to {\thif}. The problem for the system in \cite{doron03} is that if these verbs are derived using \gsc{CAUS} rather than the middle head \gsc{MID}, we have no explanation for their unaccusativity.

As a result, \citet[62]{doron03} must conclude that ``\emph{x reddened} is equivalent to \emph{Something caused x to redden}'', with \gsc{CAUS} introducing a Causer that is existentially quantified over. This kind of account is more in line with a passive analysis than a causative one.\footnote{Although Doron's morphosemantic head formed the direct inspiration for the current {\vd}, just as her \gsc{MID} and \gsc{INTNS} heads paved the way for {\vz} and {\va}.}

Assume for the sake of the argument that a silent element fills Spec,{\vd} in inchoatives. One would need to specify the exact featural makeup of this element, for example a null subject \emph{pro}. The result would be a transitive structure where \emph{pro} should be assigned Nominative case and the internal argument should be assigned Accusative case. Definite accusative objects in Hebrew take the direct object marker \emph{et}, so we would predict that \emph{et} appears before inchoatives in \thif. But this is incorrect: the generic Cause cannot be a silent pronoun in a transitive relationship with the internal argument.
\pex
	\a \ljudge{*} \begingl
		\gla heʃmin \textbf{et} ha-xatul//
		\glb fattened.\gsc{CAUS} \gsc{ACC} the-cat//
	\endgl
	
	\a \ljudge{*} \begingl
		\gla \textbf{et} ha-xatul heʃmin//
		\glb \gsc{ACC} the-cat fattened.\gsc{CAUS}//
		\glft (int. `The cat grew fat')//
	\endgl
\xe

Another tack would be to say that instead of \emph{pro}, the silent external argument is a Weak Implicit Argument in the sense of \cite{landau10}: a bundle of phi-features with no [D] feature (cf.~\citealt{legate14,bhattpancheva17}), distinguishing it from a Strong Implicit Argument such as \emph{pro}. If there is no [D] feature on the weak EA, it does not participate in the calculus of case and the IA will receive unmarked case, i.e.~Nominative. This kind of analysis ends up being very similar to the one in the current proposal: the external argument is not taking part in any relevant syntactic process, and whatever requirements {\vd} has still need to be satisfied. Furthermore, This bundle of phi-features would then have no detectable effects in the syntax or phonology, rendering it purely stipulative. In the absence of a convincing account for implied causers, I reject this analysis.

	\subsection{Contextual allomorphy}
Another possible analysis of inchoatives is strictly morphological in nature. Under this account, unaccusative inchoatives are true unaccusatives derived with {\vz}, except that the allomorphic rule in~(\ref{ex:vd:allo-unacc}) causes {\vz} to be pronounced like {\thif} rather than like {\tnif}.
\pex
  \a \vz~\lra~{\thif} / \trace~\{\root{lbn}, \root{'dm}, \root{xl\dgs{k}}, \root{xvr}, \root{ʃmn}, \dots \}\label{ex:vd:allo-unacc}
  \a \vz~\lra~{\tnif}
\xe

One question which arises is whether we would like to postulate this rule for a list of just over 30 roots. Furthermore, the mystery would remain of why it is specifically {\thif} that houses inchoatives: why doesn't the rule in~(\lastx a) insert the form of any other template, such as {\tkal}, {\tnif} or {\tpie}? This solution is technically possible but conceptually unenlightening.

That being said, it does correctly predict that the roots to which this rule applies cannot surface in {\tnif}, only in {\thit}. The reason is that forms in {\tnif} are generated using {\vz}, but {\vz} is pronounced as {\thif}.
\ex\label{ex:vd:allo-pred}
\begin{tabular}{lllll}
	a. & \emph{helbin} & $\sim$ & *\emph{nilban} & `whitened' \\
	b. & \emph{heedim} & $\sim$ & *\emph{nidam} & `reddened'\\
	c. & \emph{heʃmin} & $\sim$ & \emph{*niʃman} & `fattened'\\
\end{tabular}
\xe

	\subsection{Verbalizing affix} \label{vd:others:borer}
In the last alternative to be considered, \cite{borer91} presents an analysis of {\thif} alternations couched in Parallel Morphology (which I have translated into comparable terms in the current theory). Her account consists of two main parts. In the first, she argues that inchoative forms are derived from adjectives while causative forms are derived from a root/verb. In the second, she presents an analysis showing why it must be the case---given certain assumptions---that causatives are formed in the lexicon and inchoatives in the syntax. The current analysis is similar to hers in adopting separate structures for causatives and inchoatives, albeit using different argumentation. The content of the analysis is different, though, since for \cite{borer91} {\thif} is a single verbalizing morpheme which subcategorizes for an adjective.

This approach takes Hebrew {\thif} and English -\emph{en} to be verbalizers subcategorizing for an adjectival stem, be it a property root or an adjective \citep[136]{borer91}. %Both the adjective and the verbalizer are able to---but need not---assign an external theta-role \citep[142]{borer91}.
When this is done in the ``lexicon'' by verbalizing a root, the result is a causative verb:
\ex\label{ex:vd:thif-borer-caus}{[}$_{\text{v}}$ \root{\gsc{wide}} -\emph{en}]
\xe
When this is done in the syntax by verbalizing an adjective, the result is an inchoative verb:
\ex\label{ex:vd:thif-borer-inch}{[}$_{\text{v}}$ [$_{\text{a}}$ \root{\gsc{wide}} a] -\emph{en}]
\xe

Crucially for us, the analysis does not answer the questions posed earlier on in the discussion of inchoatives: why this template and why these roots. Here {\thif} is assumed to be a de-adjectival verbalizer, just like -\emph{en}, without discussion of this template's role in the overall morphosyntax of the language. While it is stipulated that \thif~as a verbalizer subcategorizes for an adjective, this is not always the case: as shown above, many inchoatives are derived from underlying nouns. More importantly, even run-of-the-mill causatives such as \emph{hexnis} `inserted', \emph{heexil} `fed' and \emph{helbiʃ} `dressed' are not derived from underlying adjectives.

The causative \emph{hexnis} `inserted' is derived from \root{kns}, but without a simple adjective *[$_{a}$ \root{kns} a]. One could posit an abstract adjective that is never lexicalized, but it is unclear what this non-existent adjective would be like or what its phonological form would have been (*\emph{kanus}?).
\ex \begingl
	\gla ha-nasix \textbf{hexnis} et ha-sefer la-tik//
	\glb the-prince inserted.\gsc{CAUS} \gsc{ACC} the-book to.the-bag//
	\glft `The prince put the book in the bag.'//
	\endgl
\xe

The causative \emph{heexil} `fed' is derived from \root{'kl}, but probably not from \emph{axul} `consumed', a rare adjectival passive of \emph{axal} `ate'.
\pex
	\a \begingl
		\gla ha-nasix \textbf{heexil} et ha-kivsa//
		\glb the-prince fed.\gsc{CAUS} \gsc{ACC} the-sheep//
		\glft `The prince fed the sheep.'//
		\endgl
		
	\a \ljudge{$\ne$} \begingl
		\gla ha-nasix garam la-kivsa lihiot \textbf{axula}//
		\glb the-prince caused to.the-sheep to.be consumed//
		\glft `The prince caused the sheep to be consumed.'//
		\endgl
\xe

And the causative \emph{helbiʃ} `dressed' is derived from \root{lbʃ}, but probably not from \emph{lavuʃ} `dressed up', the adjectival passive of \emph{lavaʃ} `wore', which seems to be reserved for descriptions of a full costume.
\pex
	\a \begingl
		\gla ha-ima \textbf{helbiʃ-a} et ha-jeled (be-)xalifa jafa//
		\glb the-mom dressed.\gsc{CAUS}-\gsc{F.SG} \gsc{ACC} the-boy in-suit pretty//
		\glft `The mother put the boy's pretty suit on (him).//
		\endgl
	
	\a ``On making his discovery, the astronomer had presented it to the International Astronomical Congress, in a great demonstration, \dots\\
	\begingl
%		\gla le'axar ʃe-gila et taglit-o, hetsig ota ha-astronom ha-turki bifne-j ha-kongres ha-astronomi ha-benleumi, be-tetsuga marʃima,
		\gla aval iʃ lo he'ezin le-dvara-v, miʃum ʃe-haja \textbf{lavuʃ} be-tilobʃet turkit. ka'ele hem ha-mevugarim//
%		\glb after \gsc{COMP}-discovered \gsc{ACC} discovery-his presented it the-astronomer the-Turkish in.front.of the-congress the-astronominal the-international in-display impressive
		\glb but nobody \gsc{NEG} listened to-words-his, since \gsc{COMP}-was dressed.up in-outfit Turkish such \gsc{3PL} the-grown.ups//
		\glft But he was in Turkish costume, and so nobody would believe what he said. Grown-ups are like that.''\hfill {(Antoine de Saint-Exup\'ery, \emph{The Little Prince}, Chapter 4. Hebrew by Jude Shva\footnotemark)}//
	\endgl
\footnotetext{\url{http://www.oocities.org/sant\_exupery/c4.htm}}
\xe

%To be clear, \cite{borer91} did not claim that the sole function of {\thif} is to verbalize adjectives. But even if this is only one of its functions, we have seen that not all verbs in {\thif} show the alternation. As her system stands, it is not clear how it could allow for a certain root to be instantiated only in a ``syntactic'' (inchoative) derivation but not in a ``lexical'' (causative) one.

%Similarly, and as shown above, not all inchoatives in this template are de-adjectival: \emph{heki} `threw up' comes from the noun \emph{ki} `vomit', \emph{hekrim} `clotted' from the noun \emph{krem} `cream', and \emph{hetsxin} `smelled pungent' from the noun \emph{tsaxana} `pungent smell'.
%Another question is whether -\emph{en} and \thif~are productive enough as de-adjectival verbal forms to merit a derivational rule as in the Parallel Morphology system, rather than what is in effect a list of exceptional forms as in mine.

The analysis in \cite{borer91} did not attempt to find an underlying reason for why {\thif} is used for both causatives and inchoatives, as well as for general causativization in the rest of the system. Nevertheless, it remains the only in-depth study of this alternation that I know of. Recall, for the last part of this discussion, that this analysis also postulates a structural difference between {\thif} causatives, (\ref{ex:vd:thif-borer-caus}), and inchoatives, (\ref{ex:vd:thif-borer-inch}). I review this distinction next.

The logic works as follows: if an adjective passes certain diagnostics, and the inchoative does but the causative does not, then the adjective must be embedded in the inchoative \citep[130]{borer91}. Starting with an English example, the adjective \emph{wide} is said to license comparisons with \emph{as}/\emph{like} and comparative forms, whereas the inchoative \emph{widen} does not. \citeauthor{borer91}'s claim is that comparison adverbials and the comparative must be licensed by an adjective (judgments hers).
\pex
  \a The canal is \{as \underline{wide} as a river / \underline{wider} than a river.\}
  \a The canal \underline{widened} \{like a river / more than a river\}.\\
	  (int. `The canal became as wide as a river is wide / became more wide than a river is wide')
  \a \ljudge{*} The flood \underline{widened} the canal \{like a river / more than a river\}.\\
	  (int. `The flood made the canal as wide as a river is wide / made the canal wider than a river is wide')
\xe
I suspect that there is much more variation in acceptability for the utterances in~(\lastx), and that an adverbial reading normally overpowers the scalar one (`The flood widened the canal like a river widens it'). Three native speaker linguists I have consulted do not share these contrasts but I am more interested in the argumentation involved with this approach.

Taking the adjective \emph{ʃmen-a} `fat-\gsc{F.SG}', it is claimed to license comparatives,~(\nextx a). Inchoatives license comparatives too,~(\nextx b), but causatives do not,~(\nextx c). Judgments are as in \cite{borer91}; example (\nextx c) does not sound as degraded to me, but it does to another speaker I consulted informally.
\pex
  \a Adjective:\\ \begingl
    \gla ha-xatula \textbf{ʃmena} \emph{\{}kmo xazir / joter mi-xazir\emph{\}}//
    \glb the-cat fat like pig {} more than-pig//
    \glft `The cat is fat as a pig / fatter than a pig.'//
  \endgl
  
  \a Inchoative:\\ \begingl
    \gla \underline{ha-xatula} \textbf{heʃmina} \emph{\{}kmo xazir / joter mi-xazir\emph{\}}//
    \glb the-cat fattened like pig {} more than-pig//
    \glft `The cat grew as fast as a pig / fatter than a pig.'//
  \endgl

  \a Causative:\\ \begingl
    \gla \ljudge{*}\underline{ha-zrika} \textbf{heʃmina} et ha-xatula \emph{\{}*kmo xazir / *joter me-xazir\emph{\}}//
    \glb the-injection fattened \gsc{OM} the-cat.\gsc{F} like pig {} more than-pig//
    \glft (int. `The injection made the cat fat as a pig / more than a pig is fat.')//
  \endgl
\xe

Similarly, some adverbs (\emph{haxi ʃe-efʃar} `as much as possible') must be licensed by an adjective and accordingly only appear with inchoatives, not causatives.

It seems to me that the success of this diagnostic depends to a large extent on the lexical items chosen. For example, using the antonym \emph{herza} `grew thin', my judgments are slightly different:
\pex
  \a Adjective:\\ \begingl
    \gla ha-xatula \textbf{raza} \emph{\{}kmo makel / ?joter mi-makel\emph{\}}//
    \glb the-cat thin like stick {} more than-stick//
    \glft `The cat is as thin as a rail / skinnier than a rail.'//
  \endgl
  
  \a Inchoative:\\ \begingl
    \gla\ljudge{?}\underline{ha-xatula} \textbf{herzeta} \emph{\{}kmo makel / ??joter mi-makel\emph{\}}//
    \glb the-cat thinned like stick {} more than-stick//
    \glft (int. `The cat became as thin as a rail / skinnier than a rail.')//
  \endgl

  \a Causative:\\ \begingl
    \gla \ljudge{??}\underline{ha-zrika} \textbf{herzeta} et ha-xatula \emph{\{}kmo makel / joter me-makel\emph{\}}//
    \glb the-injection thinned \gsc{OM} the-cat like stick {} more than-stick//
    \glft (int. `The injection made the cat as thin as a rail / skinnier than a rail.')//
  \endgl
\xe

With \emph{heet} `slowed down' I judge inchoatives unacceptable and causatives slightly better though still degraded. These judgments are meant to highlight the variance, not to be taken as categorical for all alternations or all speakers.
\pex
	\a Adjective:\\ \begingl
		\gla ha-mexonit ha-zo \textbf{itit} \emph{\{}kmo {ts}av / joter mi-{ts}av\emph{\}}//
		\glb the-car the-this slow like turtle {} more than-turtle//
		\glft `This car is as slow as a turtle / slower than a turtle.'//
	\endgl

	\a Inchoative:\\ \begingl
		\gla\ljudge{*}ha-mexonit ha-zo \textbf{heeta} \emph{\{}kmo {ts}av / joter mi-{ts}av\emph{\}}//
		\glb the-car the-this slowed like turtle {} more than-turtle//
		\glft (int. `This car slowed down to turtle speed / to sub-turtle speed.')\\
			(More acceptable on a reading of `The car slowed down like a turtle slowed down'.)//
	\endgl

	\a Causative:\\ \begingl
		\gla\ljudge{??}\underline{ha-baaja} \underline{ba-hiluxim} \textbf{heeta} et ha-mexonit \emph{\{}kmo {ts}av / joter mi-{ts}av\emph{\}}//
		\glb the-problem in.the-gears slowed \gsc{ACC} the-car like turtle {} more than-turtle//
		\glft (int. `The problem with the gear box slowed the car down to turtle speed / to sub-turtle speed.')//
	\endgl
\xe

It is also left vague what precisely this diagnostic is probing. In~(\nextx), for instance, there is no underlying adjective `beloved' but the utterance is completely acceptable:\footnote{Thanks to Idan Landau for pointing this out to me.}
\ex
	\begingl
	\gla ani \textbf{ohev} otxa \underline{kmo} \underline{ax}//
	\glb I love.\gsc{SMPL}.\gsc{PTCP} you.\gsc{M} like brother//
	\glft `I love you like a brother.'//
	\endgl
\xe

%The remainder of the article is devoted to working through the different possible structures generated by the framework and discussing whether they are licit or not. The final few pages \citep[150]{borer91} raise the issue of whether these verbs give comparative (change on a scale) or absolute (result) readings, concluding that both are in principle possible but are constrained by the root. That is to say, both English \emph{reddened} and Hebrew \emph{heedim} `reddened' are as compatible with a `became redder' meaning as with a `became red' meaning, but some verbs like \emph{quicken} are only compatible with a `became quicker' meaning (see \citealt[ch.~5]{bobaljik12} and the literature on degree achievements referenced in Section~\ref{vd:inch:template} for relevant discussion).

Since I am not sure that the argument from comparatives generalizes, and given that no explicit syntax or semantics for this modification was put forward, I do not endorse the arguments for distinct structures put forward in \cite{borer91}. Nevertheless, the current proposal has recast that intuition in contemporary terms and supported it using different arguments. Hopefully these were explicit enough to be similarly challenged in future work.
%Future experimental studies could test the predictions that these theories of {\thif} make for speakers' usage of nonce verbs in this template.


\section{Conclusion} \label{vd:sum}
This chapter developed the theory of [\!+\!D] in Voice based on an in-depth analysis of various verb types in {\thif}.\footnote{I do not posit a [\!+\!D] variant of \textit{p} since there is no evidence for such a head, be it syntactic, semantic, or phonological.} This template predominantly instantiates active verbs, usually causatives. It is also reasonably productive. Yet a number of roots derive inchoative verbs in this template.

\hammer{
\pex \label{ex:gen-thif-sum}\textbf{Generalizations about {\thif}}
	\a \textbf{Configurations:} Verbs appear in transitive and unergative configurations; a small class of verbs forms unaccusative degree achievements.
	\a \textbf{Alternations:} Some verbs are causative or active versions of verbs in other templates, especially {\tkal}. A small class of verbs creates a labile alternation within {\thif}.
\xe
}

The analysis proposed here showed how the influence of a certain class of roots can be accommodated in the grammar, while keeping constant the overall behavior of the functional head. The existence of unmarked and marked causatives was discussed with respect to the leeway different roots have within similar structures. The feature [+D] must have some semantic content beyond the unmarked causative. 

The two factors conspiring to create a labile alternation in a language that otherwise does not allow such an alternation are the root and the syntactic structure. The roots fall under various lexical semantic classes but all appear to derive degree achievements from underlying nouns or adjectives, as suggested by \cite{lev16}. The syntax which facilitates this derivation is one in which a noun or adjective is first formed before it is verbalized, and then combined with a specific causative head {\vd}. This theoretical approach allows us to ask more specific questions about how the idiosyncratic information associated with roots interacts with the syntactic structure in which they are embedded.

Taken together, these last three chapters cashed out the trivalent theory of Voice which is at the core of this book. The next chapter rounds off the empirical picture by examining cases in which these heads interact with additional structure: passivization, adjectival passives and nominalization.



%@entailment tests (Louise: the teacher dictated the essay cannot mean he stood menacingly over the students until they started writing, i.e. made-write)