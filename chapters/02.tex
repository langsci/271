\chapter{Unspecified Voice}
\label{chap:voice}

\section{Overview} \label{voice:intro}
We begin by examining the ``simple'' template {\tkal} and the ``intensive'' template {\tpie}. Section~\ref{voice:tkal} reviews the empirical picture for {\tkal} and distills a number of generalizations, followed by a formal analysis using the Unspecified Voice head in Section~\ref{voice:voice}. I then move on to the template {\tpie} in Section~\ref{voice:tpie}, showing what it teaches us about an agentive modifier which I call {\va} in Section~\ref{voice:va}. Section~\ref{voice:conc} summarizes and outlines how the rest of the Hebrew system will inform the theory developed in the first part of this book.

\section{\tkal: Descriptive generalizations} \label{voice:tkal}
This chapter introduces the first part of a theory of Voice which makes room for an unspecified variant, one which neither requires nor prohibits a specifier. We will first consider morphological marking which is compatible with a variety of syntactic structures, namely the template {\tkal}.

As we have already seen briefly in the previous chapter, Hebrew has dedicated active and non-active morphology. For example, verbs in {\tnif} are usually non-active and those in {\thif} are active. Verbs in {\tkal} are unique within the verbal system in that they are underspecified with regards to their argument structure. Simply knowing the morphological form (the template) is not enough to indicate what kind of verb we are dealing with. Let us examine the different possibilities, introducing the diagnostics to be used throughout this book.

	\subsection{Active verbs}
With some roots, the verb is transitive. The examples in~(\ref{ex:voice-intro-tr}) contain strongly transitive verbs, which require an internal argument, and assign accusative case.\footnote{There is a substantial literature on \emph{et} and what kind of syntactic element it is \citep{siloni97,danon01,borer13oup}. What is uncontroversial is that it occurs before specific accusative objects.}
\pex\label{ex:voice-intro-tr}
	\a \begingl
		\gla teo \textbf{taraf} *(et ha-laxmanja)//
		\glb Theo devoured.\gsc{SMPL} \gsc{ACC} the-bread.roll//
		\glft `Theo devoured the bread roll.'//
	\endgl
	\a \begingl
		\gla ha-balʃan \textbf{katav} *(et) ha-maamar ha-arox//
		\glb the-linguist wrote.\gsc{SMPL} \gsc{ACC} the-article the-long//
		\glft `The linguist wrote the long article.'//
	\endgl
\xe

With other roots, verbs in {\tkal} are unergative. The examples in (\ref{ex:voice-intro-unerg}) show activities which can be repeated or modified with atelic adverbials.
\pex\label{ex:voice-intro-unerg}
	\a \begingl
		\gla teo \textbf{rakad} ve-rakad ve-rakad (kol ha-boker)//
		\glb Theo danced.\gsc{SMPL} and-danced.\gsc{SMPL} and-danced.\gsc{SMPL} all the-morning//
		\glft `Theo danced and danced and danced (all morning long).'//
	\endgl
	\a \begingl
		\gla teo \textbf{halax} kol ha-boker//
		\glb Theo walked.\gsc{SMPL} all the-morning//
		\glft `Theo walked all morning long.'//
	\endgl
\xe

Other roots give rise to ditransitive verbs, including strong ditransitives in which the goal cannot be omitted (\ref{ex:voice-intro-ditr}).
\pex\label{ex:voice-intro-ditr}
	\a \begingl
		\gla teo \textbf{natan} *(le-marsel) et ha-xatif//
		\glb Theo gave.\gsc{SMPL} to-Marcel \gsc{ACC}  the-snack//
		\glft `Theo gave Marcel the treat.'//
	\endgl
	\a \begingl
		\gla teo \textbf{ʃaal} et ha-sefer me-ha-sifria//
		\glb Theo borrowed.\gsc{SMPL} \gsc{ACC} the-book from-the-library//
		\glft `Theo borrowed the book from the library.'//
	\endgl
\xe

	\subsection{Non-active verbs} \label{voice:tkal:nact}
Unaccusative verbs are also possible in this template. Whereas the diagnostics mentioned for active verbs of different kinds are well-established and fairly intuitive, the unaccusative ones are worth introducing in more depth. These are: compatibility with agent-oriented adverbs including `by itself’ (Section~\ref{voice:tkal:nact:adv}) and the two standard unaccusativity diagnostics for Hebrew (Section~\ref{voice:tkal:nact:unacc}).

		\subsubsection{Adverbial modifiers} \label{voice:tkal:nact:adv}
A common assumption in studies of anticausativity is that the existence of an Agent can be probed using certain adverbial modifiers, or the phrase 'by itself' if there is no Agent \citep{unaccusativity95,alexiadouanagnostopoulou04,layering15,alexiadoudoron12,koontzgarboden09,kastner17gjgl}.\footnote{\cite{layering15} emphasize that for cases where `by itself' agrees with the internal argument as in Hebrew and English, what it diagnoses is the absence of an implicit (external) argument which may be an Agent or a Cause, rather than simply being sensitive to agentivity.} 

Unaccusatives are incompatible with \emph{by}-phrases, which would otherwise refer to an Agent, (\nextx).
\ex \begingl
	\gla ha-bakbuk \textbf{nafal} (*al-jedej ha-jeled)//
	\glb the-bottle fell.\gsc{SMPL} by the-boy//
	\glft (int.~`The boy dropped the bottle')//
	\endgl
\xe

Agent-oriented adverbs are fine with transitive verbs~(\nextx a) but not with unaccusatives~(\nextx b).
\pex
	\a \begingl
		\gla teo \textbf{taraf} et ha-laxmanja \underline{be-raavtanut}//
		\glb Theo devoured.\gsc{SMPL} \gsc{ACC} the-bread.roll.\gsc{F} in-voracity//
		\glft `Theo devoured the roll ravenously.'//
		\endgl
	\a \ljudge{*} \begingl
		\gla ha-bakbuk \textbf{nafal} \underline{be-mejomanut}//
		\glb the-bottle fell.\gsc{SMPL} in-skill//
		\glft (int.~`The bottle fell skillfully') //
		\endgl
\xe	

The Hebrew equivalent of `by itself', \emph{me-a{\ts}mo} (lit.~`from himself/itself'), diagnoses the non-existence of an external argument. The phrase is not compatible with direct objects of transitive verbs~(\nextx a) but is valid with unaccusatives~(\nextx b).
\pex
	\a \ljudge{*} \begingl
		\gla teo \textbf{taraf} et ha-laxmanja \{me-a{\ts}mo / me-a{\ts}ma\}//
		\glb Theo devoured.\gsc{SMPL} \gsc{ACC} the-bread.roll.\gsc{F} from-itself {} from-herself//
		\glft (int.~`Theo devoured the roll of its own accord')//
		\endgl
	\a \begingl
		\gla ha-bakbuk \textbf{nafal} \underline{me-a{\ts}mo}//
		\glb the-bottle fell.\gsc{SMPL} from-itself//
		\glft `The bottle fell of its own accord.'//
		\endgl
\xe


		\subsubsection{Unaccusativity diagnostics} \label{voice:tkal:nact:unacc}
The syntactic literature on Hebrew has identified two main unaccusativity diagnostics. These are verb-subject order (VS) and the possessive dative, although it is important to acknowledge that their status as robust tests has been challenged in recent years \citep{gafter14li,linzen14pd,kastner17gjgl}. I will also discuss a third diagnostic, one that is less commonly adopted, namely the episodic plural.

\textbf{The first test} is the ordering of the subject and the verb. Modern Hebrew is typically SV(O), but promoted subjects may appear after the verb, resulting in VS order. This is true for both unaccusatives and passives, presumably because the underlying object remains in its original vP-internal position. Unergatives do not allow VS, with the exception of a marked structure referred to as ''stylistic inversion''. For additional discussion see \cite{shlonsky87}, to whom the test is attributed, as well as \cite{shlonskydoron91}, \cite{borer95} and \cite{preminger10} for other aspects. Transitive configurations are only possible in this kind of inversion~(\nextx a), whereas unaccusative verbs are unmarked~(\nextx b).

\pex
	\a  \ljudge{\#}\begingl
	\gla \textbf{kaf{\ts}-u} \underline{ʃloʃa} \underline{klavim} be-ʃmone ba-boker//
	\glb jumped.\gsc{SMPL}-\gsc{3PL} three dogs in-eight in.the-morning//
	\glft `And thence jumped three dogs at 8am.' (Marked variant)//
	\endgl
	\a  \begingl
	\gla \textbf{nafl-u} \underline{ʃaloʃ} \underline{kosot} be-ʃmone ba-boker//
	\glb fell.\gsc{SMPL}-\gsc{3PL} three glasses in-eight in.the-morning//
	\glft `Three glasses fell at 8am.'//
	\endgl
\xe

Passive verbs~(\nextx a) and active verbs in other templates~(\nextx b) pattern as expected; we will return to these templates later on, but the diagnostics are consistent.
\pex
	\a \begingl
	\gla huʃlex-u ʃaloʃ kosot be-ʃmone ba-boker//
	\glb throw.\gsc{CAUS}.\gsc{PASS}-\gsc{3PL} three glasses in-eight in.the-morning//
	\glft `Three glasses were discarded at 8am.'//
	\endgl
	\a  \ljudge{\#} \begingl
	\gla \textbf{jilel-u} ʃloʃa xatulim be-ʃmone ba-boker//
	\glb whined.\gsc{INTNS}-\gsc{3PL} three cats in-eight in.the-morning//
	\glft `And thence whined three cats at 8am.' (Marked variant)//
	\endgl
\xe

\textbf{The second unaccusativity diagnostic} is the possessive dative, a construction in which the possessor appears in a prepositional phrase in a separate constituent from the possessee (possessor raising). This construction is taken to be unique to internal arguments in the language \citep{borergrodzinsky86,borer98csli}.

A transitive construction is compatible with the possessive dative~(\nextx a), as is a non-active construction~(\nextx b), whereas an unergative verb leads to an affected interpretation of the kind discussed by \cite{arieletal15} and \cite{barashersiegalboneh16}, (\nextx c). 

\pex
	\a \begingl
	\gla dana \textbf{ʃavr-a} \underline{l-i} et ha-ʃaon//
	\glb Dana broke.\gsc{SMPL}-\gsc{F.SG} to-me \gsc{ACC} the-watch//
	\glft `Dana broke my watch.'//
	\endgl
	
	\a \begingl
		\gla ha-maftexot \textbf{nafl-u} \underline{l-i}//
		\glb the-keys fell.\gsc{SMPL}-\gsc{PL} to-me//
		\glft `My keys fell.'//
	\endgl

	\a \ljudge{\#} \begingl
		\gla ha-klavim \textbf{kaf{\ts}-u} \underline{l-i}//
		\glb the-dogs jumped.\gsc{SMPL}-\gsc{PL} to-me//
		\glft `The dogs jumped and I was adversely affected.' (int.~`My dogs jumped')//
	\endgl
\xe

Typical change of state predicates can also be found as unaccusatives in this template:
\pex
	\a \begingl
		\gla ha-bakbuk \textbf{kafa} ba-makpi//
		\glb the-bottle froze.\gsc{SMPL} in.the-freezer//
		\glft `The bottle froze in the freezer.'//
	\endgl
	\a \begingl
		\gla \textbf{kafa} \underline{le-teo} ha-bakbuk//
		\glb froze.\gsc{SMPL} to-Theo the-bottle//
		\glft `Theo's bottle froze.'//
	\endgl
\xe

%\ex \begingl
%	\gla\ljudge{\#}\rightcomment{\xmark~External Argument}jilel-u l-i ʃloʃet ha-xatulim//
%	\glb whined.\gsc{INTNS}-\gsc{3PL} to-me three the-cats//
%	\glft `The three cats whined and I was adversely affected.' (int. 'My three cats whined')//
%	\endgl
%\xe

\textbf{The third diagnostic} is what I call the ``episodic plural'', proposed by \cite{borer98csli,borer05vol2}. This diagnostic tests whether a covert subject (\emph{pro} in the original formulation) is compatible with plural verbs in episodic contexts. Since this diagnostic has not been subjected to the same scrutiny as others in the literature, I will briefly sketch its strengths and weaknesses as I see them.

Hebrew can express an impersonal reading by using the plural (masculine) form of the verb. When the resulting reading is generic, the argument structure makes no difference: in~(\nextx a) an unergative is followed by a passive, and in~(\nextx b) an unergative is followed by an unaccusative. All are possible (the template does not matter for present purposes).
\pex Generic, unergative/unaccusative/passive equally acceptable \citep[86]{borer98csli}
	\a \begingl
		\gla im \textbf{mafgin-im} bli riʃajon \textbf{neesar-im} al-jedej ha-miʃtara//
		\glb if demonstrate.\gsc{CAUS.PRES}-\gsc{PL.M} without license arrest.\gsc{MID.PRES}-\gsc{PL.M} by the-police//
		\glft `For all x, if x demonstrates without a license, x is arrested by the police.'//
	\endgl
	\a \begingl
		\gla kʃe-\textbf{kofts-im} me-ha-gag \textbf{nofl-im} lemata//
		\glb when-jump.\gsc{SMPL.PRES}-\gsc{PL.M} from-the-roof fall.\gsc{MID.PRES}-\gsc{PL.M} down//
		\glft `For all x, when x jumps from the roof, x falls down.'//
	\endgl
\xe

\cite{borer98csli} notes the following contrast. In episodic contexts---unlike the generic ones in~(\lastx)---verbs with an external argument are possible~(\nextx), whereas unaccusative and passive ones are not~(\anextx). this is what I call here the episodic plural.

\pex Episodic, unergative/transitive, acceptable
	\a \begingl
		\gla (lex tiftax, ) \textbf{dofk-im} ba-delet//
		\glb go open {} knock.\gsc{SMPL.PRES}-\gsc{PL.M} in.the-door//
		\glft `(Go open up,) someone's knocking at the door.'//
	\endgl
	\a \begingl
		\gla (lex tire ma kore, ) \textbf{{\ts}oak-im} ba-xuts//
		\glb go see what happens {} yell.\gsc{SMPL.PRES}-\gsc{PL.M} in-outside//
		\glft `(Go see what's happening,) someone's yelling outside.'//
	\endgl
	\a \begingl
		\gla \textbf{hef{\ts}i{\ts}-u} et levanon ha-boker//
		\glb bomb.\gsc{CAUS}-\gsc{3PL} \gsc{ACC} Lebanon the-morning//
		\glft `Lebanon was bombed this morning.'  \trailingcitation{\citep[83]{borer98csli}}//
	\endgl
\xe

\pex Episodic, unaccusative/passive, unacceptable
	\a	\ljudge{*} \begingl
		\gla \textbf{nofl-im} / \textbf{nafl-u} ba-xatser ha-boker//
		\glb fall.\gsc{MID.PRES}-\gsc{PL.M} {} fall.\gsc{MID.PAST}-\gsc{3PL} in.the-yard the-morning//
		\glft (int.~`someone is falling in the yard this morning') \trailingcitation{\citep[85]{borer98csli}}//
	\endgl
	\a  \ljudge{*} \begingl
		\gla \textbf{mitkalkel-im} ba-geʃem axʃav//
		\glb ruin.\gsc{INTNS.MID}-\gsc{PL.M} in.the-rain now//
		\glft (int.~`things are getting ruined in the rain now')//
	\endgl
\xe
		
In his discussion of the possessive dative, \cite{gafter14li} shows that while inanimate arguments with the possessive dative are fine, animate arguments with the possessive dative are less acceptable. His critique of this diagnostic is thus based on the argument that what it diagnoses is prominence on an animacy or definiteness scale, rather than structure. Taking this work as our cue, we can try to find an animacy confound here too: animates with are fine as the reference of \emph{pro}, but what about inanimates? It turns out that non-humans are enough to trip up the test. In~(\nextx) we see examples of unergative verbs with non-human arguments; the examples are ungrammatical.

\pex Episodic, unergative, non-human, unacceptable
		\a \ljudge{*} \begingl
			\gla \textbf{metsajts-ot} ba-xu{\ts}//
			\glb chirp.\gsc{INTNS.PRES}-\gsc{PL.F} in-outside//
			\glft (int.~`some birds.\gsc{F} are chirping outside')//
		\endgl
		
		\a \ljudge{*} \begingl
			\gla \textbf{nigmer-u} axarej aruxat ha-{\ts}ohoraim//
			\glb ended.\gsc{MID}-\gsc{PL.M} after meal.of the-noon//
			\glft (int.~`we ran out of some desserts after lunch')//
		\endgl
		
		\a \ljudge{*} \begingl
			\gla \textbf{mefahak-im} ba-kalbia//
			\glb yawn.\gsc{INTNS.PRES}-\gsc{PL.M} in.the-dog.pound//
			\glft (int.~`some dogs are yawning in the dog pound')//
		\endgl
		
		\a \ljudge{*} \begingl
			\gla \textbf{mehavhev-im} ba-xu{\ts}//
			\glb flicker.\gsc{INTNS.PRES}-\gsc{PL.M} in-outside//
			\glft (int.~`some car lights are flickering outside')//
		\endgl
\xe

Similarly, I think it is possible to find unaccusative contexts with human arguments which are more acceptable, something which should not be allowed:
\pex Episodic, unaccusative, human, degraded but acceptable
	\a \ljudge{?} \begingl
		\gla (lex tire im tsarix ezra, ) \textbf{mitalf-im} ba-xu{\ts}//
		\glb go see if need help {} faint.\gsc{INTNS.MID}-\gsc{PL.M} in-outside//
		\glft `(Go see if help is needed, ) people are fainting outside.'//
	\endgl
	
	\a \ljudge{?} \begingl
		\gla (axʃav adain xaʃux aval) be-ʃmone \textbf{mitorer-im} / \textbf{ji-torer-u}//
		\glb now still dark but in-eight wake.up.\gsc{INTNS.MID}-\gsc{PL.M} {} \gsc{3M}-wake.up.\gsc{INTNS.MID}-\gsc{3PL}//
		\glft `(It's still dark now, ) but at eight o'clock everyone will wake up.'//
	\endgl
\xe

%On an even more tentative note, this construction might not be limited to verbal contexts at all. The following two examples use adjectives (keeping in mind that the ``present tense'' form is ambiguous between a nominal, adjectival and verbal reading). 
%
%
%reevim babajt o ba-misrad? mamos miSloxim maviim et ha-oxel Sel gila mamo ad elexem
%% \url{https://www.facebook.com/gilamamos/photos/%D7%A8%D7%A2%D7%91%D7%99%D7%9D-%D7%91%D7%91%D7%99%D7%AA-%D7%90%D7%95-%D7%91%D7%9E%D7%A9%D7%A8%D7%93%D7%9E%D7%9E%D7%95%D7%A1-%D7%9E%D7%A9%D7%9C%D7%95%D7%97%D7%99%D7%9D-%D7%9E%D7%91%D7%99%D7%90%D7%99%D7%9D-%D7%90%D7%AA-%D7%94%D7%90%D7%95%D7%9B%D7%9C-%D7%A9%D7%9C-%D7%92%D7%99%D7%9C%D7%94-%D7%9E%D7%9E%D7%95-%D7%A2%D7%93-%D7%90%D7%9C%D7%99%D7%9B%D7%9D%EF%B8%8F%D7%98%D7%9C%D7%A4%D7%95%D7%9F-%D7%9C%D7%9E%D7%A9%D7%9C%D7%95/894558993974695/}
%nirdamim al ha-hege? itsru le-hitraanenut
%* neesarim al-jedej ha-miStara? anaxnu jexolim laazor!

In sum, the episodic plural also has its pitfalls and is not possible in all contexts. Since I attempt to use non-human arguments in the unaccusative examples, I will set it aside until the discussion of figure reflexives and canonical reflexives in Chapter~\ref{chap:vz}.


		\subsubsection{Non-active recap}
Before proceeding, it is important to note that the unaccusativity diagnostics do not always converge for a given datapoint (see again the critiques cited above). Throughout the book, I have attempted to provide examples where more than one test diagnoses the example as unaccusative. VS order and the possessive dative often go together, and these are combined with `by itself' when possible. In addition, inanimate arguments have been chosen in order to further rule out agentive readings of events, although not for all examples. But even with all these caveats, it is still fairly easy to see that there is no morphological difference between transitives, unergatives, ditransitives and unaccusatives in {\tkal}.



	\subsection{Summary}

Recall that Hebrew templates can be viewed through two lenses: the configurations they are compatible with, and their canonical alternations with other templates. The generalization about verbs in {\tkal} is a negative one: there are no syntactic constraints on the kind of verb that appears in this template. For this reason, \cite{doron03} does not associate it with any specific functional heads and \cite{borer13oup,borer15roots} treats it as a verbalized root with no additional syntactic functors. Alternations will be discussed once we engage with the other templates of the language. Generalizations regarding the two questions will be summarized as in~(\nextx).

\hammer{
\pex \label{ex:gen-tkal}\textbf{Generalizations about {\tkal}}
	\a \textbf{Configurations} Verbs appear in all possible argument structure configurations.
	\a \textbf{Alternations:} Participates in alternations with the other templates, as will be reviewed throughout the book.
\xe
}

I look into the patterns of {\tkal} in more depth in Section~\ref{voice:voice}, where I situate them within the current theory of Voice.

\section{Unspecified Voice} \label{voice:voice}
This monograph promotes a theory of argument structure in which Voice can have one of three values: [\!+\!D], [--D] or unspecified for [$\pm$D]. As foreshadowed in the introductory chapter, the idea is that {\vd} requires an external argument and {\vz} prohibits one. We will now focus on what it means for Voice not to have a preference on the matter, thereby accounting for the patterns in Section~\ref{voice:tkal}.

First, let me define Unspecified Voice in (\nextx). All definitions of Voice heads in this book take the same form: (a) syntactic definition, (b) semantic denotation, and (c) basic spell-out rules. I give these here and expand upon them in turn.
\pex \textbf{Unspecified Voice}
	\a A Voice head with no specification for a [D] feature. It has no requirements regarding whether its specifier must be filled. In transitive verbs, Voice is the locus of accusative case assignment, either itself by feature checking \citep{chomsky95} or through the calculation of dependent case \citep{marantz91}.
	\a \denote{Voice} = $\begin{cases}
		\lambda P.P & \text{/ \trace~ \{ \root{npl} `\root{\gsc{FALL}}', \root{kpa} `\root{\gsc{FREEZE}}' , \dots \} }\\
		\lambda x \lambda e.Agent(x,e) & \\
		\end{cases}$

%	\denote{Voice} = $\begin{cases}
%		\lambda x \lambda e.Agent(x,e) & \text{/ \trace~\{ \root{axl} `\root{\gsc{EAT}}', \root{ktb} `\root{\gsc{WRITE}}', \root{ntn} `\root{\gsc{GIVE}}',}\\
%			& \text{\root{ʃal} `\root{\gsc{BORROW}}', \root{\gsc{r\dgs{k}d}} `\root{\gsc{dance}}', \root{\gsc{hlx}} `\root{\gsc{WALK}}', \dots\} }\\
%		\lambda e.e & \text{/ \trace~ \{ \root{npl} `\root{\gsc{FALL}}', \root{kpa} `\root{\gsc{FREEZE}}' , \dots \} }
%		\end{cases}$
	\a Voice \lra~{\tkal} \hfill  (with the allomorph {\tpie} to follow in Section \ref{voice:va})
\xe

		\subsection{Syntax} \label{voice:voice:syn}
The view of argument and event structure adopted here (see Chapter~\ref{intro:sketch}) builds up the verbal domain in ``layers''. Taking the root \root{trf}, which has to do with devouring, we first build up a verb by adjoining the root to the verbal category head v, and then merge the DP required as an internal argument. This gives us a function over events of devouring that DP. Adding the traditional Voice head would do little to change the event but would add an agent role to the semantics. The current Voice head is slightly different.
\ex
\Tree
[.{v\\$\lambda$x$\lambda$e.devour(e) \& Theme(x,e)}
	[.\root{trf} ]
	[.v ]
]
$\Rightarrow$
\Tree
[.{vP\\$\lambda$e.devour(e) \& Theme(x,DP_{1})}
	[.{v\\$\lambda$x$\lambda$e.devour(e) \& Theme(x,e)}
		[.\root{trf} ]
		[.v ]
	]
	[.DP_{1} ]
]
$\Rightarrow$

\Tree
[.{VoiceP\\$\lambda$e.devour(e) \& Theme(DP_{1},e) \& Agent(DP_{2},e)}
	[.DP_{2} ]
	[.
		[.{Voice\\$\lambda$x$\lambda$e.Agent(x,e)} ]
		[.{vP\\$\lambda$e.devour(e) \& Theme(x,DP)}
			[.{v\\$\lambda$x$\lambda$e.devour(e) \& Theme(x,e)}
				[.\root{trf} ]
				[.v ]
			]
			[.DP_{1} ]
		]
	]
]
\xe

In the current system, the lack of a feature on Voice means that the head is not specified for any syntactic feature constraining Spec,VoiceP. That position can be filled or left unprojected, as far as the Voice head is concerned. In this state of affairs, the expectation is that differences between verbs will result from the requirements of individual roots, rather than anything in the structure. In other words, some roots will give rise to transitive verbs, other roots to unaccusative verbs, and so on.

This is exactly what we have seen in the template {\tkal}. There are no structural restrictions on argument structure in this template: verbs in {\tkal} might be transitive, unergative, ditransitive or unaccusative. Some of the examples from Section~\ref{voice:tkal} are repeated below with minimal syntactic structures (leaving out material above VoiceP, such as Tense).

In~(\nextx) we see the core transitive verb \emph{taraf}, which requires an internal argument. The accusative/DOM marker \emph{et} must also appear, indicating that this is a transitive construction.
\pex\label{ex:voice-intro-tr2}
	\a 
	\begingl
		\gla teo \textbf{taraf} *(et ha-laxmanja)//
		\glb Theo devoured \gsc{ACC} the-bread.roll//
		\glft `Theo devoured the bread roll.'//
	\endgl
	\a \Tree
	[.VoiceP
		[.\emph{teo} ]
		[.
			[.Voice ]
			[.vP
				[.v
					[.\root{trf} ]
					[.v ]
				]
				[.DP\\\emph{et ha-laxmanja} ]
			]
		]
	]
\xe

Unergative verbs are also possible, as with \emph{rakad} `danced' in~(\nextx). No internal argument is necessary, the event is an activity which can go on over a certain period of time with no concrete telos, and agent-oriented adverbs are possible.
\pex\label{ex:voice-intro-unerg2}
	\a \begingl
		\gla teo \textbf{rakad} ve-rakad ve-rakad (be-mejomanut) (kol ha-boker)//
		\glb Theo danced and-danced and-danced in-skill all the-morning//
		\glft `Theo danced and danced and danced (skillfully) (all morning long).'//
	\endgl
	\a \Tree
	[.VoiceP
		[.\emph{teo} ]
		[.
			[.Voice ]
			[.vP
				[.v
					[.\root{rkd} ]
					[.v ]
				]
			]
		]
	]
\xe

Ditransitive verbs are also possible, as in~(\nextx). We do not need to commit to any specific analysis of ditransitive verbs, so I give a general structure headed by low Appl \citep[18]{pylkkanen08}.\footnote{In Chapter~\ref{vz:pz} we switch to a specific implementation of PP arguments using \emph{p}, a PP-licenser \citep{koopman97,svenonius03,gehrke08phd,wood15springer}.}
\pex\label{ex:voice-intro-ditr2}
	\a \begingl
		\gla teo \textbf{natan} *(le-marsel) et ha-xatif //
		\glb Theo gave to-Marcel \gsc{ACC} the-snack//
		\glft `Theo gave Marcel the treat.'//
	\endgl
	\a \Tree
	[.VoiceP
		[.\emph{teo} ]
		[.
			[.Voice ]
			[.vP
				[.v
					[.\root{ntn} ]
					[.v ]
				]
				[.ApplP
					[.PP\\\emph{le-marsel} ]
					[.
						[.Appl ]					
						[.DP\\\emph{et ha-xatif} ]
					]
				]
			]
		]
	]
\xe

Lastly, unaccusative verbs are also possible. The two traditional diagnostics are fronting of the verb and the possibility of using a possessive dative, both evident in~(\nextx). I return to discussing these diagnostics in some more depth when we focus on unaccusative verbs in Chapter~\ref{vz:tnif:nact:unacc}. The tree in~(\anextx b) does not present the final word order, on which see \cite{preminger10}.
\pex\label{ex:voice-intro-unacc2}
	\a \begingl
		\gla \textbf{nafal} le-teo ha-bakbuk//
		\glb fell to-Theo the-bottle//
		\glft `Theo's bottle fell.'//
	\endgl
	\a \Tree
	[.VoiceP
		[.Voice ]
		[.ApplP/\emph{p}P
			[.PP\\\emph{le-teo} ]
			[.
				[.Appl/\emph{p} ]	
				[.vP
					[.v
						[.\root{nfl} ]
						[.v ]
					]
					[.DP\\\emph{he-bakbuk} ]
				]
			]
		]
	]
\xe		

This is how Unspecified Voice captures the underspecified nature of the template {\tkal. Since there are no restrictions in the syntax, the root is free to require any interpretation from v and Voice (save for reflexive readings, which are discussed in Chapter~\ref{vz:va:vzva}). The question does arise of what exactly the status of Merge is in such a system, a point of discussion I postpone till Chapter~\ref{chap:i}.

%% check database for breakdown by type
		
		\subsection{Semantics} \label{voice:voice:sem}
The underspecification of this head---and of the resulting template---can be implemented in the semantics using contextual allosemy of Voice. As explained in Chapter~\ref{intro:arch}, the meaning of a functional head can depend on the syntactic and semantic context it appears in, a situation of conditioned allosemy. This formal mechanism allows us to state which meanings arise in which contexts.

Assuming that the active variant is the elsewhere case, certain roots will be said to require a non-active alloseme of Voice~(\nextx a) and others will be compatible with agentive verbs~(\nextx b):\footnote{Chapter~\ref{chap:aas} contains a brief comparison of contextual allosemy with one alternative, namely postulating homophonous heads.}
\pex \denote{Voice} = 
	\a $\lambda$P.P \phantom{agent(x,e)xxx} / \trace~ \{ \root{npl} `\root{\gsc{FALL}}', \root{kpa} `\root{\gsc{FREEZE}}' , \dots \}
	\a $\lambda$x$\lambda$e.Agent(x,e)
%		 & \text{/ \trace~\{ \root{trf} `\root{\gsc{DEVOUR}}', \root{ktb} `\root{\gsc{WRITE}}', \root{ntn} `\root{\gsc{GIVE}}',}\\
%			& \text{\root{ʃal} `\root{\gsc{BORROW}}', \root{\gsc{r\dgs{k}d}} `\root{\gsc{dance}}', \root{\gsc{hlx}} `\root{\gsc{WALK}}', \dots\} }\\
\xe

Other allosemes are also possible, as when \cite{kratzer96}---and in the current formalism, \cite{woodmarantz17}---suggest that Voice can introduce either the Agent or Holder role, depending on the vP it combines with. While the syntax and semantics are flexible and root-specific, the phonology is consistent, uniquely identifying this head for learner and analyst alike.
		
		\subsection{Phonology} \label{voice:voice:phono}
The basic paradigm looks as in~(\nextx); for more examples see \cite{schwarzwald08}, \cite{faust12} or \cite{kastner18nllt}.\footnote{Some verbs in this template are phonologically marked. Verbal stems are normally longer than one syllable, except for some in {\tkal} which flout this restriction:
\ex \emph{ba} `came', \emph{ʃav} `returned', \emph{tsats} `appeared'.
\xe
The intuition is that if there are no overt affixes, the root will have free reign in the phonology. See discussion of such verbs and their roots in \cite{ussishkin05}, \cite{laks11}, \cite{borer13oup}, \cite{borer15roots}, \cite{tucker15roots} and \cite{kastner18nllt}.}

\ex
\raisebox{-4.5em}{
\begin{small}
	\begin{tabular}{|l||l|l||l|l||l|l|} \hline
		& \multicolumn{2}{c||}{Past} & \multicolumn{2}{c||}{Present} &  \multicolumn{2}{c|}{Future} \\\hline
		& \gsc{M} & \gsc{F} & \gsc{M} & \gsc{F} & \gsc{M} & \gsc{F} \\\hline\hline
		1\gsc{SG} & \multicolumn{2}{c||}{XaYaZ-ti} & XoYeZ & XoYeZ-et & \multicolumn{2}{c|}{e-XYoZ/ji-XYoZ}\\\hline
		1\gsc{PL} & \multicolumn{2}{c||}{XaYaZ-nu} & XoYZ-im & XoYZ-ot & \multicolumn{2}{c|}{ni-XYoZ}  \\\hline\hline
		2\gsc{SG} & XaYaZ-ta & XaYaZ-t & XoYeZ & XoYeZ-et & ti-XYoZ & ti-XYeZ-i\\\hline
		2\gsc{PL} & XaYaZ-tem & XaYaZ-ten/tem & XoYZ-im & XoYZ-ot & \multicolumn{2}{c|}{ti-XYeZ-u}\\\hline\hline
		3\gsc{SG} & XaYaZ & XaYZ-a & XoYeZ & XoYeZ-et & ji-XYoZ & ti-XYoZ\\\hline
		3\gsc{PL} & \multicolumn{2}{c||}{XaYZ-u} & XoYZ-im & XoYZ-ot & \multicolumn{2}{c|}{ji-XYeZ-u}\\\hline
	\end{tabular}
\end{small}
}
\xe

What I assume throughout is that the stem vowels spell out Voice and that affixes spell out T+Agr (Chapter~\ref{intro:arch:inter}). Since Voice is local to T+Agr, T+Agr can condition allomorphy on the vowels, symbolized by the dotted arrow in~(\ref{tree:locality1}). As a result, different phi-feature values condition different stem vowels as in~(\lastx). This aspect of the theory is based on \cite{katie13} and explored more fully in \cite{kastner18nllt}.
\pex \label{tree:locality1}
\Tree
    [.TP
        [.\tikz{\node (TAgr) {T+Agr};} ]
        [
            [.\tikz{\node (Voice) {Voice};} ]
            [.vP
            	[.v
            		[.\root{root} ]
            		[.v ]
            	]
            	[.(DP) ]
            ]
         ]
     ]
%                [.v\\(covert) ]
%                [.\tikz{\node (Root) {\root{root}};} ]
%            ]
%        ]
%    ]
    \begin{tikzpicture}[overlay]
    %\draw[semithick,->] (TAgr) to[out=south west,in=south west] (Voice);
%    \draw[dotted,thick,->] (Root) .. controls +(south:1) and +(south:2) .. (Voice);
    %\draw[dotted,thick,->] (TAgr) to[out=225,in=270] (Voice);
    \draw[dotted,thick,->] (TAgr) .. controls +(south west:1) and +(south west:1) .. (Voice);
    \end{tikzpicture}
\xe % \vskip 1em

The relevant Vocabulary Items for two verbs, \emph{taraf} `devoured' and \emph{katav} `wrote', are given in~(\nextx). The verbalizer v is silent by hypothesis. The final /b/ of \root{ktb} spirantizes to [v], a productive process in the language \citep{temkinmartinzemuellner16,kastner17gjgl,kastner18nllt}, yielding /katab/ $\rightarrow$ [katav].
\pex \emph{taraf} `devoured', \emph{katav} `wrote':
	\a \root{trf} \lra~\emph{trf}
	\a \root{ktb} \lra~\emph{ktb}
	\a v \lra~(covert)
	\a Voice \lra~\emph{a,a} / T[Past] \trace
\xe

Various other processes might apply, too. Next we will see derivations with the \gsc{3SG.F} suffix -\emph{a} as well as a process of syncope, in which a vowel is deleted (annotated \del{\emph{a}}). Recall that spell-out proceeds cyclically, first within the VoiceP domain and then within the TP domain. First, \emph{tarfa} `she wrote':
\ex Cycle 1, Syntax:\\
	 T[Past, 3\gsc{SG.F}]-Voice-\root{trf}
\xe
\pex Cycle 1, VIs:
	\a \root{trf} \lra~\emph{trf}
	\a Voice \lra~\emph{a,a} / T[Past] \trace
\xe
\pex Cycle 1, Phonology:
	\a \emph{a,a}-\emph{trf}
	\a /taraf/ $\Rightarrow$ \emph{taraf}
\xe

\ex Cycle 2, Syntax:\\
	T[Past, \gsc{3SG.F}]-\emph{taraf}
\xe
\ex Cycle 2, VIs:\\
	\gsc{3SG.F} \lra~\emph{a} / Past \trace
\xe
\pex Cycle 2, Phonology:
	\a \emph{a}-\emph{taraf}
	\a /a-taraf/ $\Rightarrow$ /tar\del{a}f-a/ $\Rightarrow$ \emph{tarfa}
\xe

And now \emph{katva} `she wrote':
\ex Cycle 1, Syntax:\\
	T[Past, 3\gsc{SG.F}]-Voice-\root{ktb}
\xe
\pex Cycle 1, VIs:
	\a \root{ktb} \lra~\emph{ktb}
	\a Voice \lra~\emph{a,a} / T[Past] \trace
\xe
\pex Cycle 1, Phonology:
	\a \emph{a,a}-\emph{ktb}
	\a	/katab/ $\Rightarrow$ /katav/ $\Rightarrow$ \emph{katav}.
\xe

\ex Cycle 2, Syntax:\\
	T[Past, \gsc{3SG.F}]-\emph{katav}
\xe
\ex Cycle 2, VIs:\\
	\gsc{3SG.F} \lra~\emph{a} / Past \trace
\xe
\pex Cycle 2, Phonology:
	\a \emph{a}-\emph{katav}
	\a /a-katav/ $\Rightarrow$ /kat\del{a}v-a/ $\Rightarrow$ \emph{katva}
\xe

How exactly these exponents are concatenated will not be derived here; in \cite{kastner18nllt} I give full derivations within an OT grammar. Importantly, the derivation proceeds modularly and cyclically: first the syntax builds up structure, then VI inserts exponents, then the phonology takes over and derives the most harmonic surface forms. But for future tense forms like \emph{ti-xtov} `she will write', we will require a different contextual allomorph for Voice such as that in~(\nextx b).
\ex \label{vi:voice} Voice \lra $\begin{cases}
		\text{a.~\emph{a},\emph{a}} & \text{/ T[Past] \trace}\\
		\text{b.~\emph{o}} & \text{/ T[Fut] \trace}\\
		\end{cases}$
\xe

Abstracting away from the spell-out of specific inflectional variants within a given template, a general schematic can be stated as in~(\nextx b). In Section~\ref{voice:va} below I introduce a modifier which constrains both the semantics and phonology of Voice, giving us the possibility of~(\nextx a).
\pex Voice {\lra}
	\a {\tpie} / {\trace} {\va}
	\a {\tkal}
\xe

The generalized spell-out rules in~(\lastx) provide only a crude approximation of how Voice is handled at PF, but it is important to keep in mind that there is no one ``suffix'' {\tkal}. Rather, there is an intricate morphophonological system of inflectional variants which needs to be taken into account. With that in mind, my focus in this book will be more in setting up basic schemas like those in~(\lastx), whereby different syntactic configurations---mostly reflecting different values of Voice---trigger different templatic shapes. The templates themselves, then, have no independent status in the theory and serve only as useful morphophonological mnemonics.

	\subsection{Interim summary}
The template {\tkal} is unrestricted in terms of argument structure: verbs with this morphological marking might be unergative, unaccusative, monotransitive or ditransitive, all depending (idiosyncratically) on the underling root. Yet the morphophonology is consistent across all possible verbs in this template, regardless of their syntax and semantics.

In contrast to the traditional Voice head which introduces an external argument, the Voice head I use to capture this behavior is unspecified with regards to the EPP feature [D]. This head does not place any constraints on its specifier. As a result, there are no restrictions on the argument structure of verbs which are derived using Unspecified Voice. Since every Hebrew verb must be instantiated in one of the seven verbal templates, the appearance of Voice can be traced in the morphology as the template {\tkal} (all Hebrew verbs require Voice by assumption; \citealt{arad05}).

In other frameworks, \cite{doron03} does not introduce any special heads in order to account for verbs in {\tkal}. \cite{borer13oup,borer15roots} takes {\tkal} to be a verbalized root without functional material attaching to it. The two main reasons for this are the wide range of nominalizations possible in this template and the idiosyncratic phonology. I will return to nominalizations in Chapter~\ref{passn:n}, after covering the other variants of Voice, but all three frameworks are compatible in their treatment of the {\tkal}: all allow for {\tkal} to be as idiosyncratic as it needs to be.

\section{\tpie: Descriptive generalizations} \label{voice:tpie}
The next template to be examined is {\tpie}. As can be seen from the notation, there are no unique affixes to this template, but the stem vowels are different than in~{\tkal}. In addition, the middle root consonant Y blocks the process of spirantization mentioned briefly earlier. I borrow the non-syllabic diacritic \dgs{Y} to indicate this.

In this section I lay out the basics of verbs in {\tpie}, basically supporting the generalizations established by \cite{doron03}. In terms of possible constructions, verbs in this template are always active, and what's more, they are agentive in a weak sense which I will identify informally. In terms of alternations, they sometimes provide ``intensive'' alternants of verbs in {\tkal}, again in a way I will explain below. This secion provides an overview of the data; the next section gives a formal analysis, based on the head Unspecified Voice we have just seen and an agentive modifier, {\va}.

First, let me reestablish the terminology used here. I take \emph{causes} to be any kind of external argument. \emph{Agents} are a subset of causers, typically understood as animate and volitional causers. In the discussion below, ``Agent'' will be used more or less interchangeably with ``actor'', ``direct cause'', and the other labels used in the literature. So, throughout this book, when I say \emph{Agent} what I mean is a stronger type of causer, a distinction which as far as I can see is vague precisely because it is rooted in the semantics of various kinds of events rather than in syntactic features. The discussion which follows should make these distinctions clear.

To understand the syntax-semantics of {\tpie}, consider the pairs in~(\ref{ex:va-piel1}). In~(\nextx a), both Agents and Causes are possible. In~(\ref{ex:va-piel1}b) only the Agent is possible. The (a) example has the verb in {\tkal}, the (b) example in {\tpie}.

\pex \label{ex:va-piel1}
	\a \begingl
		\gla \emph{\{}\cmark~ha-jeladim / \cmark~ha-tiltulim ba-argaz\emph{\}} \glemph{ʃavr}-u et ha-kosot//
		\glb \phantom{\{\cmark~}the-children {} \phantom{\cmark~}the-shaking in.the-box \glemph{broke.\gsc{SMPL}}-\gsc{PL} \gsc{ACC} the-glasses//
		\glft `\{The children / Shaking around in the box\} broke the glasses.'//
		\endgl
	
	\a \begingl
		\gla \emph{\{}\cmark~ha-jeladim / \xmark~ha-tiltulim ba-argaz\emph{\}} \glemph{ʃibr}-u et ha-kosot//
		\glb \phantom{\{\cmark~}the-children {} \phantom{\xmark~}the-shaking in.the-box \glemph{broke.\gsc{INTNS}}-\gsc{PL} \gsc{ACC} the-glasses//
		\glft `\{The children / *Shaking around in the box\} broke the glasses to bits.' \trailingcitation{\citep[20]{doron03}}//
		\endgl
\xe

%\pex \label{ex:va-piel2}
%	\a \begingl
%		\gla ha-xelbon \glemph{jatsar} ba-guf nogdanim//
%		\glb the protein produced.\gsc{SMPL} in the body antibodies//
%		\glft `The protein produced antibodies in the body.'//
%	\endgl
%	
%	\a \begingl 
%	\gla ha-xelbon \glemphf{jitser} ba-guf nogdanim//
%	\glb the protein produced.\gsc{INTNS} in the body antibodies//
%	\glft `The protein manufactured antibodies in the body.' \trailingcitation{\citep[21]{doron03}}//
%	\endgl
%\xe

What other readings do verbs in {\tpie} have? This template is traditionally called the ``intensive'' because of alternations such as those above and in~(\nextx a--c), but it can also house pluractional verbs, (\nextx d--f), and various others, (\nextx g--i):
\ex\label{ex:voice:piel-meanings}Pretheoretical classification of some verbs in \tpie:\\
	\begin{tabular}{lll|ll|ll}
	& & & \multicolumn{2}{c|}{\tkal} &  \multicolumn{2}{c}{\tpie}\\\hline
	\multirow{3}{*}{Intensive} & a.& \root{ʃbr} & ʃavar & `broke' & ʃiber & `broke to pieces'\\
		& b.& \root{jtsr} & jatsar & `produced' & jitser & `produced'\\
	    & c.& \root{'kl} & axal & `ate' & ikel & `corroded, consumed'\\\hline

 	\multirow{3}{*}{Pluractional} & d.& \root{hlx} & halax & `walked' & hilex & `walked around'\\
 	    & e.& \root{r\dgs{k}d} & rakad & `danced' & riked & `danced around'\\
  	    & f.& \root{\dgs{k}fts} & kafats & `jumped' & kipets/kiftsets & `jumped around'\\\hline

  		\multirow{3}{*}{Non-derived} & g. & \root{tps} & \multicolumn{2}{c|}{---} & tipes & `climbed'\\
	    & h. & \root{ltf} & \multicolumn{2}{c|}{---} & litef & `petted'\\
		  & i. & \root{\dgs{k}bl} & \multicolumn{2}{c|}{---} & kibel & `received'\\
	\end{tabular}
\xe

In all cases, the verbs are active: either unergative or transitive. And in all cases, the external argument is agentive. In some examples this contrast is clear: a storm cannot ``intensively'' break a window to bits.
\pex
	\a \begingl
		\gla ha-sufa \glemph{ʃavr-a} et ha-xalon//
		\glb the-storm broke.\gsc{SMPL}-\gsc{F} \gsc{ACC} the-window//
		\glft `The storm broke the window.'//
		\endgl
	\a \begingl
		\gla ha-jeladim \textbf{ʃivr-u} et ha-xalon le-xatixot \underline{be-xavana}//
		\glb the-children broke.\gsc{INTNS}-\gsc{PL} the-window to-pieces in-purpose//
		\glft `The children broke the window to bits on purpose.'//
		\endgl
	\a \begingl
		\gla\ljudge{*}ha-sufa \glemph{ʃibr-a} et ha-xalon (le-xatixot)//
		\glb the-storm broke.\gsc{INTNS}-\gsc{F} \gsc{ACC} the-window to-pieces//
		\glft (int.~`The storm broke the window to pieces')//
		\endgl
\xe

But as \citet[21]{doron03} points out, even inanimate entities can be the subjects of verbs in {\tpie}. She gives the following pair of examples. As she put it: ``\textit{The simple verb \emph{produce} in \emph{[(\nextx a)]} has a reading where the protein is the trigger for antibodies being produced. The intensive-template verb in \emph{[(\nextx b)]} can only be interpreted such that the protein actually participates in the production process itself}.''
\pex
	\a \begingl
		\gla ha-xelbon \glemph{jatsar} ba-guf nogdanim//
		\glb the-protein produced.\gsc{SMPL} in.the-body antibodies//
		\glft `The protein produced antibodies in the body.'//
		\endgl
	\a \begingl
		\gla ha-xelbon \glemph{jitser} ba-guf nogdanim (*be-xavana)//
		\glb the-protein produced.\gsc{INTNS} in.the-body antibodies in-purpose//
		\glft `The protein produced antibodies in the body (*on purpose).'//
		\endgl
\xe

The generalizations for {\tpie}, then, are as follows:
\hammer{
\pex \label{ex:gen-tpie}\textbf{Generalizations about {\tpie}}
	\a \textbf{Configurations:} Verbs appear in active (transitive/unergative) configurations.\\
		Readings are weakly agentive.
	\a \textbf{Alternations:} When alternating with {\tkal}, provides a more ``intensive'' or agentive version.
\xe
}

Making reference to ``weak agentivity'' and ``intensive'' readings is a fine semantic line to tread. In what follows I review what I think are some similar phenomena across languages and empirical domains, before turning to the formal analysis.

	\subsection{Agentive modifiers crosslinguistically} \label{voice:tpiel:act}

		\subsubsection{Agentivity $\ne$ animacy}
%As far as the semantics is concerned, the difference in possible interpretations between~(\lastx a) and~(\lastx b) reduces to whether or not overt {\va} is there to force an agentive reading. \cite{doron03} proposed that this modifier carries the semantics of Action, which is slightly weaker than that of Agent.
A number of recent works on argument and event structure have identified a component of meaning that can be broadly described as agentive, volitional, or a ``direct cause''. The most straightforward view of agentivity equates it with animacy. For example, Italian \emph{fare}-causatives require the causee to be animate, as in~(\nextx). Similar considerations are familiar from control phenomena as discussed in a range of work from \cite{farkas88} to \cite{zu18phd}.
\pex
	\a \begingl
		\gla Gianni ha fatto rompere la finestra a Maria//
		\glb John has made break the window to Maria//
		\glft `John had Maria break the window.'//
		\endgl
	\a \begingl
		\gla \ljudge{\#}Gianni ha fatto rompere la finestra al ramo//
		\glb John has made break the window to.the branch//
		\glft (int.~`John had the branch break the window)'\trailingcitation{\citep[196]{folliharley08}}//
		\endgl
\xe	

In their study of animacy in English, Italian, Greek and Russian, \cite{folliharley08} considered a range of data in which the acceptability of an external argument depends on whether it is \emph{teleologically capable} of causing the event (as opposed to an agency or animacy restriction). Even though animacy is the relevant factor within the teleological capability of the relevant argument in many cases, \cite{folliharley08} identified cases of sound emission, possession, causation, permission and consumption where the licensing conditions on external arguments cannot be understood in terms of animacy, but in terms of whether the internal properties of the external argument can bring about the relevant event.

For example, in Italian causatives without \emph{fare}, inanimate causers vary with respect to how acceptable they are. A branch is fine, but a storm is not. The explanation is that the branch is a direct causer but the storm is not a proximate enough cause; it is not teleologically capable.\footnote{See \cite{irwin19tlr} for an explication of some teleological properties in terms of body parts.}
\pex
	\a \begingl
		\gla Il ramo ha rotto la finestra//
		\glb the branch has broken the window//
		\glft `The branch broke the window.'//
		\endgl
	\a \begingl
		\gla \ljudge{?}Il vento ha rotto la finestra//
		\glb the wind has broken the window//
		\glft `The wind broke the window.'//
		\endgl
	\a \begingl
		\gla \ljudge{\#}Il temporale ha rotto la finestra//
		\glb the storm has broken the window//
		\glft (int.~`The storm broke the window')\trailingcitation{(Italian, \citealt[195]{folliharley08})}//
		\endgl
\xe		

A further dissociation of animacy from agentivity (in the current sense) comes from a study of manner and causation in English by \cite{beaverskoontzgarboden12}, who showed that an animate cause is still not necessarily an agent. The term they use is \emph{actor}, employed to discuss events in which an animate causer is or is not responsible for the consequences of its act. For them, causation is compatible with negligence but actorhood (agentivity) is not. That is why even the animate causer in~(\nextx) is not an actor (cf.~\citealt{rappaporthovav14}) :
\ex Kim broke my DVD player, but didn’t move a muscle—rather, when I let her borrow it a disc was spinning in it, and she just let it run until the rotor gave out!\trailingcitation{\citep[347]{beaverskoontzgarboden12}}
\xe

		\subsubsection{Agentivity in nominalizations}
What I would like to highlight next is that these kinds of readings can also be triggered by particular morphemes. Moving on to a different empirical domain, recent studies of external arguments in nominalization \citep{sichel10n,alexiadouetal13,ahdout18nom} similarly differentiate agentivity from \emph{direct causation}. The external arguments of Complex Event Nominals are often taken to exhibit \emph{agent exclusivity}, whereby only agents are possible. Examples~(\nextx)--(\anextx) show a typical instantiation of this effect: the animate Agent can serve as the external argument of a nominalization, (\nextx), but an inanimate Cause cannot, (\anextx).
\pex
	\a \textbf{The Allies} separated East and West Germany.
	\a \textbf{The Allies'} separation of East and West Germany
\xe
\pex
	\a \textbf{The cold war} separated East and West Germany.
	\a \ljudge{\#}\textbf{The cold war's} separation of East and West Germany
\xe

\cite{sichel10n} points out, however, that animacy is not always the relevant factor, as observed already in different ways by \cite{pesetsky95} and \cite{marantz97}. The core of her argument is based on natural causers, which are compatible with some nominalizations but not with others (the following judgments are hers). She takes this to mean that direct \emph{causation} is insufficient if it lacks direct \emph{participation}.
\pex
	\a The hurricane's \textbf{destruction} of our crops
	\a The hurricane's \textbf{devastation} of ten coastal communities in Nicaragua
\xe
\ex \ljudge{\#} The approaching hurricane's \textbf{justification} of the abrupt evacuation of the inhabitants
\xe

\cite{alexiadouetal13,alexiadouetal13jcgl} build on \citeauthor{sichel10n}'s proposal and propose that depending on the language and construction, the restriction can depend on either agentivity or direct participation.

		\subsubsection{Agentive morphemes}
Syntactic environments other than nominalization can give rise to similar effects. There are cases where a specific, overt morpheme can be identified as triggering these agentivity-like effects. In Hebrew itself, the external arguments of passive verbs can only be Agents, not Causes \citep{doron03}. I mention two more cases from other langauges here, before we return to a similar phenomenon in Hebrew which I attribute to the element {\va}.

In their studies of the prefix \emph{afto-} in Greek, \cite{alexiadouafto} and \cite{spathasetal15} identified it as an \emph{anti-assistive} modifier, triggering agentive readings regardless of syntactic category, (\nextx).
\pex Agentive readings of \emph{afto-} \citep[61]{alexiadouafto}:
	\a \emph{afto-katastrefome} `self-destroy' (v.)
	\a \emph{afto-kritiki} `self-criticism' (n.)
	\a \emph{afto-didaktos} `self-educated' (a.)
\xe

Given its meaning and its similarity to an analytic paraphrase, (\nextx), \cite{spathasetal15} propose the denotation in~(\anextx).
\pex
	\a \begingl
		\gla O Janis katigori-\textbf{te}//
		\glb the John accuses-\gsc{NACT}//
		\glft `John is accused.'//
	\endgl
	\a \begingl
		\gla O janis katigori \textbf{ton} \textbf{eafto} \textbf{tu}//
		\glb the John accuses the self his//
		\glft `John accuses himself.'//
	\endgl
	\a \begingl
		\gla O janis \textbf{afto}-katigori-\textbf{te}//
		\glb the John self-accuse-\gsc{NACT}//
		\glft `John accuses himself.' \trailingcitation{\citep[63--64]{alexiadouafto}}//
	\endgl
\xe
\ex \denote{\emph{afto}$_{\text{anti-assistive}}$} = $\lambda$f$\lambda$y$\lambda$e.f(y,e) \& $\forall$e'$\forall$x.(e'$\le$e \& Agent(x,e')) $\rightarrow$ x=y \trailingcitation{\citep[1335]{spathasetal15}}
\xe

Additional elaboration on these complex constructions can be found in these works and the previous works they cite. The technical conclusion is that \emph{afto-} is an adjunct which attaches to Voice, triggering agentive meaning.

A comparable (although still distinct) phenomenon can be found in Tamil, were the suffix -\emph{koɭ} adds ``affective semantics'' which are otherwise hard to pin down. \cite{sundaresanmcfadden17} discuss the difference in meaning between verbs with and without -\emph{koɭ} as one of ``affectedness'' in a way that can be exemplified using the data in~(\nextx). With -\emph{koɭ}, the event affects the agent.
\pex
	\a \begingl
	\gla Mansi paal- æ uutt- in- aaɭ//
	\glb Mansi milk \gsc{ACC} pour.\gsc{TR} Past \gsc{3SG.F}//
	\glft `Mansi poured the milk.'//
	\endgl

	\a 	\begingl
	\gla Mansi paal- æ uutti- \textbf{kko-} ɳɖ- aaɭ//
	\glb Mansi milk \gsc{ACC} pour.\gsc{TR} \gsc{koɭ} Past \gsc{3SG.F}//
	\glft `Mansi poured the milk for herself.' (Reading 1)\\
		`Mansi poured the milk on herself.' (Reading 2)//
	\endgl
\xe

As \citet[165]{sundaresanmcfadden17} put it, ``\emph{the end result of some event comes back to affect one of the arguments of that same event}'', where the relevant argument is the external argument if there is one, otherwise the internal one (as with unaccusatives). In any case, the semantics of -\emph{koɭ} are such that it forces some kind of agent-oriented reading at least in clauses with external arguments.

Where does this cross-linguistic review leave us? The pretheoretical picture which emerges from these works is that natural language has a way of making a fine-grained distinction between different degrees of ``direct participation'' or agentivity. To the extent that this triggering of agentive semantics is the same phenomenon across languages, it seems highly unlikely that it has the same syntactic underpinnings in all of these cases. A more appropriate explanation would be given in semantic terms (that is, within the denotation of certain morphemes) or in pragmatic terms (world knowledge). As alluded to above, it seems clear that in at least some cases the effect is clearly grammatical, i.e.~should be encoded in the semantics of individual morphemes directly, as with agent exclusivity in nominalizations, the anti-assistive modifier in Greek and the affective modifier in Tamil. Such a proposal for Hebrew follows.

\section{Agentive modification: \va} \label{voice:va}
In this section I introduce another syntactic primitive, the agentive modifier {\va}. Strictly speaking, this modifier is not part of the theory of trivalent Voice. The reason it is introduced early on in this book is because it is necessary to capture the full empirical picture; specifically, it will return in the discussion of {\vz} in Chapter~\ref{chap:vz}. Unspecified Voice and the template {\tkal} have already been addressed, but the behavior of the template {\tpie} indicates that we need to account for additional forms.

In order to explain the behavior of verbs in {\tpie} I propose to use a special root {\va}, which enforces agentive (or weakly agentive) readings.\footnote{\cite{doron03} uses a syntactic head $\iota$; see Chapter \ref{vz:others:ed} on some differences between the theories.} I assume that {\va} attaches to the verbal spine at the vP level, thereby triggering the agentive alloseme of Voice (following \citealt{doron03,doron14adj}). The morphophonology produces the templates {\tpie} and {\thit}, as I return to momentarily. Here is the basic proposal, followed by a deep dive into each part (syntax, semantics and phonology).
\pex {\va}:
	\a A modifier which attaches to vP.
	\a \denote{Voice} = $\lambda$x$\lambda$e.Agent(x,e) / \trace~\va
	\a Voice {\lra} {\tpie} / {\trace} {\va}
	\a {\vz} {\lra} {\thit} / {\trace} {\va}
\xe

As a root, this element has phonological and semantic content but no syntactic features or requirements. Not much hinges on whether this element is a root or a functional head in this language; since it has no syntactic influence, but combines predictable phonology with semantics that can be difficult to characterize formally, it behaves like any other root.\footnote{For these reasons I do not consider it to be a ``flavor'' of v, for example.} The question of what other such ``underspecified'' roots might exist in natural language remains an open one for further crosslinguistic research.

	\subsection{Syntax} \label{voice:va:syn}
I propose that a transitive verb like \emph{pirek} `dismantled' has the basic structure in~(\nextx a), and an unergative verb like \emph{riked} `danced around' has the basic structure in~(\nextx b).
\pex 
	\a Transitive {\tpie}:\\
	\Tree
	[.VoiceP
		[.DP ]
		[.
			[.Voice ]
			[.vP
				[.{\va} ]
				[.vP
					[.v
						[.\root{pr\dgs{k}} ]
						[.v ]
					]
					[.DP ]
				]
			]
		]
	]

	\a Unergative {\tpie}:\\
	\Tree
	[.VoiceP
		[.DP ]
		[.
			[.Voice ]
			[.vP
				[.{\va} ]
				[.vP
					[.v
						[.\root{r\dgs{k}d} ]
						[.v ]
					]
				]
			]
		]
	]
\xe

The agentive modifier forces an agentive reading, which necessarily requires either a transitive or unergative structure. This much captures the syntactic distribution of {\tpie}.

Consider what this means in terms of alternations. Returning to the examples in~(\ref{ex:voice:piel-meanings}), we saw an ``intensive'' alternation between \emph{ʃavar} `broke' and \emph{ʃiber} `broke to pieces'. Assuming a layering view of argument structure \citep{layering15}, we first build up a core vP consisting of a breaking event:
\ex \Tree
[.vP
	[.v
		[.\root{ʃbr} ]
		[.v ]
	]
	[.DP ]
]
\xe

What happens next? The grammar has two options. It can either merge Voice (\nextx a), in which case we get the verb in {\tkal}, or it can merge {\va} and then Voice (\nextx b), in which case we get the verb in {\tpie}.

\pex
	\a \Tree
	[.
		[.Voice ]
		[.vP
			[.v
				[.\root{ʃbr} ]
				[.v ]
			]
			[.DP ]
		]
	]

	\a \Tree
	[.
		[.Voice ]
		[.vP
			[.{\va} ]
			[.vP
				[.v
					[.\root{ʃbr} ]
					[.v ]
				]
				[.DP ]
			]
		]
	]
\xe

As noted in all of the major works on Hebrew morphology, alternations are not always the norm: there is no guarantee that a verb in {\tkal} will alternate with one in {\tpie}, as many verbs in {\tkal} have no counterpart in {\tkal} (and vice versa). This property is idiosyncratic and must be listed with every root. But when {\tkal} and {\tpie} do alternate, this is how: if a given root is instantiated in both templates, then the {\tpie} version will always be an ``intensive'', agentive version of the {\tkal} verb, since {\tpie} is the spell out of adding a {\va} layer to the core event which otherwise would be spelled out as {\tkal}. 

The derivation of verbs in {\tpie} which do not alternate with {\tkal} is identical. For (\nextx), we first build up the core vP, then attach {\va}, and then attach the external argument. The fact that the core vP cannot combine with Voice directly must be listed with the root, in whatever way regulates which functional heads can appear with which root. Now the meaning of the root is chosen by {\va}, rather than by v (since there is no verb in {\tkal}) or Voice (since {\va} is closer to the root, \citealt{arad03,marantz13,elenasamioti14}).
\pex
	\a \ljudge{*} ʃabaʃ (\root{ʃbʃ} in {\tkal})
   \a \begingl
    \gla \underline{ha-xom} \textbf{ʃibeʃ} et ha-medidot//
    \glb the-heat disrupted.\gsc{INTNS} \gsc{ACC} the-measurements//
    \glft `The heat messed up the measurements.//
  \endgl
	\a \Tree
	[.VoiceP
		[.DP\\{\emph{ha-xom}} ]
		[.
			[.Voice\\{\emph{i,e}} ]
			[.vP
				[.{\va} ]
				[.vP
					[.v
						[.\root{ʃbʃ}\\{\emph{ʃbʃ}} ]
						[.v ]
					]
					[.DP\\{\emph{et ha-medidot}} ]
				]
			]
		]
	]
\xe

Again, what ``intensive'' means is left intentionally vague. A few options are sketched next, after a technical aside about the height of attachment for {\va}.

		\subsubsection{Height of attachment} \label{voice:va:syn:wonk}
In principle, {\va} could be argued to adjoin to v/vP, Voice or even to the root. The benefit of adjoining it to v is that the alternations between {\tkal} and {\tpie} follow cleanly, as do those between {\tpie} and {\thit}. Here is a preview of what this looks like, to be further explored in the next chapter. Both causative \emph{pirek} `dismantled' and anticausative \emph{hitparek} are built from the core vP in~(\nextx a). If Voice is merged, we get causative \emph{pirek} in {\tpie} (\nextx b). If {\vz} is merged, we get anticausative \emph{hitparek} in {\thit} (\nextx c).
\pex
	\a \Tree
		[.vP
			[.{\va} ]
			[.vP
				[.v
					[.\root{pr\dgs{k}} ]
					[.v ]
				]
				[.DP ]
			]
		]
	\a \emph{pirek} `dismantled'\\
		\Tree
		[.VoiceP
			[.DP ]
			[.
				[.{Voice\\\emph{i,e}} ]
				[.vP
					[.{\va} ]
					[.vP
						[.v
							[.\root{pr\dgs{k}} ]
							[.v ]
						]
						[.DP ]
					]
				]
			]
		]
		\a \emph{hitparek} `fell apart'\\
			\Tree
			[.VoiceP
				[.DP ]
				[.
					[.{\vz\\\emph{hit-,a,e}} ]
					[.vP
						[.{\va} ]
						[.vP
							[.v
								[.\root{pr\dgs{k}} ]
								[.v ]
							]
							[.DP ]
						]
					]
				]
			]
\xe

In previous work \citep{kastner16phd,kastner17gjgl,kastner18nllt} I assumed that {\va} modifies Voice, and not vP as it does here. There were three reasons for this. The first was that placing {\va} between Voice and a higher element such as T correctly derives certain allomorphic patterns under the strict linear adjacency hypothesis for contextual allomorphy \citep{embick10,marantz13}, as developed in \cite{kastner18nllt}. While I am fond of this argument, much current work argues that this restriction needs to be weakened (see e.g.~\citealt{kastnermoskal18,choiharley19}). The second is that adjoining {\va} to Voice renders it similar to Greek \emph{afto}. However, it is not crucial for the theory that these two elements merge in similar locations in different languages. The third is that since {\va} influences the interpretation of the external argument, adjoining it to Voice seemed most appropriate. Yet it is clear that agentive semantics can be generated low: verbs like \emph{murder} and \emph{devour} are strongly agentive \citep{haspelmath93,unaccusativity95,marantz97,layering15}, a requirement which originates within the vP (at the root). For these reasons, I now think that {\va} adjoins to vP, although there are no clinching arguments either way.\footnote{As pointed out to me by Yining Nie (p.c.), adjoining {\va} to Voice would render this combination structurally similar to prepositional roots adjoining to Voice in the i* system of \cite{woodmarantz17}, discussed in Chapter~\ref{i:i:i}.} See \cite{ahdout19phd} for additional benefits of adjoining {\va} to vP in the domain of nominalization.

	\subsection{Semantics} \label{voice:va:sem}
Given that {\va} has been just argued to be a root, assigning a semantics to it without a theory of root semantics is difficult. What we can do is see its effects on the external argument, formalized as follows:
%\ex \denote{Voice} = $\lambda$x$\lambda$e.e \& \text{Agent}(x,e) / \trace~\va
%\xe
\pex \denote{Voice} = 
	\a $\lambda$P.P \phantom{agent(x,e)xxx} / \trace~ \{ \root{npl} `\root{\gsc{FALL}}', \root{kpa} `\root{\gsc{FREEZE}}' , \dots \}
	\a $\lambda$x$\lambda$e.Agent(x,e) or $\lambda$x$\lambda$e.Cause(x,e)
	\a $\lambda$x$\lambda$e.Agent(x,e) / \trace~\va
%		 & \text{/ \trace~\{ \root{trf} `\root{\gsc{DEVOUR}}', \root{ktb} `\root{\gsc{WRITE}}', \root{ntn} `\root{\gsc{GIVE}}',}\\
%			& \text{\root{ʃal} `\root{\gsc{BORROW}}', \root{\gsc{r\dgs{k}d}} `\root{\gsc{dance}}', \root{\gsc{hlx}} `\root{\gsc{WALK}}', \dots\} }\\
\xe
While this formalization aims to be explicit, I have taken a few shortcuts. As already argued for by \cite{layering15}, it is the vP which provides the causative component, not Voice. The formalization in~(\lastx) is meant to indicate that both Causes and Agents are compatible with Voice, but that only Agents are possible once {\va} is in the structure. 

Let us expand the analysis a bit more: what readings does {\va} make available? Some examples are given in~(\ref{ex:voice:piel-meanings2}), repeated from~(\ref{ex:voice:piel-meanings}). While {\tpie} is traditionally called the ``intensive'' template, it can also house pluractional verbs, (\nextx d--f), and various others which do not alternate with forms in {\tkal}, (\nextx g--i):
\ex\label{ex:voice:piel-meanings2}Pretheoretical classification of some verbs in \tpie:\\
	\begin{tabular}{lll|ll|ll}
	& & & \multicolumn{2}{c|}{\tkal} &  \multicolumn{2}{c}{\tpie}\\\hline
	\multirow{3}{*}{Intensive} & a.& \root{ʃbr} & ʃavar & `broke' & ʃiber & `broke to pieces'\\
		& b.& \root{jtsr} & jatsar & `produced' & jitser & `produced'\\
	    & c.& \root{'kl} & axal & `ate' & ikel & `corroded, consumed'\\\hline

 	\multirow{3}{*}{Pluractional} & d.& \root{hlx} & halax & `walked' & hilex & `walked around'\\
 	    & e.& \root{r\dgs{k}d} & rakad & `danced' & riked & `danced around'\\
  	    & f.& \root{\dgs{k}fts} & kafats & `jumped' & kipets/kiftsets & `jumped around'\\\hline

  		\multirow{3}{*}{Non-derived} & g. & \root{tps} & \multicolumn{2}{c|}{---} & tipes & `climbed'\\
	    & h. & \root{ltf} & \multicolumn{2}{c|}{---} & litef & `petted'\\
		  & i. & \root{\dgs{k}bl} & \multicolumn{2}{c|}{---} & kibel & `received'\\
	\end{tabular}
\xe

The pluractional readings and underived verbs have potentially interesting theoretical consequences, which will be touched on here before moving on to the phonological contribution of {\va}.

		\subsubsection{Pluractionality} \label{voice:va:sem:plural}
One possible way to describe the semantics of {\va} is by extended reference to pluractionality. The intuition as is follows. Assume that {\va} is a pluractional (and perhaps also agentive) affix. Building on recent work by \cite{henderson12phd,henderson17nllt}, pluractionality can be seen as a way of pluralizing an event. This pluralization can hold spatially as well as temporally. For the ``intensive'' forms in~(\ref{ex:voice:piel-meanings2}a--c), the underlying core vP has a direct object. The corresponding pluralized events in {\tpie} can be individuated with respect to the direct objects, e.g.~many broken pieces in (\ref{ex:voice:piel-meanings2}a) or many different simultaneous corrosions of parts of the material's surface in (\ref{ex:voice:piel-meanings2}c). This extension is admittedly less obvious for ``production'' in~(\ref{ex:voice:piel-meanings2}b). \cite{greenberg10} makes a similar claim for verbs in {\tpie} that are derived from reduplicated roots.

For the ``pluractional'' forms in (\ref{ex:voice:piel-meanings2}d)--(\ref{ex:voice:piel-meanings2}f), the underlying core events are unergative. The pluralizing operation has no direct object to operate on, and so I would suggest that it pluralizes the spatio-temporal event itself in {\tpie}.

Lastly, in (\ref{ex:voice:piel-meanings2}g)--(\ref{ex:voice:piel-meanings2}i) there is no underlying form and hence nothing to pluralize. The resulting verbs are still agentive but not necessarily pluractional.

%The database of verbs we are employing contains over 900 forms in {\tpie}, so this line of inquiry faces a serious amount of empirical corroboration.
This way of thinking about {\tpie} is speculative at this point. A number of potential counterexamples can be conjured up fairly easily. These are cases where the alternation does not plausibly result in a plural event:
\pex
	\a \emph{lamad} `learned' $\sim$ \emph{limed} `taught'
	\a \emph{ratsa} `wanted' $\sim$ \emph{ritsa} `satisfied'
\xe

In the examples in~(\lastx) the event does not entail change of state, unlike with breaking and eating/corroding. So perhaps there is a tripartite division of roots to be made, as follows:
\pex
	\a \textbf{Other-oriented roots (change of state)} such as \root{\gsc{BREAK}} and \root{\gsc{PRODUCE}}: pluralization of the object.
	\a \textbf{Activity roots} or \textbf{self-oriented roots} such as \root{\gsc{RUN}} and \root{\gsc{JUMP}}: pluralization of the spatio-temporal aspects of the event.
	\a \textbf{Other cases:} no pluralization.
\xe

Since our current focus is not on the lexical semantics of root classes and how they integrate into the syntax, I will leave proper testing of the hypothesis in~(\lastx) for future work. Evaluating this proposal will need to proceed along the lines laid out above, testing whether each root instantiated in this template does indeed fit into one of the three cases in~(\lastx).

		\subsubsection{Underived forms} \label{voice:va:sem:underived}
A number of verbs in {\tpie} stretch the notion of ``agentivity'' to the point where even a weak definition is no longer tenable. In the examples in~(\nextx), the verb can hardly be described as agentive since the subject is inanimate, while in~(\anextx) the subject is animate but non-volitional. These verbs are compatible with agentive subjects as well, but clearly do not require them.
\pex
  \a \begingl
    \gla \underline{ha-midgam} \textbf{ʃikef} et totsot ha-emet//
    \glb the-poll reflected.\gsc{INTNS} \gsc{ACC} results.\gsc{CS} the-truth//
    \glft `The polls (correctly) reflected the results.'//
  \endgl
    
  \a \begingl
    \gla be-ritsa axat \underline{ha-ʃaon} ʃel garmin kimat \textbf{diek} kaaʃer hetsig stia kimat xasrat maʃmaut ʃel axuz ve-ktsat//
    \glb in-run one the-watch of Garmin almost was.accurate.\gsc{INTNS} when showed deviation almost devoid.of meaning of percent and-little//
    \glft `In one run, the Garmin watch was precise as it showed an almost insignificant deviation of just over one percent.'\trailingcitation{\url{www.haaretz.co.il/sport/active/.premium-1.2309128}}//
  \endgl
\xe

\ex \begingl
  \gla \underline{hu} \textbf{kibel} maka xazaka ba-regel//
  \glb he received.\gsc{INTNS} hit strong in.the-leg//
  \glft `He got hit hard in the leg.'//
  \endgl
\xe

In these examples an external argument is still required, regardless of whether it can felicitously be called an agent or not. What these examples show is that a rigid denotation of {\va} is difficult to specify, beyond some general notion of a direct cause. I believe it is significant, though, that the verbs in~(\blastx)--(\lastx) do not have correspondents in {\tkal}: \emph{ʃikef} $\nless$ *\emph{ʃakaf}, \emph{diek} $\nless$ *\emph{dajak}, \emph{kibel} $\nless$ *\emph{kabal} and \emph{ʃibeʃ} $\nless$ *\emph{ʃabaʃ} from earlier. They would fit with the underived group of~(\ref{ex:voice:piel-meanings2}): generated when {\va} selects the meaning of the root directly without having to agentivize an event in vP/{\tkal}. If {\va} really is a root rather than a functional head, its partially unpredictable contributions to the meaning of the verb are not unexpected.

	\subsection{Phonology} \label{voice:va:phono}
The morphophology of {\tpie} consist of two parts that distinguish it from other templates: different stem vowels and the way it bleeds a regular phonological process of spirantization. In Modern Hebrew, /p/, /b/ and /k/ spirantize to [f], [v] and [x] postvocalically \citep{adam02,temkinmartinez08wccfl,gouskova12nllt}, a process that applies to nonce words as well \citep{temkinmartinzemuellner16}. An example of this process was seen above in~(\ref{ex:va-piel1}a--b), where /b/ spirantizes to [v] after a vowel except if {\va} is also in the structure. The inflectional paradigm for {\tpie} in three tenses is given in~(\nextx).

\ex
\raisebox{-4.5em}{
\begin{small}
	\begin{tabular}{|l||l|l||l|l||l|l|} \hline
		& \multicolumn{2}{c||}{Past} & \multicolumn{2}{c||}{Present} &  \multicolumn{2}{c|}{Future} \\\hline
		& \gsc{M} & \gsc{F} & \gsc{M} & \gsc{F} & \gsc{M} & \gsc{F} \\\hline\hline
		1\gsc{SG} & \multicolumn{2}{c||}{Xi\dgs{Y}aZ-ti} & me-Xa\dgs{Y}eZ & me-Xa\dgs{Y}eZ-et & \multicolumn{2}{c|}{a-Xa\dgs{Y}eZ/je-Xa\dgs{Y}eZ}\\\hline
		1\gsc{PL} & \multicolumn{2}{c||}{Xi\dgs{Y}aZ-nu} & me-Xa\dgs{Y}Z-im & me-Xa\dgs{Y}Z-ot & \multicolumn{2}{c|}{ne-Xa\dgs{Y}eZ}  \\\hline\hline
		2\gsc{SG} & Xi\dgs{Y}aZ-ta & Xi\dgs{Y}aZ-t & me-Xa\dgs{Y}eZ & me-Xa\dgs{Y}eZ-et & te-Xa\dgs{Y}eZ & te-Xa\dgs{Y}Z-i\\\hline
		2\gsc{PL} & Xi\dgs{Y}aZ-tem & Xi\dgs{Y}aZ-ten/m & me-Xa\dgs{Y}Z-im & me-Xa\dgs{Y}Z-ot & \multicolumn{2}{c|}{te-Xa\dgs{Y}Z-u}\\\hline\hline
		3\gsc{SG} & Xi\dgs{Y}eZ & Xi\dgs{Y}Z-a & me-Xa\dgs{Y}eZ & me-Xa\dgs{Y}eZ-et & je-Xa\dgs{Y}eZ & te-Xa\dgs{Y}eZ\\\hline
		3\gsc{PL} & \multicolumn{2}{c||}{Xi\dgs{Y}Z-u} & me-Xa\dgs{Y}Z-im & me-Xa\dgs{Y}Z-ot & \multicolumn{2}{c|}{je-Xa\dgs{Y}Z-u}\\\hline
	\end{tabular}
\end{small}
}
\xe

VIs can be assigned similarly to how this was done for Voice in Section~\ref{voice:voice:phono}. The difference is that since {\va} is adjacent to Voice, it can condition allomorphy of the vowels; this is what we see in~(\lastx), where the stem vowels are different for {\tpie} than for {\tkal}. However, {\tpie} is separated from T by overt Voice (the vowels), so the agreement affixes are correctly predicted to be identical across the templates.

\pex \label{tree:locality-tpie}
\Tree
    [.TP
        [.\tikz{\node (TAgr) {T+Agr};} ]
        [
            [.\tikz{\node (Voice) {Voice};} ]
            [.vP
            	[.\tikz{\node (va) {\va};} ]
            	[.vP
            		[.\root{XYZ} ]
            		[.{\dots} ]
            	]
         	]
         ]
     ]
%                [.v\\(covert) ]
%                [.\tikz{\node (Root) {\root{root}};} ]
%            ]
%        ]
%    ]
    \begin{tikzpicture}[overlay]
    %\draw[semithick,->] (TAgr) to[out=south west,in=south west] (Voice);
%    \draw[dotted,thick,->] (Root) .. controls +(south:1) and +(south:2) .. (Voice);
    %\draw[dotted,thick,->] (TAgr) to[out=225,in=270] (Voice);
    \draw[dotted,thick,->] (TAgr) .. controls +(south:1) and +(south west:1) .. (Voice);
    \draw[dotted,thick,->] (va) .. controls +(south:1) and +(south:1) .. (Voice);
    \draw[dotted,thick,->] (va) .. controls +(south:2) and +(south west:2) .. node{\LARGE $\times$}(TAgr);
    \end{tikzpicture}
\xe \vskip 1em

The non-spirantization can be analyzed as a floating feature docking onto the medial root consonant and preventing it from acquiring a [continuant] feature. Two basic VIs are given in~(\nextx), where the floating feature still needs a constraint to dock it onto the right segment; see \cite{kastner18nllt} for the full implementation.
\pex
	\a\label{vi:voice2}	Voice \lra~\emph{i,e} / T[Past] \trace~\va
			\a\label{vi:va}\va~\lra~[--cont]$_{\gsc{ACT}}$ / \trace~ \{ \root{XYZ} $|$ Y $\in$ p, b, k \}
			
\xe

\section{Summary and outlook} \label{voice:conc}
This chapter examined the two templates {\tkal} and {\tpie}, treating them not as morphemic atoms but as combinations of functional heads, specifically v, Voice, and (in the case of {\tpie}) {\va}. The following generalizations about the argument structure of both templates are repeated here from~(\ref{ex:gen-tkal}) and~(\ref{ex:gen-tpie}).

\hammer{
\pex \label{ex:gen-tkal2}\textbf{Generalizations about {\tkal}}
	\a \textbf{Configurations:} Verbs appear in all possible argument structure configurations.
	\a \textbf{Alternations:} Participates in alternations with the other templates, as will be reviewed throughout the book.
\xe
}

\hammer{
\pex \label{ex:gen-tpie2}\textbf{Generalizations about {\tpie}}
	\a \textbf{Configurations:} Verbs appear in active (transitive/unergative) configurations.\\
		Readings are weakly agentive.
	\a \textbf{Alternations:} When alternating with {\tkal}, provides a more ``intensive'' or agentive version.
\xe
}

To account for these patterns, I began to unfold the proposed theory of trivalent Voice. This chapter concentrated on two elements: Unspecified Voice does not impose any strict constraints in the syntax but is nevertheless traceable in the morphophonology. It is compatible with whatever argument structure the root allows. The modifier {\va} enforces certain agentive or agentive-like readings which, I have argued, can be found in various other languages as well. Both elements are overt.

The next chapters of Part I examine the other templates, motivating an analysis which uses different values of Voice. In Chapter~\ref{chap:vz} we will see what happens when Voice is endowed with a [--D] feature, prohibiting the merger of DPs in its specifier. The result will be a structure that allows anticausatives and, in some cases, reflexives of different kinds. In Chapter~\ref{chap:vd} we will see the consequences of a [\!+\!D] feature appearing on Voice, requiring its specifier to be filled. And in Chapter~\ref{chap:passn} we will see how these Voice heads interact with passiviziation, nominalization and adjectivization.
