\chapter{The valence of Voice}
\label{chap:intro}

The aim of this monograph is to present a new theory of argument structure alternations, one which is anchored in the syntax but has systematic interfaces with the phonology and the semantics. Conceptually, my goal is to argue for a specific formal system. Empirically, my goal is to provide the most comprehensive description and analysis of Hebrew verbal morphology to date, one whose formal assumptions are as similar as possible to those made in work on non-Semitic languages. Let's first see why Hebrew is interesting (Section~\ref{intro:puzzles}) and then why it continues to challenge existing accounts (Section~\ref{intro:basic}, before outlining the current proposal, Trivalent Voice (Sections~\ref{intro:arch}--\ref{intro:sketch}).

\section{Identifying the puzzles} \label{intro:puzzles}
In the verbal system of Modern Hebrew, verbs appear in one of seven morphological \emph{templates}. These templates, listed in~(\nextx), are the main object of study in this book. I will go into exact notational matters and how to understand these forms momentarily; for now, all that matters is that the \emph{root consonants} can be substituted for the placeholders X, Y and Z. Templates are traditionally given in the citation form, third person masculine singular past tense.
 \begin{exe}
 \ex  
 \begin{xlist} 
 	\ex  {\tkal} 
 	\ex  {\tnif} 
 	\ex  {\tpie} 
 	\ex  {\thit} 
 	\ex  {\thif} 
 	\ex  {\tpua} 
 	\ex  {\thuf} 
 \z
\z 

The most important thing to know about the templates is that they are easy to identify based on morphophonological form (although I provide glosses just in case), and that they \emph{often} carry \emph{some kind} of meaning. Pinning down the essence of ``often'' and ``some kind'' is my main analytical task.

	\subsection{The two problems of Semitic morphology}
Because our theoretical interest is in argument structure alternations, we can start there. The following examples demonstrate three different verbs, all sharing the same root which I notate \root{ktb}. In general, it can be seen that all verbs have to do with writing in some sense. The first is a simple transitive in the template {\tkal}:
 \begin{exe}
\ex 	\label{ex:intro-tkal}Transitive {\tkal} 
		
 		\gll  ha-talmidim \glemph{katv-u} et ha-nosim\\
 		  the-students wrote-\gsc{PL} \gsc{ACC} the-topics\\
 		\glt `The students wrote the topics down.' 
	
 \z 

The second is a non-active variant in {\tnif}; this is how we would express the anticausative or passive version of~(\lastx).
 \begin{exe}
\ex  \label{ex:intro-tnif}Non-active (mediopassive) {\tnif} 
		
 		\gll  ha-xiburim \glemph{nixtev-u} (al-jedej ha-talmidim)\\
 		  the-essays were.written-\gsc{PL} by the-students\\
 		\glt `The essays were written (by the students)'. 
	
 \z 

The third is a causative version in~{\thif}.
 \begin{exe}
\ex  \label{ex:intro-thif}Causative {\thif} 
		
 		\gll  ha-mora \glemph{hextiv-a} (la-talmidim) et reʃimat ha-nosim\\
 		  the-teacher dictated-\gsc{3SG.F} to.the-students \gsc{ACC} list.of the-topics\\
 		\glt `The teacher dictated the list of topics (to the students).' 
	
 \z 

If this is what the language looked like, the system would be far less puzzling. The analytical issues begin to mount when we understand that verbs in~{\tkal} are not always transitive like in~(\ref{ex:intro-tkal}). Verbs in {\tnif} are not always non-active like those in~(\ref{ex:intro-tnif}). And verbs in {\thif} are not always causative like those in~(\ref{ex:intro-thif}); counterexamples are given in~(\nextx).
 \begin{exe}
 \ex  
 \begin{xlist} 
 	\ex  Unaccusative in {\tkal}: 
		
 		\gll  ha-bakbuk \glemph{kafa} ba-makpi\\
 		  the-bottle froze in.the-freezer\\
 		\glt `The bottle froze in the freezer.' 
		
 	\ex  Unergative in {\tnif}: 
		
 		\gll  josi \glemph{nixnas} la-xeder be-bitaxon\\
 		  Yossi entered to.the-room in-security\\
 		\glt `Yossi confidently entered the room.' 
	
 	\ex  Unergative in {\thif}: 
		
 		\gll  marsel \glemph{heezin} be-savlanut\\
 		  Marcel listened in-patience\\
 		\glt `Marcel listened patiently.' 
	
 \z
\z 

Be that as it may, it is crucial that there is also some method to the madness. It is not the case that any template can be associated with any syntactic or semantic construction. Certain configurations---unaccusative, transitive, reflexive, etc.---are only possible with certain templates. This is \textbf{the first problem} of Semitic morphology: what syntactic structures and semantic readings is a given template associated with, and why?

Additionally, sometimes we can find alternations like in~(\ref{ex:intro-tkal})--(\ref{ex:intro-thif}). Certain templates alternate with some but not with others. \textbf{The second problem} of Semitic morphology is thus: what templates does a given template alternate with, and why?

Granted, there is also a third problem: how can we tell which meaning is licensed by which root? That question deserves a monograph of its own, though I will try to flag ways in which it can be approached throughout the book.

I believe the answers to these questions can be found once we abandon the notion of a ``template'' as some kind of morphological primitive. I propose here a decomposition of the template into functional heads in the syntax, one that is able to address both problems above. What this means is that we need to engage with what alternations are and how argument structure comes about.

	\subsection{Argument structure}
Contemporary theories of argument structure often take as a starting point the ``anticausative alternation'', whereby a transitive verb (\textbf{causative}) and its intransitive equivalent (\textbf{anticausative}) stand in some morphologically mediated relationship. In some languages, such as English in~(\ref{ex:intro1}), the two verbs do not differ in their morphological marking. In other languages the predominant situation is one in which a reflexive pronoun appears in the anticausative variant, as in \ili{German}, (\ref{ex:intro2}). And in other languages, the anticausative variant has specific non-active morphological marking. Some verbs in \ili{Greek} are like this, (\ref{ex:intro3}). Other languages fall into one or more of these typological categories.
 \begin{exe}
 \ex \label{ex:intro1} 
 \begin{xlist} 
 	\ex  Meg \glemph{opened }the door. \hfill (causative) 
  	\ex  The door \glemph{opened}.			\hfill (anticausative) 
 \z

 \ex \ili{German} \label{ex:intro2} 
 \begin{xlist} 
 	\ex   
 		\gll Florian \"offnete die T\"ur.\\
		 Florian opened the door\\ \jambox{(causative)}
 		\glt `Florian opened the door.' 
	
 	\ex   
 		\gll Die T\"ur \"offnete \glemph{sich}.\\
		 the door opened \gsc{REFL}\\ \jambox{(anticausative)}
 		\glt `The door opened.' 
	
 \z

 \ex \ili{Greek} \label{ex:intro3} 
 \begin{xlist} 
 	\ex   
 		\gll o Giorgos \glemph{ekapse} ti supa\\
		 the Giorgos burned the soup\\ \jambox{(causative)}
 		\glt `Giorgos burned the soup.' 
	
 	\ex   
 		\gll i supa \glemph{kaike}\\
		 the soup burned.\gsc{NACT}\\ \jambox{(anticausative)}
 		\glt `The soup burned.' 
	
 \z
\z 

In other languages a ``causative alternation'' can be observed, where the causative variant is marked. The \ili{Japanese} example in~(\nextx) exemplifies \citep[3]{oseki17nyu}.
 \begin{exe}
 \ex  \ili{Japanese}
 \begin{xlist} 
 	\ex   
 		\gll John-ga ringo-o koor-\glemph{as}-ta\\
		 John-\gsc{NOM} apple-\gsc{ACC} freeze-\gsc{CAUS}-\gsc{Past}\\ \jambox{(causative)}
 		\glt `John froze an apple.' 
	
 	\ex   
 		\gll Ringo-ga koor-ta\\
		 apple-\gsc{NOM} freeze-\gsc{Past}\\ \jambox{(anticausative)}
 		\glt `An apple became frozen.' 
	
 \z
\z 

Various syntactic and semantic questions arise in connection with these seemingly simple patterns, many of which have been explored in influential studies such as \cite{haspelmath93}, \cite{unaccusativity95}, \cite{schaefer08}, \cite{koontzgarboden09} and \cite{layering15}: what kind of morphological marking appears on the different variants? Is there a sense in which one is derived from the other, or do the two share a common base? Which predicates are marked as causative or anticausative crosslinguistically? 

The degree of variation both within and across languages is substantial. However, most studies on argument structure have analyzed this aspect of the syntax-semantics interface through the lens of languages with relatively simple concatenative morphology. Each of these languages has contributed much to our understanding of argument structure, to be sure: the English labile alternation shines light on which predicates are likely to be marked in which way \citep{haspelmath93,unaccusativity95,koontzgarboden09}; the French, German and Spanish alternations bring in many aspects of cliticization, binding and agreement \citep{labelle08,schaefer08,cuervo14}; the Greek alternation shows consistent morphological marking for at least one class of predicates \citep{alexiadoudoron12,layering15}; and more recent work on Icelandic has further identified ways in which argument structure alternations can be correlated with morphological processes \citep{wood14nllt,wood15springer,wood16roots}. Yet this line of work has the drawback that these languages usually show only binary morphological distinctions, if any: either the causative variant is marked, or the anticausative one is marked (or neither is, as in the labile alternation). Three-way marking is a challenge which persists with some larger-scale typological surveys as well \citep{haspelmath93,arad05}.

	\subsection{Solving the two problems}
The intuition guiding my analysis is that of \cite{schaefer08}, \cite{layering15} and related work: the alternations are not alternations at all. The grammar does not derive causative forms form inchoative ones, or anticausative forms from transitive ones. Rather, what happens is that both readings are derived from one core structure (technically a vP) with a causative component in the semantics. If we add an external argument, we get a transitive/causative verb; if we do not, we simply retain the basic event and have an anticausative verb on our hands.

This book provides a way of implementing the same idea in Hebrew. Now, I am by no means the first to suggest that the templates be decomposed. Maya Arad and Edit Doron have both made seminal contributions to our understanding of these issues. But \cite{arad05} was torn between the need to acknowledge the idiosyncrasies of the system, on the one hand, and the need to encode the alternations, on the other hand. As a result, that theory had to implement conjugation classes in order to adequately describe which alternations exist. \cite{doron03} sidestepped the issue by providing a compositional semantics for the components making up the templates, but the result was that alternations could only be discussed in terms of their semantics, and not their morphology or syntax. What I propose is a way to get the alternations from contemporary syntactic assumptions.

The two problems are addressed as follows. By building up specific syntactic structures we are able to easily explain what syntactic configurations and semantic interpretations arise for a given structure, as well as how this structure is spelled out; that spell-out is what we call the template. Instead of figuring out the many-to-many mapping between form and meaning, I map one structure deterministically to form and to meaning, thereby solving the first problem. And by adopting the idea that a core vP carries the basic meaning of a verb, we can then layer additional heads above it, regulating the introduction of an external argument. The majority of work is carried out by the head Voice, which introduces the external argument. This solves the second problem. A technical innovation lies with the syntactic feature [$\pm$D] that Voice might carry, hence the valence of Voice. But we will get to that soon enough.

Part I of this book is comprised of case studies of the different templates, which together come to form the theory of trivalent Voice. Part II consists of two chapters situating this theory within contemporary theoretical debates.

The rest of this introductory chapter is structured as follows. I give a general overview of Hebrew morphology in Section~\ref{intro:basic}, including a brief account of what the traditional view is. Section~\ref{intro:arch} introduces the formal assumptions of my theory, which itself is outlined in Section~\ref{intro:sketch}.


\section{Traditional descriptions and basic generalizations} \label{intro:basic}
	\subsection{Hebrew morphology for beginners}
The first thing to note about Hebrew is that not \emph{all} morphology is non-concatenative. Agreement, for example, may consist of prefixes and suffixes, alongside non-concatenative changes to the stem. The future tense paradigm for the verb \emph{katav} `wrote' in {\tkal} is given in Table~\ref{tab:1-2-1:tense}. The stem vowel is either /o/ or /e/, depending on whether the verb is suffixed or not, but other than that all of the agreement information is affixal.
\begin{table}
	\begin{tabularx}{.75\textwidth}{XXX}
 \lsptoprule
Person/Gender	& \gsc{SG}	& \gsc{PL}\\\midrule
1				&\textbf{e-}xtov				&\textbf{ni-}xtov\\
2\gsc{M}		&\textbf{ti-}xtov				&\textbf{ti-}xtev\textbf{-u}\\
2\gsc{F}		&\textbf{ti-}xtev\textbf{-i}	&\textbf{ti-}xtev\textbf{-u}\\
3\gsc{M}		&\textbf{ji-}xtov				&\textbf{ji-}xtev\textbf{-u}\\
3\gsc{F}		&\textbf{ti-}xtov				&\textbf{ji-}xtev\textbf{-u}\\
\lspbottomrule
 \end{tabularx}
\caption{Concatenative affixation in Hebrew verbs.}
\label{tab:1-2-1:tense}
\end{table} 
I do not concern myself here with this distinction directly since my main interest is within the thematic domain, i.e.~VoiceP. In general, it is not surprising that syntactic material from a certain height and ``upwards'' in the tree is spelled out affixally rather than non-concatenatively; see \cite{harbour08} and \cite{kastnertucker19cup} for further discussion of this cross-Semitic point.

Nevertheless, linguists and non-specialists alike often find themselves scratching their heads in an attempt to come to terms with Semitic's distinctive morphological system, built around ``roots'' and ``patterns''. Many early speakers of Modern Hebrew were such head-scratchers themselves: the language was revived in the late 19th century by individuals who, for the most part, were not native speakers of Semitic languages. The language nevertheless retained the Semitic morphology of its classical predecessor.
Given that this book is a study of the verbal system of Hebrew, I will make repeated reference to ``roots'' and ``templates'' (the latter also called ``patterns'', ``measures'', ``forms'' and \emph{binyanim}) as the two main components of the verb. I reserve the terms ``templates'' for the seven systematic verbal forms and ``patterns'' for the nominal and adjectival forms. These traditional terms have been used, as far as I know, for as long as the verbal systems of Hebrew and other Semitic languages have been documented. \cite{ussishkin00phd} mentions a number of works on Hebrew which use roots and templates as integral parts of the system, including \cite{gesenius}---perhaps the best-regarded grammar of Biblical Hebrew---as well as \cite{bopp1824}, \cite{ewald1827}, \cite{harris41} and \cite{chomsky51}. For Arabic, he mentions \cite{desacy1810} as one example among many of older works which make direct reference to roots and templates.

The nature of the root was already debated by the traditional Arabic grammarians of Basra and Kufa in the 8th Century, according to \citet[563ff]{borer13oup} who herself cites \cite{owens88}. Turning to more recent works, we can add foundational contributions by \cite{rosen77}, \cite{berman78}, \cite{bolozky78,bolozky99} and \cite{ravid90}, all relying on the root and the template as descriptive notions. I cannot hope to do justice here to the vast modern-day literature on Hebrew, much of which has been published in Hebrew. The interested reader may want to consult the works of Yehoshua Blau, Reuven Mirkin, Uzzi Ornan and Haim Ros\'en, among others.

To see how the system is traditionally conceived of, let us consider first form, then meaning. The verbs in~(\lastx) were all given in the 3rd person masculine singular past tense -- the citation form. The actual conjugation of a given form across tenses and person/number/gender features is completely predictable, as Table~\ref{table:piel} exemplifies for the {\tpie} template (barring certain lexical idiosyncrasies investigated in \citealt{kastner18nllt}). That is to say, even though the meaning of a given verb cannot be immediately guessed in its entirety, the morphophonological form is predictable. Note again how agreement material is mostly affixal.

\begin{table}
\fittable{
	\begin{tabularx}{\textwidth}{lllllll}
 \lsptoprule
		& \multicolumn{2}{c}{Past} & \multicolumn{2}{c}{Present} &  \multicolumn{2}{c}{Future} \\
		& \gsc{M} & \gsc{F} & \gsc{M} & \gsc{F} & \gsc{M} & \gsc{F} \\ \midrule
		1\gsc{SG} & \multicolumn{2}{c}{XiY̯aZ-ti} & me-XaY̯eZ & me-XaY̯eZ-et & \multicolumn{2}{c}{je-XaY̯eZ}\\
		1\gsc{PL} & \multicolumn{2}{c}{XiY̯aZ-nu} & me-XaY̯Z-im & me-XaY̯Z-ot & \multicolumn{2}{c}{ne-XaY̯eZ}  \\ \tablevspace
		2\gsc{SG} & XiY̯aZ-ta & XiY̯aZ-t & me-XaY̯eZ & me-XaY̯eZ-et & te-XaY̯eZ & te-XaY̯Z-i\\
		2\gsc{PL} & \multicolumn{2}{c}{XiY̯aZ-tem} & me-XaY̯Z-im & me-XaY̯Z-ot & \multicolumn{2}{c}{te-XaY̯Z-u}\\ \tablevspace
		3\gsc{SG} & XiY̯eZ & XiY̯Z-a & me-XaY̯eZ & me-XaY̯eZ-et & je-XaY̯eZ & te-XaY̯eZ\\
		3\gsc{PL} & \multicolumn{2}{c}{XiY̯Z-u} & me-XaY̯Z-im & me-XaY̯Z-ot & \multicolumn{2}{c}{je-XaY̯Z-u}\\
\lspbottomrule
 	\end{tabularx}
}
\caption{Tense and agreement marking in \tpie.}
\label{table:piel}
\end{table}

For meaning, we may take as a starting point the essay by \cite{schwarzwald81} and the traditional classification of the seven templates in Table~\ref{tab:1-2:schwarz}.\footnote{Seven is the canonical number, but cf.~\cite{schwarzwald16} for reasons to posit novel templates.}
\begin{table}
	\begin{tabularx}{\textwidth}{lccp{0.0cm}llll}
 \lsptoprule
		& \textbf{Active} & \textbf{Passive} & && & & \\
	\textbf{Simple} & \tkal & \tnif & & \root{sgr} & \emph{sagar} & \emph{nisgar} & `closed'\\
	\textbf{Intensive} & \tpie & \tpua & & \root{tpl} & \emph{tipel} & \emph{tupal} & `treated'\\
	\textbf{Causative} & \thif & \thuf & & \root{kns} & \emph{hexnis} & \emph{huxnas} & `inserted' \\
	\textbf{Reflexive/recip.} & \multicolumn{2}{c}{\thit} & & \root{xb\dgs{k}} & \multicolumn{2}{c}{\emph{hitxabek}} & `hugged' \\
\lspbottomrule
 	\end{tabularx}	
\caption{A na\"ive classification of Hebrew templates \citep[131]{schwarzwald81}.}
\label{tab:1-2:schwarz}
\end{table}

As \citeauthor{schwarzwald81} immediately points out herself, this classification is misleading. The relationships between the templates (the argument structure alternations) are not always predictable and most templates have additional meanings beyond those listed in~(\lastx). For example, there is little way to predict what the root \root{rʃm}, which has to do with writing down, will mean when it is instantiated in a given template. In the ``simple'' template {\tkal} we substitute the consonants in \root{rʃm} for X, Y and Z and derive \emph{raʃam} `wrote down'. In the ``middle'' template \tnif, \emph{nirʃam le-} means `signed up for', against the characterization of \tnif~as ``simple passive'' in~(\lastx). In the ``intensive middle'' \thit, \emph{hitraʃem me-} means `was impressed by', challenging the characterization of \thit~as ``reflexive or reciprocal'' in~(\lastx). 

The only cells of the table which are completely predictable are the two passive templates {\tpua} (``intensive passive'') and {\thuf} (``causative passive''). The other templates constrain the possible meaning in ways that have eluded precise specification. This returns us to the two basic questions that need to be addressed, mentioned at the outset:
\begin{itemize}
	\item What are the possible readings associated with a given template (and why)?
	\item What templates does a given template alternate with (and why)?
\end{itemize}

In Part I of the book we will see that the syntax and semantics of the system can nevertheless be analyzed within a constrained theory of morphosyntax. I will make precise what the unique contribution of each template is and how that contribution comes about in the syntax. We will then be able to identify the role of the root in selecting between different possible meanings for the verb in a given template.


	\subsection{Traditional generative treatments of the system} \label{intro:basic:jjmcc}
Before we get to the meat of the book, I would like to acknowledge some of the earlier generative work on Semitic morphology. This will also help set the stage for direct comparison with alternative accounts later on. My aim is not to provide a history of ideas; for that see \cite{kastnertucker19cup}.

In a groundbreaking series of works, John McCarthy presented a purely phonological account of Semitic morphology, focusing on \ili{Arabic} \citep{jjmcc79,jjmcc81,jjmcc89li,jjmccprince90}. His original contribution lay in dividing the Semitic verb into three ``planes'' or ``tiers'': the CV skeleton (consonant and vowel slots), the root (consonants) and the melody (individual vowels).
By including the vocalism on a separate tier, \citeauthor{jjmcc81}'s theory allowed vowels to be manipulated independently of the roots or the skeleton.
The beauty of this theory is that it allowed for a separation of three morphological elements on three phonological tiers: the root (identity of the consonants), the template (the form of the CV skeleton) and additional inflectional or derivational information (the identity of the vowels).

The current work shifts the focus to the nature of the CV skeleton and the melody. \citeauthor{jjmcc81}'s approach did not attempt to model the relationships between the semantics of the different templates -- the alternations in argument structure. 
Taking these relationships into account requires a slight change in perspective. Like \cite{jjmcc81}, I believe that the consonantal root lies at the core of the lexicon. Unlike in that theory, I do not postulate independent CV skeletons and do not accord the prosody morphemic status as such. The skeletons will be a by-product of how functional heads are pronounced and regulated by the general phonology of the language. There is no skeleton CVCVCCVC (in \ili{Arabic}) giving \emph{takattab} `got written', for example \citep[392]{jjmcc81}: there would be a prefix \emph{ta}-, a number of vowels spelling out Voice, gemination spelling out an additional head, and the organization of these different segments will proceed in a way that satisfies the phonology without making reference to prosodic primitives like skeletons. Furthermore, each morpheme will have an explicit syntax and semantics associated with it.

A few more pieces of research that capture generalizations important to this book deserve mention. The seminal work by \cite{berman78} underscored the semi-predictable nature of the templates. \citet[Ch.~3]{berman78}, in particular, made the point that the combination of root and template is neither fully regular nor completely idiosyncratic. Instead, she proposed a principle of ``lexical redundancy'' to regulate the system. According to this theory, each root has a ``basic form'' in some template from which other forms are derived. Yet this theory did not formalize the relations between the templates, arbitrarily selecting one as the ``basic form'' and the others as derived from it, for each root. Nevertheless, \citeauthor{berman78}'s clear description of the regularities and irregularities in the morphology of Hebrew laid the groundwork for later works such as \cite{doron03}, \cite{arad05}, \cite{borer13oup} and the current contribution.

Alongside work that analyzed the syntactic and semantic features of roots and templates, other researchers have focused on the morphophonological properties of the system. The research program developed in a series of works by \cite{batel89,batel94} and \cite{ussishkin99,ussishkin00phd,ussishkin05}---credited by \cite{ussishkin00phd} at least in part to \cite{horvath81}---denies the existence of the root as an independent morpheme. Instead, all verbs are derived via phonological manipulation of surface forms from each other, rather than from an underlying root. The syntactic-semantic aspects of this view were developed by \cite{reinhartsiloni05} and \cite{laks11,laks13morpho,laks14,laks18}. I refer to this idea as the ``stem-based approach'' and critique it briefly when relevant; if there is no consistent way of thinking about templates as morphosyntactic primitives then this view has few legs to stand on. See \cite{kastner17gjgl,kastner18nllt} for more pointed objections, and \cite{kastnertucker19cup} for a broader perspective.

Even before the stem-based approach took form, other Semitists explored the idea of a Semitic system which diverged from the traditional descriptions. \cite{schwarzwald73} doubted the productivity of both the root and the templates, making an early argument for frequency effects in the interpretation of different templates. On that view, it is only the high frequency verbs of the language that show reliable alternations between templates. These verbs lead us as analysts to postulate relationships between templates, though when one looks at less frequent verbs, transparent alternations are less likely to hold. Unlike the stem-based hypothesis, which eschewed roots and relied on the template as a morphological primitive, the proposal in \cite{schwarzwald73} kept the root but relegated the template to morphophonological limbo: salient in the grammar but not operative in the syntax. While this early formulation of a template-less idea is intriguing, it cannot be maintained in the face of wug studies in which speakers generate argument structure alternations between templates using nonce words \citep{berman93jcl,moorecantwell13}.

A special place in the literature has been carved out by \cite{doron03,doron13voice} and \cite{arad03,arad05}. I have already mentioned some features of these theories and we will return to them in more depth as the discussion proceeds.

Finally, to pick out a few studies on Arabic (as gleaned from the helpful overview in \citealt{ussishkin00phd}), \cite{darden92} offered an analysis of Egyptian Arabic that attempted to do without verbal templates; \cite{mcomber95} developed an infixation-based system similar to that of \cite{jjmcc81} which makes crucial reference to morpheme edges; and \cite{ratcliffe97,ratcliffe98} attempted to improve on \cite{jjmccprince90} by restricting the CV skeleton and treating more phenomena as cases of infixation. But let us return to the current study.

	\subsection{Data and notation} \label{sec:data:notation}
I use the variables X, Y and Z for the tri-consonantal root: \root{XYZ}. This monograph contains little discussion of roots with more than three consonants, but nothing in the notation hinges on it. \cite{ehrenfeld12} curated a database of verbal forms in Hebrew notated for root and template; examining the roots in this database reveals 311 quadrilateral roots and three quintilateral roots\footnote{\root{xntrʃ} `bullshit', \root{snxrn} `synchronize' and \root{flrtt} `flirt'.} out of 1876 roots in total. I have adapted this database for my own use and refer to it throughout the monograph. \cite{ahdout19phd} has further annotated parts of this database with additional information related to argument structure; some of her findings are referenced in the monograph as well. Other data, in particular examples and judgments of productivity, rely on my own intuitions, published work and online resources.

As will be discussed in Chapter~\ref{voice:tpie}, Hebrew has a fairly productive process of postvocalic spirantization applying to /b/, /k/ and /p/, turning them into [v], [x] and [f] respectively. This process is blocked in certain verbal templates; to note this blocking I borrow the non-syllabicity diacritic and place it under the medial root consonant: ``\dgs{Y}''. This notation can be found in the templates \tpie~and \thit, in which this blocking holds. The same notation is used for segments which never spirantize: ``\dgs{k}''.

Transcriptions are given using the International Phonetic Alphabet with the following modifications:
\begin{itemize}
	\item ``e'' stands for /ɛ/ and /ə/.
	\item ``g'' stands for /ɡ/.
	\item ``o'' stands for /ɔ/.
	\item ``r'' stands for /ʁ/.
	\item ``x'' stands for /χ/.
	\item The apostrophe ' stands for the glottal stop. %, although I usually leave it out omitted since the glottal stop is often dropped in contemporary speech.
\end{itemize}
These changes were made purely for reasons of convenience. The syntactic literature has often used ``\v{s}'' or ``S'' for /ʃ/ and ``c'' for /ts/. In both cases I preferred to retain the IPA transcription, ``ʃ'' and ``{\ts}''. Stress is marked with an acute accent when necessary, ``\'a''. Deleted vowels are enclosed in angle brackets, ``$<>$''. \underline{Underlining} and \textbf{boldface} are used only for emphasis, never as diacritics or notation.

My notation also contains various deviations from standard forms; these will probably only be of interest to readers already familiar with the language.

The template {\thif} usually appears in the literature as \emph{h\textbf{i}XYiZ}, with an /i/-/i/ vocalic pattern. Yet contemporary speakers use /ɛ/ \citep{trachtman16}, and so I transcribe ``e'' throughout. Conversely, the initial /h/ is usually dropped in speech but I retain it for two reasons. First, /h/ is still pronounced by some older speakers and certain sociolinguistic groups, often marginalized ones \citep[cf.~][]{schwarzwald81biu,gafter14phd}. And second, the initial \emph{h}- should help non-Semitist readers to distinguish this template from other ones.

Glottal stops are often dropped in speech \citep{enguehardfaust18}. I usually omit them, but at times retain an apostrophe in order to distinguish between otherwise homophonous forms, for example \emph{hefria} `he disturbed' $\sim$ \emph{hefri\underline{'}a} `she disturbed'.

When presenting verbal paradigms I include two substandard forms. The \gsc{1SG} future form is normally prefixed with \emph{a-} or \emph{e-}, e.g.~\emph{e-xtov} `I will write'. Contemporary usage, however, syncretizes \gsc{1SG} future with \gsc{3SG.M} future: \emph{je-daber} `I/he will talk'. I include both forms when giving paradigms. And finally, contemporary usage does not distinguish between masculine and feminine plural forms in past and future tense verbs. The traditional feminine plural endings have been discarded, syncretizing instead with the masculine plural forms.

In the Hebrew glosses, \gsc{ACC} is used for the direct object marker \emph{et} and `of' for the head of a Construct State nominal, in the interest of readability. When reproducing examples from the literature I have modified the original transcriptions for consistency.

Finally, I am careful to use \emph{construction} as a term which is meant to be informal, descriptive or pre-theoretical. For example, a ``causative construction'' does not entail any specific analysis but is merely a convenient label. In contrast, I use \emph{structure} or \emph{configuration} to mean the underlying syntax, for example an unaccusative configuration. With this housekeeping out of the way, we return to the theoretical approach.


\section{Architectural assumptions} \label{intro:arch}
Since my aim is to account for the syntactic, semantic and phonological behavior of the system, I must be explicit about my assumptions in all three cases. But since the focus is on the syntax and how it feeds interpretation at Logical Form (LF) and Phonological Form (PF), I divide the overview here into syntax and the interfaces.

	\subsection{The syntax}
I assume a mainstream variety of Distributed Morphology \citep{dm} within the Minimalist framework \citep{chomsky95}. This means that all syntactic and morphological objects are built in the syntax; there is no separate grammatical module for word-building. The traditional work of the lexicon is distributed between the syntax, the semantics and the phonology: the syntax builds up binary structures via Merge of morphemes according to syntactic constraints, features, mechanism and so on. Traditional ``words'' are composed here minimally of an abstract root, lacking syntactic category, and one of the three functional heads: a for adjectives, n for nouns and v for verbs \citep{marantz01,arad03}. The core of a verb phrase therefore looks as follows, where the root modifies v and the internal argument is the complement of v.
 \begin{exe}
\ex  
	\Tree
	[.vP
		[.v
			[.\root{root} ]
			[.v ]
		]
		[.DP ]
	]
 \z 

The syntactic structure is transferred to the interfaces at Spell-Out, where it is interpreted by each of the two components. LF calculates meaning and PF calculates (morpho-)phonology. Spell-Out proceeds cyclically, that is, after a structure of certain size has been built up. The three categorizing heads are one such domain for Spell-Out \citep{arad03,embick10,marantz13,elenasamioti14}. The head Voice (see below) demarcates another domain.

Lexical information is stored in what is often called the Encyclopedia, a vaguely-defined warehouse of idiosyncratic information. To the extent that we understand the Encyclopedia, we assume that it is organized by root \citep{harley14thlia}.

This architecture means that there are no stems (as such), no paradigms (as such) and no words (as such). None of these are primitives of the system. Some are epiphenomenal (like paradigms) and some can be more accurately specified as phonological/prosodic words or morphological words, depending on the definition \citep{embicknoyer01,gouskova19jsl}; either way, this definition will be in terms of syntax or phonology, not in independent terms of morphology.

I will not argue for any of these assumptions in this book, but to the extent that the results are convincing, they provide natural support for these assumptions and against stem-based (word-based) theories. Some finer details now follow.

		\subsubsection{What is Voice?} \label{intro:arch:voice}
In the current neo-Davidsonian tradition, theories of argument structure have adopted a specific way of thinking about internal and external arguments in the syntax, based in large part on the interpretation asymmetries observed by \cite{marantz84} and discussed by \cite{kratzer96}. The theme or patient of the predicate is generated within the vP as the complement of v.\footnote{Contemporary decompositional theories do not have a ``big-V'' lexical verb, V, nor do they have lexical adjectives A and nouns N. In addition, whether or not internal arguments end up in Spec,VP/Spec,vP as in various approaches is immaterial here \citep{johnson91,alexiadouschaefer11wccfl}.} The agent is introduced in the specifier of a higher functional head, which takes the vP as its own complement. Since \cite{kratzer96} it has become common to call this head Voice and to associate it with accusative case licensing, thereby identifying it with causative ``little v'' of \cite{chomsky95}. The basics are given in~(\nextx), slightly modifying \citet[121]{kratzer96}. The relevant compositional functions invoked here are Functional Application and Event Identification. We will also make use of Predicate Modification later on in this book; see \cite{wood15springer} for an accessible introduction. I leave out the semantic types of the arguments.
 \begin{exe}
 \ex  
 \begin{xlist} 
 	\ex  Mittie fed the dog. 
 	\ex  \Tree 
	[.VoiceP\\{λe.Agent(Mittie, e) \& feed(the dog, e)}\\{\textsf{(by Functional Application})}
		[.DP\\\emph{Mittie} ]
		[.{λxλe.Agent(x,e) \& feed(the dog, e)}\\{\textsf{(by Event Identification)}}
			[.Voice\\{λxλe.Agent(x,e)} ]
			[.vP\\{λe.feed(the dog, e)}\\{\textsf{(by Functional Application)}}
				[.v\\{λxλe.feed(x,e)}
					[.\root{\gsc{FEED}} ]
					[.v ]
				]
				[.DP\\\emph{the dog} ]
			]
		]
	]
 \z
\z 

I would like to focus on two important points here as a segue into the Trivalent theory. First, this original formulation does not make any claims regarding a structural difference between agents and causers (e.g.~circumstances, inanimate objects or natural forces). While there have been some attempts to draw a structural difference between the two---at least for certain psychological predicates \citep{bellettirizzi88,harleystone13}---I join the majority of work on argument structure in making no claims to that extent \citep[7]{layering15}. For me, \emph{Agents} are a subset of \emph{Causes}, but this difference is semantic, not syntactic. What this means is that an external argument position (Spec,VoiceP) should be compatible with both Agents and Cause, but some additional element could force only a narrower, agentive reading. This we will see already in Chapter~\ref{voice:va}. When the difference between Agents and Causes matters, I will be clear about it. In any case, this architecture does not utilize traditional theta roles as primitives of argument structure.

The second point is that as a functional head, Voice might be endowed with different features. In principle, since it licenses a DP in its specifier, it should have the EPP feature [D] \citep{chomsky95}. Once we accept that it has that feature, we can begin to ask what other features it can have, and might these features get checked in the course of the derivation. Much recent work in argument structure has explored the possible values of the [$\pm$D] feature on Voice, as well as the theoretical characterization of [$\pm$D]; these issues are discussed directly in Chapters \ref{chap:aas} and \ref{i:agree}, after the current theory has been developed in depth. One recent approach is of particular importance so I introduce it next.\label{r1:g:2a1}

		\subsubsection{Layering} \label{intro:arch:layering}
A recurring question in discussions of argument structure regards the direction of derivation in (anti-)causative alternations. For an alternation like~(\nextx), is the transitive version derived from the intransitive one via causativization or is the intransitive variant derived from the transitive one via anticausativization?
 \begin{exe}
 \ex  
 \begin{xlist} 
 	\ex  Mary broke the vase. 
 	\ex  The vase broke. 
 \z
\z 

In their ``layering'' approach to transitivity alternations, \cite{layering15} summarize a number of reasons for thinking that neither answer is strictly speaking true. They propose that both variants have the same base: a minimal vP (\nextx a) containing the verb (a verbalized root) and the internal argument. The difference between the two variants is that the transitive one, (\nextx b), then has the external argument added by additional functional material (Voice).\footnote{Marked anticausatives contain their own Voice layer but still have anticausative syntax and semantics, again because the core vP is the locus of the event. We will get to this in Chapter~\ref{chap:vz} and recap the specific implementation of the Layering approach in Chapter~\ref{chap:aas}.}
 \begin{exe}
\ex  
a. 
\Tree
		[.vP
			[.\emph{broke} ]
			[.\emph{the glass} ]
		]
b. \Tree
[.VoiceP
	[.\emph{John} ]
	[.
		[.Voice ]
		[.vP
			[.\emph{broke} ]
			[.\emph{the glass} ]
		]
	]
]
 \z 

This view explains a range of facts about this alternation, chiefly that there is no dedicated direction of derivation which is marked by the morphology across languages. That is, while some languages mark the transitive variants, others mark the intransitive variants, and sometimes both variants are marked in the same language (as we have already seen for Hebrew). Even though there is much to say about which verbs or roots are marked in which way \citep{haspelmath93,unaccusativity95,arad05}, the grammar itself does not force derivation from one stem type to the other.

In addition to the morphological reasoning, \cite{layering15} provide a series of arguments showing that the core causative component of the vP is present even in the anticausative variants. For example, there is no difference in event structure between causatives and anticausatives, indicating that Agents and Causes are not introduced in a separate event to the change of state. Furthermore, the Cause PPs in~(\nextx) are possible with anticausatives but Agents are not possible, indicating that causation can take place even without an external argument. Importantly, the causative component is not simply introduced by the preposition \emph{from} \citep[30]{alexiadouetal06,alexiadouetal06nels,layering15}.
 \begin{exe}
 \ex  
 \begin{xlist} 
 	\ex  The flowers wilted \{from the heat / *from the gardener\}. 
 	\ex  The window cracked \{from the pressure / *from the worker\}. 
 \z
\z 

In sum, while there is a causative core, an actual Cause argument can only be introduced by additional structure; either in a cause-PP, or as an external argument in a higher projection; an additional layer, so to speak. Voice is the functional head enabling this layer, both in terms of licensing Spec,VoiceP in the syntax and in opening the thematic predicate Agent. The causative alternation in English can be easily explained in these terms.

	\subsection{Interfaces} \label{intro:arch:inter}
When syntactic structure is spelled-out, it is interpreted at LF (semantics) and PF (phonology). The Trivalent approach shares with other current work a certain view of the so-called autonomy of syntax \citep{marantz13,wood15springer,woodmarantz17,myler17oup}. Essentially, the grammar (the syntax) is free to generate different syntactic structures, so long as these satisfy inherently syntactic requirements (for example Case licensing or feature valuation). The syntactic object must then still be interpreted by the interfaces at Spell-Out, at which point they can be said to interpret but also ``filter'' the output. At LF the semantic composition may or may not converge, and at PF the phonological calculation may or may not yield an optimal candidate. In both cases we may expect certain kinds of crosslinguistic variation.

\label{r1:2:3a}Compositional semantics proceeds straightforwardly (the main operations are Functional Application and Event Identification, as mentioned above), as does linearization, prosodification and phonological evaluation. See the introductory chapters of \cite{wood15springer} or \cite{myler16mit} for additional details on the semantic composition. While I make repeated reference to semantic roles such as Agent, I do not assume that theta-roles are a primitive of the system. The phonological calculation may be implemented using Optimality Theory \citep{ot} as done in \cite{kastner18nllt}. Here are the other points that might require further elaboration.

		\subsubsection{Roots}
I individuate roots based on their phonology (e.g.~\root{ktb} and \root{\gsc{arrive}}), but it is more accurate to think of them as pointers to phonological and semantic information \citep{harley14thlia,faust16,kastner18nllt}. Nevertheless, I will use the phonological shorthand for convenience.

Despite the crucial role of roots in determining the reading of a word, I cannot provide a theory of root meaning here. Not every root can appear in every template, meaning that a root has to license the functional heads it combines with somehow \citep{harleynoyer00}. Exactly how this happens is left vague. Presumably, this licensing should be similar to the way that a root like \root{\gsc{murder}} requires Voice in English, but a root like \root{\gsc{arrive}} does not license Voice.

The idea that roots pick out meanings which are shared across forms will likewise not be formalized. I will be relatively comfortable talking about shared meaning in cases of alternations. In Chapter~\ref{voice:tpie} and in Chapter~\ref{vd:caus} I will discuss cases where the shared meaning is slightly less easy to pin down. Neither of these points is particular to Hebrew within root-based approaches like DM (and both require engaging more seriously with the lexical semantics literature), but they do appear more prominent because of the nature of the morphological system.

It is important to delve a bit deeper into the idea of one root across a few templates. Consider \root{p\dgs{k}d} in~(\nextx).  One could find a general semantic notion of ``counting'' or ``surveying'' running through the use of this root but the alternations are in no way obvious. 
 \begin{exe}
 \ex \label{ex:naive-pkd} 
 \begin{xlist} 
   \ex  \tkal: \emph{pakad} `ordered'. 
   \ex  \tnif: \emph{nifkad} `was absent'. 
   \ex  \tpie: \emph{piked} `commanded' (and a passive \tpua~form). 
   \ex  \thif: \emph{hefkid} `deposited' (and a passive \thuf~form). 
   \ex  \thit: \emph{hitpaked} `allied himself', `conscripted'. 
 \z
\z 
The problem is exacerbated when considering nominal forms as well: \emph{pakid} `clerk', \emph{mifkada} `headquarters', \emph{pikadon} `deposit'. Templates, then, do not provide us with deterministic mappings from phonological form (the template) to semantics (interpretation of a root), again with the exception of the passive templates.

So the question is whether verbs such as those in~(\ref{ex:naive-pkd}) do in fact share the same root. For example, it could be argued that (\lastx a,b,c,e) as well as the noun `headquarters' share one root that has to do with military concepts, and that (\lastx d) as well as the nouns `clerk' and `deposit' stem from a homophonous root that has to do with financial concepts. There are a number of reasons to reject this claim. First, there are no ``doublets''; if we were dealing with two roots, call them \root{pkd$_1$} and \root{pkd$_2$}, then each should be able to instantiate any of the templates. But \emph{hefkid} can only mean `deposited', never something like `installed into command'. The choice of verb for that root in that template has already been made. Second, experimental studies have found roots to behave uniformly across their different meanings \citep{deutsch16,deutschetal16,deutschkuperman18,kastneretal18}, although this is not a consensus yet \citep{fmdpmetal05jml,hellerbendavid15}.
		
		\subsubsection{Contextual allomorphy}
A morpheme is an abstract element, comprised of a bundle of syntactic features (or, in the case of roots, comprised of a pointer to lexical information). In DM, a morpheme is matched up with its exponent, or Vocabulary Item, in a postsyntactic process of Vocabulary Insertion. Which exponent is chosen depends on the phonological and syntactic environment the morpheme is in (see~\citealt{bonetharbour12} and~\citealt{gouskovabobaljik20cup} for overviews).

It may be the case that a morpheme has a number of contextual variants or \emph{allomorphs}. For example, the English past tense marker has a number of possible exponents, depending on the phonological environment it is inserted in.
 \begin{exe}
 \ex  
 \begin{xlist} 
 	\ex  \emph{grade}[əd] 
 	\ex  \emph{jam}[d] 
 	\ex  \emph{jump}[t] 
 \z
\z 
This can be formalized as follows:
 \begin{exe}
\ex  T[Past] \lra $\begin{cases} 
	\emph{\text{əd}} & / \trace ~ \text{[+cor --cont --son]} \\
	\text{\emph{d}} & / \trace~ \text{[+voice]} \\
	\text{\emph{t}} & \\
	\end{cases}$
 \z 

The English definite article also has two contextual allomorphs conditioned by the phonological environment (but cf.~\citealt{gouskovaetal15,pak16}).
 \begin{exe}
 \ex  
 \begin{xlist} 
 	\ex  \emph{a dog}		 
 	\ex  \emph{an apple} 
 \z
\z 
Similarly:
 \begin{exe}
\ex  D[--def] \lra $\begin{cases} 
	\emph{\text{ə}} & / \trace~ \#\text{C} \\
	\emph{\text{ən}} & / \trace ~ \#\text{V} \\
	\end{cases}$ 
 \z 

Some roots also supplete based on their environment. Here the context for allomorphy is not the phonological features of the local trigger but the syntactic features.
 \begin{exe}
 \ex  
 \begin{xlist} 
 	\ex  \emph{go} (today) 
 	\ex  \emph{went} (yesterday) 
 \z

\ex   \gsc{go}\footnote{This is a simplified version for expository purposes. The element to be spelled out should be something like \root{\gsc{go}} in the context of the verbalizer v, in addition to phi-features.} \lra $\begin{cases} 
	\text{\emph{go}} & / \trace~ \text{[\gsc{PRES}]} \\
	\text{\emph{went}} & / \trace ~ \text{[\gsc{PAST}]} \\
	\end{cases}$
 \z \label{r1:1:4}

Similarly for adjectives:
 \begin{exe}
 \ex  
 \begin{xlist} 
 	\ex  \emph{good}		 
 	\ex  \emph{better} 
 	\ex  \emph{best} 
 \z

\ex  \gsc{GOOD} \lra $\begin{cases} 
	\text{\emph{good}} & / \trace~ \text{[\gsc{NORM}]} \\
	\text{\emph{better}} & / \trace ~ \text{[\gsc{CMPR}]} \\
	\text{\emph{best}} & / \trace ~ \text{[\gsc{SPRL}]} \\
	\end{cases}$
 \z 

The question occupying many theorists is the moment is what exactly the nature of ``\trace'' is: is it linear adjacency, syntactic adjacency, or something else? In previous work I have adopted the idea that allomorphy can only be triggered under linear adjacency of overt elements \citep{embick10,marantz13}. This hypothesis helps explain a range of allomorphic interactions in Hebrew, as I argued for in \cite{kastner18nllt}. Some of these points will be mentioned in the following chapters---in particular because I think the current analysis makes the right predictions---but the discussion does not revolve around them.

In my formal analysis I will assume that the stem vowels spell out Voice and that affixes spell out higher material (this can be seen as a Mirror Principle effect following directly from cyclic spell out; \citealt{baker85,muysken88,katie13,zukoff16nels,kastner18nllt}). Alternatively, we may assume that a dissociated Theme node is projected (``sprouted'') from Voice postsyntactically \citep{oltramassuet99,embick10}; the same holds for Agr (agreement suffixes based on phi-features), be it on T or sprouted from T. But for simplicity I will represent the stem vowels as the overt spell-out of Voice and agreement as the spell out of a joint T+Agr head.

		\subsubsection{Contextual allosemy}
The phenomenology of contextual allomorphy is fairly well understood, even if the exact mechanisms are under debate. A similar concept that has only recently gained currency is contextual \emph{allosemy}. The idea is the same. One morpheme may have a number of interpretations competing for insertion at PF; this is allomorphy. One morpheme may also have a number of interpretations competing for insertion at LF; this is allosemy. Recent discussions can be found in \cite{woodmarantz17} and \cite{mylermarantz19cup}.

\cite{kratzer96} proposed that Voice introduces the Agent role for eventualities~(\nextx) and that Holder introduces the Holder role for states~(\anextx).
 \begin{exe}
 \ex  \emph{feed the dog}: 
 \begin{xlist} 
 	\ex  \denote{feed the dog} = λe.feed(the dog,e) 
 	\ex  \denote{Voice} = λxλe.Agent(x,e) 
 	\ex  \denote{Voice feed the dog} = λxλe.Agent(x,e) \& feed(the dog,e) 
 \z

 \ex  \emph{own the dog}: 
 \begin{xlist} 
 	\ex  \denote{own the dog} = λs.own(the dog, e) 
 	\ex  \denote{Holder} = λxλs.Holder(x,s) 
 	\ex  \denote{Holder own the dog} = λxλs.Holder(x,s) \& own(the dog,e) 
 \z
\z 

Yet nothing forces Voice and Holder to be separate heads; in fact, this would be surprising given that their syntax and morphology are identical. As explained by \cite{wood15springer}, we could just as well posit that Voice has two contextual allosemes: one when it combines with a dynamic event and when it combines with a stative event.
 \begin{exe}
\ex  \denote{Voice} \lra~ $\begin{cases} 
	\text{λxλe.Agent(x,e)} & / \trace~ \text{(eventuality)} \\
	\text{λxλs.Holder(x,s)} & / \trace~ \text{(state)} \\
	\end{cases}$ 
 \z 
Here the contexts are purely semantic, as they should be, given that we are now in LF.

I make extensive use of this formalism in order to formalize the semantics of functional heads in this book. An alternative could also be considered, whereby there is a proliferation of homophonous heads similar to Voice and Holder. I see no reason to adopt this perspective, especially considering how naturally contextual allosemy fits into the Trivalent framework. We can now overview what this framework does for the puzzles of Hebrew.


\section{Sketch of the system} \label{intro:sketch}
Reviewing the facts that require explanation, all templates can be described along two axes: the range of interpretations they are compatible with, and the canonical alternations they participate in.

We have already seen that transitive verbs exist both in {\tkal} and {\thif}, and that unaccusatives exist in in {\tnif}. Yet transitive verbs also exist in {\tpie} (\nextx a) and anticausatives also exist in {\thit} (\nextx b). So the syntactic configuration does not entail a given template.
 \begin{exe}
 	\ex \label{ex:counter1}
 	\begin{xlist}
 		\ex Transitive in {\tpie}: \emph{biʃel} `cooked' (not {\tkal} *\emph{baʃal})
 		\ex Anticausative in {\thit} \emph{hitparek} `fell apart' (not {\tnif} *\emph{nifrak})
 \z 
 \z

Conversely, a given template does not always entail a given syntactic configuration. Even {\tnif} appears on some unergatives, (\nextx a), and {\thit} instantiates not only anticausatives as in~(\lastx b) but also reflexives as in~(\nextx b).
\begin{exe}
	\ex \label{ex:counter2}
	\begin{xlist}
		\ex  Unergative in {\tnif} \emph{nilxam} `fought' (not anticausative)
		\ex Reflexive in {\thit} \emph{hitgaleax} `shaved' (not anticausative)
\z
\z

This section concludes the introduction by presenting a simplified overview of how the entire system can be understood. The first aspect of the analysis (what readings a given template has) is mainly accomplished using the features on Voice. The second aspect (which templates form alternations) is accomplished using hierarchical syntactic structure.

	\subsection{Simple alternations}
We can start from an alternation that works fairly intuitively, the one we saw back in~(\ref{ex:intro-tkal})--(\ref{ex:intro-thif}). In this near-minimal triplet, three verbs are found in which a given root (\root{ktb}) clearly has three different kinds of morphological marking, or templates. Again, ``template'' is a descriptive term in this book, not a formal one.

 \begin{exe}
 \ex \label{ex:general} 
 \begin{xlist} 
 	\ex  Causative verb in {\thif}: 
		
 		\gll  fabjen \glemph{hextiv-a} (la-talmidim) et reʃimat ha-nosim\\
 		  Fabienne dictated-\gsc{F} to.the-students \gsc{ACC} list.of the-topics\\
 		\glt `Fabienne dictated the list of topics (to the students).' 
		
 	\ex  Transitive verb in {\tkal}: 
		
 		\gll  ha-talmidim \glemph{katv-u} et ha-nosim\\
 		  the-students wrote-\gsc{PL} \gsc{ACC} the-topics\\
 		\glt `The students wrote the topics down.' 
	
 	\ex  Anticausative/mediopassive verb in {\tnif}: 
		
 		\gll  ha-xiburim \glemph{nixtev-u} (al-jedej ha-talmidim)\\
 		  the-essays were.written-\gsc{PL} by the-students\\
 		\glt `The essays were written (by the students)'. 
	
 \z
\z 

Relying on the idea that the external argument is introduced by the functional head Voice, I propose that it may be endowed with syntactic features, specifically the feature [$\pm$D].
 \begin{exe}
 \ex  \textbf{Trivalent Voice} 
 \begin{xlist} 
 	\ex  Voice is associated with a [$\pm$D] feature, meaning it can be valued as [+D], [--D] or unspecified with regards to [D].\footnote{A similar view of binary features as trivalent is espoused by \cite{harbour11}.} 
 	\ex  This feature indicates whether the specifier of Voice must be filled by a DP ([+D]), cannot be filled by a DP ([--D]), or is agnostic as to whether it is filled by a DP (unspecified). 
 \z
\z 

A verb with {\vd} requires an external argument; a verb with {\vz} prohibits an external argument; and if Unspecified Voice is merged, the syntax itself does not place a restriction, although the root will (lexical idiosyncrasy contained within a rigid syntax).

Importantly, these Voice heads differ in their phonological form. Assuming that {\vd} spells out as {\thif}, {\vz} as {\tnif} and Unspecified Voice as {\tkal}, the theory derives the alternations seen in~(\ref{ex:general}) as in Table~\ref{table:alternations-heb}.

\begin{table}
	\begin{tabularx}{\textwidth}{lllllll}
 \lsptoprule
	 & \multicolumn{2}{c}{\textbf{\vd}}	&	\multicolumn{2}{c}{\textbf{Voice}}	& \multicolumn{2}{c}{\textbf{\vz}}\\
	 & \multicolumn{2}{c}{Causative}		& \multicolumn{2}{c}{Transitive}		& \multicolumn{2}{c}{Anticausative}\\\midrule
	& \multicolumn{2}{c}{\thif}	&	\multicolumn{2}{c}{\tkal}	& \multicolumn{2}{c}{\tnif}\\
	\root{ktb} & \emph{hextiv}	& `dictated' &	\emph{katav}	& `wrote'	&	\emph{nixtav}	& `was written' \\
	\root{'xl} & \emph{heexil}	& `fed' &	\emph{axal}	& `ate'	&	\emph{neexal}	& `was eaten' \\
\lspbottomrule
 	\end{tabularx}
	\caption{Simple alternations in Hebrew.}
\label{table:alternations-heb} 
\end{table}

To finish this initial overview we will walk through the alternations. Following \cite{kratzer96} and \cite{layering15}, it has become fairly common to assume that a core vP contains a causative component which is semantically available even in anticausatives (Section~\ref{intro:arch:layering}). Voice can then add an external argument (an agent), but otherwise the vP already has a basic meaning. Accordingly, we can combine the root \root{ktb}, the verbalizer v and an internal argument. This vP gives us a basic event of writing something, where v is silent (all through the language, by hypothesis):
 \begin{exe}
\ex  
\Tree
	[.vP
		[.v
			[.\root{ktb} ]
			[.v ]
		]
		[.DP ]
	]
 \z 

The combinatorics are now simple. If we merge {\vz}, no external argument is added and we have a simple anticausative. I notate the ban on an element in Spec,VoiceP as ``---''' in the specifier position for explicitness.
 \begin{exe}
\ex  
	\Tree
	[.VoiceP
		[.{---} ]
		[.
			[.{\vz} ]
			[.vP
				[.v
					[.\root{ktb} ]
					[.v ]
				]
				[.DP ]
			]
		]
	]
 \z 

If we merge Voice, an external argument is added and we get the causative variant: an event of writing something with an agent doing the writing.
 \begin{exe}
\ex  
	\Tree
	[.VoiceP
		[.DP ]
		[.
			[.Voice ]
			[.vP
				[.v
					[.\root{ktb} ]
					[.v ]
				]
				[.DP ]
			]
		]
	]
 \z 

And if we merge {\vd}, we will need to specify a different kind of external argument (how this happens is explored in Chapter~\ref{vd:caus}).
 \begin{exe}
\ex  
	\Tree
	[.VoiceP
		[.DP ]
		[.
			[.{\vd} ]
			[.vP
				[.v
					[.\root{ktb} ]
					[.v ]
				]
				[.DP ]
			]
		]
	]
 \z 

There is no direct alternation between templates, only compositional interpretation of syntactic structure.

	\subsection{Beyond simple alternations}
The three-way distinction analyzed above is instructive but not deterministic, since a given syntactic configuration does not always entail a given template, and a given template does not always entail a given syntactic configuration.

Importantly, while verbs in {\thif} are generally active~(\ref{ex:general}a) and those in {\tnif} generally non-active~(\ref{ex:general}c), verbs in {\tkal} are underspecified with regards to their argument structure, cf.~(\ref{ex:general}b): with some roots, the verb might be transitive; with others, unergative; and with others still, unaccusative, (\nextx).
 \begin{exe}
 \ex \label{ex:kal} 
 \begin{xlist} 
 	\ex  Transitive {\tkal}: 
	
 		\gll  teo \glemph{axal} et ha-laxmanja\\
 		  Theo ate \gsc{ACC} the-bread.roll\\
 		\glt `Theo ate the bread roll.' 
	

 	\ex  Unergative {\tkal}: 
	
 		\gll  teo \glemph{rakad} ve-rakad ve-rakad (kol ha-boker)\\
 		  Theo danced and-danced and-danced all the-morning\\
 		\glt `Theo danced and danced and danced (all morning long).' 
	

 	\ex  Unaccusative {\tkal}: 
	
 		\gll  \glemph{nafal} le-teo ha-bakbuk\\
 		  fell to-Theo the-bottle\\
 		\glt `Theo's bottle fell.' 
	
	
 \z
\z 

This ``flexibility'' of {\tkal} can be explained if Unspecified Voice does not impose any restrictions of its own on argument structure. Then some roots like \root{'xl} in~(\lastx a) do require an external argument, some like \root{r\dgs{k}d} in~(\lastx b) require an external argument but no internal argument (save for cognate objects), and other still like \root{npl} in~(\lastx c) disallow an external argument. The summary in Table~\ref{table:alternations-heb} is augmented in Table~(\ref{table:alternations-heb2}).
\begin{table}
	\begin{tabularx}{\textwidth}{llllllll}
 \lsptoprule
	 & & \multicolumn{2}{c}{\textbf{\vd}}	&	\multicolumn{2}{c}{\textbf{Voice}}	& \multicolumn{2}{c}{\textbf{\vz}}\\
	 & & \multicolumn{2}{c}{Active}		& \multicolumn{2}{c}{Unmarked}		& \multicolumn{2}{c}{Non-active}\\\midrule
	& & \multicolumn{2}{c}{\thif}	&	\multicolumn{2}{c}{\tkal}	& \multicolumn{2}{c}{\tnif}\\
	a.& \root{ktb} & \emph{hextiv}	& `dictated' &	\emph{katav}	& `wrote'	&	\emph{nixtav}	& `was written' \\
	b.& \root{'xl} & \emph{heexil}	& `fed' &	\emph{axal}	& `ate'	&	\emph{neexal}	& `was eaten' \\\tablevspace
	c.& \root{r\dgs{k}d} & \emph{herkid} & `made dance' & \emph{rakad} & `danced' & \multicolumn{2}{c}{---}\\
	d.& \root{nfl} & \emph{hepil} & `dropped' & \emph{nafal}	& `fell' & \multicolumn{2}{c}{---}\\
	e.& \root{ʃbr} & \multicolumn{2}{c}{---} & \emph{ʃavar} & `broke' & \emph{niʃbar} & `was broken'\\
\lspbottomrule
 	\end{tabularx}
	\caption{Basic alternations in Hebrew (extended).}
\label{table:alternations-heb2} 
\end{table}

On this account, verbs in {\thif} are expected to be transitive or unergative because they require an external argument (Chapter~\ref{chap:vd}), verbs in {\tnif} are expected to be mediopassive (anticausative or passive) because they lack an external argument syntactically (Chapter~\ref{chap:vz}), and verbs in {\tkal} could go either way, depending on the idiosyncratic requirements of the root (Chapter~\ref{chap:voice}). The three values of Voice correspond to different morphological markings, but there is more than one way to get e.g.~an anticausative verb (namely with Unspecified Voice or with {\vz}). This way of looking at things dissolves the puzzle posed by examples like those above: there is no reason to expect ``transitive'' to map onto specific morphology deterministically. Any alternations that arise are a by-product of the differences between the heads, or more concretely, between prohibiting an agent, requiring one and allowing one.

This much takes care of three out of seven templates. Two additional templates are easy to explain: the two passive templates {\tpua} and {\thuf} are always derived from existing active verbs and can be analyzed as spelling out an additional passive head Pass merging above VoiceP \citep{doron03,alexiadoudoron12}. This analysis is uncontroversial, as I discuss in Chapter~\ref{passn:pass}. For example, take \emph{huxtav} `was dictated':
 \begin{exe}
\ex  
	\Tree
	[.PassP
		[.Pass ]
		[.VoiceP
			[.{\vd} ]
			[.vP
				[.v
					[.\root{ktb} ]
					[.v ]
				]
				[.DP ]
			]
		]
	]
 \z 

We are left with two more templates, namely {\tpie} and {\thit}. Here I propose that an agentive modifier {\va} combines with the vP to create a new core vP. This [{\va} vP] can then merge with Unspecified Voice, yielding {\tpie}, or with {\vz}, yielding {\thit}, with the predicted alternation between active and anticausative. How this works is the focus of Chapter~\ref{vz:va:vzva}.

	\subsection{From templates to functional heads} \label{intro:sketch:heads}
Before delving into the data we should take stock of the syntactic machinery. The functional head v introduces an event variable and categorizes a root as a verb. A higher functional head, Voice, introduces the external argument. The agentive modifier {\va} overtly introduces agentive semantics whose characterization is set aside until Chapter~\ref{voice:tpie}. I further assume that the functional head \emph{p} introduces the external argument of a preposition, also called its Figure \citep{svenonius03,svenonius07,wood14nllt}. 

Voice and \emph{p} heads introduce a DP in their specifier. In a regular, unmarked active clause, default (silent) Voice introduces the external argument. The head \emph{p} was proposed by \cite{svenonius03,svenonius07} to act in similar fashion to Voice or Chomskyan little \textit{v}: it merges above the PP, introducing the Figure (subject) of the \textbf{p}reposition. I will not attempt to motivate this structure but will simply assume it; it is meant to capture the predicative relationship between the two DPs, similarly to the PredP of \cite{bowers93,bowers01} and \emph{ann}-XP of \cite{mccloskey14}. In~(\nextx) the Figure is the DP \emph{the book} and the Ground, the object of the preposition, is \emph{the table}. Dashed arrows represent assignment of semantic roles; see Chapter~\ref{vz:pz}.
 \begin{exe}
 \ex  
 \begin{xlist} 
 	\ex   
 \Tree
	 [.\emph{p}P
	 	[.DP\\\emph{the book}\\{\tikz{\node (Fig) {\textbf{\textsc{figure}}};}} ]
	 	[
	 		[.{\tikz{\node (p) {\emph{p}};}} ]
	 		[.PP
	 			[.P\\{\tikz{\node (P) {\emph{on}};}} ]
	 			[.DP\\\emph{the table}\\{\tikz{\node (Ground) {\textbf{\textsc{ground}}};}} ]
	 		]
	 	]
	 ]
	\begin{tikzpicture}[overlay]
	\draw[dashed,thick,->] (p) .. controls +(south east:1) and +(east:1) .. (Fig);
	\draw[dashed,thick,->] (P) .. controls +(south west:1) and +(west:1) .. (Ground);
	\end{tikzpicture}
 	\ex      \denote{\emph{p}}  =  λxλs.Figure(x,s)  
 \z
\z 

To these heads I add nonactive counterparts, namely \textbf{\vz} and \textbf{\pz}. These two heads dictate that nothing may be merged in their specifiers. {\vz} blocks the introduction of an external argument and {\pz} blocks merger of a DP in the specifier of \textit{p}P. The different kinds of Voice/\emph{p} only manipulate the syntax: they dictate whether a DP may or may not be merged in their specifier. Both {\vz} and {\pz} are spelled out by the morphophonology of {\tnif}, which adds a prefix and triggers insertion of certain vowels. Voice also has the strongly active counterpart \textbf{\vd}. In its simplest definition, this head requires that a DP be merged in its specifier, behaving the opposite of {\vz}. For completeness we might also assume a covert \textbf{\pd} in some ditransitive verbs, just like overt \emph{p}, although at least in Hebrew there is no empirical motivation for distinguishing \emph{p} from {\pd}.\label{r1:g:2b}

Alongside lexical roots and these functional heads I posit \textbf{\va}. In the semantics, this {element} types the event as an Agent, an Action \citep{doron03} or ``self-propelled'' \citep{folliharley08}. In the phonology, {\va} is spelled out as a predictable set of vowels slotting between the root consonants. It also blocks a process of spirantization which would otherwise apply to the middle consonant of the root.

\textbf{The spell-out} of these heads produces templates as an epiphenomenon. Details are provided in the relevant chapters.

Table~\ref{table:summary-syn-rep} summarizes the syntactic, semantic and morphophonological effects of these heads, as well as the chapters and sections in which these fantastic beasts can be found. Special Voice/\emph{p} heads affect their specifier; see for the external argument (EA) under ``Syntax'' and as a prefix under ``Phonology.'' The effects of the special root {\va} can be seen under ``Semantics'' and as de-spirantization under ``Phonology.'' Note in particular that the {\thit} template is morphologically complex. It is prefixed (overt {\vz}/{\pz}) and de-spirantized ({\va}).
\begin{table}
\fittable{
		\begin{tabularx}{\textwidth}{llllcclc}
 \lsptoprule
				\multicolumn{4}{c}{Heads} & Syn 	& Sem & Phono & Ch\\\midrule
		
				& Voice& &	&  	& 	&  \emph{XaYaZ} & \ref{voice:voice} \\\tablevspace
		
				& Voice&\red{\va}&	& 	& \red{Action}	 & \emph{X{\red{i\dgs{Y}e}}Z}&  \ref{voice:va}	\\
		
				\olive{Pass} & Voice&\red{\va}&	& \olive{Passive}	& \red{Action}	 & \emph{X\olive{u}{\red{\dgs{Y}}}\olive{a}Z}&  \ref{passn:pass:pass}	\\\tablevspace
		
				& \blue{\vd}& &		& \blue{EA}	& 	 & \emph{{\blue{he}}-XY{\blue{i}}Z} & \ref{vd:vd} \\
		
				\olive{Pass} & \blue{\vd}& &		& \olive{Passive}, \blue{EA}	& 	 & \emph{{\blue{h}}\olive{u}-XY\olive{a}Z} & \ref{passn:pass:pass} \\\tablevspace
				
				& \blue{\vz}& &		& \blue{No EA}	& 	 & \multirow{2}{*}{\emph{{\blue{ni}}-XY{\blue{a}}Z}} & \ref{vz:vz} \\
				& Voice& &\blue{\pz}	& \blue{EA = Figure} & 	 &  & \ref{vz:pz} \\\tablevspace
				& \blue{\vz}&\red{\va}&	& \blue{No EA}	& \red{Action}	 & \multirow{2}{*}{\emph{{\blue{hit}}-X{\red{a\dgs{Y}e}}Z} } &  \ref{vz:va:vzva} \\
				& Voice&\red{\va}&\blue{\pz}	& \blue{EA = Figure} & \red{Action}	 & & \ref{vz:va:pzva} \\
\lspbottomrule
 			\end{tabularx}
}
		\caption{Functional heads in the Hebrew verb.}
		\label{table:summary-syn-rep}
	\end{table}

If this last part of the overview went by too quickly, the following chapters will guide us more smoothly through the empirical and theoretical landscape.

Part I of the book is organized as follows. I discuss the template {\tkal}, Unspecified Voice, the template {\tpie} and {\va} in Chapter~\ref{chap:voice}. The templates {\tnif} and {\thit} and their particular head {\vz} are the topic of Chapter~\ref{chap:vz}. Chapter~\ref{chap:vd} discusses {\thif} and {\vd}. With this trivalent system established, Chapter~\ref{chap:passn} embeds these structures under passivizing, adjectivizing and nominalizing heads, providing some cross-categorial context.

Part II provides a detailed comparison of the Trivalent theory with the ``Layering'' theory in Chapter~\ref{chap:aas}. Chapter~\ref{chap:i} then summarizes with general considerations for the valence of Voice.