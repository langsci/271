\chapter{Syntactic vs.~semantic transitivity}
\label{chap:aas}

\section{Introduction} \label{sec:intro}
The first part of this book developed a theory of Voice which recognizes three possible values: {\vd}, {\vz} and \isi{Unspecified Voice}. We have seen that the syntactic features of Voice are correlated with semantic interpretation in ways which themselves are informative: {\vd} introduces a thematic external argument, although this requirement can be voided in de-adjectival and de-verbal inchoatives, indicating that the semantics is still sensitive to syntactic structure (Chapter \ref{chap:vd}); {\vz} does not introduce a syntactic external argument, but it can trigger existential closure over an \isi{Agent}, and {\pz} does introduce a \isi{Figure} role, indicating that the thematic interpretation is sensitive to the extended verbal projection (Chapter \ref{chap:vz}); and Voice places no requirements either in the syntax or in the semantics, although the two are once again correlated (Chapter \ref{chap:voice}).

The Trivalent Theory assumes that every verbal projection contains Voice. In this chapter I would like to highlight some differences between this theory and the one most closely related to it, which I will call for simplicity the \textsc{\isi{Layering}} approach \citep{schaefer08,layering15}. I discuss the two basic premises behind \isi{Layering} in Section~\ref{aas:layering}, and show how they are manifested in the current theory and how they are different in Section~\ref{aas:compare}. The differences will pattern as follows: everything that can be expressed using \isi{Layering} can be expressed in the current approach, but \isi{causative} alternations are beyond the purview of \isi{Layering} and require a Trivalent system. In addition, a number of necessary stipulations are arguably less stipulative in the Trivalent Theory. Sections~\ref{aas:hebrew} and~\ref{aas:jim} verify that \isi{Layering} cannot be applied to the Hebrew data. This comparison between theories sets up expectations for additional crosslinguistic study, to which I turn in the Conclusion (Section~\ref{aas:conc}) and in the next chapter.


\section{Layering} \label{aas:layering}
Many recent syntax-based theories of argument structure adopt two central assumptions which have been most notably defended in the work of \cite{schaefer08} and colleagues \citep{alexiadouetal06,layering15}. These are the shared core for \isi{causative} and anticausative alternants (reviewed in Section~\ref{aas:layering:base}) and the dissociation of syntactic and semantic transitivity\is{transitive} (reviewed in Section~\ref{aas:layering:features}).

	\subsection{Causative core} \label{aas:layering:base}
This component is crucial for the current approach and was reviewed in Section~\ref{intro:arch:layering}. Its main parts are summarized here.

In argument structure alternations such as~(\ref{ex:6:1}), it is not accurate to think that the \isi{transitive} variant is derived from the intransitive one, nor is it accurate to think that the intransitive variant is derived from the \isi{transitive} one. 
 \begin{exe}
 \ex  \label{ex:6:1}
 \begin{xlist} 
 	\ex  Mary broke the vase. 
 	\ex  The vase broke. 
 \z
\z 

\cite{layering15} propose that both variants have the same base: a core vP~(\ref{ex:6n2}a) containing the verb (a verbalized root) and the internal argument. The difference between the two variants is that the transitive one, (\ref{ex:6n2}b), then has the external argument added by additional functional material, namely Voice.

 \begin{exe}
\ex  \label{ex:6n2}
{\columnsep=-.5cm\begin{multicols}{2}\raggedcolumns
\ea
\Tree
		[.vP
			[.\emph{broke} ]
			[.\emph{the glass} ]
		]
\columnbreak\ex \Tree
[.VoiceP
	[.\emph{Mary} ]
	[.
		[.Voice ]
		[.vP
			[.\emph{broke} ]
			[.\emph{the glass} ]
		]
	]
]
\z
\end{multicols}}
 \z 

There is therefore no single direction of derivation which is marked\is{markedness} by the morphology across languages: some languages mark the \isi{transitive} variants, others mark the intransitive variants, and sometimes both variants are marked\is{markedness} in the same language (as we have already seen for Hebrew).

In addition to the morphological reasoning, \cite{layering15} provide a series of arguments showing that the core \isi{causative} component of the vP is present even in the anticausative variants, like the anticausative examples in~(\ref{ex:6n3}) which nevertheless have \isi{Causer} PPs.
 \begin{exe}
 \ex  \label{ex:6n3}
 \begin{xlist} 
 	\ex  The flowers wilted \{from the heat / *from the gardener\}. 
 	\ex  The window cracked \{from the pressure / *from the worker\}. 
 \z
\z 

The \isi{causative} component is thereby dissociated from the external argument, the latter being introduced in an additional structural layer. Voice is the functional head enabling this layer, both in terms of \isi{licensing} Spec,VoiceP in the syntax and in opening the semantic predicate \isi{Agent}. This much suffices to account for the English alternation.

	\subsection{The transitivity of Voice} \label{aas:layering:features}\largerpage[2]
The second tenet of \isi{Layering} is informed by alternations in additional languages. The existence of marked\is{markedness} anticausatives as in~(\ref{ex:6n4}) raises the question of what their morphology is tracking in the syntax.
 \begin{exe}
\ex   
[] 	{\label{ex:6n4} \gll Die T\"ur hat \glemphu{sich} ge\"offnet.\\
 	  the door has \gsc{SICH} opened\\
 	\glt `The door opened.' \hfill (\ili{German}) } 
	
 \z 

\cite{layering15} propose a system in which Voice can be \textsc{transitive} both in syntactic and semantic terms. In the syntax, Voice might be either associated with a specifier or not; in the semantics, it might introduce a thematic Agent role or not. This conceptual innovation is implemented by using a syntactic feature [D], an EPP feature on Voice.\footnote{Unlike the Trivalent system -- in which [D] is a feature that must simply be checked, or intuitively some kind of filter -- the [D] feature in Layering is inherent to Voice and is structure-building, being the only thing that can license/project the specifier position; see Section~\ref{aas:jim}.}

The important consequence is that there are five possible configurations. Four are derived using Voice, depending on whether it is syntactically active and whether it is semantically thematic. The fifth is the complete lack of Voice, as suggested for unmarked\is{markedness} anticausatives. With this basic setup, \isi{Layering} is able to handle cases in which syntactic elements like expletives do not involve thematic roles as well as cases in which the anticausative variant is marked\is{markedness} morphologically, as we will see below. Taken together, the two main components of Layering combine to provide substantial empirical coverage and theoretical insight.

Let us get to the details. Assuming that Voice may or may not have a [D] feature, and that it may or may not introduce an \isi{Agent}, the four-celled typology of \citet[109]{layering15} emerges in Table~\ref{table:typo-layer}.

\begin{table}
\small
\begin{tabular}{cllll}
 \lsptoprule
	& \multicolumn{2}{c}{Syntax D}	& 	\multicolumn{2}{c}{Syntax {\zero}} \\\midrule
Semantics & 	a.&	Thematic active 	&	b.&	Thematic non-active\\
λx 	 & &
\Tree
[.VoiceP 
	[.DP ]
	[.VoiceP
		[.{Voice\{λx, D\}} ]
		[.{vP} ]
	]
]
& &
\Tree
[.VoiceP 
		[.{Voice\{λx, \zero\}\\\gsc{NACT}} ]
		[.{vP} ]
]
\\
&&&&\\\tablevspace
Semantics & 	c.&	Expletive active 	&	d.&	Expletive non-active\\
{\zero}	 & &
\Tree
[.VoiceP 
	[.DP\\\gsc{SE} ]
	[.VoiceP
		[.{Voice\{\zero, D\}} ]
		[.{vP} ]
	]
]
& &
\Tree
[.VoiceP 
		[.{Voice\{\zero, \zero\}\\\gsc{NACT}} ]
		[.{vP} ]
]
\\
\lspbottomrule
 \end{tabular}
	\caption{The typology of the Layering approach\label{table:typo-layer}}
\end{table}

This typology can be augmented by including the Voiceless unmarked\is{markedness} anticausative, vP, giving us the full table in Table~\ref{table:typo-layer-all}.

\begin{table}
\fittable{\begin{tabular}{cllllll}
 \lsptoprule
	& \multicolumn{2}{L{4cm}}{Syntax D}	&  \multicolumn{2}{L{1.5cm}}{vP}	& \multicolumn{2}{L{4cm}}{Syntax {\zero}} \\\midrule
Semantics	 & 		a.	&	&			b.	&& 	c. & \\
λx 	 & 
&\Tree
[.VoiceP 
	[.DP ]
	[.
		[.{Voice\{λx, D\}} ]
		[.vP ]
	]
]
& 
& --- %\phantom{Undefined.}
&& \Tree
[.VoiceP 
		[.{Voice\{λx, \zero\}\\\gsc{NACT}} ]
		[.vP ]
]
\\\tablevspace
Semantics	 & 		d.		& &			e.	& &	f. & \\
\zero	 &
& \Tree
[.VoiceP 
	[.DP\\\gsc{SE} ]
	[.VoiceP
		[.{Voice\{\zero, D\}} ]
		[.vP ]
	]
]
&
&\Tree
		[.vP ]
&
&\Tree
[.VoiceP 
		[.{Voice\{\zero, \zero\}\\\gsc{NACT}} ]
		[.vP ]
]
\\
\lspbottomrule
 \end{tabular}}
	\caption{The typology of Voice under Layering\label{table:typo-layer-all}}
\end{table}

I examine the cells one by one. The structure in cell~a is a straightforward \isi{transitive} derivation, at least since \cite{kratzer96} and \cite{pylkkanen08}. The [D] feature on Voice licenses a DP in its specifier, and the agent\is{Agent} role is introduced in the semantics (notated here simply as λx).

The structure in cell~f of Table~\ref{table:typo-layer-all} derives a marked\is{markedness} anticausative, similar to {\tnif} from Section~\ref{vz:vz}. There is no [D] feature, so no DP can be merged in Spec,Voice. The lack of a specifier is spelled out as non-active morphology, \gsc{NACT} in short, via a rule we return to in~(\ref{ex:aas:vi-nact}). No agent\is{Agent} is introduced in the semantics. The result is a marked\is{markedness} anticausative (unaccusative) construction. The unmarked\is{markedness} anticausative is derived in cell~e, by not merging any Voice head at all. Since no Voice head exists in the structure, there is no agentive semantics either, so cell~b is undefined.

The particularly interesting cases are those in which we find ``mismatches'' between the values of the syntactic feature and semantic specification, namely cells~c--d. Starting with cell~d we have a situation in which no agentive semantics is introduced but Voice still requires a specifier. The \isi{Layering} analysis proposes that this is the situation for the Romance\il{French}\il{Spanish} expletive \gsc{SE} and the \ili{German} \emph{sich}, which appear in marked\is{markedness} anticausatives but contribute nothing to the semantics. Similar analyses have been proposed for \ili{Icelandic} by \cite{wood14nllt,wood15springer} and for various phenomena in English and \ili{Quechua} by \cite{myler16mit}.

Finally, the configuration in cell~c is also possible. Here, Voice does not have a [D] feature and does not project a specifier. However, it does introduce a thematic role. \cite{layering15} propose that this is the correct analysis of \isi{passive} verbs in \ili{Greek}, which are morphologically identical to anticausatives; the analysis captures the fact that the morphology of cells~c and~f is identical, since in neither case is a specifier projected. The open predicate must presumably be closed off by existential closure later in the derivation. It is worth keeping in mind that regardless of combination with thematic non-active Voice, every root must still need to state whether the \gsc{NACT} variant will be anticausative, \isi{passive}, or compatible with both \citep[88]{alexiadouanagnostopoulou04,layering15}.


\section{Comparison} \label{aas:compare}
The Trivalent Theory differs from \isi{Layering} in two concrete ways. First, \isi{Layering} assumes that no Voice layer is projected for unmarked\is{markedness} non-active constructions. I assume that a VoiceP is always projected but that its specifier might not be filled. We have seen this in the difference between active and non-active verbs in {\tkal} (Chapter \ref{chap:voice}), and in the difference between {\vz} and {\vd} (Chapters \ref{chap:vz}--\ref{chap:vd}). The morphological reflexes of this difference are briefly highlighted in Section~\ref{aas:compare:vi-nact}. The second difference  is more substantial, building on the first: there are three possible values of the [D] features, closely associated with semantic interpretation (Section~\ref{aas:compare:features}).

	\subsection{Non-active layers} \label{aas:compare:vi-nact}
Marked anticausatives show consistent morphological marking on the anticausative member of an alternation. The \ili{Latin} example in~(\ref{ex:6n5}) is adapted from sources cited in~\cite{kastnerzu17}:
 \begin{exe}
\ex   \langinfo{Latin}{}{\citealt[662]{kastnerzu17}}\label{ex:6n5}\\
  \gll vulnus \glemph{claudi-t-ur}.\\
 	  wound.\gsc{NOM} close-\gsc{3SG}-\gsc{NACT}\\
 	\glt `The wound heals.' 
	
 \z 

To account for the appearance of non-active morphology in marked\is{markedness} anticausatives, \isi{Layering} proposes the rule in~(\ref{ex:aas:vi-nact}), following \cite{embick04}:
 \begin{exe}
\ex \label{ex:aas:vi-nact}Voice \lra~\gsc{NACT} / \trace~No Spec 
 \z 

The cases in cells~c and~f of Table~\ref{table:typo-layer-all} can both be accounted for using the rule in~(\ref{ex:aas:vi-nact}), if we assume that {\vz} is in fact the Voice head in those structures. A theory of Vocabulary Insertion which allows~(\ref{ex:aas:vi-nact}) must then be able to make reference to syntactic contexts such as ``lack of a specifier''. While this is not impossible, it does complicate the theory somewhat.\label{r1:6:1}

I have made a different claim: non-active morphology such as \gsc{NACT} and {\tnif} is the spell-out of {\vz}. In other words, it is the flavor of Voice which is spelled out as non-active morphology, not Voice when it has no specifier. Recall the reason for this preference: \isi{Unspecified Voice} in Hebrew is spelled out as {\tkal} regardless of whether it has a specifier or not. The spell-out rule in~(\ref{ex:6n6}) is thus more consistent crosslinguistically.
 \begin{exe}
\ex  \label{ex:6n6}{\vz} \lra~\gsc{NACT} \hfill (always No Spec) 
 \z 

This technical difference aside, in what follows I address more substantive differences between the theories.

	
	\subsection{The trivalency of transitivity} \label{aas:compare:features}
The two systems ended up looking as shown in Tables~\ref{table:typo-layer-all2} and~\ref{table:typo-feat}.

\begin{table}
\small
	\begin{tabular}{cllllll}
		\lsptoprule
		& \multicolumn{2}{L{4cm}}{Syntax D}	&  \multicolumn{2}{L{1.5cm}}{vP}	& \multicolumn{2}{L{4cm}}{Syntax {\zero}} \\\midrule
		Sem	 & 		a.	&	&			b.	&& 	c. & \\
		λx 	 & 
		&\Tree
		[.VoiceP 
		[.DP ]
		[.
		[.{Voice\{λx, D\}} ]
		[.vP ]
		]
		]
		& 
		& --- %\phantom{Undefined.}
		&& \Tree
		[.VoiceP 
		[.{Voice\{λx, \zero\}\\\gsc{NACT}} ]
		[.vP ]
		]
		\\\tablevspace
		Sem	 & 		d.		& &			e.	& &	f. & \\
		\zero	 &
		& \Tree
		[.VoiceP 
		[.DP\\\gsc{SE} ]
		[.VoiceP
		[.{Voice\{\zero, D\}} ]
		[.vP ]
		]
		]
		&
		&\Tree
		[.vP ]
		&
		&\Tree
		[.VoiceP 
		[.{Voice\{\zero, \zero\}\\\gsc{NACT}} ]
		[.vP ]
		]
		\\
		\lspbottomrule
	\end{tabular}
	\caption{The typology of Voice under Layering\label{table:typo-layer-all2}}
\end{table}

\begin{table}
\fittable{\begin{tabular}{cllllll}
 \lsptoprule
	& \multicolumn{2}{l}{\vd}	&  \multicolumn{2}{l}{Voice}	& \multicolumn{2}{l}{\vz} \\\midrule
Sem	 & 		a.	&	&			b.	&& 	c. & \\
λx 	 & 
&\Tree
[.VoiceP 
	[.DP ]
	[.
		[.{\vd} ]
		[.vP ]
	]
]
& 
&\Tree
[.VoiceP 
	[.DP ]
	[.
		[.Voice ]
		[.vP ]
	]
]
&& (Figure reflexives) 
\\\tablevspace
Sem	 & 		d.		& &			e.	& &	f. & \\
\zero/∃x	 & 
& ({\vd} inchoatives) 
&
&\Tree
[.VoiceP
	[.(\gsc{SE}) ]
	[.
		[.Voice ]
		[.vP ]
	]
]
&
&\Tree
	[.VoiceP
		[.{\vz} ]
		[.vP ]
	]
\\
\lspbottomrule
 \end{tabular}}
	\caption{The Trivalent typology\label{table:typo-feat}}	
\end{table}

While the semantics of the cells in \isi{Layering} is deterministic (modulo existential closure), the Trivalent Theory relies on contextual \isi{allosemy} of the Voice head.\footnote{\isi{Layering} can also be formalized using \isi{allosemy}, as in~\cite{schaefer17oup}.} Its semantics looks broadly as in~(\ref{ex:6n8}), summarizing what we have seen in previous chapters:

 \begin{exe}
 \ex  \label{ex:6n8}Semantics (abstracting away from Agent $\neq$ Causer): 
 \begin{xlist} 
 	\ex  \denote{\vd} = λxλe.Agent(x,e) 
 	\ex  \denote{Voice}\phantom{.......} = $\begin{cases} 
		\text{λxλe.Agent(x,e)} & \text{/ \trace \{\root{\gsc{eat}}, \dots\} }\\
		\text{λe.e} & \text{/ \trace \{\root{\gsc{fall}}, \dots\} }\\
	\end{cases}$
 	\ex  \denote{\vz}\phantom{.} = $\begin{cases} 
		\text{λe∃x.Agent(x,e)} & \text{/ \trace \{\root{\gsc{write}}, \dots\} }\\
		\text{λe.e} & \\
	\end{cases}$
 \z
\z 

There are two points to be made about how powerful the current approach is: first, that it has all the empirical coverage necessary for the \isi{Layering} patterns. Very little has to be said in order to maintain the coverage of \isi{Layering} as applied to English, German and Greek. For instance, expletive constructions in Germanic and Romance are derived by simply adding the expletive in Spec,VoiceP, as in cell~e of Table~\ref{table:typo-feat}.

\label{r1:6:2}The second is that this power actually comes from a system that is just as constrained as Layering (if not more so). While in Layering all features may combine freely, in the Trivalent Theory semantic interpretation tracks the syntactic feature on the functional head: barring exceptional cases, the active head {\vd} has an agentive reading and the non-active head {\vz} has either a non-active reading or a passive reading. So {\vd} does not have a straightforward non-active alloseme, and {\vz} is the only one with a passive alloseme. In this sense, at least, the interpretation of these heads is natural.\largerpage[-2]

I take this correlation to be a welcome result, though I will not attempt to derive how the syntax feeds the semantics in this way. With that in mind, however, one might still wonder whether cells~c and~e in Table~\ref{table:typo-feat} could be possible. In fact, we have already seen that these configurations are possible, but only when additional \emph{syntactic} constraints are at play.

Section \ref{vz:pz} analyzed \isi{figure reflexives} in {\tnif}. This was a situation in which a [\textminus{}D] head introduced an external argument, specifically the \isi{Figure} role of {\pz}. As noted in Section~\ref{vz:interim}, {\pz} and {\vz} may be considered contextual variants of each other and of the generalized head \emph{\isi{i*}}; I return to this point in Chapter~\ref{chap:i}. Still, why can {\pz} introduce a thematic role? One answer can be found in the work of \cite{wood15springer}. He suggests that the allosemic sensitivity of a head depends on its place in the extended projection of the verb. Metaphorically speaking, since {\pz} ``knows'' that it is not the last head in the VoiceP, it can introduce a \isi{Figure} role knowing that this role can be saturated later on. Importantly, this is not a case of lookahead: the derivation will crash if no DP is merged to saturate that role. So a generalized version of cells~c and~e is possible after all, if the syntactic configuration is just right (as expected).\footnote{As mentioned earlier, \cite{legate14} and \cite{akkus19jl} suggest that the \isi{Agent} role can be introduced and then closed off. Perhaps deponent verbs can also be treated in similar fashion.}

The second case is cell~d in Table~\ref{table:typo-feat}, a situation in which a \isi{causative}-marked\is{markedness} verb turns out to be inchoative. As we saw in Section~\ref{vd:vd}, this is no hypothetical. Inchoatives do exist in {\thif}, but only in specific syntactic configurations (when the verb is de-adjectival or de-nominal). The allosemic rule in Chapter~\ref{chap:vd}, (\ref{ex:vd:sem-full}), stated this explicitly.\footnote{One crosslinguistic correlate might be the \textsc{adversity \isi{causative}} of \ili{Japanese} \citep{pylkkanen08,woodmarantz17}, where the Voice head itself is potentially {\vd} (see Section~\ref{i:i:jap}) but does not have its own agentive semantics, instead taking a possessor role passed up from lower in the tree.}

As a final point of comparison before returning to Hebrew, it is important to consider how \isi{Layering} allows languages to pick and choose between features to be combined. For example, \ili{Greek} passives are derived as in Table~\ref{table:typo-layer-all2}, where existential closure applies to the open \isi{Agent} role. \cite{schaefer17oup} later adopts a position similar to the one here, whereby ∃x is another possible semantic value for Voice heads, thus removing the need for existential closure of λx. 

Either way, given that existential closure can apply at some level as in Greek, the question arises of why it does not apply in situations where an overt DP appears. Specifically, there is nothing to prevent an expletive such as German \emph{sich} from being the DP in cell~a of Table~\ref{table:typo-layer-all2}. The expletive would not have a semantic role to saturate, but an \isi{Agent} would still be entailed. The result should be a construction with an expletive whose reading is not anticausative but \isi{passive}. Yet this is impossible in \ili{German}:\largerpage[-2]

 \begin{exe}
\ex  
[] 	{\label{ex:6n9} \gll Die T\"ur hat sich (*absichtlich) ge\"offnet, (*um das Zimmer zu l\"uften).\\
 	  the door has \gsc{SICH} \phantom{(*}on.purpose opened \phantom{(*}in.order the room to air.out\\
 	\glt (int. `The door opened on purpose and/or in order to air out the room.') } 
	
 \z 

What is relevant in this regard is that Greek and German might avail themselves of different cells of the typology. Specifically, German can be argued not to have ∃x Voice heads (passivization applies above the VoiceP in German; cf. Section~\ref{passn:pass}). \cite{schaefer17oup} discusses similar cases of passivization in depth, concluding that ∃x is a necessary semantic possibility for Voice heads (as already mentioned above) and providing an analysis explaining why \ili{French} and other languages do allow passives as in~(\ref{ex:6n9}), albeit without \emph{by}-phrases. 

The problem is, then, that e.g.~French might have Voice\{∃x, D\} for these passives but does not have Voice\{∃x, \zero\}; the selection of features from the universal pool appears arbitrary. A similar problem arises for \isi{Layering} when turning to \isi{causative} marking: we would expect that a language with \isi{causative} marking could combine it with an expletive. This does not seem to be correct, although I have not conducted enough crosslinguistic work to make this assertion conclusively. 

In the Trivalent Theory, these issues do not arise, because the dichotomy of thematic/expletive Voice is abandoned, as is the idea that languages pick only a subset of features to instantiate across cells. Instead, Voice is allosemic in ways which are constrained both by the root and by the feature [D].


\section{Hebrew with Layering} \label{aas:hebrew}
The last issue to be tackled here is whether the Trivalent Theory is a necessary development. Could we tweak \isi{Layering} to account for the patterns analyzed in this book?

Recall that Hebrew has trivalent morphological marking, and, crucially, that verbs in {\tkal} might be unaccusative or \isi{transitive} (Chapter~\ref{chap:voice}); see Table~\ref{table:aas:alternations-heb2}.

\begin{table}
	\begin{tabular}{llllll}
 \lsptoprule
	 \multicolumn{2}{c}{\vd}	&	\multicolumn{2}{c}{Voice}	& \multicolumn{2}{c}{\vz}\\\midrule
	\multicolumn{2}{c}{\thif}	&	\multicolumn{2}{c}{\tkal}	& \multicolumn{2}{c}{\tnif}\\
	\emph{heexil}	& `fed' &	\emph{axal}	& `ate'	&	\emph{neexal}	& `was eaten' \\
	\emph{hextiv}	& `dictated' &	\emph{katav}	& `wrote'	&	\emph{nixtav}	& `was written' \\ 
	\emph{hepil} & `dropped' & \emph{nafal}	& `fell' & \multicolumn{2}{c}{---}\\
\lspbottomrule
 	\end{tabular}
		\caption{Basic analysis of the templates as proposed in this book\label{table:aas:alternations-heb2}}
\end{table}

Trying to analyze Hebrew using the machinery of \isi{Layering} will require us to take {\tkal} as the spell-out of v, not Voice as in Chapter~\ref{chap:voice}. Then, the distinction between active and non-active Voice would derive the distinction between active verbs in {\tkal} and verbs in {\tnif}. To derive the active verbs in {\thif}, additional functional structure would be necessary (since there are only two Voice heads under \isi{Layering}, regular/\isi{transitive} and non-active). This alternative approach to Hebrew is summarized in Table~\ref{tab:6-4:layer}.

\begin{sidewaystable}
\begin{tabularx}{\textwidth}{lcccc}
 \lsptoprule
			&	unmarked anticausative	&	unmarked transitive &	marked anticausative	& marked transitive\\\midrule
		Derivation					& \Tree [.vP ] 		&	\Tree [.VoiceP [.DP ] [ [.Voice ] [.vP ] ] ]	&	\Tree [.VoiceP [.{Voice\{\zero, \zero\}} ] [.vP ] ] 	& \Tree [.\gsc{CAUS}P [.\gsc{CAUS} ] [. [.DP ] [ [.Voice ] [.vP ] ] ] ] \\
		Spell-out					& \multicolumn{1}{c}{\tkal}	&	{\tkal}					& {\tnif}	& \thif\\
\lspbottomrule
 	\end{tabularx}
	\caption{Layering-style analysis of Hebrew (to be rejected)}
	\label{tab:6-4:layer}
\end{sidewaystable}

I can identify a number of problems with this approach. First, it is not possible to treat {\tkal} akin to English or Greek unmarked\is{markedness} alternations because {\tkal} does not have the zero-alternation. If \emph{kafa} `froze' is an unaccusative verb derived without Voice, adding Voice should simply give us \isi{transitive} `froze' with identical pronunciation, contrary to fact. While it is true that various constraints dictate whether zero-derivation is possible in a given language, it is striking that the alternation is not possible in Hebrew (setting aside the discussion in Section~\ref{vd:caus:labile}). This version of \isi{Layering} predicts that zero-alternation should be fairly prevalent. Note that this point is crucial to maintaining the current view. If I am wrong about this, the way is paved for a theory of Hebrew-as-Greek consisting of vP, VoiceP and {\vz}P (without {\vd}).

Second, there is no convincing argument for positing extra structure in {\thif} (``lexical'' causatives, Section~\ref{vd:caus:mrkd}). This template seems to be as integrated into the morphological system as any other, meaning that {\vd} is as integrated into the system as the other heads. In \cite{kastner18nllt} I explained how the current system derives a number of allomorphic interactions correctly. The behavior of {\vd} with regard to constraints such as \isi{locality} in \isi{allomorphy} is qualitatively identical to that of {\vz} and Voice. Adding structure for {\thif} would lose a number of morphophonological generalizations regarding the interplay of roots and functional structure.

Third, it is unclear what the relevant function of an additional head would be. As a  distinct \isi{causative} head, it would be an odd type of causativizer, since it would not necessarily add any argument; it would take a \isi{transitive} structure and turn it into a different \isi{transitive} structure. As discussed in Section~\ref{vd:caus}, \isi{transitive} verbs in {\thif} are lexical causatives, not analytic causatives.

And fourth, \isi{nominalizations} clearly contain the morphology of the underlying verbal template (Section~\ref{passn:n}). But then why should deverbal nouns be derived from Voice when they can all be derived from v?

We could also imagine a mixed view, under which what I have called {\vz} is not a syntactic requirement on Spec,Voice but a semantic one: {\vz} simply has only the non-active allosemes, but does not ban DPs in its specifier as such. This would mean, for example, that in the reflexive derivation of Section~\ref{vz:va:vzva:refl} the \isi{Theme} raises to Spec,{\vz} and not to Spec,TP. Such a view also opens the door for derivations in which an internal argument raises through Spec,VoiceP, as has been proposed for some ergative languages \citep{deal19li}. The problem with such a system is conceptual, in that {\vz} now has no syntactic feature distinguishing it from \isi{Unspecified Voice}. Given that \isi{Unspecified Voice} has a non-active alloseme, it is not clear what this semantic {\vz} would be signaling to the learner. This view also severs the similarity between modified {\vz} (which has no syntactic requirements and does not introduce a thematic role) and {\pz} or a modified {\pz} (which has a syntactic requirement and does introduce a thematic role).\footnote{Thanks to Yining Nie for noting this possibility and its implications.}

One final alternative would maintain the basics of \isi{Layering} while placing an emphasis on processes of \isi{Impoverishment}. This possibility is discussed next.

\section{An alternative with Impoverishment} \label{aas:jim}

\label{r1:g:2c2}As mentioned already, in the Trivalent system [D] acts as a feature that needs to be checked or as a sort of ``filter'', rather than as a structure-building feature. Work in the Layering tradition -- most explicitly that of \cite{schaefer08} and \cite{wood15springer} -- differentiates between Voice with \{D\}, which necessarily projects a specifier, and Voice with empty \{\}, which does not. Recent discussions with Jim Wood (p.c.~Sep--Nov 2019) helped clarify how this specific view of [D] and Merge could be maintained in the face of the Hebrew data. I present this alternative first in Section~\ref{aas:jim:pros}, and then list my reasons for rejecting it in Section~\ref{aas:jim:cons}.

Since much of the discussion will have to do with \textsc{triplets}, let us recall the empirical picture. The most complicated cases are those where a given root occurs in the three templates {\tkal}, {\tnif} and {\thif}. These are the ones I take to be simplest structurally, as they do not involve {\va} or Pass. As far as I know, there is no curated list of all such triplets in Hebrew. Searching the database of \cite{ehrenfeld12}, I found 147 roots that are instantiated in all three templates out of 1,875 roots in total. Not all of these make for clear triplets, and so I searched for good examples by hand. Table~\ref{table:aas:triplets} (page \pageref{table:aas:triplets}) lists the ten clearest cases I have found, in which a semantic relationship holds between all three forms and at least two of these are transparently related.

\begin{table}
\fittable{\begin{tabular}{ll>{\itshape}ll>{\itshape}ll>{\itshape}ll}
\lsptoprule
& Root & \multicolumn{2}{c}{\tkal} &	\multicolumn{2}{c}{\tnif} & \multicolumn{2}{c}{\thif}\\\midrule
a.& \root{axl} & axal 	& `ate' 	& neexal 	& `was eaten' 	& heexil 	& `fed' \\
b.& \root{xʃb} & xaʃav 	& `thought' 	& nixʃav 	& `was considered' 	& hexʃiv 	& `considered'\\
c.& \root{jda} & jada 	& `knew' 	& noda 	& `was known' 	& hodia 	&`announced'\\
d.& \root{ktb} & katav 	& `wrote' 	& nixtav 	& `was written' 	& hextiv 	& `dictated'\\
e.& \root{m{\ts}a} & matsa 	& `found' 	& nimtsa 	& `was found' 	& hemtsi 	& `invented'\\
f.& \root{sgr} & sagar 	& `closed' 	& nisgar 	& `was closed' 	& hesgir 	& `extradited'\\
g.& \root{ark} & arax 	& `edited' 	& neerax 	& `was edited' 	& heerix 	& `estimated'\\
h.& \root{pnj} & pana 	& `faced' 	& nifna 	& `turned towards' 	& hifna 	& `directed to'\\
i.& \root{krj} & kara 	& `read' 	& nikra 	& `was read' 	& hekri 	& `read out'\\
j.& \root{raj} & raa 	& `saw' 	& nira 	& `was seen' 	& hera 	& `showed'\\
\lspbottomrule
\end{tabular}}
    \caption{Derivational triplets in Hebrew\label{table:aas:triplets}}
\end{table}

	\subsection{An Impoverished Layering Theory of Hebrew} \label{aas:jim:pros}
		\subsubsection{Basics}
This alternative attempts to maintain the structure-building view of [D], whereby there is no \isi{Unspecified Voice}, only {\vds} which projects a specifier, and {\vzs} which does not. The [D] feature, like any other feature, can undergo \isi{Impoverishment}.

The most basic spell-out rules, to be revised immediately, are in~(\ref{ex:6n10}). The \isi{transitive} Voice head is spelled out as the ``\isi{causative}'' template {\thif}, and the non-active Voice head as the non-active template (\tnif).

 \begin{exe}
 \ex  \label{ex:6n10}Initial VIs (to be revised) 
 \begin{xlist} 
 	\ex  {\vds} \lra~{\thif} 
 	\ex  {\vzs} \lra~{\tnif} 
 \z
\z 

We also know that some roots simply need to appear in certain templates. In particular, some agentive verbs do not appear in {\thif} but in {\tkal} (Chapter~\ref{chap:vd}), and some unaccusative verbs do not appear in {\tnif} but in {\tkal} (Chapter~\ref{chap:vz}). Calling these simply \root{Root1}, \root{Root2} and so on for the time being, we have the revised VIs in~(\ref{ex:6n11}).\footnote{The listed roots could appear under either the marked\is{markedness} or unmarked\is{markedness} template in each case; I give them in the marked\is{markedness} cases here.}

 \begin{exe}
 \ex  \label{ex:6n11}Revised VIs (to be revised further) 
 \begin{xlist} 
 	\ex  {\vds} \lra~$\begin{cases} 
		\text{\thif} & \text{/ \trace~\root{Root1}, \root{Root2}, \dots}\\
		\text{\tkal} & \\
		\end{cases}$
 	\ex  {\vzs} \lra~$\begin{cases} 
		\text{\tnif} & \text{/ \trace~\root{Root3}, \root{Root4}, \dots}\\
		\text{\tkal} & \\
		\end{cases}$
 \z
\z 


		\subsubsection{Appl}
It is now that the challenge posed by triplets can be re-introduced. Here the theory makes use of the \isi{applicative} head Appl\is{applicative}. The intuition is that when an alternation holds between {\tkal} and {\thif}, the latter form can be derived by using an Appl\is{applicative} head.

Since I have spent some time discussing the relationship between {\tkal} and {\thif} in Chapter~\ref{chap:vd}, I will not repeat the details here; see the list in Table~\ref{table:vd:triplets-caus} for some examples. This idea does receive empirical support from pairs like those in rows~a and~d of Table~\ref{table:aas:triplets}, where `feed' and `dictate' arguably take an additional argument when compared to `eat' and `write'. But it is a bit more of a stretch with cases like row~e, \emph{daxak} `shoved' $\sim$ \emph{hedxik} `suppressed (emotions)'. In this case, the latter would have to involve some kind of low Appl\is{applicative}-\gsc{INTO}, and it would be far less clear what it means to be an applied argument.

In any case, if we were to accept this premise, we would have the revised VIs in~(\ref{ex:6n12}), again not a final proposal.

 \begin{exe}
 \ex  \label{ex:6n12}Revised VIs (pre-final version) 
 \begin{xlist} 
 	\ex  {\vds} \lra~$\begin{cases} 
		\text{\thif} & \text{/ \trace~\textbf{Appl}, \root{Root1}, \root{Root2}, \dots}\\
		\text{\tkal} & \\
		\end{cases}$
 	\ex  {\vzs} \lra~$\begin{cases} 
		\text{\tnif} & \text{/ \trace~\root{Root3}, \root{Root4}, \dots}\\
		\text{\tkal} & \\
		\end{cases}$
 \z
\z 

		\subsubsection{Additional diacritic}
Finally, we need to account for the triplets that cannot be handled with Appl\is{applicative}. Consider a triplet like that in row~e of Table~\ref{table:aas:triplets}, where it is highly doubtful that `invent' is an \isi{applicative} version of `find'. Since this root appears with both templates, they cannot be differentiated by listing a \root{Root5} in both cases for {\vds}. Some additional diacritic (or feature) would be necessary, call it simply F like in~(\ref{aas:ex:jim-vis}).

 \begin{exe}
 \ex  \label{aas:ex:jim-vis}VIs in the Impoverishment alternative (final version) 
 \begin{xlist} 
 	\ex  {\vds} \lra~$\begin{cases} 
		\text{\thif} & \text{/ \trace~\textbf{F}, Appl, \root{Root1}, \root{Root2}, \dots}\\
		\text{\tkal} & \\
		\end{cases}$
 	\ex  {\vzs} \lra~$\begin{cases} 
		\text{\tnif} & \text{/ \trace~\root{Root3}, \root{Root4}, \dots}\\
		\text{\tkal} & \\
		\end{cases}$
 \z
\z 

	
	\subsection{Discussion} \label{aas:jim:cons}
I have tried to lay out this alternative as explicitly as possible so that its strengths and weaknesses may be evaluated. The main gain would be a theory-internal one: as mentioned at the outset, a theory with only {\vds} and {\vzs} preserves a specific conceptualization of \isi{Merge}, one which is no longer available once \isi{Unspecified Voice} enters the picture in a Trivalent system. An additional benefit is a closer connection with extant theories insofar as Appl\is{applicative} can be used in similar fashion.

These strengths are outweighed by the weaknesses, in my eyes, and these are of a conceptual as well as empirical nature. Starting with the use of \isi{Impoverishment}, the following points arise. First, since the choice of template for a given root has syntactic an semantic effects, this means that \isi{Impoverishment} would have to apply in the syntax proper and not early in Spell-Out, as commonly assumed, where it does not have semantic effects (e.g.~\citealt{harbour03}). Second, \isi{Impoverishment} would need to be triggered by particular roots and not by marked\is{markedness} features or feature combinations. Third, this would only happen some of the time, because many roots can appear in more than one template, e.g.~in both {\thif} and {\tkal}. This last point could be discounted due to the use of Appl\is{applicative}, which is the next point of discussion.

As already mentioned, the definition of the semantics of Appl\is{applicative} might be stretch\-ed fairly thin, depending on which specific cases it should be applied to. In addition, even in the cases in which an \isi{applicative} \emph{semantics} is easier to motivate, an \isi{applicative} \emph{syntax} is only optional. That is to say, even in row~i of Table~\ref{table:aas:triplets} -- \emph{hekri} `read out' -- a Goal argument does not have to be expressed. This issue could be addressed by assuming that an expletive Appl\is{applicative}$_{\text{\{\}}}$ is possible (\citealt{wood15springer}; p.c.) with no argument introduced in its specifier.

As a last conceptual point, the system as a whole makes reference to an additional mechanism sorting out triplets, namely the diacritic/feature [F]. Taken together, this approach is increasingly reminiscent of \cite{arad05}, where the grammar lists the conjugation\is{conjugation class} classes a given root participates in. It is also noteworthy that Appl\is{applicative} and [F] form a natural class somehow. 

Finally, there is also one important empirical point: this alternative is able to derive labile alternations in {\tkal} as a general rule. Imagine a root \root{Root5} which is not on either of the lists for {\vds} and {\vzs} in~(\ref{aas:ex:jim-vis}). Then it would be spelled out as {\tkal} for {\vds}, but also as {\tkal} for {\vzs}. As noted in Section~\ref{vd:thif:inch}, this is not the case with any verb except for perhaps \emph{a{\ts}ar} `stopped'.

In conclusion, even though I have identified various reasons to doubt a \isi{Layering} approach to Hebrew (whether implemented as in this section or as in Section~\ref{aas:hebrew}), it is important to acknowledge that not all of the explanations given here are particularly deep. For instance, I have implicitly assumed that all Hebrew verbs need Voice, in contrast to existing assumptions for certain verbs in English, German and Greek. This assumption raises the question of whether Voice should be obligatory for all verbs in all languages, a point leading us to the concluding remarks for this comparison.


\section{Conclusion} \label{aas:conc}
This chapter presented a direct comparison of the theory developed in this book with what I have called the \isi{Layering} approach, the prevalent syntax-based theory of transitivity\is{transitive} alternations as implemented by \cite{schaefer08,schaefer17oup} and \cite{layering15}. I have identified a number of weaknesses with the \isi{Layering} approach and illustrated how its considerable explanatory power can be mirrored in the Trivalent approach. Furthermore, I have identified cases which require a concrete departure from the features of \isi{Layering}.

Aside from the specific weaknesses discussed here, the main empirical difference underlying the most substantial need for a revised theory is that the \isi{Layering} Theories were based on an exploration of anticausative marking, not of \isi{causative} marking (see the discussion in Chapter \ref{chap:vd}). The languages on which this approach is based show anticausative marking, including English \citep{myler16mit}, German \citep{schaefer17oup} and Greek \citep{spathasetal15}, but also Albanian \citep{kallulli13}, Icelandic \citep{wood15springer}, Latin \citep{embick04,kastnerzu17} and Spanish \citep{schaefervivanco16}.

The theory developed here on the basis of Hebrew makes explicit room to accommodate \isi{causative} marking. The Trivalent View of Voice is most useful when considering languages that show reflexes of this marking, including Japanese \citep{oseki17nyu} as well as a number of Austronesian and Polynesian languages \citep{nie17}. In other words, it becomes clear that \isi{causative} marking has much to tell us about argument structure alternations, alongside anticausative marking and, ideally, in a joint theory like the one presented thus far.

To conclude, let us put the pieces together and speculate on what the Voice inventory of a given language might be. I see three possibilities.

On the one hand, it is possible that all languages have the Trivalent system of Table~\ref{table:typo-feat}. We would then assume that in English, German and so on {\vd} and Voice are syncretic. On the other hand, it might be the case that only Voice heads that are morphophonologically distinct can be argued to exist in a given language. This is essentially the view of \cite{layering15}, who proposed that learners of English do not hypothesize the existence of expletive Voice because there is no morphological evidence for it. If this is the case, then languages with marked\is{markedness} anticausatives and marked\is{markedness} causatives are Trivalent languages, whereas languages with only marked\is{markedness} anticausatives are \isi{Layering} languages. Finally, one could also come up with a hybrid view, in which all languages are at least active/non-active \isi{Layering} languages, even when there is no morphological evidence (as in English), following from the basic active/non-active distinction that came up in the context of \isi{causative} marking (Section~\ref{vd:caus}).

I will not argue for any of these views explicitly, although I do maintain that the Trivalent Theory is simultaneously the most constrained and the most flexible (whether this flexibility means that Trivalent Voice should be hard-coded into the grammar is debatable). In addition, treating transitivity\is{transitive} alternations in terms of various features on Voice -- extending the original \isi{Layering} view -- paves the way for a more nuanced view of what these features might be. In Section~\ref{vd:caus:markvoice} I speculated that French could be treated as a Trivalent language if certain prefixes spell out {\vd}. I do not know if that view can be maintained -- it probably cannot -- but it does highlight what new perspectives can be gained by looking at different argument structure phenomena in terms of Layering with a certain feature set. In the next chapter I consider recent proposals that extend the coverage of feature-based approaches beyond transitivity\is{transitive} marking, by considering their interaction with \isi{case} and agreement.
