\chapter{The features of Voice}
\label{chap:i}

This book presents a theory of argument structure and its relation to morphology which was developed with two opposing goals in mind: to remain as close as possible to existing analyses of non-Semitic languages (theoretical parsimony) and to explain the templatic system of verbs in Modern Hebrew (empirical adequacy). I have contended that this theory should be seen not only as a theory of Hebrew but as a theory of argument structure crosslinguistically. Accordingly, two questions ought to be distinguished when we ask about the crosslinguistic validity of the current theory:
\begin{enumerate}
	\item Does the syntactic inventory of every language always contain these heads (Voice, v and \emph{p})?
	\item Does every language have the kinds of features on these heads that Hebrew does, i.e. {\vd}, \pz, etc.?
\end{enumerate}

In the current chapter I summarize the generalizations drawn out here and the means used to explain them (Section~\ref{i:sum}). I then discuss some related issues in the theory of Voice, chiefly the identity of this head (Section~\ref{i:i}, addressing the first question) and the identity of its features (Section~\ref{i:agree}, addressing the second feature). Section~\ref{i:conc} concludes the chapter, and with it the monograph, although it will surely not be the final word (or template) on Hebrew morphology.


\section{Summary of the current approach} \label{i:sum}
The summary of this work is divided into two: the empirical generalizations which an adequate theory of Hebrew morphology ought to derive, and the formal system I have implemented to do so. Readers who have jumped directly to this section may want to consult Chapter~\ref{intro:arch} for some formal background.

	\subsection{Summary of generalizations}
Every description or discussion of Hebrew morphology begins with the templatic system. The current book was also organized this way, laying out the empirical landscape template by template. When we are careful about the date, we quickly find that some of the generalizations are quite unnatural. To take one example, {\tnif} has both unaccusative and unergative verbs, but no transitive verbs; what would one need to say in a formal analysis? Having acknowledged the unergative and unaccusative verbs have different structures (the Unaccusativity Hypothesis), contemporary theories can hardly refer to ``intransitivity'' as a theoretical primitive.

The list below recaps the generalizations made about each template.

\pex \textbf{Generalizations about {\tkal}}
	\a \textbf{Constructions:} Verbs appear in all possible argument structure configurations.
	\a \textbf{Alternations:} Participates in alternations with the other templates, as will be reviewed throughout the book.
\xe

\pex \textbf{Generalizations about {\tpie}}
	\a \textbf{Constructions:} Verbs appear in active (transitive/unergative) configurations.\\
		Readings are weakly agentive.
	\a \textbf{Alternations:} When alternating with {\tkal}, provides a more ``intensive'' or agentive version.
\xe

\pex \textbf{Generalizations about {\tnif}}
	\a \textbf{Constructions:} Verbs appear in unaccusative, passive and figure reflexive constructions; but never in a simple transitive configuration.
	\a \textbf{Alternations:} Some verbs are anticausative or passive versions of verbs in {\tkal}.
\xe

\pex \textbf{Generalizations about {\thit}}
	\a \textbf{Constructions:} Verbs appear in unaccusative, figure reflexive and reflexive constructions; but not in a simple transitive configuration.
	\a \textbf{Alternations:} Some verbs are anticausative or reflexive versions of verbs in {\tpie}.
\xe

\pex \textbf{Generalizations about {\thif}}
	\a \textbf{Constructions:} Verbs appear in transitive and unergative configurations; a small class of verbs forms unaccusative degree achievements.
	\a \textbf{Alternations:} Some verbs are causative or active versions of verbs in other templates, especially {\tkal}. A small class of verbs creates a labile alternation within {\thif}.
\xe

\pex \textbf{Generalizations about {\tpua} and {\thuf}}
	\a \textbf{Constructions:} Verbs appear in passive configurations only.
	\a \textbf{Alternations:} Verbs in {\tpua} are always the passive version of an active verb in {\tpie}. Verbs in {\thuf} are always the passive version of an active verb in {\thif}.
\xe

These generalizations are summarized in~(\nextx), where a na\"ive view of the templates defies economical description -- at least at first sight.\footnote{The table in~\ref{table:sum-naive} does not distinguish plain unergatives from figure reflexives, precisely because traditional views do not make this distinction.}
\ex \label{table:sum-naive}
\xe
\begin{tabular}{ll|ccccc|l}
& Template	& Unaccusative	& Unergative	& Transitive	& Reflexive	& Passive  & Alternations\\\hline
a.& {\tkal}			& \cmark			& \cmark			& \cmark		& \xmark	& \xmark & b, c, e\\
b.& {\tpie}			& \xmark			& \cmark			& \cmark		& \xmark	& \xmark & a, d, f\\
c.& {\tnif}			& \cmark			& \cmark			& \xmark		& \xmark	& \cmark & a, e\\
d.& {\thit}			& \cmark			& \cmark			& \xmark		& \cmark	& \xmark & b, g\\
e.& {\thif}			& \cmark			& \cmark			& \cmark		& \xmark	& \xmark & a, c, e\\\hline
f.& {\tpua}			& \xmark		& \xmark			& \xmark			& \xmark	& \cmark & b\\
g.& {\thuf}			& \xmark		& \xmark			& \xmark			& \xmark	& \cmark & g\\
\end{tabular}

The table in~(\lastx) clearly shows the many-to-many mapping problem that arises when trying to link morphology and syntax/semantics in Hebrew (with the exception of the two passive templates, perhaps the sole consensus in the morphosyntactic literature on Hebrew). What is the reading of a verb in {\thit}? The answer is not deterministic: sometimes unaccusative, sometimes unergative, sometimes reflexive. What morphology do you choose if you want to express an unaccusative verb? The answer is again not deterministic: sometimes {\tkal}, sometimes {\tnif}, sometimes {\thit} and sometimes {\thif}. So form cannot map directly to meaning, if by ``form'' we mean templates. My critique of \cite{arad05} in previous chapters hopefully showed just how important that work was in clearing up this point.

And once you have chosen a template, how do you know which template to derive an alternation in? Should you go from an anticausative in {\tnif} to {\tkal} or to {\thif}? We have seen that both are possible:
\ex
	\begin{tabular}{l|lcl|ccc}
		&	&	&	&	{\tnif}	& {\tkal} & {\thif}\\\hline
	\root{ʃbr} & broke & $\sim$ & broke	& \emph{niʃbar} & \emph{ʃavar} & --- \\
	\root{xkd} & went extinct & $\sim$ & eradicated & \emph{nikxad} & --- & \emph{hekxid}\\
	\end{tabular}
\xe
Again, some patterns exist but looking at the templates in terms of primitives makes it hard to understand why precisely these patterns arise, above and beyond the idiosyncrasy of individual roots.

Some of the generalizations tallied above are novel. The entire formal conception of figure reflexives in Hebrew \citep{kastner16phd} and indeed in this general framework is recent \citep{wood12phd,wood14nllt}. The characterization of unaccusatives in {\thif} as degree achievements is recent, although not my own proposal \cite{lev16}. The observation that {\thif} is not a causative template but a general active one, which might also have lexical causatives, is novel as far as I can tell, at least when put in these terms. The idea that the distinction between anticausatives and reflexives in {\thit} reflects the lexical semantics of the roots is also very recent \cite{kastner17gjgl}. And the allomorphic interactions which I have mentioned have also only been noticed recently in a unified way \cite{kastner18nllt}. Even if the analysis ends up being inadequate, I believe that these generalizations are important and hope that future work can engage with them rigorously.

	\subsection{D-composing the templates}
Instead of talking about templates, I have proposed that we talk about syntactic structure and cyclic derivations. This was done using the head Voice and allowing it to take one of three values: [+D], [--D] or unspecified for [D]. Informal definitions are given in~(\nextx).
\pex Overt Voice heads in Hebrew
\a \textbf{{\vd}:} requires a DP in in its specifier.
\a \textbf{{\vz}:} prohibits a DP in its specifier.
\a \textbf{Unspecified Voice:} places no restrictions on its specifier.
\xe

Acknowledging the existence of an overt agentive modifier whose semantics is difficult to pin down but which does not seem to be active in the syntax adds {\va} to our toolbox.
\ex \textbf{{\va}:} an overt modifier adding ``action'' (agentive) semantics to an event. In practice, often creates a requirement for an Agent role to be saturated.
\xe

Assuming that prepositional phrases are derived similarly to verbs gives us the functional head \textit{p} and its [--D] variant {\pz} (more on this in the next section).
\pex
	\a \textbf{\textit{p}:} the prepositional equivalent of Voice; introduces the Figure role (the subject of a preposition).
	\a \textbf{{\pz}:} prohibits a DP in its specifier.
\xe

The passive head Pass rounds off the picture.
\ex \textbf{Pass:} prevents the projection of an argument in Spec,VoiceP and closes off the Agent role existentially in the semantics.
\xe

When these elements are combined, the resulting picture is not the confusing one in~(\ref{table:sum-naive}) but the principled one in~(\nextx). These elements and the values of their features explain what structures give rise to what spell-outs, or in other words, what \textbf{constructions} a given template is possible in.
\ex
\xe
\begin{center}
\begin{small}
%	\begin{table}[ht] \centering \small
		\begin{tabular}{|llll||c|c|l||c|}\hline
				\multicolumn{4}{|c||}{Heads} & Syntax 	& Semantics & Phonology & Chapter\\\hline\hline
		
				& Voice& &	& (underspecified) 	& (underspecified)	&  \emph{XaYaZ} & \ref{voice:voice} \\\hline
		
				& Voice&\red{\va}&	& (underspecified)	& \red{Action}	 & \emph{X{\red{i\dgs{Y}e}}Z}&  \ref{voice:va}	\\
		
				\olive{Pass} & Voice&\red{\va}&	& \olive{Passive}	& \red{Action}	 & \emph{X\olive{u}{\red{\dgs{Y}}}\olive{a}Z}&  \ref{passn:pass:pass}	\\\hline
		
				& \blue{\vd}& &		& \blue{EA}	& (underspecified)	 & \emph{{\blue{he}}-XY{\blue{i}}Z} & \ref{vd:vd} \\
		
				\olive{Pass} & \blue{\vd}& &		& \olive{Passive}, \blue{EA}	& (underspecified)	 & \emph{{\blue{h}}\olive{u}-XY\olive{a}Z} & \ref{passn:pass:pass} \\\hline
				
				& \blue{\vz}& &		& \blue{No EA}	& (underspecified)	 & \multirow{2}{*}{\emph{{\blue{ni}}-XY{\blue{a}}Z}} & \ref{vz:vz} \\
				& Voice& &\blue{\pz}	& \blue{EA = Figure} & (underspecified)	 &  & \ref{vz:pz} \\\hline
				& \blue{\vz}&\red{\va}&	& \blue{No EA}	& \red{Action}	 & \multirow{2}{*}{\emph{{\blue{hit}}-X{\red{a\dgs{Y}e}}Z} } &  \ref{vz:va:vzva} \\
				& Voice&\red{\va}&\blue{\pz}	& \blue{EA = Figure} & \red{Action}	 & & \ref{vz:va:pzva} \\\hline
			\end{tabular}
%		\caption{The requirements of functional heads in the Hebrew verb.\label{table:summary-syn-rep}}
%	\end{table}
\end{small}
\end{center}
The combinatorics of these heads---specifically unattested combinations---were addressed in the individual chapters. See \citet[Ch.~2.4.1.1]{kastner16phd} for a summary.

The hierarchical structure---that is to say, the layering of Voice above vP---explains what \textbf{alternations} are possible. Since Voice merges above a core vP, the three values of Voice can give rise to three templates in transitivity alternations ({\tkal}, {\tnif} and {\thif}). Because {\va} merges with the core vP, {\tpie} has a less transparent, but still fairly consistent, relationship with {\tkal}. And given that the core vP with {\va} can then combine with either unspecified Voice or {\vz}, we derive the alternation between {\tpie} and {\thit}. The final alternation, between active {\tpie} and {\thif} and the one side and passive {\tpua} and {\thuf} on the other side, follows from merging Pass with the active VoiceP. Alternations are therefore not listed extrinsically nor are they the property of templates. They are what we see if a given root is instantiated in a number of structures which share a core vP, just like \citeauthor{schaefer08}'s (\citeyear{schaefer08}) conclusion for the causative alternation in European languages.

Let me reiterate the role of \textbf{roots} in this theory and in any theory of Semitic: at some level it must be listed what functional heads are licensed by what roots. I have not attempted to formalize this matter, since at present I do not see whether there are any generalizations to account for or predictions worth making. \cite{arad05} makes a strong case for the view under which the combination is almost entirely arbitrary, in that each combination of root and template must be listed. I do not know whether this is necessarily true, but currently have no better way of describing the system. As emphasized throughout the book, the same issue arises regardless of theoretical framework.

In the next section I try to take a step back from Semitic, considering the conceptual and crosslinguistic inventory of syntactic heads.


\section{Voice heads} \label{i:i}
The framework as a whole allows for Voice heads and \emph{p} heads, alongside verbalizers (little v, n and a). I also assume that applicatives are introduced using the head Appl, although I make no claims about applicatives in Hebrew. These heads all appear to be empirically necessary, but recent work suggests that they need not be distinguished theoretically. In what comes next I review how the current proposal can be applied to an unrelated language, Japanese (Section~\ref{i:i:jap}); a recent theoretical proposal which dovetails nicely with the current proposal (Section~\ref{i:i:i}; and crosslinguistic prospects for the trivalent approach to Voice (Section~\ref{i:i:ay}).

	\subsection{Japanese} \label{i:i:jap}
The morphology of transitivity alternations in Japanese has received significant attention over the years \citep{suga80,jacobsen92,miyagawa98,nishiyama98,volpe05,harley08}. Recently, \cite{oseki17nyu} proposed an analysis of argument structure alternations and their expression in the morphology of Japanese which builds on the trivalent system. Here is an outline and quick evaluation.

Some verbs like \emph{hirak} `open' have the labile alternation:
\pex
	\a \begingl
		\gla John-ga doa-o hirak-\zero-ta//
		\glb John-\gsc{NOM} door-\gsc{ACC} open-\zero-Past//
		\glft `John opened the door.'//
	\endgl
	\a \begingl
		\gla Doa-ga hirak-\zero-ta//
		\glb door-\gsc{NOM} open-\zero-Past//
		\glft `The door opened.'//
	\endgl
\xe

With many others, overt transitivity markers can be found, namely -\gsc{S}- (and allomorphs) for marked transitives and -\gsc{R}- (and allomorphs) for marked intransitives:
\pex
	\a \begingl
		\gla John-ga posutaa-o hag-as-ta//
		\glb John-\gsc{NOM} poster-\gsc{ACC} peel-\gsc{S}-Past//
		\glft `John took down a poster.'//
	\endgl
	\a \begingl
		\gla Syatsu-ga chijim-ar-ta//
		\glb shirt-\gsc{NOM} shrink-\gsc{R}-Past//
		\glft `A shirt shrank.'//
	\endgl
\xe

Tellingly, Japanese has ``minimal triplets'' similar to those which motivated the current proposal for Hebrew. \cite{oseki17nyu} reproduces data such as the following from \cite{suga80}:
\ex
\begin{tabular}{lllll}
	& & Marked intransitive & Unmarked intransitive & Marked transitive\\\hline
	a.& \root{\gsc{PEEL}} & hag-er-u & hag-\zero-u & hag-as-u\\
	b.& \root{\gsc{CUT}} & kur-er-u & kir-\zero-u & kir-as-u\\
\end{tabular}
\xe

These patterns can be understood if we assume something like the following for Japanese:
\pex
	\a Voice \lra~\zero
	\a {\vz} \lra~-\gsc{R}-
	\a {\vd} \lra~-\gsc{S}-
\xe

\cite{oseki17nyu} then goes on to show how this system predicts the behavior of the causative \emph{-sase} applicatives as \emph{-s-} over \emph{-s-}---potentially {\vd} over {\vd}---and of the passive \emph{rare} \emph{-r} over \emph{-r-} (potentially {\vz} over {\vz}).\footnote{In Hebrew this specific combination cannot be examined because Hebrew does not have a ``morphological'' Appl affix; benefactive and malefactive arguments are introduced using the preposition \emph{le-} `to'. 
\ex \begingl
    \gla ha-arje biʃel ʃu'it (\underline{la}-jeladim)//
    \glb the-lion cooked beans (\underline{to.the}-children)//
    \glft `The lion cooked (the children) beans'//
    \endgl
\xe
}

Appealing as this analysis of Japanese may be, I must note two aspects in which it diverges from the current proposal for Hebrew. First, \gsc{S}-marked verbs are always transitive. Hebrew {\vd} does not enforce transitivity, only an external argument. In fact, \citet[26]{nie17} notes that the alternation between a marked and unmarked transitive seems like a Differential Object Marking pattern, correlating with the appearance of the object marker -\emph{o} (glossed \gsc{ACC}, just like Hebrew \emph{et}). Marked transitives are only possible with -\emph{o}, but zero-marking is possible regardless of whether the object marker appears.
\pex
	\a \begingl
		\gla John-ga posutaa-\textbf{o} hag-\zero/as-ta//
		\glb John-\gsc{NOM} poster-\gsc{ACC} peel-\zero/\gsc{S}-Past//
		\glft `John took down (a specific) poster.'//
	\endgl
	\a \begingl
		\gla John-ga posutaa hag-\zero/(*as)-ta//
		\glb John-\gsc{NOM} poster-\gsc{ACC} peel-\zero/\gsc{S}-Past//
		\glft `John took down a poster (some poster).'//
	\endgl
\xe

Second, these patterns are not necessarily the entire story. \citeauthor{oseki17nyu} also gives the following patterns from \cite{suga80}, in which \emph{-er-} is now the transitivity marker and the marked intransitive -\gsc{R}- form is -\emph{ar}-.
\ex
\begin{tabular}{lllll}
	& & Marked intransitive & Unmarked transitive & Marked transitive\\\hline
	a.& \root{\gsc{SHRINK}} & chijim-ar-u & chijim-\zero-u & chijim-er-u\\
	b.& \root{\gsc{MOVE}} & tsutaw-ar-u	& tsutaw-\zero-u & tsutaw-er-u\\
\end{tabular}
\xe

To what extent these patterns are phonologically predictable and to what extent there is accidental syncretism are important questions. But they are tangential to my current reservation, which is simply that more work needs to be done in order to understand Japanese transitivity alternations. The trivalent system therefore provides a good starting point for investigating Japanese, especially if it can provide novel explanations for the morphology of applicatives and passives, and if it can explain DOM-like patterns (Section~\ref{i:agree:nie}).

	\subsection{\emph{i}*} \label{i:i:i}
In a recent account of the way argument structure is derived and interpreted, \cite{woodmarantz17} propose to reduce the overall inventory of functional heads. Working within a similar framework, they suggest that non-internal arguments (external and applied arguments introduced by Voice, Appl, \emph{p} and P) are in fact introduced by contextual variants of the same predicational head. This head is called \emph{i*}. 

If \cite{woodmarantz17} are correct, the difference between \emph{p}, Appl and Voice is an illusion: they are all the same predicational head underlyingly, albeit in different contexts. Voice is but \emph{i*} that merges with a vP. Little \emph{p} is but \emph{i*} that merges with a PP. And P itself is \emph{i*} modified by a (prepositional) root; see the work cited for full details. The analytical possibilities opened up by this view have already been pursued in a range of recent work, including Appl and Voice in German psych-predicates \citep{hirsch18phd}, and P and Appl in Russian datives \citep{bonehnash17}.

My goal here is not to evaluate this proposal, which is supported by conceptual considerations as well as empirical study of figure reflexives in Icelandic, the Adversity Causative in Japanese and possession in Quechua and other languages. Instead, I want to highlight one welcome point of convergence between the \emph{i*} hypothesis and my proposal for Hebrew. In the inventory of functional heads I have laid out, {\vz} and {\pz} are conspicuously similar: they do similar work in the syntax and have the same spell-out. If we follow the \emph{i*} hypothesis, the two \emph{should} be similar: they are the same functional head, only in different contexts (\nextx).
\pex
	\a Anticausative in {\tnif}\\
	\Tree
	[.{\textit{i}*P\\(VoiceP)}
		[.{---} ]
		[.
			[.{\textit{i}*$_{\text{[--D]}}$\\ (\vz)\\ \emph{ni-}} ]
			[.vP ]
		]
	]
	\a Figure reflexive in {\tnif}\\
	\Tree
	[.{\textit{i}*P\\ (VoiceP)}
		[.DP ]
		[.
			[.{\emph{i*}\\ (Voice)} ]
			[.vP
				[.v ]
				[.{\textit{i*}P\\ (\textit{p}P)}
					[.{---} ]
					[.
						[.{\textit{i}*$_{\text{[--D]}}$\\ (\pz)\\ \emph{ni-}} ]
						[.PP ]
					]
				]
			]
		]
	]
\xe

To be clear, I do not believe that the \emph{i*} hypothesis must be true for the current account to go through. But if this hypothesis is on the right track, a strong version can be formulated under which all exponents of \emph{i*} (as well as its variants \emph{i*}$_{\text{[--D]}}$ and \emph{i*}$_{\text{[+D]}}$) should be identical to each other. Such a hypothesis would immediately predict the similarity between Voice and \emph{p}---both default and silent---and that between {\vz} and {\pz}.

	\subsection{Prospects for trivalent Voice/\textit{i*}} \label{i:i:ay}
This chapter began asking whether the syntactic inventory of every language always contain the heads Voice, v, \textit{p} and potentially Appl? My working hypothesis is that the answer is yes. I assume that Voice, v and \emph{p} are an inherent part of the syntactic system of every language. I am less certain about the applicative head Appl, although under the \textit{i*} hypothesis there is no difference between Appl and Voice or \textit{p}. 

In Kinyarwanda, the causative and instrumental applicative suffixes are spelled out identically \citep{jerro17} and in the Algonquian language Penobscot, many ``relational predicates'' have similar if not identical morphology for causatives and applicatives \citep[Ch.~2.3.7.1]{quinn06phd}. The \emph{i*} hypothesis would lead us to expect similar correlations between different heads crosslinguistically. Combining the outlined analysis of Japanese above with the basics of \emph{i*}, we arrive at the picture in~(\nextx).

%\ex\label{ex:heads-langs}Exponents of syntactic elements in a number of unrelated languages:\\
%	\begin{tabular}{l|lll}
%	Head 	& Hebrew 		& Greek  		& Japanese\\\hline
%	Voice   & \tkal     & (silent)      & -\emph{e}-\\
%	{\vz} 	& \tnif 	& \emph{-thike}	& -\emph{r}-\\
%	{\vd}	& \thif		& ?		& -\emph{s}-\\
%	{\va}	& \tpie		& \emph{afto-}	& -\emph{ak}-?\\
%	{\pz}	& \tnif		& ?		& ?\\
%	\end{tabular}
%\xe

%\ex\label{ex:heads-langs2}Syntactic elements in a number of unrelated languages, adopting \cite{woodmarantz17}:\\
%	\begin{tabular}{l|lll}
%	Head 		& Hebrew 		& Greek  		& Japanese\\\hline
%	\emph{i*}   	& \tkal     & (silent)      & -\emph{e}-\\
%	\emph{i*}$_{\text{[--D]}}$ 	& \tnif 	& \emph{-thike}	& -\emph{r}-\\
%	\emph{i*}$_{\text{[+D]}}$	& \thif		& ?		& -\emph{s}-\\
%%	{\va}	& \tpie		& \emph{afto-}	& -\emph{ak}-?\\
%	\end{tabular}
%\xe

\ex\label{ex:heads-langs2}
%Syntactic elements in a number of unrelated languages, adopting \cite{woodmarantz17}:\\
	\begin{tabular}{l|ll}
	Head 		& Hebrew 	  		& Japanese\\\hline
	\emph{i*}   	& \tkal          & -\emph{e}-\\
	\emph{i*}$_{\text{[--D]}}$ 	& \tnif 		& -\emph{r}-\\
	\emph{i*}$_{\text{[+D]}}$	& \thif		& -\emph{s}-\\
%	{\va}	& \tpie		& \emph{afto-}	& -\emph{ak}-?\\
	\end{tabular}
\xe

How far can this idea extend? Within Semitic, Standard Arabic and some Arabic dialects might be informative. In these languages an ``applicative template'' can introduce an applied argument without need for a preposition \citep{alkaabi15phd}. Other templates do not seem to have syntactic and semantic requirements which are as stringent as those in Hebrew, indicating that perhaps an \citeauthor{arad05}-style analysis might be more appropriate, that additional semi-lexical roots similar to {\va} need to be invoked, or that ``flavors'' of v might be necessary after all \citep{katie13}. It remains to be seen how the current framework can accommodate patterns in other Semitic languages. Similarly, a strong view restricting structure-building of verbs to v, Voice/\textit{i*} and Pass means there is no room for specialized heads such as Reflexive (\citealt{ahn15phd}, cf.~\citealt{spathas17camvoice,spathas17debrecen}) or Reciprocal \citep{bruening04}. I believe that exploring this strong claim will lead to new discoveries and refute the dedicated-heads approaches (although tough talk comes cheap). There is also the question of languages with dedicated ``slots'' for different argument structure affixes, like the CARP template of Bantu and similar phenomena \citep{hyman03,paster05}. Yet I am optimistic, since data which had received analysis in terms of affix slots was also shown to be analyzable in more derivational terms; here I have in mind the Nimboran debate between \cite{inkelas93} and \cite{noyer98}.

It thus appears to be potentially useful to adopt this framework for additional languages and map out which heads and features are instantiated in which language. The heads were discussed in this section; features are the topic of the next section, which is also the last one of this book before concluding.


%
%This framework not only allows us to describe different languages with similar tools, it allows us to ask more fine-grained questions about crosslinguistic variation. Consider the following differences between Hebrew and Greek: The first can lead to new discoveries about Greek. The second can form part of a general theory of lexical semantics and how it interacts with the syntax.
%\pex
%	\a Hebrew uses \thit~for both reflexives and reciprocals (\S\S\ref{syn:middle:refl}--\ref{syn:middle:recip}). Greek uses \emph{afto}- with a nonactive base for reflexives (\S\ref{syn:middle:refl}) and \emph{alilo}- with a nonactive base for reciprocals \citep{alexiadouafto}.
%	\a Hebrew Other-Oriented roots are not compatible with \va; their semantics is impoverished, (\ref{sem:thit-incho}). In Greek, only Other-Oriented roots are compatible with \emph{afto-}.
%\xe
%
%The first difference opens up the following line of questioning: are Greek reflexives and reciprocals different in ways that can inform our understanding of unaccusativity, in the way that Hebrew reflexives, reciprocals and anticausatives in \thit~mask structural differences? To the best of my knowledge, this question has not been explored yet. It is possible to ask this question now, though, since we have a vocabulary with which to describe the system and make predictions. For example, if there is no structural difference between Greek reflexives and reciprocals save for the form of the prefix, it should be the case that the same root can instantiate both constructions, though only if its lexical semantics is compatible with them. See \cite{alexiadouborerschaefer14} for related discussion of crosslinguistic interactions between the lexical semantics of roots and syntactic structure.
%
%The second difference shows that even if we find similar classes of roots crosslinguistically (in this case Other-Oriented roots such as \root{\gsc{BREAK}} or \root{\gsc{HIT}}), they might not interact with similar elements in the same way. Again, we can develop this theory now that we have a framework for roots of different semantic classes and the syntactic elements they combine with.


\section{Valuing features on Voice} \label{i:agree}
The second question opening this chapter had to do with the inventory of features which might exist on \textit{i*}/Voice. It is interesting to note that virtually all work on the features of Voice assumes the [D] feature and/or phi-features.

In Hebrew, I have made the case that Voice can be [+D] as in {\vd}, [--D] as in {\vz} or not inherently valued as in unspecified Voice. In principle however, the architecture allows any syntactic feature to appear on Voice. Nothing in the theory prohibits Voice$_{\text{[wh]}}$, for instance, which would require a \emph{wh}-phrase in Spec,VoiceP. Now granted, any theory of syntax must stipulate in one way or another which features are possible on which functional elements. One way to restrict the theory is to require only \emph{uninterpretable} features \citep{chomsky95}, being purely syntactic features, to exist on Voice. The EPP feature [D] is one such feature. This kind of solution would rely on a certain view of which features are interpretable and which are not (the notion of uninterpretable features and whether they are necessary has itself been questioned in recent work, e.g.~\citealt{preminger14mit}).

Ideally, our theory of features on argument-introducing heads would be part of a general theory of argument structure, feeding processes such as case assignment and specifying the triggers for A-movement. Some recent theories do exactly that by recourse to [D] and phi-features.

@@@

	\subsection{Other approaches to the content of Voice} \label{i:agree:other}
\cite{schaefer08,schaefer12,schaefer17oup}: D for spec, phi for dependent case
(case in Hebrew? no se/sich)


Legate

Wurmbrand

\pex Invites comparison with \cite{wurmbrandshimamura17}.
	\a Voice features on Voice: active/passive (and also Agent/Cause semantics). %This theory has Voice encode either Agent or Passive. Not sure what the motivation is. And Agent/Cause is a separate (allosemic) issue.
	\a Phi-features on Voice: identify a DP.
	\a Additional consideration/result: unvalued Voice is restructuring-Voice.
\xe

\pex To summarize:
	\a Parts of the current approach are compatible with translation to phi-features, others (\vz) less so.
	\a Right now the question of whether there's parametric variation or a unified account is open \citep[191ff12]{wurmbrandshimamura17}.
\xe


	\subsection{Trivalent Voice as feature valuation} \label{i:agree:nie}

%AM: We've been re-visiting the account of Japanese +D causative -s marking, with the possible correlation with obligatory -o marking (and thus no -o drop), precisely because Yining's use of +D on voice is designed to force obligatory phi agreement with the/an object (which is straightforward for ergative languages, in which the phi agreement goes along with the absolute, but less intuitive for nom/acc language, in which the spec of the +D voice will generally be able to provide phi features for T; nevertheless, the prediction is that +D voice should go along with object marking of the specific/definite/animate sort, and a type of transitivity that elevates the direct object to discourse representation status as a potential topic or focus).



\pex\label{ex:puyuma}\cite{nie17} proposes trivalent Voice for Puyuma (Formosan); similarly for Niuean.
	\a {\vd} is marked with an infix; obligatory external argument.\\
		\begingl
		\gla s\caus{<em>}a-senay i baeli//
		\glb {<}AV>-sing \gsc{NOM.SG} my.elder.sibling//
		\glft `My elder sister is/was singing.'\trailingcitation{\citep[68]{teng07}}//
	\endgl
	\a Unmarked Voice is spelled out by a prefix; external argument optional.\\
		\begingl
		\gla \unmark{mu}-atel la na ladru (dra balri)//
		\glb \gsc{MU}-fall \gsc{PERF} \gsc{NOM.DEF} mango \gsc{OBL.IND} wind//
		\glft `The mango fell. / Wind made the mango fall.'\trailingcitation{\citep[5]{chenfukuda17}}//
	\endgl
	\a {\vz} is unmarked morphologically.\\
		\begingl
		\gla \anticaus{drua} nantu lalak//
		\glb came her.\gsc{NOM} child//
		\glft `Her child came.'\trailingcitation{\citep[222]{teng07}}//
	\endgl
\xe


\pex \cite{nie17}:
	\a Voice{[}$\pm$D] for introduction of an external argument.
	\a Phi-features on Voice which can be valued from below.
\xe

\pex This derives Puyuma~(\ref{ex:puyuma}) but also:
	\a The traditional \textbf{``voices''} of Austronesian---Agent Voice and Patient Voice---are both {\vd}. The choice \emph{between} them depends on case and agreement of the internal argument.
	\a Simliarly for Niuean and other languages, with additional technical details.
\xe




interplay of factors associated with ergativity, obligatory transitive verbs, and the featural EPP.


\begin{itemize}
	\item Various approaches to the content of Voice.
	\item Trivalent Voice with overt spell-out: [+D], [--D], unspecified for D..
	\item Ongoing: [D]? [phi]? More? Less?
\end{itemize}


\section{Conclusion} \label{i:conc}

being honest about the data/generalizations - didn't check every single verb

acquisition?


No templates. Arad was explicit and unsatisfying. Doron wasn't connected to the syntax. So we need a modern morphophonemics of Hebrew, if you will.

%\section{Notes}
%Triplets.
%Legate conversation at CamVoice: she's against underspecified Voice. Instead, have active Voice, nonactive Voice, and then choose the zero allomorph based on some other structural difference between the transitive constructions. So one practical question is, is there some generalization about the internal arguments of thif vs transitive tkal, or of tnif vs unaccusative tkal.
%	I don't think so, but it's worth looking at systematically. And worst case, I'll have a foil to argue against.
%	The relevant question for me is point where vP is one event and then Voice does something predictable. Because triplets show that this is usually not the case: sagar-nisgar-hesgir. So it's not really the case that there's a "closing" vP and then you either block an external argument (tnif) or not (tkal), and also can require one (thif).
%	So really what I'm forced to say is that the connection between sagar and nisgar is just as tight as between nisgar and hesgir.
%	There aren't that many unaccusative doublets with tkal-tnif either (if any).
%	I think this really requires figuring out the empirical state of affairs, quantitatively even (causativization isn't the main function of thif anyway).
%	Now have a few examples in slides.

%AM:  In short, it's not clear to me that you're going in a different direction than the agreement/valuation one -- you just have not yet emphasized the interplay of factors associated with ergativity, obligatory transitive verbs, and the featural EPP.
%In filling out your typology of structures, with possible allosemes for the different voice heads, you might include the +D -s suffix of Japanese as one that, in the case of adversity causatives, does not have the agentive/causer semantics (but still has a thematic external argument, since the possessor role is "passed up" the tree).
%And of course there are the deponent verbs --  -D voice that nevertheless has agentive semantics.