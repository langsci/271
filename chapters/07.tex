\chapter{The features of Voice}
\label{chap:i}


summary of generalizations

summary of proposal

summary of claim

being honest about the data/generalizations - didn't check every single verb

acquisition?


\section{i*}
In a recent account of the way argument structure is derived and interpreted, \cite{woodmarantz15} propose to reduce the overall inventory of functional heads. Working within a similar framework, in which much of the burden of deriving properties of the event lies in the semantic component, \citeauthor{woodmarantz15} suggest that non-internal arguments (external and applied arguments introduced by Voice, Appl, \emph{p} and P) are in fact variants of the same predicational head. This head is called \emph{i*}.

If \cite{woodmarantz15} are correct, the difference between \emph{p}, Appl and Voice is an illusion: they are all the same predicational head underlyingly, albeit in different contexts. Voice is but \emph{i*} that merges with a vP. Little \emph{p} is but \emph{i*} that merges with a PP. And P itself is \emph{i*} modified by a (prepositional) root.

Our goal here is not to evaluate their proposal, which is supported by conceptual considerations as well as empirical study of figure reflexives in Icelandic, the Adversity Causative in Japanese and possession in Quechua and other languages. Instead, I want to highlight one welcome point of convergence between the \emph{i*} hypothesis and my proposal for Hebrew. In the inventory of functional heads I have laid out, \vz~and \pz~are conspicuously similar: they do similar work in the syntax and have the same spell-out. If we follow the \emph{i*} hypothesis, the two \emph{should} be similar: they are the same functional head, only in different contexts.

To be clear, I do not believe that the \emph{i*} hypothesis must be true for my account to go through. But if this hypothesis is on the right track, a strong version can be formulated under which all exponents of \emph{i*} (as well as its variants \emph{i*}$_{\text{\zero}}$ and \emph{i*}$_{\{\text{D}\}}$) should be identical to each other. Such a hypothesis would immediately predict the similarity between Voice and \emph{p}---both default and silent---and that between \vz~and \pz. {Adopting the \emph{i*} hypothesis we may modify~(\ref{ex:heads-langs}) as~(\ref{ex:heads-langs2}).}
\ex\label{ex:heads-langs2}Syntactic elements in a number of unrelated languages, adopting \cite{woodmarantz15}:\\
	\begin{tabular}{l|lll}
	Head 		& Hebrew 		& Greek  		& Japanese\\\hline
	\emph{i*}   	& \tkal     & (silent)      & -\emph{e}-\\
	\emph{i*}$_{\text{\zero}}$ 	& \tnif 	& \emph{-thike}	& -\emph{r}-\\
	\emph{i*}$_{\text{\{D\}}}$	& \thif		& ?		& -\emph{s}-\\
	{\va}	& \tpie		& \emph{afto-}	& -\emph{ak}-?\\
	\end{tabular}
\xe

The \emph{i*} hypothesis would also lead us to expect similar correlations crosslinguistically, a prediction which remains to be tested as the tools employed in this dissertation and in related works such as \cite{schaefer08}, \cite{spathasetal15} and \cite{wood15springer} are extended to additional languages.


i*: CAUS=APPL in Kinyarwanda https://doi.org/10.1017/S0022226717000044
p. 64 onwards: http://www.conormquinn.com/CQuinn2006Referential-AccessDependencyInPenobscot.pdf
Mithun on Iroquian

\section{Japanese}
Oseki
The framework developed in \S\S\ref{syn:middle}--\ref{syn:templates} makes use of a number of different syntactic heads. In this section I show how these fit into a crosslinguistic theory of argument structure.

Two questions ought to be distinguished when we ask about the crosslinguistic validity of this theory:
\begin{enumerate}
	\item Does the syntactic inventory of every language always contain these heads (Voice, v and \emph{p})?
	\item Does every language have the kinds of features on these heads that Hebrew does, i.e. {\vd}, \pz, \va, etc?
\end{enumerate}
The answer to the first question is yes. I assume that Voice, v and \emph{p} are an inherent part of the syntactic system of every language. I am less certain about the applicative head Appl, but the recent proposal by \cite{woodmarantz15} which I discuss in \S\ref{syn:crosslx:woodmarantz} below allows us to reconceptualize Appl as a variant of Voice.

The answer to the second question is less clear cut and potentially more interesting. First, we need to ask what features are possible on different heads, for example on Voice.

In Hebrew, I have made the case that Voice can be [+D] as in \vd, [--D] as in {\vz} or underspecified as in default Voice. But the architecture allows any syntactic feature to appear on Voice. Nothing in the setup prohibits Voice$_{\text{[wh]}}$, for instance, which would require a \emph{wh}-phrase in Spec,VoiceP. Now granted, any theory of syntax must stipulate in one way or another which features are possible on which functional elements. One way to restrict our theory is to require only \emph{uninterpretable} features \citep{chomsky95}, being purely syntactic features, to exist on Voice. The EPP feature [D] is one such feature. This kind of solution would rely on a certain view of which features are interpretable and which are not; the notion of whether uninterpretable features are necessary has itself been questioned in recent work \citep{preminger14mit}. Ideally, our theory of features on argument-introducing heads would be part of a general theory of argument structure, feeding processes such as case assignment and specifying the triggers for A-movement.

Expanding the crosslinguistic envelope, then, is every language predicted to have the same features on the same heads as Hebrew does? Not necessarily. It is certainly possible that English, for instance, has only one Voice head, so that argument structure alternations such as those in~(\nextx) arise through the general underspecification of Voice. \pex {[}Voice [v \root{\gsc{BREAK}}~\!]]
	\a John broke the vase.
	\a The vase broke.
\xe
Another possibility is pursued by \cite{schaefer08} for German and \cite{wood15springer} for Icelandic, where similar variants of Voice are used as in our system but at times without any phonological indication. I remain neutral with regards to specific claims about these languages, though my preference is to only postulate a variant of a head (meaning a head with a marked feature on it) when there is morphophonological reason to do so.

In some languages it is possible to find overt evidence for these heads. Icelandic \vz~fits the bill \citep{wood15springer} and the Greek data discussed by \cite{spathasetal15}, reviewed in \S\ref{syn:middle:refl}, led to a theory making use of \vz~(pronounced -\emph{thike}) and \va~(pronounced \emph{afto-}). In recent {unpublished }work following similar lines, \cite{oseki16nyu} proposes that Japanese makes use of Voice, {\vz} and \vd. The following table summarizes Hebrew, Greek and Japanese. The table is simplified: \tnif~is technically epiphenomenal rather than an exponent of \vz, to name one example{ (\S\ref{syn:middle:nonactive})}.
\ex\label{ex:heads-langs}Exponents of syntactic elements in a number of unrelated languages:\\
	\begin{tabular}{l|lll}
	Head 	& Hebrew 		& Greek  		& Japanese\\\hline
	Voice   & \tkal     & (silent)      & -\emph{e}-\\
	{\vz} 	& \tnif 	& \emph{-thike}	& -\emph{r}-\\
	{\vd}	& \thif		& ?		& -\emph{s}-\\
	{\va}	& \tpie		& \emph{afto-}	& -\emph{ak}-?\\
	{\pz}	& \tnif		& ?		& ?\\
	\end{tabular}
\xe
It thus appears to be potentially useful to adopt this framework for additional languages and map out which heads and features are instantiated in which language.

This framework not only allows us to describe different languages with similar tools, it allows us to ask more fine-grained questions about crosslinguistic variation. Consider the following differences between Hebrew and Greek: The first can lead to new discoveries about Greek. The second can form part of a general theory of lexical semantics and how it interacts with the syntax.
\pex
	\a Hebrew uses \thit~for both reflexives and reciprocals (\S\S\ref{syn:middle:refl}--\ref{syn:middle:recip}). Greek uses \emph{afto}- with a nonactive base for reflexives (\S\ref{syn:middle:refl}) and \emph{alilo}- with a nonactive base for reciprocals \citep{alexiadouafto}.
	\a Hebrew Other-Oriented roots are not compatible with \va; their semantics is impoverished, (\ref{sem:thit-incho}). In Greek, only Other-Oriented roots are compatible with \emph{afto-}.
\xe

The first difference opens up the following line of questioning: are Greek reflexives and reciprocals different in ways that can inform our understanding of unaccusativity, in the way that Hebrew reflexives, reciprocals and anticausatives in \thit~mask structural differences? To the best of my knowledge, this question has not been explored yet. It is possible to ask this question now, though, since we have a vocabulary with which to describe the system and make predictions. For example, if there is no structural difference between Greek reflexives and reciprocals save for the form of the prefix, it should be the case that the same root can instantiate both constructions, though only if its lexical semantics is compatible with them. See \cite{alexiadouborerschaefer14} for related discussion of crosslinguistic interactions between the lexical semantics of roots and syntactic structure.

The second difference shows that even if we find similar classes of roots crosslinguistically (in this case Other-Oriented roots such as \root{\gsc{BREAK}} or \root{\gsc{HIT}}), they might not interact with similar elements in the same way. Again, we can develop this theory now that we have a framework for roots of different semantic classes and the syntactic elements they combine with.

%AM: We've been re-visiting the account of Japanese +D causative -s marking, with the possible correlation with obligatory -o marking (and thus no -o drop), precisely because Yining's use of +D on voice is designed to force obligatory phi agreement with the/an object (which is straightforward for ergative languages, in which the phi agreement goes along with the absolute, but less intuitive for nom/acc language, in which the spec of the +D voice will generally be able to provide phi features for T; nevertheless, the prediction is that +D voice should go along with object marking of the specific/definite/animate sort, and a type of transitivity that elevates the direct object to discourse representation status as a potential topic or focus).


	\subsubsection{Logical possibilities for combinations} \label{syn:crosslx:combinatorics}
Sections \S\S\ref{syn:middle}--\ref{syn:templates} attempted to account for a range of data by exploring the combinations of different heads in the structure. Table~\ref{table:summary-syn-rep} above provided a summary of how different heads combine in the structure. I will address two points here: what combinations are not attested within Hebrew and what combinations should be attested crosslinguistically.

The unattested combinations are listed in Table~\ref{table:unattested-syn}. The final column notes whether the \emph{non-existence} of each form is predicted under our theory (check marks show a good match of theory to data).
\begin{table}[hb] \centering
\begin{tabular}{|lllll||c|c|c|}\hline
	\multicolumn{5}{|c||}{Heads} & Syntax 	& Semantics & Predicted? \\\hline\hline
	
	a. & & \blue{\vd} &\red{\va}&	& \blue{EA}	& \red{Action} & \xmark \\

	b. & & \blue{\vd} &\red{\va}& \blue{\pz}	& \blue{EA}, \blue{EA = Figure}	& \red{Action} & \xmark \\\hline
			
	c. & & \blue{\vd} & &\blue{\pz}	& \blue{EA}, \blue{EA = Figure} & (underspecified)  & \xmark \\\hline
	
	d.& \olive{Pass} & Voice& &		& \olive{Passive}	& (underspecified) & \cmark/\xmark \\
		
	e.& \olive{Pass} & Voice& &	\blue{\pz}	& \olive{Passive}, \blue{EA = Figure}	& (underspecified) & \cmark/\xmark \\
	
	f.& \olive{Pass} & \blue{\vz}& &		& \olive{Passive}, \blue{No EA}	& (underspecified) & \cmark \\\hline	
	
	g. & & \blue{\vz} & &\blue{\pz}	& \blue{No EA}, \blue{EA = Figure} & (underspecified) & \cmark \\\hline
\end{tabular}
\caption{Unattested combinations of syntactic elements in Hebrew.\label{table:unattested-syn}}
\end{table}

The combinations in \textbf{(a)} and \textbf{(b)} predict a strongly agentive template: \vd~requires an external argument and \va~ensures that this argument has agentive semantics. But we have no evidence for such a template, which would have a prefix from \vd~and non-spirantization from \va. This gap has no principled explanation: perhaps both heads perform similar enough work functionally that such a template is not necessary. There already exists a head that requires an external argument (\vd), and there already exists a head that requires agentive semantics (\va~\!), so the system is in no need of a redundant combination of the two.

It is also possible for (a) and (b), at least technically, that when \vd~and \va~combine a null allomorph of \va~is selected: thus when a verb in \thif~is agentive, there is a silent \va~in the structure. This proposal is unfalsifiable and I will not attempt to defend it.

The structure in \textbf{(c)} would consist of an external argument saturating the Figure role passed up by \pz. In effect, this is what happens in ordinary figure reflexives. Regular Voice carries out this role since if it did not, the derivation would not converge. But I do not think that the combination in (c) is blocked for any principled reason.

It is worth noting at this point that the combinations in (a)--(c) all involve an overt variant of Voice and an overt variant of \emph{p}. Time will tell if this is a spurious correlation or whether this pattern indicates an issue with multiple overt argument-introducing heads.

The combinations in \textbf{(d)--(f)} all pertain to the structures Pass can combine with. Pass can only merge with structures that have a guaranteed external argument: {\vd} and {\va} are fine, but as seen in the table, the other heads are not. The intuition is that Pass must ``know'' that it has an external argument to quantify over. This intuition is at least partly compatible with the structures in (d)--(f), and in fact Biblical Hebrew did have a passive counterpart to the ``simple'' (Voice+v) template. That template has since been lost.

The combination in (f) is correctly predicted not to be possible since Pass has no external argument to quantify over.

Finally, the combination in \textbf{(g)} is predicted not to exist, and correctly so. If \vz~and \pz~were to combine, the Figure role of \pz~would be passed up but no external argument would be able to saturate it. The derivation would crash at interpretation; this gap is predicted.

On the crosslinguistic angle now, the theory did predict that certain heads could combine. Since \va~is by definition a modifier of Voice, it may combine with Voice, yielding \tpie, or with \vz/\pz, yielding \thit. Because the individual elements all have their own syntax and semantics, the result of combining them is to a large extent predictable.

The combination of \vz~and \va~has already been explored for Greek in \S\ref{syn:middle:refl}, where it was shown that the two bring about reflexive readings. This is exactly the kind of pattern that is predicted to arise when these heads are in the structure, though as discussed, the rest of the grammar (including the root) needs to be compatible with the construction. At some point along the line the theory should be able to predict how heads combine in a given language, depending on other aspects of the grammar.

\cite{oseki16nyu} does suggest that the Japanese causative \emph{-sase} should be decomposed into an Applicative head \emph{-s-} and the \vd~morpheme -\emph{s}-{, in a system which allows for various combinations of Voice heads}. Other combinations thus do seem to be attested in one language where another cannot combine them; in Hebrew this {specific }combination cannot be examined because Hebrew does not have a ``morphological'' Appl affix; benefactive and malefactive arguments are introduced using the preposition \emph{le-} `to'.
\ex \begingl
    \gla ha-arje biʃel ʃu'it (\underline{la}-jeladim)//
    \glb the-lion cooked beans (\underline{to.the}-children)//
    \glft `The lion cooked (the children) beans'//
    \endgl
\xe

Standard Arabic and some Arabic dialects might be more informative. In these languages an ``Applicative template'' can introduce an applied argument (an extra indirect object) without need for a preposition \citep{alkaabi12}. It remains to be seen how the Hebrew framework can accommodate patterns in Arabic.

Applicatives bring up a more general point which relates to the kinds of argument-introducing heads in the syntax: the framework as a whole now allows for Voice heads, \emph{p} heads and Appl heads. All are empirically necessary, but recent work suggests that they need not be distinguished theoretically.

\section{Valuing features} \label{i:nie}
Nie

Legate

Wurmbrand

\section{Notes}
Triplets.
Legate conversation at CamVoice: she's against underspecified Voice. Instead, have active Voice, nonactive Voice, and then choose the zero allomorph based on some other structural difference between the transitive constructions. So one practical question is, is there some generalization about the internal arguments of thif vs transitive tkal, or of tnif vs unaccusative tkal.
	I don't think so, but it's worth looking at systematically. And worst case, I'll have a foil to argue against.
	The relevant question for me is point where vP is one event and then Voice does something predictable. Because triplets show that this is usually not the case: sagar-nisgar-hesgir. So it's not really the case that there's a "closing" vP and then you either block an external argument (tnif) or not (tkal), and also can require one (thif).
	So really what I'm forced to say is that the connection between sagar and nisgar is just as tight as between nisgar and hesgir.
	There aren't that many unaccusative doublets with tkal-tnif either (if any).
	I think this really requires figuring out the empirical state of affairs, quantitatively even (causativization isn't the main function of thif anyway).
	Now have a few examples in slides.

AM:  In short, it's not clear to me that you're going in a different direction than the agreement/valuation one -- you just have not yet emphasized the interplay of factors associated with ergativity, obligatory transitive verbs, and the featural EPP.
In filling out your typology of structures, with possible allosemes for the different voice heads, you might include the +D -s suffix of Japanese as one that, in the case of adversity causatives, does not have the agentive/causer semantics (but still has a thematic external argument, since the possessor role is "passed up" the tree).
And of course there are the deponent verbs --  -D voice that nevertheless has agentive semantics.