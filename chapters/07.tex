\chapter{The features of Voice}
\label{chap:i}

This book presents a theory of argument structure and its relation to morphology which was developed with two goals in mind: to remain as close as possible to existing analyses of non-Semitic languages (theoretical parsimony) and to explain the templatic system of verbs in Modern Hebrew (empirical adequacy). I have contended that this theory should be seen not only as a theory of Hebrew but as a theory of argument structure crosslinguistically. Accordingly, two questions should be distinguished when we ask about the crosslinguistic validity of the Trivalent Theory:
\begin{enumerate}
	\item Does the syntactic inventory of every language always contain these heads (Voice, v and \emph{p})?
	\item Does every language have the kinds of features on these heads that Hebrew does, i.e. {\vd}, \pz, etc.?
\end{enumerate}

In the current chapter I summarize the generalizations pointed out throughout the book and the means used to explain them (Section~\ref{i:sum}). I then discuss some related issues in the theory of Voice, chiefly the identity of this head (Section~\ref{i:i}, addressing the first question) and the identity of its features (Section~\ref{i:agree}, addressing the second question). Section~\ref{i:conc} concludes the chapter, and with it the book, although it will surely not be the final word (or template) on Hebrew morphology.


\section{Summary of the Trivalent approach} \label{i:sum}
The two questions I posed regarding verbal templates were as follows:
\begin{itemize}
	\item What are the possible readings associated with a given template (and why)?
	\item What templates does a given template alternate with (and why)?
\end{itemize}

The summary of this work is divided into the empirical generalizations which an adequate theory of Hebrew morphology must derive, and the formal system I have implemented to do so. Readers who have jumped directly to this section may want to consult Section~\ref{intro:arch} for some formal background.

\subsection{Summary of generalizations}
Every description or discussion of Hebrew morphology begins with the templatic system. The current book was also organized this way, laying out the empirical landscape template by template. The list below recaps the generalizations made about each template.

 \begin{exe}
 \ex  Generalizations about {\tkal}
 \begin{xlist} 
 	\ex  \textit{Configurations:} Verbs appear in all possible argument structure configurations. 
 	\ex  \textit{Alternations:} {\tkal} participates in alternations with the other templates. 
 \z

 \ex  Generalizations about {\tpie}
 \begin{xlist} 
 	\ex  \textit{Configurations:} Verbs appear in active (transitive/unergative) configurations. 
		Readings are weakly agentive.
 	\ex  \textit{Alternations:} When alternating with {\tkal}, {\tpie} provides a more ``intensive'' or agentive version. 
 \z

 \ex  Generalizations about {\tnif}
 \begin{xlist} 
 	\ex  \textit{Configurations:} Verbs appear in unaccusative, passive and figure reflexive structures, but never in a simple transitive configuration. 
 	\ex  \textit{Alternations:} Some verbs are anticausative or passive versions of verbs in {\tkal}. 
 \z

 \ex  Generalizations about {\thit}
 \begin{xlist} 
 	\ex  \textit{Configurations:} Verbs appear in unaccusative, figure reflexive and reflexive structures, but not in a simple transitive configuration. 
 	\ex  \textit{Alternations:} Some verbs are anticausative or reflexive versions of verbs in {\tpie}. 
 \z

 \ex  Generalizations about {\thif}
 \begin{xlist} 
 	\ex  \textit{Configurations:} Verbs appear in transitive and unergative configurations; a small class of verbs forms unaccusative degree achievements. 
 	\ex  \textit{Alternations:} Some verbs are causative or active versions of verbs in other templates, especially {\tkal}. A small class of verbs creates a labile alternation within {\thif}. 
 \z

 \ex  Generalizations about {\tpua} and {\thuf}
 \begin{xlist} 
 	\ex  \textit{Configurations:} Verbs appear in passive configurations only. 
 	\ex  \textit{Alternations:} Verbs in {\tpua} are always the passive version of an active verb in {\tpie}. Verbs in {\thuf} are always the passive version of an active verb in {\thif}. 
 \z
\z 

Some of these generalizations are quite unnatural. To take one example, {\tnif} has both unaccusative and unergative verbs, but no \isi{transitive} verbs; how should a formal analysis account for this? Having acknowledged the unergative and unaccusative verbs have different structures (the Unaccusativity hypothesis), contemporary theories can hardly refer to ``intransitivity'' as a theoretical primitive. These generalizations are summarized in Table~\ref{table:sum-naive}, where a na\"ive view of the templates defies economical description -- at least at first sight.\footnote{This table does not distinguish plain unergatives from \isi{figure reflexives}, precisely because traditional views do not make this distinction.}

\begin{table}
\begin{tabularx}{\textwidth}{llcccccl}
 \lsptoprule
& Template	& Unacc	& Unerg	& Transitive	& Reflexive	& Pass  & Alternations\\\midrule
a.& {\tkal}			& \cmark			& \cmark			& \cmark		& \xmark	& \xmark & b, c, e\\
b.& {\tpie}			& \xmark			& \cmark			& \cmark		& \xmark	& \xmark & a, d, f\\
c.& {\tnif}			& \cmark			& \cmark			& \xmark		& \xmark	& \cmark & a, e\\
d.& {\thit}			& \cmark			& \cmark			& \xmark		& \cmark	& \xmark & b, g\\
e.& {\thif}			& \cmark			& \cmark			& \cmark		& \xmark	& \xmark & a, c, e\\\tablevspace
f.& {\tpua}			& \xmark		& \xmark			& \xmark			& \xmark	& \cmark & b\\
g.& {\thuf}			& \xmark		& \xmark			& \xmark			& \xmark	& \cmark & g\\
\lspbottomrule
 \end{tabularx}
	\caption{A descriptive view of the templates}
	\label{table:sum-naive} 
\end{table}

Table~\ref{table:sum-naive} clearly shows the many-to-many mapping problem that arises when trying to link morphology and syntax/semantics in Hebrew (with the exception of the two \isi{passive} templates, perhaps the sole topic of consensus in the morphosyntactic literature on Hebrew). What is the reading of a verb in {\thit}? The answer is not deterministic: sometimes unaccusative, sometimes unergative, sometimes reflexive. What morphology do you choose if you want to express an unaccusative verb? The answer is again not deterministic: sometimes {\tkal}, sometimes {\tnif}, sometimes {\thit} and sometimes {\thif}. So form cannot map directly to meaning, if by ``form'' we mean templates. My critique of \cite{arad05} in previous chapters hopefully showed just how important that work was in clearing up this point.

And once you have chosen a template, how do you know which template to derive an alternation in? Should you go from an anticausative in {\tnif} to {\tkal} or to {\thif}? We have seen that both are possible, as shown in Table~\ref{tab:7-1:both}.
\begin{table}
	\begin{tabularx}{\textwidth}{llclccc}
 \lsptoprule
		&	&	&	&	Anticausative & \multicolumn{2}{c}{Causative}\\
		&	&	&	&	{\tnif}	& {\tkal} & {\thif}\\\midrule
	\root{ʃbr} & broke & $\sim$ & broke	& \emph{niʃbar} & \emph{ʃavar} & --- \\
	\root{kxd} & went extinct & $\sim$ & eradicated & \emph{nikxad} & --- & \emph{hekxid}\\
\lspbottomrule
 	\end{tabularx}
	\caption{Causative and anticausative alternations}
	\label{tab:7-1:both}
\end{table}

Again, some patterns exist but looking at the templates in terms of primitives makes it hard to understand why precisely these patterns arise, above and beyond the idiosyncrasy of individual roots.

Some of the generalizations tallied above are novel. The entire formal conception of \isi{figure reflexives} in Hebrew \citep{kastner16phd} and indeed in this general framework is recent \citep{wood12phd,wood14nllt}. The characterization of unaccusatives in {\thif} as \isi{degree achievements} is recent, although not my own \citep{lev16}. The observation that {\thif} is not a \isi{causative} template but a general active one, which might also have lexical causatives, is novel as far as I can tell, at least when put in these terms. The idea that the distinction between anticausatives and reflexives in {\thit} reflects the lexical semantics of the roots is also very recent \citep{kastner17gjgl}. And the allomorphic interactions which I have identified have also only been noticed recently in a unified way \citep{kastner18nllt}. Even if the analysis ends up being inadequate, I believe that these generalizations are important and hope that future work can engage with them rigorously.

	\subsection{D-composing the templates}
Instead of talking about templates, I have proposed that we talk about syntactic structure and cyclic derivations. This was done using the head Voice and allowing it to take one of three values: [+D], [\textminus{}D] or unspecified for [D]. Informal definitions are given in~(\ref{ex:7n7}).
 \begin{exe}
 \ex  \label{ex:7n7}Overt Voice heads in Hebrew 
 \begin{xlist} 
 \ex  {\vd}: requires a DP in its specifier. 
 \ex  {\vz}: prohibits a DP in its specifier. 
 \ex  Unspecified Voice: places no restrictions on its specifier. 
 \z
\z 

Acknowledging the existence of an overt agentive modifier whose semantics is difficult to pin down but which does not seem to be active in the syntax adds {\va} to our toolbox.
 \begin{exe}
\ex  {\va}: an overt modifier adding \textsc{action} (agentive) semantics to an event. In practice, this often creates a requirement for an Agent role to be saturated. 
 \z 

Assuming that \isi{prepositional phrases} are derived similarly to verbs gives us the functional head \textit{p} and its [\textminus{}D] variant {\pz} (more on this in Section~\ref{i:i:i}).
 \begin{exe}
 \ex  
 \begin{xlist} 
 	\ex  \textit{p}: the prepositional equivalent of Voice; introduces the Figure role (the subject of a preposition). 
 	\ex  {\pz}: prohibits a DP in its specifier. 
 \z
\z 

The \isi{passive} head Pass rounds off the picture.
 \begin{exe}
\ex  Pass: prevents the projection of an argument in Spec,VoiceP and closes off the Agent role existentially in the semantics. 
 \z 

When these elements are combined, the resulting picture is not the confusing one in Table~\ref{table:sum-naive} but the principled one in Table~\ref{table:summary-syn-rep2}. These elements and the values of their features explain what structures give rise to what spell-outs, or in other words, what \textsc{configurations} a given template is possible in.
\begin{table}
	\fittable{
		\begin{tabularx}{\textwidth}{llllcclc}
			\lsptoprule
			\multicolumn{4}{c}{Heads} & Syn 	& Sem & Phono & Ch\\\midrule
			
			& Voice& &	&  	& 	&  \emph{XaYaZ} & \ref{voice:voice} \\\tablevspace
			
			& Voice&\red{\va}&	& 	& \red{Action}	 & \emph{X{\red{i\dgs{Y}e}}Z}&  \ref{voice:va}	\\
			
			\olive{Pass} & Voice&\red{\va}&	& \olive{Passive}	& \red{Action}	 & \emph{X\olive{u}{\red{\dgs{Y}}}\olive{a}Z}&  \ref{passn:pass:pass}	\\\tablevspace
			
			& \blue{\vd}& &		& \blue{EA}	& 	 & \emph{{\blue{he}}-XY{\blue{i}}Z} & \ref{vd:vd} \\
			
			\olive{Pass} & \blue{\vd}& &		& \olive{Passive}, \blue{EA}	& 	 & \emph{{\blue{h}}\olive{u}-XY\olive{a}Z} & \ref{passn:pass:pass} \\\tablevspace
			
			& \blue{\vz}& &		& \blue{No EA}	& 	 & \multirow{2}{*}{\emph{{\blue{ni}}-XY{\blue{a}}Z}} & \ref{vz:vz} \\
			& Voice& &\blue{\pz}	& \blue{EA = Figure} & 	 &  & \ref{vz:pz} \\\tablevspace
			& \blue{\vz}&\red{\va}&	& \blue{No EA}	& \red{Action}	 & \multirow{2}{*}{\emph{{\blue{hit}}-X{\red{a\dgs{Y}e}}Z} } &  \ref{vz:va:vzva} \\
			& Voice&\red{\va}&\blue{\pz}	& \blue{EA = Figure} & \red{Action}	 & & \ref{vz:va:pzva} \\
			\lspbottomrule
		\end{tabularx}
	}
	\caption{Functional heads in the Hebrew verb}
	\label{table:summary-syn-rep2}
\end{table}

The combinatorics of these heads -- specifically unattested combinations -- were addressed in the individual chapters. See~\citet[Ch.~2.4.1.1]{kastner16phd} for a summary.

The hierarchical structure -- that is to say, the layering of Voice above vP -- explains what \textsc{alternations} are possible. Since Voice merges above a core vP, the three values of Voice can give rise to three templates in transitivity\is{transitive} alternations ({\tkal}, {\tnif} and {\thif}). Because {\va} merges with the core vP, {\tpie} has a less transparent, but still fairly consistent, relationship with {\tkal}. And given that the core vP with {\va} can then combine with either unspecified Voice or {\vz}, we derive the alternation between {\tpie} and {\thit}. The final alternation, between active {\tpie} and {\thif} on one side and \isi{passive} {\tpua} and {\thuf} on the other, follows from merging Pass with the active VoiceP. Alternations are therefore not listed extrinsically nor are they the property of templates. They are what we see if a given root is instantiated in a number of structures which share a core vP, similar to \citeauthor{schaefer08}'s (\citeyear{schaefer08}) conclusion for the \isi{causative} alternation in European languages.

Let me reiterate the role of \textsc{roots} in this theory and in any theory of Semitic: at some level it must be listed what functional heads are licensed\is{licensing} by what roots. I have not attempted to formalize this matter. \cite{arad05} makes a strong case for the view under which the combination is almost entirely arbitrary, in that each combination of root and template must be listed. I do not know whether this is necessarily true, but currently have no better way of describing the system. As emphasized throughout the book, the same issue arises regardless of language or theoretical framework.

This analysis aims to be simple: the child only posits variants of basic elements if she has evidence for them. Hebrew has phonological (and of course syntactic and semantic) evidence for all and only the elements in Table~\ref{table:summary-syn-rep2}, obviating the need for additional silent structure. In the next section I try to take a step back from Semitic, considering the conceptual and crosslinguistic inventory of syntactic heads.


\section{Voice heads} \label{i:i}
The framework as a whole allows for Voice heads and \emph{p} heads, alongside verbalizers (little v, n and \textit{a}). I also assume that applicatives are introduced using the head Appl\is{applicative}, although I made no claims about applicatives in Hebrew. These heads all appear to be empirically necessary, but recent work suggests that they need not be distinguished theoretically. In what comes next I review how the Trivalent proposal can be applied to an unrelated language, Japanese (Section~\ref{i:i:jap}), as well as a recent theoretical proposal which dovetails nicely with the current proposal (Section~\ref{i:i:i}), and the crosslinguistic prospects for the Trivalent approach to Voice (Section~\ref{i:i:ay}).

	\subsection{Japanese} \label{i:i:jap}
The morphology of transitivity\is{transitive} alternations in \ili{Japanese} has received significant attention over the years \citep{suga80,jacobsen92,miyagawa98,nishiyama98,volpe05,harley08}. Recently, \cite{oseki17nyu} proposed an analysis of argument structure alternations and their expression in the morphology of Japanese which builds on the Trivalent system. Here is an outline and quick evaluation.

Some verbs like \emph{hirak} `open' have the labile alternation:
 \begin{exe}
 \ex  \langinfo{Japanese}{}{\citealt{oseki17nyu}}
 \begin{xlist} 
 	\ex   
[] 		{ \gll John-ga doa-o \glemph{hirak-\zero-ta}.\\
 		  John-\gsc{NOM} door-\gsc{ACC} open-\zero-Past\\
 		\glt `John opened the door.' } 
	
 	\ex   
[] 		{ \gll Doa-ga \glemph{hirak-\zero-ta}.\\
 		  door-\gsc{NOM} open-\zero-Past\\
 		\glt `The door opened.' } 
	
 \z
\z 

With many other verbs, overt transitivity\is{transitive} markers can be found, namely -\gsc{S}- (and allomorphs) for marked\is{markedness} transitives and -\gsc{R}- (and allomorphs) for marked\is{markedness} intransitives:
 \begin{exe}
 \ex  
 \begin{xlist} 
 	\ex   
[] 		{ \gll John-ga posutaa-o \glemph{hag-as-ta}.\\
 		  John-\gsc{NOM} poster-\gsc{ACC} peel-\gsc{S}-Past\\
 		\glt `John took down a poster.' } 
	
 	\ex   
[] 		{ \gll Syatsu-ga \glemph{chijim-ar-ta}.\\
 		  shirt-\gsc{NOM} shrink-\gsc{R}-Past\\
 		\glt `A shirt shrank.' } 
	
 \z
\z 

Tellingly, Japanese has ``minimal triplets'' similar to those which motivated the current proposal for Hebrew. \cite{oseki17nyu} reproduces data such as those in Table~\ref{tab:7-2:j3} from \cite{suga80}:
\begin{table}
\begin{tabularx}{\textwidth}{llll}
 \lsptoprule
	& Marked intransitive & Unmarked intransitive & Marked transitive\\\midrule
	\root{\gsc{PEEL}} & hag-er-u & hag-\zero-u & hag-as-u\\
	\root{\gsc{CUT}} & kur-er-u & kir-\zero-u & kir-as-u\\
\lspbottomrule
 \end{tabularx}
	\caption{Minimal triplets in Japanese}
	\label{tab:7-2:j3}
\end{table}

These patterns can be understood if we assume something like the following for Japanese:
 \begin{exe}
 \ex  
 \begin{xlist} 
 	\ex  Voice \lra~\zero 
 	\ex  {\vz} \lra~-\gsc{R}- 
 	\ex  {\vd} \lra~-\gsc{S}- 
 \z
\z 

\cite{oseki17nyu} then goes on to show how this system predicts the behavior of the causative \emph{-sase} applicatives as \emph{-s-} over \emph{-s-} (potentially {\vd} over {\vd}) and of the passive \emph{rare} as \emph{-r-} over \emph{-r-} (potentially {\vz} over {\vz}).\footnote{In Hebrew this specific combination cannot be examined because Hebrew does not have a ``morphological'' Appl affix; benefactive and malefactive arguments are introduced using the preposition \emph{le-} `to'. 
 \begin{exe}
\ex   
[]     { \gll ha-arje biʃel ʃuit (\glemphu{la}-jeladim).\\
       the-lion cooked beans (to.the-children)\\
     \glt `The lion cooked (the children) beans.' } 
    
 \z 
}

Appealing as this analysis of Japanese may be, I must note two aspects in which it diverges from the Trivalent proposal for Hebrew. First, \gsc{S}-marked\is{markedness} verbs are always \isi{transitive}. In fact, \citet[26]{nie17} notes that the alternation between a marked\is{markedness} and unmarked\is{markedness} \isi{transitive} resembles a Differential Object Marking pattern, correlating with the appearance of the object marker -\emph{o} (glossed \gsc{ACC}) for some speakers. Marked transitives are only possible with -\emph{o}, but zero-marking is possible regardless of whether the object marker appears. Yet Hebrew {\vd} does not enforce transitivity\is{transitive}, only an external argument.
 \begin{exe}
 \ex  
 \begin{xlist} 
 	\ex   
[] 		{ \gll John-ga posutaa-\glemphu{o} \glemph{hag-\zero/as-ta}.\\
 		  John-\gsc{NOM} poster-\gsc{ACC} peel-\zero/\gsc{S}-Past\\
 		\glt `John took down (a specific) poster.' } 
	
 	\ex   
[] 		{ \gll John-ga posutaa \glemph{hag-\zero/(*as)-ta}.\\
 		  John-\gsc{NOM} poster-\gsc{ACC} peel-\zero/\gsc{S}-Past\\
 		\glt `John took down a poster (some poster).' } 
	
 \z
\z 

Second, these patterns are not necessarily the entire story. \citet[9]{oseki17nyu} also gives the patterns in Table~\ref{tab:7-2:j32} from \cite{suga80}, in which \emph{-er-} is now the transitivity\is{transitive} marker and the marked\is{markedness} intransitive -\gsc{R}- form is -\emph{ar}-.
\begin{table}
\begin{tabularx}{\textwidth}{llll}
 \lsptoprule
 	Marked intransitive & Unmarked transitive & Marked transitive\\\midrule
	\root{\gsc{SHRINK}} & chijim-ar-u & chijim-\zero-u & chijim-e-ru\\
	\root{\gsc{MOVE}} & tsutaw-ar-u	& tsutaw-\zero-u & tsutaw-e-ru\\
\lspbottomrule
 \end{tabularx}
	\caption{Alternative minimal triplets in Japanese}
	\label{tab:7-2:j32}
\end{table}

To what extent these patterns are phonologically predictable and to what extent there is accidental syncretism are important questions. But they are tangential to my current reservation, which is simply that more work needs to be done in order to understand Japanese transitivity\is{transitive} alternations. The Trivalent system therefore provides a good starting point for investigating Japanese, especially if it can provide novel explanations for the morphology of applicatives and passives, and if it can explain DOM-like patterns (Section~\ref{i:agree:nie}).

	\subsection{\emph{i}*} \label{i:i:i}
In a recent account of the way argument structure is derived and interpreted, \cite{woodmarantz17} propose to reduce the overall inventory of functional heads. Working within a similar framework, they suggest that non-internal arguments (external and applied arguments introduced by Voice, Appl\is{applicative}, \emph{p} and P) are in fact introduced by contextual variants of the same predicational head. This head is called \emph{\isi{i*}}. 

If \cite{woodmarantz17} are correct, the difference between \emph{p}, Appl\is{applicative} and Voice is an illusion: they are all the same predicational head underlyingly, albeit in different contexts. Voice is but \emph{\isi{i*}} that merges with a vP. Little \emph{p} is but \emph{\isi{i*}} that merges with a PP\is{prepositional phrases}. And P itself is \emph{\isi{i*}} modified by a (prepositional) root; see the work cited for full details. The analytical possibilities opened up by this view have already been pursued in a range of recent work, including Appl\is{applicative} and Voice in German psych-predicates \citep{hirsch18phd}, and P and Appl\is{applicative} in Russian datives \citep{bonehnash17}.

My goal here is not to evaluate this proposal, which is supported by conceptual considerations as well as empirical study of \isi{figure reflexives} in Icelandic, the Adversity Causative in Japanese and possession in Quechua and other languages. Instead, I want to highlight one welcome point of convergence between the \emph{\isi{i*}} hypothesis and my proposal for Hebrew. In the inventory of functional heads I have laid out, {\vz} and {\pz} are conspicuously similar: they do similar work in the syntax and have the same spell-out. If we follow the \emph{\isi{i*}} hypothesis, the two \emph{should} be similar: they are the same functional head, only in different contexts (\ref{ex:7n15}).
 \begin{exe}
 \ex  \label{ex:7n15}
 \begin{xlist} 
 	\ex  Anticausative in {\tnif} \\
	\Tree
	[.{\textit{i}*P\\(VoiceP)}
		[.{---} ]
		[.
			[.{\textit{i}*$_{\text{[\textminus{}D]}}$\\ (\vz)\\ \emph{ni-}} ]
			[.vP ]
		]
	]
 	\ex  Figure reflexive in {\tnif} \\
	\Tree
	[.{\textit{i}*P\\ (VoiceP)}
		[.DP ]
		[.
			[.{\emph{i*}\\ (Voice)} ]
			[.vP
				[.v ]
				[.{\textit{i*}P\\ (\textit{p}P)}
					[.{---} ]
					[.
						[.{\textit{i}*$_{\text{[\textminus{}D]}}$\\ (\pz)\\ \emph{ni-}} ]
						[.PP ]
					]
				]
			]
		]
	]
 \z
\z 

To be clear, I do not believe that the \emph{\isi{i*}} hypothesis must be true for the Trivalent account to go through. But if this hypothesis is on the right track, a strong version can be formulated under which all exponents of \emph{\isi{i*}} (as well as its variants \emph{\isi{i*}}$_{\text{[\textminus{}D]}}$ and \emph{\isi{i*}}$_{\text{[+D]}}$) should be identical to each other. Such a hypothesis would immediately predict the similarity between Voice and \emph{p} -- both default and silent -- and that between {\vz} and {\pz}.

	\subsection{Trivalent Voice/\textit{i*} crosslinguistically} \label{i:i:ay} \label{r1:g:1}
This chapter began by asking whether the syntactic inventory of every language always contain the heads Voice, v, \textit{p} and potentially Appl\is{applicative}. My working hypothesis is that the answer is yes. I assume that Voice, v and \emph{p} are an inherent part of the syntactic system of every language. I am less certain about the \isi{applicative} head Appl\is{applicative}, although under the \textit{\isi{i*}} hypothesis there is no difference between Appl\is{applicative} and Voice or \textit{p}. 

In \ili{Kinyarwanda}, the \isi{causative} and instrumental \isi{applicative} suffixes are spelled out identically \citep{jerro17} and in the Algonquian language \ili{Penobscot}, many ``relational predicates'' have similar if not identical morphology for causatives and applicatives \citep[Ch.~2.3.7.1]{quinn06phd}. The \emph{\isi{i*}} hypothesis would lead us to expect similar correlations between different heads crosslinguistically. Combining the outlined analysis of Japanese above with the basics of \emph{\isi{i*}}, we arrive at the picture in Table~\ref{table:heads-langs2}.

\begin{table}
	\begin{tabular}{lll}
 \lsptoprule
	Head 		& Hebrew 	  		& Japanese\\\midrule
	\emph{i*}   	& \tkal          & -\emph{e}-\\
	\emph{i*}$_{\text{[\textminus{}D]}}$ 	& \tnif 		& -\emph{r}-\\
	\emph{i*}$_{\text{[+D]}}$	& \thif		& -\emph{s}-\\
\lspbottomrule
 	\end{tabular}
	\caption{Flavors of \textit{i*} in Hebrew and \ili{Japanese}\label{table:heads-langs2}}
\end{table}

How far can this idea extend? Within Semitic, Standard \ili{Arabic} and some Arabic dialects might be informative. Little contemporary work has investigated the morphosyntax of templates in Semitic languages in depth \citep{kastnertucker19cup}: even though a significant number of studies have explored morphosyntactic processes and their interaction with phonological exponence in languages such as Amharic, Mehri and Maltese (e.g.~\citealt{kramer14,kramer16li,doronkhan16,faust16,faust18gjgl,faust19,rood17phd,winchester17nels,winchester19phd,kalin18,akkus19jl}), these works do not usually discuss verbal morphology as such.

One recent contribution to the study of Semitic verbal morphology is that of \cite{alkaabintelitheos19}, who put forward a formal account of verbal morphology in Emirati Arabic. This DM analysis recasts the heads proposed by \cite{doron03} as features on the heads v and Voice. One drawback of this approach is that features such as [\gsc{caus}] and [\gsc{appl}], which are spelled out as templates in a fashion similar to the Trivalent account, do not have predictable syntactic or semantic properties; the \isi{causative} feature does carry out any syntactic work (e.g.~introducing an additional argument) or semantic work~(e.g.~introducing \isi{causative} semantics) consistently. I believe that this specific work showcases what may well be the case in many other Semitic languages: that templates do not have syntactic and semantic requirements which are as stringent as those in Hebrew.

If that is the case, then a Trivalent analysis would not be suitable and a number of options would deserve a closer look. Perhaps an \citeauthor{arad05}-style analysis might be more appropriate, where conjugation\is{conjugation class} classes must be listed (beyond perhaps maintaining a Passive head and a Non-Active Voice head). Some work might alternatively be done by invoking semi-lexical roots similar to {\va}. Or maybe ``flavors'' of v might be necessary after all \citep{katie13}, listing different verbalizers instead of making syntactic commitments, as with \citeauthor{alkaabintelitheos19}' features.

\label{r1:6:5}Generalizing beyond Semitic, a strong view restricting structure-building of verbs to v, Voice/\textit{i*} and Pass means there is no room for specialized heads such as Reflexive (\citealt{ahn15phd}, cf.~\citealt{spathas17camvoice,spathas17debrecen}) or Reciprocal \citep{bruening04}. There is also the question of languages with dedicated ``slots'' for different argument structure affixes, like the CARP template of Bantu\il{Bantu languages} and similar phenomena \citep{hyman03,paster05}. I believe that exploring this strong claim will lead to new discoveries and refute the dedicated-heads approaches (although tough talk comes cheap), aiming for a constrained and therefore more Minimalist inventory of functional heads. One reason for my optimism is that data which had received analysis in terms of lexicalist affix slots was later shown to be analyzable in more decompositional terms; here I have in mind the \ili{Nimboran} debate between \cite{inkelas93} and \cite{noyer98}.

It thus appears to be potentially useful to adopt this framework for additional languages and map out which heads and features are instantiated in which language. The heads were discussed in this section; features are the topic of the next section, which is also the last one of this book before concluding.


\section{Features on Voice} \label{i:agree}
The second question opening this chapter had to do with the inventory of features which might exist on \textit{\isi{i*}}/Voice. It is interesting to note that virtually all work on the features of Voice assumes the [D] feature and/or phi-features.

In Hebrew, I have made the case that Voice can be [+D] as in {\vd}, [\textminus{}D] as in {\vz} or not inherently valued\is{Agree} as in unspecified Voice. In principle however, the architecture allows any syntactic feature to appear on Voice. Nothing in the theory prohibits Voice$_{\text{[wh]}}$, for instance, which would require a \emph{wh}-phrase in Spec,VoiceP. Now granted, any theory of syntax must stipulate in one way or another which features are possible on which functional elements. One way to restrict the theory is to require only \textsc{uninterpretable} features \citep{chomsky95}, being purely syntactic features, to exist on Voice. The \isi{EPP} feature [D] is one such feature. This kind of solution would rely on a certain view of which features are interpretable and which are not (the notion of uninterpretable features and whether they are necessary has itself been questioned in recent work, e.g.~\citealt{preminger14mit}).

\label{r1:g:2a2}Another issue that I have not addressed so far is the extent to which [D] really is an EPP feature. Even though I have referred to it as an EPP feature in passing -- see Sections~\ref{intro:arch:voice}, \ref{voice:voice:sum} and~\ref{aas:layering:features} -- the original characterization of the EPP on Voice by \cite{chomsky00,chomsky01} cast it as the strong feature v*, providing a landing site for arguments not selected by the head itself: a position for shifted objects in Spec,v*. In this strict sense, then, the EPP on T or contemporary Voice and the EPP on Chomskian v* are not the same. Current theories that make use of [D] in order to regulate the syntactic projection of the external argument have a different conception of the EPP on v*/Voice, converging on the \textsc{strong feature} use of \cite{adger03}. See \cite{adgersvenonius11} for related general discussion.

\label{r1:g:2c1}If [D] is a feature like all others, or at least an EPP feature like the one often postulated on T, then we would be led to expect trivalent [$\pm$D] on T and perhaps other heads as usual. It is not clear whether such cases exist, in which case we would have additional reason to think that the verbal domain (i.e.~VoiceP) is privileged in the ways it can introduce arguments \citep{grimshaw00,woodmarantz17}. The existence of [\textminus{}D] as a feature prohibiting Merge is also a theoretical innovation, although \cite{harbour11,harbour14} has already argued for the relevance of [\textminus{}$\alpha$] features on logical as well as empirical grounds.\footnote{The idea of a feature on specific heads banning merge might be reminiscent of the anti-locality account of \cite{grohmann03}, and see also \cite{baier18phd}, although the details differ considerably.}

Ideally, a theory of features on argument-introducing heads would be part of a general theory of argument structure, feeding processes such as \isi{case} assignment and specifying the triggers for A-\isi{movement}. Some recent theories do exactly that by recourse to [D] and phi-features.

	\subsection{Layering}
Within the standard \isi{Layering} approach \citep{schaefer08,schaefer12,schaefer17oup}, two sets of features have been associated with Voice. The first involves syntactic and semantic transitivity\is{transitive}, as discussed in Chapter~\ref{chap:aas}. The second is a set of phi-features. These are used mainly to explain the behavior of reflexive/expletive pronouns such as \ili{French} \emph{se} or \ili{German} \emph{sich}, \isi{case} assignment and agreement. Since Hebrew does not have expletives such as these, I have not invoked phi-features alongside [$\pm$D].
 \begin{exe}
 \ex  Features of Voice under Layering: 
 \begin{xlist} 
 	\ex  {[}D] regulating Spec,VoiceP. 
 	\ex  λx/∃x/\zero~regulating the semantics of the external argument (if any). 
 	\ex  Phi-features regulating case assignment (primarily with expletives). 
 \z
\z 
In this theory, [D] is still the feature regulating most of the argument structure; the discussion is mostly based on data from Germanic and Romance.

	\subsection{Restriction}
\cite{legate12lang,legate14} presents a different theory, one in which phi-features on Voice serve to Restrict the arguments with which they are associated. \textsc{Restrict} is meant in both an intuitive sense and in the formal sense of \cite{chungladusaw04}: a 3rd person specification on Voice means that a 2nd person argument will not be possible. The bundle of phi-features may appear either on Voice or in Spec,VoiceP (instead of a regular external argument). Object licensing features are generated on Voice and inherited by v.

If the features are on Voice, they serve to restrict the \isi{Agent}. This can be either the subject of an active clause (merged in Spec,VoiceP), or the \isi{Agent} of a \isi{passive} clause, regardless of whether it is implied or overt in a \emph{by}-phrase.\footnote{The exact mechanics require existential closure to apply to the \isi{Agent} role if no \emph{by}-phrase appears, but this issue arises for all mainstream analyses of the \isi{passive}; see \cite{williams15} for background.} Crosslinguistically, these features appear overtly as a prefix denoting person and familiarity on the verb (in \ili{Acehnese}), as a dedicated form of the \isi{passive} prefix alternating according to number (in \ili{Chamorro}), or as a specific suffix for \gsc{3SG} \isi{passive} verbs (in \ili{Balinese}).

If the features are merged in Spec,VoiceP they can only restrict the DP in the \emph{by}-phrase of a \isi{passive} and are covert (in \ili{Ukranian} and the \ili{Icelandic} New Passive). This is a kind of Weak Implicit Argument in the terminology of \cite{landau10}.

Impersonal constructions are analyzed as structures in which a silent pronoun in Spec,VoiceP has both a [D] feature and phi-features; this is essentially \textit{pro}, or the Strong Implicit Argument of \cite{landau10}.
 \begin{exe}
 \ex  Features of Voice under the restriction theory inspired by Acehnese: 
 \begin{xlist} 
 	\ex  λx for Agent semantics. 
 	\ex  Phi-features restricting the Agent. 
 	\ex  Phi-features can be merged either directly on Voice (overt) or in Spec,VoiceP (covert). 
 \z
\z 
In this theory, phi-features regulate argument structure, at least in the relevant constructions. The data come from three Austronesian languages (features on Voice) and from Ukranian and Icelandic (features on Spec,VoiceP).

	\subsection{Restructuring}
Another kind of theory was proposed by \cite{wurmbrandshimamura17}. For them, Voice has two sets of features. \textsc{Voice features} regulate whether the clause is active or \isi{passive}, and introduce \isi{Agent}/\isi{Causer} semantics. Voice also carries phi-features which may be inserted valued\is{Agree} or unvalued. Object \isi{licensing} features (\gsc{ACC}) are also generated on Voice.

In an active clause, unvalued phi-features on Voice need to be valued\is{Agree} by the external argument DP. Its merger is thus triggered by the phi-features and not by the feature Voice[\gsc{\isi{Agent}}]. In a \isi{passive} clause, Voice carries valued\is{Agree} phi-features as proposed by~\cite{legate14} and the \gsc{Pass} feature. Feature checking between Voice and v/V spells out the appropriate participial morphology on V (in English), and incorporation of v/V to Voice results in affixal morphology.

This is the basic architecture. Things get interesting when \cite{wurmbrandshimamura17} propose to incorporate restructuring into their theory. This happens if Voice is unvalued for the Voice feature, i.e.~gets merged without an \gsc{Agent\is{Agent}} or \gsc{pass} feature, in another sort of trivalent setup. The unvalued phi-features on Voice will then be valued\is{Agree} by the higher Agent\is{Agent}/Voice. Importantly for this analysis, there is also a V feature on v which needs to be valued\is{Agree}. There are two possibilities for how this is accomplished, each predicting a different kind of restructuring language (voice matching languages and default voice languages). Empirical details aside, the main contribution of this work to the general issue of Voice lies in bringing in considerations from long object \isi{movement} and restructuring, motivating a series of agreement interactions between Voice, v, V and higher elements in the clause.
 \begin{exe}
 \ex  Features of Voice under the restructuring-inspired theory: 
 \begin{xlist} 
 	\ex  \gsc{agent/passive}/\zero~on Voice. 
 	\ex  Valued or unvalued phi-features on Voice. 
 	\ex  V feature on v. 
 \z
\z 
Feature valuation regulates argument structure. Since the focus of this work was on accounting for restructuring patterns in languages as diverse as \ili{Acehnese}, \ili{Chamorro}, \ili{Japanese} and Mayrinax \ili{Atayal}, the combinatorics of the theory were not investigated in full.

	\subsection{Valuation} \label{i:agree:nie}
The final theory to be mentioned here is the Trivalent Valuation Theory developed by \cite{nie17}. Following cues from all this recent work, including the Trivalent proposal, \citeauthor{nie17} proposes that Voice carries a trivalent [D] feature and phi-features. Taking a page out of \citeauthor{wurmbrandshimamura17}'s playbook, she assumes that features may be \textsc{lexical} (inherent, merged as-is) or \textsc{derived}, i.e.~valued\is{Agree} in the course of the derivation. \citeauthor{nie17} then explores this idea in detail, making the case for a lexical [$\pm$D] feature, regulating the external argument, and derived phi-features, which are normally valued\is{Agree} by the internal argument but which may remain unvalued. This combination allows her, first, to apply the Trivalent approach to additional languages. For example, she analyzes \ili{Niuean} and the Formosan language \ili{Puyuma} as exhibiting the Trivalent distinction:
 \begin{exe}
 \ex \label{ex:puyuma} \langinfo{Puyuma}{Formosan}{\citealt{nie17}}
 \begin{xlist} 
 	\ex {  {\vd} is marked with an infix; obligatory external argument. \\
		 \gll s\glemph{<em>}a-senay i baeli.\\
 		  {<}AV>-sing \gsc{NOM.SG} my.elder.sibling\\
 		\glt `My elder sister is/was singing.' \hfill \citep[68]{teng07} }
	
 	\ex {  Unmarked Voice is spelled out by a prefix; external argument optional. \\
		 \gll \glemph{mu}-atel la na ladru (dra balri).\\
 		  \gsc{MU}-fall \gsc{PERF} \gsc{NOM.DEF} mango \gsc{OBL.IND} wind\\
 		\glt `The mango fell. / Wind made the mango fall.' \hfill \citep[5]{chenfukuda17} } 
	
 	\ex { {\vz} is unmarked morphologically. \\
	 \gll \glemph{drua} nantu lalak.\\
 		  came her.\gsc{NOM} child\\
 		\glt `Her child came.' \hfill \citep[222]{teng07} } 
 \z
\z 

Second, the Austronesian distinction between what are traditionally called \isi{Agent} Voice and Patient Voice can be seen as a difference in transitivity, and hence is analyzed as a difference in phi-features. In these languages, the pivot receives privileged marking: nominative\is{case} in Puyuma, as in~(\ref{ex:7n20}), or \emph{ang} in \ili{Tagalog}.
 \begin{exe}
 \ex \label{ex:7n20} % Puyuma \citep{nie17} 
 \begin{xlist} 
 	\ex   
[] 		{ \gll tr<em>akaw dra paisu i Isaw.\\
 		  {<}\gsc{AV}{>}steal \gsc{OBL.IND} money \gsc{NOM.SG} Isaw\\
 		\glt `Isaw stole money.' } 
	
 	\ex   
[] 		{ \gll tu=trakaw-aw na paisu kan Isaw.\\
 		  3.\gsc{GEN}=steal-\gsc{PV} \gsc{NOM.DEF} money \gsc{OBL.SG} Isaw\\
 		\glt `Isaw stole the money.' } 
	
 \z
\z 

Both \isi{Agent} Voice and Patient Voice are assumed to be reflexes of {\vd} (see \citealt{nie17} for details on the other ``Voices''). However, the choice \emph{between} them depends on the \isi{case} and agreement\is{Agree} properties of the internal argument. If the internal argument values the phi-features on v, then it is marked\is{markedness} as the pivot. These phi-features are then transferred from v to Voice, valuing Voice. This is ``Patient Voice''. If the internal argument does not agree with v, then v does not transfer phi-features to Voice. Instead, Voice has them valued\is{Agree} by the external argument, with predictable consequences: Voice gets ``\isi{Agent} Voice'' and the subject is marked\is{markedness} as pivot. \cite{nie17} equates this kind of interaction with patterns of \isi{case} and agreement in ergative languages and with DOM effects in nominative\is{case} languages. She argues, for example, that a similar difference in transitivity\is{transitive} marking explains the distribution of the object marker -\emph{o} and \isi{transitive} morphology in Japanese.
 \begin{exe}
 \ex  Features of Voice under the valuation theory: 
 \begin{xlist} 
 	\ex  {[}D] regulates the licensing of an external argument. 
 	\ex  Phi-features on Voice can be valued from either below (internal argument) or above (external argument). 
 \z
\z 
Broadly speaking, [D] regulates argument structure and the phi-features regulate \isi{case} and agreement marking.

	\subsection{Towards a uniform inventory}
Where does this proliferation of recent theories leave us? It seems unlikely to me that the Trivalent proposal can be restated in terms of approaches that do not assume Trivalent Voice. While many parts of the current approach are compatible with translation to phi-features, others are less so: the restriction enforced by {\vz} is not one that can be easily captured in terms of phi-features on Voice, unless it is specified that these features can only be valued\is{Agree} by the \emph{internal} argument. Similarly, \isi{Unspecified Voice} (or {\tkal}) cannot be the deterministic spell-out of configurations whose phi-features are at times valued\is{Agree} by an external argument and at times are not valued\is{Agree} (or are valued\is{Agree} by the internal argument).

Generally speaking, the question whether a unified account of all these Voice-related phenomena would be possible is still open \citep[191fn12]{wurmbrandshimamura17}. The various approaches to the content of Voice have converged on [D] and phi-features being the most important pieces in the syntax, with various possible interactions with \isi{case} and agreement. How much or how little of each is needed, what other possible values of Voice might be relevant, and what elements we expect to be overt are questions that will likely occupy syntacticians for some time to come.


\section{Conclusion} \label{i:conc}
This book set out to analyze the verbal system in Hebrew in a way that is formally explicit, internally consistent, and easily translatable to accounts implemented for other languages and phenomena. The basic idea was that a core vP can be modified by an agentive modifier (or not) and combined with a Voice head specified as [+D], specified as [\textminus{}D], or unspecified for [D]. As a result, what looks like templates and alternations between templates in Hebrew can be reduced to the interpretation of well-understood syntactic structures by the semantics and the phonology (fed by some idiosyncrasy associated with lexical roots). The templates have been decomposed.

In my inherently biased view these goals have been reached; the generalizations, analyses and predictions of Part~\ref{part:1} were summarized at the outset of this chapter. Where possible, the connection to the phonology was also elaborated on. And Part~\ref{part:2} laid out how I see this theory's place in the current landscape of theories investigating argument structure. I will conclude with some open questions for research, followed by a few final words.

	\subsection{Open questions}
The study of Hebrew does not conclude with this book, of course; even if everything I proposed were correct -- which it is surely not -- there would still remain many open questions for the study of this language and Semitic as a whole, not to mention argument structure in general. Here are the most pressing ones.

First, the data and generalizations. Even though I have tried to base my generalizations on as much data as possible, I was not always able to quantitatively examine the entire verbal lexicon of Hebrew. When this was possible, as with {\tnif} in Section~\ref{vz:tnif} or {\thif} in Section~\ref{vd:thif}, I have provided exact counts. But it is important to keep in mind that even these numbers cannot be considered authoritative, since the way I classify a given verb might be different than another analyst's. The trailblazing works of \cite{doron03} and \cite{arad05} make it clear just how much variation one can find between different roots; so when I suggest that verbs in {\thit} are unambiguous between readings whereas those in {\tnif} might be associated with different readings, the door is always open for counterexamples.

This is particularly the case given that a few such observations are not predicted by the current system. I have no explanation to offer for this difference between {\thit} and {\tnif}. Furthermore, {\tnif} verbs allow for \isi{passive} readings but those in {\thit} might not. While I can encode this difference in the rules of semantic interpretations, I have not offered an explicit analysis.

\label{r1:6:4}Zooming out even more, one major component of the system which has been relegated to arbitrary listing in the current theory is the role of roots. Formally, I have not provided an explicit theory of how a given root licenses a given functional head (or vice versa) beyond encoding co-occurrence in VI lists. And empirically, even though authors throughout the ages have emphasized that Semitic roots are simply idiosyncratic, a number of possible avenues for further exploration remain. Here and there I have noted places where the lexical semantics of the root seems to influence its place in a formal ontology. These include the pluractional events of {\tpie}, the divide between anticausatives and reflexives in {\thit}, the degree achievements and the forms of causation in {\thif}, the change-of-state adjectival passives of Section~\ref{passn:adjpass}, and quite clearly the passives in {\tnif}. It remains to be seen whether other cases can be attributed to the lexical semantics of the underlying event  \citep[114]{kastner16phd}.

Similarly, the question of root meaning in general is one we are only beginning to scratch the surface of. I will not belabor the point here, as it has recently come back to the fore; see \cite{harley14thlia} for a good starting point and \cite{kastnertucker19cup} for an outline of the current state of affairs.
Beyond formal work, computational modeling might provide new insights into how roots cluster in an abstract n-dimensional space.

Finally, looking beyond theoretical analyses, we would ideally like to know how well our formal accounts allow us to understand processing, neurophysiological behavior and acquisition. In all cases the level of formal analyses has been far more detailed than the level of granularity currently tracked in these allied fields. Nevertheless, \cite{kastneretal18} have attempted to generate predictions from this kind of decompositional approach in a neurolinguistic experiment on Hebrew, and the work of \cite{frostetal97}, \cite{fmdpmetal05jml}, \cite{deutschkuperman18} and colleagues can facilitate further work probing more specific theories of roots. Hebrew is no stranger to the developmental literature either (\citealt{berman82,berman93jcl,ashkenazietal16,ravidetaltilar,havronarnon17jcl}, a.m.o), but even there discussion based on contemporary theories has not relied on formal specifics \citep{borer04,kastneradriaans17}. Yet.

	\subsection{Epilogue: From templates to heads}
Morphological systems like the templatic system of Hebrew pose an immediate puzzle for formal analyses: the mapping between form and meaning is not one-to-one, with phonology, morphology, syntax and semantics intertwining in ways which seem to defy explanation. It has traditionally been assumed that each template is simply its own construction, and that relations hold -- sometimes -- between specific templates. This view, in which each template is in effect its own morpheme, is unsatisfactory, leaving more questions than answers. Why does a given template have the meanings it has, and not others? Why do two (or three, or four) templates stand in an ``alternation''? What does the morphology actually track?

And so a new wave of analyses emerged on the scene, which in my mind were no less than visionary. These accounts attempted to decompose the templates and see what made them tick. Maya Arad's theory of Hebrew was explicit about how this can be done, but ultimately unsatisfactory because it had to rely on extrinsically listing alternations between semantic atoms which had no grounding in the morphology. Edit Doron's theory was explicit about how the semantic primitives might combine but lacked clear syntactic underpinnings, and furthermore was not fully committed to mapping these elements to the morphology. So we needed a modern morphophonemics of Hebrew, if you will.

In this book, I have proposed a formal theory of argument structure which explains why each ``template'' is the way it is, and why templates stand in certain relationships to each other. The account ended up looking a lot like accounts of other phenomena and languages within the same framework. The fact that one can understand crosslinguistic variation in more concrete terms -- for example, the phonological constraints that linearize\is{linearization} the same structures might look different in Semitic than in non-Semitic languages -- is, to me, a clear sign of progress. And even though we still need to understand exactly how lexical information fits into the rigid syntax, the resulting picture is one which is fairly easy to navigate: the syntax generates, and the interfaces interpret.
