\label{chap:voice}
\section{Overview} \label{voice:intro}
This chapter introduces the first argument for a theory of Voice which makes room for an underspecified variant, one which neither requires nor prohibits a specifier. We will first consider morphological marking which is compatible with a variety of verbal forms, namely the template {\tkal}.

As we have already seen briefly in the previous chapter, Hebrew has dedicated active and non-active morphology (a characterization we will refine in the following chapters). Nevertheless, while verbs in {\tnif} are non-active and those in {\thif} are active, verbs in {\tkal} are underspecified with regards to their argument structure. With some roots, the verb might be transitive, (\ref{ex:voice-intro-tr}), with others, unergative, (\ref{ex:voice-intro-unerg}), with others, ditransitive, (\ref{ex:voice-intro-ditr}), and with others still, unaccusative, (\ref{ex:voice-intro-unacc}).
\pex\label{ex:voice-intro-tr}
	\a \begingl
		\gla teo \textbf{taraf} et ha-laxmanja//
		\glb Theo devoured \gsc{ACC} the-bread.roll//
		\glft `Theo devoured the bread roll.'//
	\endgl
	\a \begingl
		\gla ha-balʃan \textbf{katav} et ha-maamar ha-arox//
		\glb the-linguist wrote \gsc{ACC} the-article the-long//
		\glft `The linguist wrote the long article.'//
	\endgl
\xe

\pex\label{ex:voice-intro-unerg}
	\a \begingl
		\gla teo \textbf{rakad} ve-rakad ve-rakad (kol ha-boker)//
		\glb Theo danced and-danced and-danced all the-morning//
		\glft `Theo danced and danced and danced (all morning long).'//
	\endgl
	\a \begingl
		\gla teo \textbf{halax} kol ha-boker//
		\glb Theo walked all the-morning//
		\glft `Theo walked all morning long.'//
	\endgl
\xe

\pex\label{ex:voice-intro-ditr}
	\a \begingl
		\gla teo \textbf{natan} et *(le-marsel) ha-xatif//
		\glb Theo gave \gsc{ACC} to-Marcel the-snack//
		\glft `Theo gave Marcel the treat.'//
	\endgl
	\a \begingl
		\gla teo \textbf{ʃaal} et ha-sefer me-ha-sifria//
		\glb Theo borrowed \gsc{ACC} the-book from-the-library//
		\glft `Theo borrowed the book from the library.'//
	\endgl
\xe

\pex\label{ex:voice-intro-unacc}
	\a \begingl
		\gla \textbf{nafal} le-teo ha-bakbuk//
		\glb fell to-Theo the-bottle//
		\glft `Theo's bottle fell.'//
	\endgl
	\a \begingl
		\gla ha-bakbuk \textbf{kafa} ba-makpi//
		\glb the-bottle froze in.the-freezer//
		\glft `The bottle froze in the freezer.'//
	\endgl
\xe

I look into these patterns in more depth in Section~\ref{voice:voice}, where I situate them within the current theory of Voice. I then show how this head can be constrained by introducing an agentive modifier which I call {\va} in Section~\ref{voice:va}. Section~\ref{voice:conc} summarizes and outlines how the rest of the chapters in the first part of this book develop the theory.

\section{Voice} \label{voice:voice}
	\subsection{What is Voice?}
In the current neo-Davidsonian tradition, theories of argument structure have adopted a specific way of thinking about internal and external arguments in the syntax, based in large part on the interpretation asymmetries observed by \cite{marantz84} and discussed by \cite{kratzer96}. The theme or patient of the predicate is generated within the VP as the complement of V.\footnote{whether or not internal arguments end up in Spec,VP as in various approaches is immaterial here \citep{johnson91,alexiadouschaefer11wccfl}.} The agent is introduced in the specifier of a higher functional head, which takes the VP as its own complement. Since \cite{kratzer96} it has become common to call this head Voice and to associate it with accusative case licensing, thereby identifying it with causative ``little v'' of \cite{chomsky95}. The basics are given in~(\nextx), slightly modifying \citet[121]{kratzer96}. The relevant compositional functions invoked here are Functional Application and Event Identification. We will also make use of Predicate Modification later on in this book; see \cite{wood15springer} for an accessible introduction. I leave out the semantic types of the arguments.
\pex
	\a Mittie fed the dog.
	\a \Tree
	[.VoiceP\\{λe.Agent(Mittie, e) \& feed(the dog, e)}\\{\textsf{(by Functional Application})}
		[.DP\\\emph{Mittie} ]
		[.{λxλe.Agent(x,e) \& feed(the dog, e)}\\{\textsf{(by Event Identification)}}
			[.Voice\\{λxλe.Agent(x,e)} ]
			[.vP\\{λe.feed(the dog, e)}\\{\textsf{(by Functional Application)}}
				[.v\\{λxλe.feed(x,e)}
					[.\root{\gsc{FEED}} ]
					[.v ]
				]
				[.DP\\\emph{the dog} ]
			]
		]
	]
\xe

I would like to focus on two important points as a segue into the current theory. First, this original formulation does not make any claims regarding a structural difference between agents and causers (e.g.~circumstances, inanimate objects or natural forces). While there have been some attempts to draw a structural difference between the two---at least for certain psychological predicates \citep{bellettirizzi88,harleystone13}---I join the majority of work on argument structure in making no claims to that extent \citep{layering15}. For me, agents are a subset of causers, but this difference is semantic, not syntactic. What this means is that an external argument position (Spec,VoiceP) should be compatible with both agents and causers, but some additional element could force only a narrower, agentive reading. This is we will see already in Section~\ref{voice:va}. I will try to be clear about 

The second point is that as a functional head, Voice might be endowed with different features. In principle, since it licenses a DP in its specifier, it should have the EPP feature [D] \citep{chomsky95}. Once we accept that it has that feature, we can begin to ask what other features it can have, and might these features get checked in the course of the derivation. As recapped in the previous chapter, much recent work in argument structure has explored the possible values of the [$\pm$D] feature on Voice; recent approaches are discussed directly in Chapters \ref{chap:aas} and \ref{i:nie}, once the current theory has been developed in depth. To begin, I will now explain what it means for Voice to be underspecified for this feature.

	\subsection{Underspecified Voice}
This monograph promotes a theory of argument structure in which Voice can have one of three values: [+D], [--D] or underspecified for [$\pm$D]. As foreshadowed in the introductory chapter, the idea is that {\vd} requires an external argument and {\vz} prohibits one. We will focus on what it means for Voice not to have a preference on the matter, thereby accounting for the patterns in (\ref{ex:voice-intro-tr})--(\ref{ex:voice-intro-unacc}).

First, let me define underspecified Voice in (\nextx). All definitions of Voice heads in this chapter and the following ones will take the same form: (a) syntactic definition, (b) semantic denotation, and (c) informal spell-out rule. I give these here and expand upon them in turn.
\pex \textbf{Voice}
	\a A Voice head no specified for a [D] feature. It has no requirements regarding whether its specifier must be filled. In transitive verbs, Voice is the locus of accusative case assignment, either itself by feature checking \citep{chomsky95} or through the calculation of dependent case \citep{marantz91}.
	\a \denote{Voice} = $\begin{cases}
		\lambda e.e & \text{/ \trace~ \{ \root{npl} `\root{\gsc{FALL}}', \root{kpa} `\root{\gsc{FREEZE}}' , \dots \} }\\
		\lambda x \lambda e.Agent(x,e) & \\
		\end{cases}$

%	\denote{Voice} = $\begin{cases}
%		\lambda x \lambda e.Agent(x,e) & \text{/ \trace~\{ \root{axl} `\root{\gsc{EAT}}', \root{ktb} `\root{\gsc{WRITE}}', \root{ntn} `\root{\gsc{GIVE}}',}\\
%			& \text{\root{ʃal} `\root{\gsc{BORROW}}', \root{\gsc{r\dgs{k}d}} `\root{\gsc{dance}}', \root{\gsc{hlx}} `\root{\gsc{WALK}}', \dots\} }\\
%		\lambda e.e & \text{/ \trace~ \{ \root{npl} `\root{\gsc{FALL}}', \root{kpa} `\root{\gsc{FREEZE}}' , \dots \} }
%		\end{cases}$
	\a Voice \lra~{\tkal} \hfill  (with the allomorph {\tpie} to follow in Section \ref{voice:va})
\xe

		\subsubsection{Syntax}
In the current system, the lack of a feature on Voice means that the head is not specified for any syntactic feature constraining Spec,VoiceP. That position can be filled or left unprojected, as far as the Voice head is concerned. In this state of affairs, the expectation is that differences between verbs will result from the requirements of individual roots, rather than anything in the structure. In other words, some roots will give rise to transitive verbs, other roots to unaccusative verbs, and so on.

This is exactly what we find in the template {\tkal}, which spells out Voice. There are no structural restrictions on argument structure in this template: verbs in {\tkal} might be transitive, unergative, ditransitive or unaccusative. Some of the examples in (\ref{ex:voice-intro-tr})--(\ref{ex:voice-intro-unacc} are repeated below with the relevant diagnostics.

In~(\nextx) we see the core transitive verb \emph{taraf}, which requires an internal argument. The accusative/DOM marker \emph{et} must also appear, indicating that this is a transitive construction.\footnote{There is a substantial literature on \emph{et} and what it tracks. I assume for present purposes, rather uncontroversially, that it is only compatible with accusative objects.}
\pex\label{ex:voice-intro-tr2}
	\a 
	\begingl
		\gla teo \textbf{taraf} *(et ha-laxmanja)//
		\glb Theo devoured \gsc{ACC} the-bread.roll//
		\glft `Theo devoured the bread roll.'//
	\endgl
	\a \Tree
	[.VoiceP
		[.\emph{teo} ]
		[.
			[.Voice ]
			[.vP
				[.v
					[.\root{trf} ]
					[.v ]
				]
				[.DP\\\emph{et ha-laxmanja} ]
			]
		]
	]
\xe

Unergative verbs are also possible, as with \emph{rakad} `danced' in~(\nextx). No internal argument is necessary, the event is an activity which can go on over a certain period of time with no concrete telos, and agent-oriented adverbs are possible.
\pex\label{ex:voice-intro-unerg2}
	\a \begingl
		\gla teo \textbf{rakad} ve-rakad ve-rakad (be-mejomanut) (kol ha-boker)//
		\glb Theo danced and-danced and-danced in-skill all the-morning//
		\glft `Theo danced and danced and danced (skillfully) (all morning long).'//
	\endgl
	\a \Tree
	[.VoiceP
		[.\emph{teo} ]
		[.
			[.Voice ]
			[.vP
				[.v
					[.\root{rkd} ]
					[.v ]
				]
			]
		]
	]
\xe

Ditransitive verbs are also possible, as in~(\nextx). We do not need to commit to any specific analysis of ditransitive verbs, so I give a general structure headed by Appl or \emph{p}, a PP-licenser \citep{koopman97,svenonius03,gehrke08phd,wood15springer} I return to in Chapter~\ref{vz:figrefl}.
\pex\label{ex:voice-intro-ditr2}
	\a \begingl
		\gla teo \textbf{natan} *(le-marsel) et ha-xatif //
		\glb Theo gave \gsc{ACC} to-Marcel the-snack//
		\glft `Theo gave Marcel the treat.'//
	\endgl
	\a \Tree
	[.VoiceP
		[.\emph{teo} ]
		[.
			[.Voice ]
			[.ApplP/\emph{p}P
				[.PP\\\emph{le-marsel} ]
				[.
					[.Appl/\emph{p} ]
					[.vP
						[.v
							[.\root{ntn} ]
							[.v ]
						]
						[.DP\\\emph{et ha-xatif} ]
					]
				]
			]
		]
	]
\xe

Lastly, unaccusative verbs are also possible. The two traditional diagnostics are fronting of the verb and the possibility of using a possessive dative, both evident in~(\nextx). I return to discussing these diagnostics in some more depth when we focus on unaccusative verbs in Chapter~\ref{vz:nact}. The tree in~(\anextx b) does not present the final word order, on which see \cite{preminger10}.
\pex\label{ex:voice-intro-unacc2}
	\a \begingl
		\gla \textbf{nafal} le-teo ha-bakbuk//
		\glb fell to-Theo the-bottle//
		\glft `Theo's bottle fell.'//
	\endgl
	\a \Tree
	[.VoiceP
		[.Voice ]
		[.ApplP/\emph{p}P
			[.PP\\\emph{le-teo} ]
			[.
				[.Appl/\emph{p} ]	
				[.vP
					[.v
						[.\root{nfl} ]
						[.v ]
					]
					[.DP\\\emph{he-bakbuk} ]
				]
			]
		]
	]
\xe		

The current analysis captures the underspecified nature of {\tkal} straightforwardly. Since there are no restrictions in the syntax, the root is free to require any interpretation from v and Voice (save for reflexive readings, which are discussed in Chapter~\ref{vz:va:refl}).

%% check database for breakdown by type
		
		\subsubsection{Semantics}
The underspecification of this head---and of the resulting template---can be implemented in the semantics using contextual allosemy of Voice, allowing different meanings to arise in different contexts.\footnote{Chapter~\ref{chap:aas} contains a brief comparison of contextual allosemy with one alternative, namely postulating homophonous heads. There is little to choose between the two options.} Assuming that the causative variant is the elsewhere case, certain roots will be said to require a non-active alloseme of Voice, (\nextx):
\pex \denote{Voice} = 
	\a $\lambda$e.e \phantom{agent(x,e)xxx} / \trace~ \{ \root{npl} `\root{\gsc{FALL}}', \root{kpa} `\root{\gsc{FREEZE}}' , \dots \}
	\a $\lambda$x$\lambda$e.Agent(x,e)
%		 & \text{/ \trace~\{ \root{trf} `\root{\gsc{DEVOUR}}', \root{ktb} `\root{\gsc{WRITE}}', \root{ntn} `\root{\gsc{GIVE}}',}\\
%			& \text{\root{ʃal} `\root{\gsc{BORROW}}', \root{\gsc{r\dgs{k}d}} `\root{\gsc{dance}}', \root{\gsc{hlx}} `\root{\gsc{WALK}}', \dots\} }\\
\xe
		
		
		\subsubsection{Phonology}
While this template is underspecified in the syntax and semantics, the lack of overt heads constraining the structure means that it can be \emph{more} marked in the phonology. The intuition is that if there are no overt affixes, the root will have free reign in the phonology. For example, verbal stems are normally longer than one syllable except for some roots in {\tkal}:
\ex \emph{ba} `came', \emph{ʃav} `returned', \emph{{\texttslig}a{\texttslig}} `appeared'.
\xe

The phonological markedness of this template has been discussed in contemporary work by \cite{ussishkin05} in terms of Emergence of the Unmarked, with similar observations made by \cite{laks11} and \cite{borer13oup,borer15roots}. I will not have much to add on this point here. In Section~\ref{voice:va} below I introduce a modifier which constrains both the semantics and phonology of Voice. The relevant Vocabulary Items can be schematized in~(\nextx).
\pex Voice {\lra}
	\a {\tpie} / {\trace} {\va}
	\a {\tkal}
\xe

The spell-out rules in~(\lastx) provide a crude approximation of how Voice is handled at PF, but it is important to keep in mind that there is no one ``suffix'' {\tkal} or {\tpie} which spells out this head. Rather, there is an intricate morphophonological system of inflectional variants which needs to be taken into account. The full paradigm of {\tpie}, for example, is given by \cite{kastner15nels} as~(\nextx).

\ex
\raisebox{-4.5em}{
\begin{small}
	\begin{tabular}{|l||l|l||l|l||l|l|} \hline
		& \multicolumn{2}{c||}{Past} & \multicolumn{2}{c||}{Present} &  \multicolumn{2}{c|}{Future} \\\hline
		& \gsc{M} & \gsc{F} & \gsc{M} & \gsc{F} & \gsc{M} & \gsc{F} \\\hline\hline
		1\gsc{SG} & \multicolumn{2}{c||}{Xi\dgs{Y}aZ-ti} & me-Xa\dgs{Y}eZ & me-Xa\dgs{Y}eZ-et & \multicolumn{2}{c|}{a-Xa\dgs{Y}eZ/ye-Xa\dgs{Y}eZ}\\\hline
		1\gsc{PL} & \multicolumn{2}{c||}{Xi\dgs{Y}aZ-nu} & me-Xa\dgs{Y}Z-im & me-Xa\dgs{Y}Z-ot & \multicolumn{2}{c|}{ne-Xa\dgs{Y}eZ}  \\\hline\hline
		2\gsc{SG} & Xi\dgs{Y}aZ-ta & Xi\dgs{Y}aZ-t & me-Xa\dgs{Y}eZ & me-Xa\dgs{Y}eZ-et & te-Xa\dgs{Y}eZ & te-Xa\dgs{Y}Z-i\\\hline
		2\gsc{PL} & Xi\dgs{Y}aZ-tem & Xi\dgs{Y}aZ-ten/m & me-Xa\dgs{Y}Z-im & me-Xa\dgs{Y}Z-ot & \multicolumn{2}{c|}{te-Xa\dgs{Y}Z-u}\\\hline\hline
		3\gsc{SG} & Xi\dgs{Y}eZ & Xi\dgs{Y}Z-a & me-Xa\dgs{Y}eZ & me-Xa\dgs{Y}eZ-et & ye-Xa\dgs{Y}eZ & te-Xa\dgs{Y}eZ\\\hline
		3\gsc{PL} & \multicolumn{2}{c||}{Xi\dgs{Y}Z-u} & me-Xa\dgs{Y}Z-im & me-Xa\dgs{Y}Z-ot & \multicolumn{2}{c|}{ye-Xa\dgs{Y}Z-u}\\\hline
	\end{tabular}
\end{small}
}
\xe

A more comprehensive elucidation of this part of the theory can be found in \cite{faust12} and \cite{kastner18nllt}.

	\subsection{Interim summary}
In contrast to the traditional Voice head which introduces an external argument, the current Voice head is underspecified with regards to the EPP feature [D] and does not place any constraints on its specifier. As a result, there are no restrictions on the argument structure of verbs which have only this head. Since every Hebrew verb must be instantiated in one of the seven verbal templates, the appearance of Voice can be traced in the morphology as the template {\tkal}.

In other frameworks, \cite{doron03} does not introduce any special heads in order to account for verbs in {\tkal}. \cite{borer13oup,borer15roots} takes {\tkal} to be a verbalized root, without functional material attaching to it. The two main reasons for this are the wide range of nominalizations possible in this template and the idiosyncratic phonology. I will return to nominalizations in Chapter~\ref{passn:n}, after covering the other variants of Voice, but all three frameworks are compatible in their treatment of the {\tkal}: all allow for {\tkal} to be as idiosyncratic as it needs to be, in the phonology as in the syntax.


\section{Agentive modification} \label{voice:va}
In this section I introduce another syntactic primitive, the agentive modifier {\va}. Strictly speaking, this modifier is not part of the theory of trivalent Voice. As I explain immediately below, we are formalizing an element similar to those which have either been assumed implicitly in other work or been given other names. The reason it is introduced so early on in this book is because it is necessary to capture the full empirical picture. Throughout this book and especially in the first part, we begin each chapter by explaining the theoretical proposal, verifying its immediate predictions, and then expanding the picture by looking at related phenomena in the language. Underspecified Voice and the template {\tkal} have already been addressed, but the behavior of the template {\tpie} indicates that the theory needs to be augmented as I do next. Section~\ref{voice:va:act} sketches the general crosslinguistics phenomenon, Section~\ref{voice:va:heb} formally characterizes the element {\va} in Hebrew, and Section~\ref{voice:va:plural} suggests an additional way of looking at the phenomenon in terms of pluractionality.

	\subsection{Agentive modifiers crosslinguistically} \label{voice:va:act}
%As far as the semantics is concerned, the difference in possible interpretations between~(\lastx a) and~(\lastx b) reduces to whether or not overt {\va} is there to force an agentive reading. \cite{doron03} proposed that this modifier carries the semantics of Action, which is slightly weaker than that of Agent.
I believe that {\va} enforces a reading that has appeared in a number of recent works on argument structure. In their study of animacy in English, Italian, Greek and Russian, \cite{folliharley08} considered a range of data in which the acceptability of an external argument depends on whether it is \emph{teleologically capable} of causing the event (as opposed to an agency or animacy restriction). In a study of manner and causation in English, \cite{beaverskoontzgarboden12} used the notion of \emph{actor} and \emph{non-actor} to discuss events in which an animate causer is or is not responsible for the consequences of its act, distinguishing causation from actorhood. In two studies of external arguments in nominalizations, \cite{sichel10n} and \cite{alexiadouetal13} similarly differentiated agentivity from \emph{direct causation}. In a study of reflexives in Greek (which we return to in Section \ref{sec:others-theory:afto}), \cite{spathasetal15} identified the prefix \emph{afto-} as an \emph{anti-assistive} modifier, again performing a similar semantic function. And in Tamil, the suffix -\emph{koɭ} adds affective semantics in a way that is otherwise difficult to pin down immediately \citep{sundaresanmcfadden17}.

To be clear, the crosslinguistic claim is not that {\va} is the sole element responsible for all of these cases. Instead, the pretheoretical picture which emerges from these works is that natural language has a way of making this fine-grained distinction, a distinction we are not yet fully able to explain. Since this phenomenon appears to be semantic in nature (rather than demonstrably syntactic), it is formalized in Hebrew using {\va}. As a root, this element has phonological and semantic content but no syntactic requirements. Not much hinges on whether this element is a root or a functional head in this language; I take the simple view that it has no syntactic influence, and so is root-like. The question of what other such ``underspecified'' roots might exist in natural language remains an open one for further crosslinguistic research.


	\subsection{\va} \label{voice:va:heb}

For this part of the puzzle I will modify the suggestion made by \cite{doron03} according to which Hebrew has an agentive modifier with predictable spell-out and consistent semantics. The current section formalizes this element as {\va} and draws comparisons with other languages.

In Hebrew the modifier {\va} is attested in {\thit} and in the ``intensive'' template {\tpie}. Consider first the typical difference between verbs in {\tkal} (with unmarked Voice) and {\tpie} (with Voice and \va~\!). In~(\ref{ex:piel1}) both agents and causes are possible with the ``simple'' {\tkal} verb \emph{ʃavru} `broke', but in~(\ref{ex:piel2}) only the agent is available with the ``intensive'' {\tpie} verb \emph{ʃibru} `broke to bits'.
\pex \citet[20]{doron03}
	\a \label{ex:piel1}\begingl
	\gla \{\cmark~ha-jeladim / \cmark~ha-tiltulim ba-argaz \} ʃa\glemph{v}r-u et + \phantom{\{\cmark~}ha-kosot.//
	\glb \phantom{\{\cmark~}the-children {} \phantom{\cmark~}the-shaking in.the-box {} broke.\gsc{SMPL}-\gsc{PL} \gsc{ACC} \phantom{\{\cmark~}the-glasses//
	\glft `\{The children / Shaking around in the box\} broke the glasses.'//
	\endgl

	\a \label{ex:piel2}\begingl
	\gla \{\cmark~ha-jeladim / \xmark~ha-tiltulim ba-argaz \} ʃi\glemph{b}r-u et + \phantom{\{\cmark~}ha-kosot.//
	\glb \phantom{\{\cmark~}the-children {} \phantom{\xmark~}the-shaking in.the-box {} broke.\gsc{INTNS}-\gsc{PL} \gsc{ACC} \phantom{\{\cmark~}the-glasses//
	\glft `\{The children / *Shaking around in the box\} broke the glasses to bits.'//
	\endgl
\xe

This element is phonologically overt. I follow \cite{doron03} and \cite{kastner16nllt} in assuming that {\tkal} is derived morphophonologically through the combination of Voice, v and the root, whereas {\tpie} is the result of adding {\va}. The Hebrew consonants /p/, /b/ and /k/ normally spirantize to [f], [v] and [x] following a vowel, but not when {\va} is in the structure. This element has the phonological property that spirantization of the middle root consonant is blocked. In~(\lastx a), the medial /b/ of \root{ʃbr} spirantizes to [v] following a vowel. But in~(\lastx) it remains [b], as discussed elsewhere in the phonological literature \citep{temkinmartinez08wccfl,gouskova12nllt,kastner16nllt}. 

An anonymous reviewer asserts that this morphophonological process has no bearing on the internal structure of these verbs. Two considerations lead me to disagree: whether the spirantization process is productive, and whether the blocking is grammatical. On both counts, the answer is affirmative. \cite{temkinmartinzemuellner16} conducted a nonce word study for Hebrew and found that native speakers normally spirantize the three stops, but do not spirantize them in medial position of {\tpie}, as would be expected. The results were not categorical, however, in line with previous work; \cite{adam02} previously identified patterns of variation in the application of the phonological rule and the morphophonological one. Clearly, then, there is a variable phonological process which is blocked by grammatical means, indicating that these grammatical means should part of the the analysis.




I proposed above that the template {\tpie} is derived by use of a special root, \va. There are three points to be made about this element: what does it do, why should it be a root, and what is its crosslinguistic validity.

As mentioned in \S\ref{syn:middle:refl}, \va~introduces agentive or ``self-propelled'' \citep{folliharley08} semantics. Examples~(\ref{ex:piel1})--(\ref{ex:piel2}) are repeated here to show that \tpie~verbs involve agentive entailments. Inanimate causers are possible in~(\nextx a) where the verb is in \tkal~but not in~(\nextx b) where the verb is in \tpie~with \va. That these are the two templates is evidenced by the vocalism in the stem and by the spirantization contrast between [b]$\sim$[v].
\pex
	\a \begingl
		\gla \emph{\{}\cmark~ha-jeladim / \cmark~ha-tiltulim ba-argaz\emph{\}} \textbf{ʃa\underline{v}r}-u et ha-kosot//
		\glb \phantom{\{\cmark~}the-children {} \phantom{\cmark~}the-shaking in.the-box \textbf{broke.\gsc{SMPL}}-\gsc{PL} \gsc{ACC} the-glasses//
		\glft `\{The children / Shaking around in the box\} broke the glasses.'//
		\endgl
	
	\a \begingl
		\gla \emph{\{}\cmark~ha-jeladim / \xmark~ha-tiltulim ba-argaz\emph{\}} \textbf{ʃi\underline{b}r}-u et ha-kosot//
		\glb \phantom{\{\cmark~}the-children {} \phantom{\xmark~}the-shaking in.the-box \textbf{broke.\gsc{INTNS}}-\gsc{PL} \gsc{ACC} the-glasses//
		\glft `\{The children / *Shaking around in the box\} broke the glasses to bits.' \trailingcitation{\citep[20]{doron03}}//
		\endgl
\xe

The \tpie~template, made up of a lexical root, \va~and Voice, is traditionally called the ``intensive'', but it can also house pluractional verbs (c--e) and various others (f--g):
\ex\label{table:piel-meanings}Pretheoretical classification of some verbs in \tpie:\\
	\begin{tabular}{lll|ll|ll}
	& & & \multicolumn{2}{c|}{\tkal} &  \multicolumn{2}{c}{\tpie}\\\hline
	\multirow{2}{*}{Intensive} & a.& \root{ʃbr} & ʃavar & `broke' & ʃiber & `broke to pieces'\\
	    & b.& \root{'kl} & axal & `ate' & ikel & `corroded, consumed'\\\hline

 	\multirow{3}{*}{Pluractional} & c.& \root{hlx} & halax & `walked' & hilex & `walked around'\\
 	    & d.& \root{r\dgs{k}d} & rakad & `danced' & riked & `danced around'\\
  	    & e.& \root{\dgs{k}fts} & kafats & `jumped' & kipets/kiftsets & `jumped around'\\\hline
  		
  		\multirow{2}{*}{Non-derived} & f. & \root{tps} & \multicolumn{2}{c|}{---} & tipes & `climbed'\\
	    & g. & \root{ltf} & \multicolumn{2}{c|}{---} & litef & `petted'\\
	\end{tabular}
\xe
There is no obvious interpretation of this element beyond agentive semantics; verbs in \tpie~are not simply verbs in \tkal~with added agentivity entailments. In fact, ``agentive semantics'' is too gross a generalization itself; \cite{doron03} describes the relevant thematic role as an Actor, though \cite{doron14adj} returns to Agent. For general discussion of related notions of a direct causing participant crosslinguistically see \cite{folliharley08} mentioned above, as well as \cite{sichel10n}, \cite{beaverskoontzgarboden12} and \cite{alexiadouetal13}.


Setting the pluractionality hypothesis aside and keeping things uniform for the time being, I assume that \va~triggers an agentive alloseme of Voice, as in \S\ref{syn:middle:refl} and following \cite{doron03,doron14adj}.
\ex \denote{Voice} = $\lambda$e$\lambda$x.e \& \text{Agent}(x,e) / \trace~\va
\xe
It has been pointed out to me by Hagit Borer that a number of verbs in \tpie~would stretch the notion of Agent/Actor to the point where such a denotation is no longer tenable. In the examples in~(\nextx), the verb can hardly be described as agentive since the subject is inanimate, while in~(\anextx) the subject is animate but non-volitional. These verbs are compatible with agentive subjects as well, but clearly do not require them.
\pex
  \a \begingl
    \gla \textbf{ha-midgam} \underline{ʃikef} et totsot ha-emet//
    \glb the-poll reflected.\gsc{INTNS} \gsc{ACC} results.\gsc{CS} the-truth//
    \glft `The polls (correctly) reflected the results.'//
  \endgl
    
  \a \begingl
    \gla be-ritsa axat \textbf{ha-ʃaon} ʃel garmin kimat \underline{diek} kaaʃer hetsig stia kimat xasrat maʃmaut ʃel axuz ve-ktsat//
    \glb in-run one the-watch of Garmin almost was.accurate.\gsc{INTNS} when showed deviation almost devoid.of meaning of percent and-little//
    \glft `In one run, the Garmin watch was precise as it showed an almost insignificant deviation of just over one percent.'\trailingcitation{\url{www.haaretz.co.il/sport/active/.premium-1.2309128}}//
  \endgl
  \a \begingl
    \gla \textbf{ha-xom} \underline{ʃibeʃ} l-i et ha-medidot//
    \glb the-heat disrupted.\gsc{INTNS} to-me \gsc{ACC} the-measurements//
    \glft `The heat messed up my measurements.//
  \endgl
\xe

\ex \begingl
  \gla \textbf{hu} \underline{kibel} maka xazaka ba-regel//
  \glb he received.\gsc{INTNS} hit strong in.the-leg//
  \glft `He got hit hard in the leg.'//
  \endgl
\xe

In these examples an external argument is still required, regardless of whether it can felicitously be called an Agent or not. What these examples show is that a rigid denotation of \va~is difficult to specify, beyond some general notion of a direct cause. I believe it is significant, though, that the verbs in~(\blastx)--(\lastx) do not have correspondents in \tkal. That is, they are not derived by adding \va~to an existing verb or via some process of causativization: \emph{ʃikef} $\nless$ *\emph{ʃakaf}, \emph{diek} $\nless$ *\emph{dajak}, \emph{ʃibeʃ} $\nless$ *\emph{ʃabaʃ}, and \emph{kibel} $\nless$ *\emph{kabal}. They are derived when \va~selects the alloseme of the root directly without having to agentivize a verb in \tkal.

Let us summarize the semantic contributions of \va:
\pex
  \a When attaching to an already-existing verb in \tkal, \va~adds Agent/Actor semantics.
  \a When attaching to an already-existing verb in \tkal, \va~might change the meaning as in~(\ref{table:piel-meanings}).
  \a When attaching to the root without a verb already existing in \tkal, agentive semantics need not be entailed (but an external argument is still obligatory).
\xe

Why, then, should this element be a root? The same job could in principle be accomplished using a  ``flavor'' of little v{, but I see no empirical need for ``flavors'' of v in my theory. }{Some other} functional head could be invoked, but it would have to be independently motivated. {I contend instead that} this morpheme behaves more like a root than a functional head: it does not {do syntactic work or }introduce an argument as syntactic heads do, and it has a range of possible meanings and phonological information that can be associated with it, more in line with a lexical element than a functional one. If \va~is a root rather than a functional head, its partially unpredictable contributions to the meaning of the verb are to be expected.

Even if it were a functional head, the equivalent question would be asked: what kind of functional head is it? As things stand, there is little to choose from in terms of empirical support for one view over the other.

In the phonology, \va~blocks spirantization of the middle root consonant and triggers the insertion of vowels specific to \tpie/\thit. Neither of these behaviors is problematic: \va~is local to both root and Voice, so it can derive unpredictable semantics from the former and condition allomorphy on both. For a more explicit discussion of the morphophonology see \S\ref{sec:t-phi}.



	\subsection{Pluractionality} \label{voice:va:plural}
One possible way to describe the semantics of \va~is by extended reference to pluractionality. The intuition as is follows. Assume that \va~is a pluractional (and perhaps also agentive) affix. Building on recent work by \cite{henderson12phd,henderson16nllt}---whom I thank for discussing this data with me---pluractionality can be seen as a way of pluralizing an event. This pluralization can hold spatially as well as temporally. In the data in~(\lastx a--b), the underlying verb in \tkal~has a direct object. The corresponding pluralized events in \tpie~can be individuated with respect to the direct objects: many broken pieces in (\ref{table:piel-meanings}a), many different simultaneous corrosions of the material in (\ref{table:piel-meanings}b).

For the forms in (\ref{table:piel-meanings}c)--(\ref{table:piel-meanings}e), observe that the underlying verbs in \tkal~are unergative. The pluralizing operation has no direct object to operate on, and so it pluralizes the spatio-temporal event itself in \tpie.

Lastly, in (\ref{table:piel-meanings}f)--(\ref{table:piel-meanings}g) there is no underlying form and hence nothing to pluralize.

The database of verbs from \cite{ehrenfeld12} contains over 900 forms in \tpie, so this line of inquiry faces a serious amount of empirical corroboration. A number of potential counterexamples can be conjured up fairly easily, though. These are cases where the alternation does not plausibly result in a plural event:
\pex
	\a \emph{lamad} `learned' $\sim$ \emph{limed} `taught'
	\a \emph{ratsa} `wanted' $\sim$ \emph{ritsa} `satisfied'
\xe
In the examples in~(\lastx) the event does not entail change of state, unlike with breaking and eating/corroding. So perhaps there is a tripartite division of roots to be made, as follows:
\pex
	\a \textbf{Other-oriented (change of state):} pluralization of the object.
	\a \textbf{Self-oriented:} pluralization of the spatio-temporal aspects of the event.
	\a \textbf{Other cases:} no pluralization.
\xe
Proper evaluation of this novel proposal will proceed along the lines laid out above, testing whether each root instantiated in this template does indeed fit into one of the three cases in~(\lastx).

@@\cite{greenberg10}@@




\section{Conclusion and outlook} \label{voice:conc}

%\section{Psych-verbs}