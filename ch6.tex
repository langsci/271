\section{Introduction} \label{sec:intro}
The first part of this book developed a theory of Voice which recognizes three possible values: {\vd}, {\vz} and underspecified Voice. We have seen that the syntactic features of Voice are only loosely correlated with the semantic interpretation in ways which themselves are informative {\vd} introduces a thematic external argument, although this requirement can be voided in certain contexts, indicating that the semantics is still sensitive to syntactic structure (Chapter @); {\vz} does not introduce a thematic external argument, but it can trigger existential closure, and {\pz} does introduce an external argument, indicating that the thematic interpretation is sensitive to the extended verbal projection (Chapter @); and Voice places no requirements either in the syntax or in the semantics, although the two are once again correlated (Chapter @).

In this chapter I would like to highlight some differences between this theory and the one most closely related to it, which I will call for simplicity the Layering approach \citep{schaefer08,layering15}. I discuss the two basic premises behind Layering in Section @@ and show how they are manifested in the current theory and how they are different in Section @@. The differences will pattern as follows: everything that can be expressed using Layering can be expressed in the current approach, but causative alternations are beyond the purview of Layering and require a trivalent system. This comparison sets up expectations for additional crosslinguistic study, to which I turn in Section @ and in the next chapter.


\section{Layering}
Many recent syntax-based theories of argument structure adopt two core assumptions which have been most notably defended in the work of \cite{schaefer08} and colleagues \citep{alexiadouetal06,layering15}. These are the shared base for causative and anticausative alternations, and the dissociation of syntactic and semantic transitivity.

	\subsection{Causative base}
A recurring question in discussions of argument structure regards the direction of derivation in (anti-)causative alternations. For an alternation like~(\nextx), is the transitive version derived from the intransitive one via causativization or is the intransitive variant derived from the transitive one via anticausativization?
\pex
	\a John broke the vase.
	\a The vase broke.
\xe

\cite{layering15} summarize a number of reasons for thinking that neither answer is strictly speaking true. They propose that both variants have the same base: a minimal vP containing the verb (a verbalized root) and the internal argument. The difference between the two variants is that the transitive one then has the external argument added by additional functional material (Voice).
\ex
a. \Tree
[.VoiceP
	[.\emph{John} ]
	[.
		[.Voice ]
		[.vP
			[.\emph{broke} ]
			[.\emph{the glass} ]
		]
	]
]
b. 
\Tree
		[.vP
			[.\emph{broke} ]
			[.\emph{the glass} ]
		]
\xe

This view explains a range of facts about the alternations, chiefly that there is no dedicated direction of derivation which is marked by the morphology across languages. That is, while some languages mark the transitive variants, others mark the intransitive variants, and sometimes both variants are marked in the same language (as we have already seen for Hebrew). Even though there is much to say about which verbs or roots are marked in which way \citep{haspelmath93,unaccusativity95,arad05}, the grammar itself does not force derivation from one stem type to the other.

In addition to the morphological reasoning, \cite{layering15} provide a series of arguments showing that the core causative component of the vP is present even in the anticausative variants. For example, @@

In sum, while there is a causative base, an actual causer can only be introduced by additional structure; in an additional layer, so to speak. Voice is the functional head enabling this layer, both in terms of licensing Spec,VoiceP in the syntax and in opening the thematic predicate agent (abstracting away here from the distinction between agents and causers, on which see Chapter @@ and \citet[7]{layering15}. This component suffices to explain the English facts.

	\subsection{The transitivity of Voice}
The second component of Layering is informed by alternations in additional languages. The existence of marked anticausatives raises the question of what their morphology of tracking. \cite{layering15} propose a system in which Voice can be ``transitive'' both in syntactic and semantic terms. In the syntax, Voice might be either associated with a specifier or not; in the semantics, it might introduce a thematic agent role or not. This conceptual innovation is implemented by using a syntactic feature [D], an EPP feature on Voice.

Precise structures are given in the next section; at this point the important consequence is that there are four possible configurations, depending on whether Voice is syntactically active and whether it is semantically thematic. With this basic assumption, Layering is able to handle cases in which syntactic elements like expletives do not involve thematic roles as well as cases in which the anticausative variant is marked morphologically, as we will see below. Taken together, these two components combine to provide substantial empirical coverage and theoretical insight.


	\subsection{Implementation}
Let us get to the details. Assuming that Voice may or may not have a [D] feature, and that it may or may not introduce an agent, the four-celled typology of \citet[109]{layering15} emerges:
\ex\label{ex:typo-layer}
\begin{tabular}{c|ll|ll}
	& \multicolumn{2}{c|}{Syntax D}	& 	\multicolumn{2}{c}{Syntax {\zero}} \\\hline
%&&&&\\
Semantics & 	a.&	Thematic active 	&	b.&	Thematic non-active\\
$\lambda$x 	 & &
\Tree
[.VoiceP 
	[.DP ]
	[.VoiceP
		[.{Voice\{$\lambda$x, D\}} ]
		[.{\dots~vP \dots} ]
	]
]
& &
\Tree
[.VoiceP 
		[.{Voice\{$\lambda$x, \zero\}\\\fbox{\gsc{NACT}}} ]
		[.{\dots~vP \dots} ]
]
\\
&&&&\\
& & \ding{228} Transitive verb.	& & \ding{228} Passives in Greek.\\
&&&&\\\hline
Semantics & 	c.&	Expletive active 	&	d.&	Expletive non-active\\
{\zero}	 & &
\Tree
[.VoiceP 
	[.DP\\\fbox{\gsc{SE}} ]
	[.VoiceP
		[.{Voice\{\zero, D\}} ]
		[.{\dots~vP \dots} ]
	]
]
& &
\Tree
[.VoiceP 
		[.{Voice\{\zero, \zero\}\\\fbox{\gsc{NACT}}} ]
		[.{\dots~vP \dots} ]
]
\\
&&&&\\
& & \ding{228} Marked anticausatives (German).	& & \ding{228} Anticausatives in Greek.\\
\end{tabular}
\xe

I examine the cells one by one. The structure in~(\lastx a) is a straightforward transitive derivation, at least since \cite{kratzer96} and \citep{pylkkanen08}. The [D] feature on Voice licenses a DP in its specifier, and the agent role is introduced in the semantics (notated here simply as $\lambda$x).

The structure in~(\lastx d) derives a marked anticausative, similar to {\tnif} from Chapter @. There is no [D] feature, so no DP can be merged in Spec,Voice. The lack of a specifier is spelled out as non-active morphology, \gsc{NACT} in short, via a rule we return to in~(@@). In the semantics, no external argument is introduced. The result is an anticausative (unaccusative) construction.

The particularly interesting cases are those in which we find ``mismatches'' between the values of the syntactic feature and semantic specification. Starting with~(\lastx c), we have a situation in which no agentive semantics are introduced but Voice still requires a specifier. The Layering analysis proposes that this is the situation for the Romance expletive \gsc{SE} and the German \emph{sich}, which appear in marked anticausatives but contribute nothing to the semantics. Similar analyses have been proposed for Icelandic by \cite{wood14nllt,wood15springer} and for various phenomena in English@ and Quechua@ by \cite{myler16mit}.

Finally, the configuration in~(\lastx b) is also possible. Here, Voice does not have a [D] feature and does not project a specifier. However, it does introduce a thematic role. \cite{layering15} propose that this is the correct analysis of passive verbs in Greek, which are identical morphologically to anticausatives; the analysis captures the fact that the morphology of~(\lastx b) and~(\lastx d) is identical, since in neither case is a specifier projected. The open predicate must presumably be closed off by existential closure later in the derivation.


\section{Comparison}
The current theory differs from Layering in two concrete ways. First, Layering assumes that no Voice layer is projected for non-active constructions, although the Voice head (with certain features) does head a VoiceP in these structures. I assume that a VoiceP is always projected, except that its specifier might not be filled. We have seen this in the difference between active and non-active verbs in {\tkal} (Chapter @), and in the difference between {\vz} and {\vd} (Chapters @@). This difference is, to some extent, a matter of theoretical taste. The second feature is more substantial, although it does build on the first: there are three possible values of the [D] features, associated more closely with semantic interpretation.

	\subsection{Non-active layers}
Marked anticausatives show consistent morphological marking on the anticausative member of an alternation. The example in~(\nextx) is from Latin:
\ex @@
\xe

To account for this suffix, Layering proposes the rule in~(\nextx), following \cite{embick04}:
\ex Voice \lra~\gsc{NACT} / \trace No Spec
\xe

@@What if the verb moves to T and then there's a specifier?@@

I make a different assumption: at least for Hebrew, the morphology of {\tnif} was the spell-out of {\vz}. In other words, it is the flavor of Voice which is spelled out as non-active morphology, not Voice when it has no specifier. Recall the reason for this preference: underspecified Voice in Hebrew is spelled out as {\tkal} regardless of whether it has a specifier or not. The spell-out rule in~(\nextx) is thus more consistent crosslinguistically:
\ex {\vz} \lra~\gsc{NACT} \hfill (always No Spec)
\xe

The cases in~(\ref{ex:typo-layer}b),(\ref{ex:typo-layer}d) can both be accounted for using the rule in~(\lastx), if we assume that {\vz} is in fact the Voice head in those structures.
	
	\subsection{The trivalency of Voice}


\ex\label{ex:alternations-heb}Hebrew has \textbf{trivalent} morphological marking \citep{kastner18nllt}:\\
	\begin{tabular}{cll|ll|ll}
	& \multicolumn{2}{P{4.2cm}|}{causative} &	\multicolumn{2}{P{4cm}|}{underspecified}	& \multicolumn{2}{P{4.2cm}}{anticausative}\\\cline{2-7}
	\phantom{Semantics} & \multicolumn{2}{c|}{\thif}	&	\multicolumn{2}{c|}{\tkal}	& \multicolumn{2}{c}{\tnif}\\
	& \emph{heexil}	& `fed' &	\emph{axal}	& `ate'	&	\emph{neexal}	& `was eaten' \\
	& \emph{hextiv}	& `dictated' &	\emph{katav}	& `wrote'	&	\emph{nixtav}	& `was written' \\
	\end{tabular}
\xe

\ex\label{typo-feat}\textbf{The typology of Featural:}\\
\begin{tabular}{c|ll|ll|ll}
	& \multicolumn{2}{P{4cm}|}{\vd}	&  \multicolumn{2}{P{4cm}|}{Voice}	& \multicolumn{2}{P{4cm}}{\vz} \\\hline
%&&&&\\
Semantics	 & 		a.	&	&			b.	&& 	c. & \\
$\lambda$x 	 & 
&\Tree
[.VoiceP 
	[.DP ]
	[.
		[.{\vd} ]
		[.vP ]
	]
]
& 
&\Tree
[.VoiceP 
	[.DP ]
	[.
		[.Voice ]
		[.vP ]
	]
]
&& \phantom{Undefined.}
\\\hline
Semantics	 & 		d.		& &			e.	& &	f. & \\
\zero	 &
& \phantom{Undefined.}
&
&\Tree
	[.VoiceP
		[.Voice ]
		[.vP ]
	]
&
&\Tree
	[.VoiceP
		[.{\vz} ]
		[.vP ]
	]\\
\end{tabular}
\xe


\pex\label{intro-layer}The \textbf{Layering} theory \citep{schaefer08,layering15}.
	\a Proposed on the basis of English and German (mostly).\\
		Further developed for Latin, Albanian, Greek, Icelandic, Quechua and more English:\\ \cite{embick04,kallulli13,spathasetal15,wood15springer,schaefervivanco16,myler17oup,schaefer17oup}.
	\a Syntactic and semantic features combine freely.
\xe

\pex\label{intro-feat}The \textbf{Featural} theory \citep{kastner16phd}.
	\a Proposed on the basis of Hebrew.\\
		Further developed for Japanese, Austronesian, Polynesian and more Hebrew:\\ \cite{kastner17gjgl,kastner18tlr,kastner18nllt,oseki17nyu,nie17}.
	\a Syntactic features feed semantic features.
\xe

\ex The empirical difference underlying the difference: the Layering theories considered anticausative marking, not causative marking.
\xe



\pex The missing cells (\ref{typo-feat}c)--(\ref{typo-feat}d):
	\a This is an attempt to constrain the system.
	\a Not the case that everything goes.
	\a The syntactic heads do have semantic implications, (\ref{sem-feat}).
	\a Could also be done in the Layering way.
\xe

\pex What about \textbf{passives} in Greek (\ref{typo-layer}b)--(\ref{typo-feat}c,f)?
	\a For every root we need to state whether the \gsc{NACT} variant will be anticausative, passive, or compatible with both.
	\a We have to do this regardless of formalization.
	\a \cite{layering15} for Greek, \cite{ahdoutkastner18} for Hebrew.
\xe
	
\pex What about \textbf{expletives} in Romance and Germanic, (\ref{typo-layer}c)--(\ref{typo-feat}b,d)?
	\a Regular Voice with an expletive in its Spec.
	\a But see \cite{wood15springer} for more variants.
\xe
	
\pex \textbf{To sum up}, I would say:
	\a {\vd} is always $\lambda$x.
	\a {\vz} is always $\lambda${\zero}.
\xe
\ding{228} (But my figure reflexives are different... hmm...)



\section{Crosslinguistic expectations}

{\vz} - Voice
Voice - {\vd}

\ex\label{ex:alternations-heb2}Featural analysis of the templates:\\
	\begin{tabular}{cll|ll|ll}
	& \multicolumn{2}{P{4cm}|}{\textbf{\vd}}	&	\multicolumn{2}{P{4cm}|}{\textbf{Voice}}	& \multicolumn{2}{P{4cm}}{\textbf{\vz}}\\
	\phantom{Semantics} & \multicolumn{2}{c|}{causative} &	\multicolumn{2}{c|}{transitive}	& \multicolumn{2}{c}{anticausative}\\\cline{2-7}
	& \multicolumn{2}{c|}{\thif}	&	\multicolumn{2}{c|}{\tkal}	& \multicolumn{2}{c}{\tnif}\\
	& \emph{heexil}	& `fed' &	\emph{axal}	& `ate'	&	\emph{neexal}	& `was eaten' \\
	& \emph{hextiv}	& `dictated' &	\emph{katav}	& `wrote'	&	\emph{nixtav}	& `was written' \\
	\end{tabular}
\xe

\pex\label{sem-feat}Semantics (abstracting away from Agent $\neq$ Cause):
	\a \denote{\vd} = $\lambda x \lambda e$.Agent($x,e$)
	\a \denote{Voice}\phantom{.......} = $\begin{cases}
		\lambda x \lambda e.\text{Agent}(x,e) & \text{/ \trace \{\root{\gsc{eat}}, \dots\} }\\
		\lambda e.e & \text{/ \trace \{\root{\gsc{fall}}, \dots\} }\\
	\end{cases}$
	\a \denote{\vz}\phantom{.} = $\begin{cases}
		\lambda e.e & \text{/ \trace \{\root{\gsc{burn}}, \dots\} }\\
		\lambda e \exists x.\text{Agent}(x,e) & \\
	\end{cases}$
\xe

\ex The \textbf{typological predictions} are clear: find me a language with overt \emph{causative} marking and an expletive. Currently predicted not to exist.
\xe



\pex Which language has what?
	\a Possibility 1: All languages are Featureal languages like in (\ref{typo-feat}).
	\a Possibility 2: The flavors of Voice need to be morphophonologically distinct.\\
		\cite{layering15}: English learners don't hypothesize (\ref{typo-layer}c) or (\ref{typo-layer}d) because they have no phonological evidence for expletive Voice.
	\a Possibility 3: All languages are at least active/non-active Voice languages, even without phonological evidence.
\xe



\section{Conclusion}
\pex Layering $\Rightarrow$ Featural.
	\a More natural(?) way to model the syntax/semantics interface.
	\a No Layering $\prec$ Layering $\prec$ Featural.
	\a Necessary for causativer alternations.
\xe

\pex Causative alternations:
	\a Different than anticausative alternations.
	\a Asymmetry is captured(?) by the syntactic analysis.
\xe