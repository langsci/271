\section{Introduction} \label{sec:intro}
\ex \cite{schaefer08} provided two kinds of innovation for the way we analyze argument structure within minimalism/DM.
	(also \citealt{alexiadouetal06,layering15})
\xe

\pex\label{florian-conce}\textbf{Conceptual} innovation:
	\a Syntactic transitivity (Spec,VoiceP).
	\a Semantic transitivity (Agent).
\xe

\pex\label{florian-tech}\textbf{Technical} innovation:
	\a Active/Non-active Voice.
	\a Thematic/expletive Voice.
\xe

\textbf{Today:} contrast two specific implementations.

\pex\label{intro-layer}The \textbf{Layering} theory \citep{schaefer08,layering15}.
	\a Proposed on the basis of English and German (mostly).\\
		Further developed for Latin, Albanian, Greek, Icelandic, Quechua and more English:\\ \cite{embick04,kallulli13,spathasetal15,wood15springer,schaefervivanco16,myler17oup,schaefer17oup}.
	\a Syntactic and semantic features combine freely.
\xe

\pex\label{intro-feat}The \textbf{Featural} theory \citep{kastner16phd}.
	\a Proposed on the basis of Hebrew.\\
		Further developed for Japanese, Austronesian, Polynesian and more Hebrew:\\ \cite{kastner17gjgl,kastner18tlr,kastner18nllt,oseki17nyu,nie17}.
	\a Syntactic features feed semantic features.
\xe

\ex The empirical difference underlying the difference: the Layering theories considered anticausative marking, not causative marking.
\xe

\newpage
\section{Syntactic vs semantic transitivity}
	\subsection{Formalisms}
\ex\label{typo-layer}The four-celled typology in \citet[109]{layering15}:\\
%\begin{tabular}{c|ll|ll}
%	& \multicolumn{2}{c|}{Semantics $\lambda$x}	& 	\multicolumn{2}{c}{Semantics {\zero}} \\\hline
%%&&&&\\
%Syntax & 	a.&	Thematic active 	&	c.&	Thematic non-active\\
%+D 	 & &
%\Tree
%[.VoiceP 
%	[.DP ]
%	[.VoiceP
%		[.{Voice\{$\lambda$x, D\}} ]
%		[.{\dots~vP \dots} ]
%	]
%]
%& &
%\Tree
%[.VoiceP 
%	[.DP\\\fbox{\gsc{SE}} ]
%	[.VoiceP
%		[.{Voice\{\zero, D\}} ]
%		[.{\dots~vP \dots} ]
%	]
%]
%\\
%&&&&\\
%& & \ding{228} Business as usual.	& & \ding{228} Marked anticausatives (German).\\
%&&&&\\\hline
%Syntax & 	b.&	Thematic non-activex 	&	d.&	Expletive non-active\\
%{\zero}	 & &
%\Tree
%[.VoiceP 
%		[.{Voice\{$\lambda$x, \zero\}\\\fbox{\gsc{NACT}}} ]
%		[.{\dots~vP \dots} ]
%]
%
%& &
%\Tree
%[.VoiceP 
%		[.{Voice\{\zero, \zero\}\\\fbox{\gsc{NACT}}} ]
%		[.{\dots~vP \dots} ]
%]
%\\
%&&&&\\
%& & \ding{228} Passives in Greek.	& & \ding{228} Anticausatives in Greek.\\
%\end{tabular}

\begin{tabular}{c|ll|ll}
	& \multicolumn{2}{c|}{Syntax D}	& 	\multicolumn{2}{c}{Syntax {\zero}} \\\hline
%&&&&\\
Semantics & 	a.&	Thematic active 	&	b.&	Thematic non-active\\
$\lambda$x 	 & &
\Tree
[.VoiceP 
	[.DP ]
	[.VoiceP
		[.{Voice\{$\lambda$x, D\}} ]
		[.{\dots~vP \dots} ]
	]
]
& &
\Tree
[.VoiceP 
		[.{Voice\{$\lambda$x, \zero\}\\\fbox{\gsc{NACT}}} ]
		[.{\dots~vP \dots} ]
]
\\
&&&&\\
& & \ding{228} Business as usual.	& & \ding{228} Passives in Greek.\\
&&&&\\\hline
Semantics & 	c.&	Expletive active 	&	d.&	Expletive non-active\\
{\zero}	 & &
\Tree
[.VoiceP 
	[.DP\\\fbox{\gsc{SE}} ]
	[.VoiceP
		[.{Voice\{\zero, D\}} ]
		[.{\dots~vP \dots} ]
	]
]
& &
\Tree
[.VoiceP 
		[.{Voice\{\zero, \zero\}\\\fbox{\gsc{NACT}}} ]
		[.{\dots~vP \dots} ]
]
\\
&&&&\\
& & \ding{228} Marked anticausatives (German).	& & \ding{228} Anticausatives in Greek.\\
\end{tabular}
\xe

\pex \textbf{Mismatches} (the interesting cases -- not evaluating these):
	\a Passives~(\ref{typo-layer}b) have a semantic external argument but no syntactic external argument.\\
	$\Rightarrow$ Need to allow existential closure on an open $\lambda$x. \hfill \citep[124]{layering15}
	\a Marked anticausatives~(\ref{typo-layer}c) predict no semantic effect of \gsc{SE}.\\
	$\Rightarrow$ different derivation than reflexive \gsc{SE}. \hfill \citep[111]{layering15}
\xe

\ex\label{ex:alternations-heb}Hebrew has \textbf{trivalent} morphological marking \citep{kastner18nllt}:\\
	\begin{tabular}{cll|ll|ll}
	& \multicolumn{2}{P{4.2cm}|}{causative} &	\multicolumn{2}{P{4cm}|}{underspecified}	& \multicolumn{2}{P{4.2cm}}{anticausative}\\\cline{2-7}
	\phantom{Semantics} & \multicolumn{2}{c|}{\thif}	&	\multicolumn{2}{c|}{\tkal}	& \multicolumn{2}{c}{\tnif}\\
	& \emph{heexil}	& `fed' &	\emph{axal}	& `ate'	&	\emph{neexal}	& `was eaten' \\
	& \emph{hextiv}	& `dictated' &	\emph{katav}	& `wrote'	&	\emph{nixtav}	& `was written' \\
	\end{tabular}
\xe

\ex\label{typo-feat}\textbf{The typology of Featural:}\\
\begin{tabular}{c|ll|ll|ll}
	& \multicolumn{2}{P{4cm}|}{\vd}	&  \multicolumn{2}{P{4cm}|}{Voice}	& \multicolumn{2}{P{4cm}}{\vz} \\\hline
%&&&&\\
Semantics	 & 		a.	&	&			b.	&& 	c. & \\
$\lambda$x 	 & 
&\Tree
[.VoiceP 
	[.DP ]
	[.
		[.{\vd} ]
		[.vP ]
	]
]
& 
&\Tree
[.VoiceP 
	[.DP ]
	[.
		[.Voice ]
		[.vP ]
	]
]
&& \phantom{Undefined.}
\\\hline
Semantics	 & 		d.		& &			e.	& &	f. & \\
\zero	 &
& \phantom{Undefined.}
&
&\Tree
	[.VoiceP
		[.Voice ]
		[.vP ]
	]
&
&\Tree
	[.VoiceP
		[.{\vz} ]
		[.vP ]
	]\\
\end{tabular}
\xe
%\begin{tabular}{c|l|l}
%	& Semantics $\lambda$x	&  Semantics \zero	\\\hline
%%&&&&\\
%{\vd}	 & 		a.		&			d.	\\
% 	 & 
%\Tree
%[.VoiceP 
%	[.DP ]
%	[.
%		[.{\vd} ]
%		[.vP ]
%	]
%]
%& 
%\phantom{Undefined.}
%\\\hline
%
%Voice & b.		&		e. \\
%	&
%\Tree
%[.VoiceP 
%	[.DP ]
%	[.
%		[.Voice ]
%		[.vP ]
%	]
%]
%&
%\Tree
%	[.VoiceP
%		[.Voice ]
%		[.vP ]
%	]
%\\\hline
%
%{\vz}	 & 		c.		&			f. \\
%	 &
%\phantom{Undefined.}
%&
%\Tree
%	[.VoiceP
%		[.{\vz} ]
%		[.vP ]
%	]\\
%\end{tabular}

\ex\label{ex:alternations-heb2}Featural analysis of the templates:\\
	\begin{tabular}{cll|ll|ll}
	& \multicolumn{2}{P{4cm}|}{\textbf{\vd}}	&	\multicolumn{2}{P{4cm}|}{\textbf{Voice}}	& \multicolumn{2}{P{4cm}}{\textbf{\vz}}\\
	\phantom{Semantics} & \multicolumn{2}{c|}{causative} &	\multicolumn{2}{c|}{transitive}	& \multicolumn{2}{c}{anticausative}\\\cline{2-7}
	& \multicolumn{2}{c|}{\thif}	&	\multicolumn{2}{c|}{\tkal}	& \multicolumn{2}{c}{\tnif}\\
	& \emph{heexil}	& `fed' &	\emph{axal}	& `ate'	&	\emph{neexal}	& `was eaten' \\
	& \emph{hextiv}	& `dictated' &	\emph{katav}	& `wrote'	&	\emph{nixtav}	& `was written' \\
	\end{tabular}
\xe

\pex\label{sem-feat}Semantics (abstracting away from Agent $\neq$ Cause):
	\a \denote{\vd} = $\lambda x \lambda e$.Agent($x,e$)
	\a \denote{Voice}\phantom{.......} = $\begin{cases}
		\lambda x \lambda e.\text{Agent}(x,e) & \text{/ \trace \{\root{\gsc{eat}}, \dots\} }\\
		\lambda e.e & \text{/ \trace \{\root{\gsc{fall}}, \dots\} }\\
	\end{cases}$
	\a \denote{\vz}\phantom{.} = $\begin{cases}
		\lambda e.e & \text{/ \trace \{\root{\gsc{burn}}, \dots\} }\\
		\lambda e \exists x.\text{Agent}(x,e) & \\
	\end{cases}$
\xe

\ex The \textbf{typological predictions} are clear: find me a language with overt \emph{causative} marking and an expletive. Currently predicted not to exist.
\xe

	\subsection{Comparison}
\pex No Voice in Layering = {\vz} in Featural.
	\a 	Voice \lra~\gsc{NACT} / \trace No Spec
	\a  {\vz} \lra~\gsc{NACT} \hfill (always No Spec)
	\a A little cleaner closslinguistically?
\xe
\ding{228} Not a substantive difference.

\pex The missing cells (\ref{typo-feat}c)--(\ref{typo-feat}d):
	\a This is an attempt to constrain the system.
	\a Not the case that everything goes.
	\a The syntactic heads do have semantic implications, (\ref{sem-feat}).
	\a Could also be done in the Layering way.
\xe

\pex What about \textbf{passives} in Greek (\ref{typo-layer}b)--(\ref{typo-feat}c,f)?
	\a For every root we need to state whether the \gsc{NACT} variant will be anticausative, passive, or compatible with both.
	\a We have to do this regardless of formalization.
	\a \cite{layering15} for Greek, \cite{ahdoutkastner18} for Hebrew.
\xe
	
\pex What about \textbf{expletives} in Romance and Germanic, (\ref{typo-layer}c)--(\ref{typo-feat}b,d)?
	\a Regular Voice with an expletive in its Spec.
	\a But see \cite{wood15springer} for more variants.
\xe
	
\pex \textbf{To sum up}, I would say:
	\a {\vd} is always $\lambda$x.
	\a {\vz} is always $\lambda${\zero}.
\xe
\ding{228} (But my figure reflexives are different... hmm...)

\pex Which language has what?
	\a Possibility 1: All languages are Featureal languages like in (\ref{typo-feat}).
	\a Possibility 2: The flavors of Voice need to be morphophonologically distinct.\\
		\cite{layering15}: English learners don't hypothesize (\ref{typo-layer}c) or (\ref{typo-layer}d) because they have no phonological evidence for expletive Voice.
	\a Possibility 3: All languages are at least active/non-active Voice languages, even without phonological evidence.
\xe


\ruler
\section{The causative-causativer alternation}
\pex
	\a Most of the existing work is on marking of anticausatives.
	\a Compare the typological spread of languages in (\ref{intro-layer}a)--(\ref{intro-feat}a).
\xe

\pex The causative-inchoative alternation.
	\a \emph{John} \textbf{Voice} [_{\text{vP}} \emph{broke the glass}].
	\a \textbf{\vz} [_{\text{vP}} \emph{The glass broke}].
	\a What's important is that the vP is a predicate over eventualities. Voice can add an external argument if it ``wants'' to \citep{layering15}.
\xe

\pex This section:
	\a Regular Voice allows us to add an external argument.
	\a What's left for an additional grammatical device (\vd) to do?
	\a Difference in productivity between the anticausative and the causative-r alternations.
\xe

	\subsection{Marked causatives}
\pex What about \textbf{marked causatives}?
	\a Japanese, Hebrew, \dots
	\a \citet[62f3]{layering15}: causative version $\Rightarrow$ thematic voice.
	\a Featural: causative version $\Rightarrow$ {\vd}.
\xe

\ex Well, that was easy!
\xe

	\subsection{It wasn't}
\pex \textbf{How productive} are these alternations?
	\a Very, in the sense that there are lots of roots with both {\tkal} (Voice) and {\thif} (\vd) forms.
	\a About 300 out of 500-550 verbs in {\thif} in general \citep{kastner18tlr}.
\xe

\pex \textbf{How predictable} are these alternations? Not very. Recall (\ref{ex:alternations-heb2}).
	\a The causative of `wrote' is `dictated'.
	\a \begingl
		\gla ha-talmidim \textbf{katvu} et ha-nosim//
		\glb the-students wrote \gsc{ACC} topics//
		\glft `The students wrote down the list of topics.'//
		\endgl
	\a \begingl
		\gla ha-more \textbf{hextiv} et ha-nosim (la-talmidim)//
		\glb the-teacher dictated \gsc{ACC} topics to.the-students//
		\glft `The teacher dictated the list of topics (to the students).//
	\endgl
	\a \Tree [. [.students ] [. [.Voice ] [. [.\root{\gsc{WROTE}} ] [.topics ] ] ] ]		\Tree [. [.teacher ] [. [.{\vd} ] [. [.\root{\gsc{WROTE}} ] [.topics ] ] ] ]
\xe

\pex
	\a The causative of \emph{axal} `ate': \emph{heexil et (be-)} `fed s.o (s.th)'.
	\a Why not `made eat'?
	\a What about the added argument syntactically?\\
		\Tree [. [.John ] [. [.Voice ] [. [.\root{\gsc{ATE}} ] [.cake ] ] ] ]		\Tree [. [.Mary ] [. [.{\vd} ] [. [.\root{\gsc{ATE}} ] [.John ] ] ] ]
\xe

\pex Some more:\\
		\begin{tabular}{lll@{ $\sim$ }ll}
		a.& \emph{kara} & `read'		& \emph{hekri}	& `read out'\\
		b.& \emph{laxa\texttslig} & `pressed'	& \emph{helxi\texttslig} & `pressured'\\
		c.&	\emph{daxak}	& `shoved'	& \emph{hedxik}	& `suppressed'\\
		d.& \emph{sagar} & `closed'		& \emph{hesgir} & `extradited'\\
		\end{tabular}
\xe

\pex Predictable meaning?\\
	\begin{tabular}{lcc}
		& Make X Verb	& make X be Verb-ed\\
	a. `ate' - `fed'	& \cmark & \xmark\\
	b. `read' - `read out' & \xmark & \cmark\\
	c. `closed' - `extradited' & \xmark	& \textbf{?}\\
	\end{tabular}
\xe
\ding{228} Other factors might be at play, depending on the animacy of the object and so on.
	
\ex Or maybe your language has Voice-over-Voice \citep{nie17}.
\xe

	\subsection{Generalization}
\pex The anticausative-transitive alternation seems more transparent:\\
		\begin{tabular}{lll@{ $\sim$ }lc@{ $\sim$ }ll}
		a.& \textbf{\emph{nikra}} & \textbf{`was read'} & \emph{kara} & `read'		& \emph{hekri}	& read out\\
		b. & \textbf{\emph{nilxa\texttslig}} & \textbf{`was pressured'} & \emph{laxa\texttslig} & `pressed'	& \emph{helxi\texttslig} & `pressured'\\
		c.&	\textbf{\emph{nidxak}} & \textbf{`was shoved (aside)'} & \emph{daxak}	& `shoved'	& \emph{hedxik}	& `suppressed'\\
		d.& \textbf{\emph{nisgar}} & \textbf{`was closed'} & \emph{sagar} & `closed'		& \emph{hesgir} & `extradited'\\
		\end{tabular}
\xe

\pex \textbf{The super-causative generalization for transitivity alternations}\\
	(alternative names are welcome):
	\a The anticausative alternation is transparent.
	\a The causativer alternation is not: root-specific.
\xe

\pex So would we even call this a generalization? Why not different roots/verbs?
	\a No doublets: there are no additional verbs in {\tnif} that alternate with {\thif}.
	\a E.g.~additional \emph{nikra} *`was read out' alternating with \emph{hekri} `read out' for (\blastx a).
\xe

	\subsection{Analysis}
\pex This pattern follows directly from the general layering approach to transitivity alternations.
	\a If the basic vP already has causation, then it's clear what \textbf{not} adding an external argument would mean.
	\a It's also relatively clear what adding an external argument would mean.
	\a But adding an external argument in another, differentiating way would be hard.
	\a $\Rightarrow$ Necessitates change in meaning as well.
\xe

\pex 
	\a Say we have an event of causation:\\
\Tree
	[.vP
		[.v ]
		[.DP ]
	]

	\a \textbf{Not} adding a causer is easy:\\
\Tree
[.VoiceP
	[.{\vz} ]
	[.vP
		[.v ]
		[.DP ]
	]
]

	\a And you can have various kinds of external arguments to go with different causation events:\\
\Tree
[.VoiceP
	[.DP_1 ]
	[.
		[.Voice ]
		[.vP
			[.v ]
			[.DP ]
		]
	]
]
\Tree
[.VoiceP
	[.DP_2 ]
	[.
		[.{\vd} ]
		[.vP
			[.v ]
			[.DP ]
		]
	]
]

	\a But how would you have various kinds of lack of external arguments? Different events of minimal causation?
\xe

\pex Prediction:\\
	Few triplets of the form [marked \textbf{unaccusative} $\sim$ unmarked \textbf{uanccusative} $\sim$ marked causative].
	\a Correct!
	\a \root{xrv} `\root{\gsc{DEMOLISH}}': \emph{nexrav} $\sim$ ??\emph{xarav} $\sim$ \emph{hexriv}
	\a \root{ʃlm} `\root{\gsc{BE.WHOLE}}': \emph{niʃlam} $\sim$ ??\emph{ʃalam} $\sim$ \emph{heʃlim}
	\a Not productive in {\tkal}, speakers prefer the {\tnif} form for the anticausative.
\xe


\ruler
\section{Summary}
\pex Layering $\Rightarrow$ Featural.
	\a More natural(?) way to model the syntax/semantics interface.
	\a No Layering $\prec$ Layering $\prec$ Featural.
	\a Necessary for causativer alternations.
\xe

\pex Causative alternations:
	\a Different than anticausative alternations.
	\a Asymmetry is captured(?) by the syntactic analysis.
\xe