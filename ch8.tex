summary

future work
	- interaction of Voice and roots
	- psych verbs


The main issue addressed in this chapter was the following: how are semantic roles distributed in the syntax and how are they reflected in the morphology. The overarching question is a familiar one from linguistic theorizing: what does the learner need to know about each and every construction (root, verb) and what can be underspecified? If the learner knows that a verb in \tnif~has one fewer external argument, for example, then she does not need to memorize that fact about each and every verb in that template.  

I have argued for a system in which neither θ-roles nor valency-reducing operations are necessary. Instead, an active syntactic structure and a nonactive syntactic structure---that is, one with an external argument and one without---can both result in the same ``middle'' morphological marking. The correct interpretation of the verb is a result of functional heads combining with an idiosyncratic root.

The analysis presented here attempted to answer a general question and a specific question relating to the morphology of Modern Hebrew. Generally, it is the case that one cannot predict the meaning of a verb from its morphophonological form (its template), nor can one predict what template a verb will have based solely on its meaning. The solution to this mapping problem was implemented in a system that builds syntactic structure and then interprets said structure at PF and LF. For a consistent system to be set up, templates must be viewed as emergent from functional heads in the structure and not as morphemes, which is the traditional view.

Once the structure is set up correctly, roots have the power to influence the interpretation at the semantics. I have sketched what a framework looks like that allows us to discuss the empirical and theoretical consequences of root semantics: the structure also needs to be interpreted under certain locality constraints. In this chapter it was argued that strict adjacency of the nearest contentful element is the deciding factor in selecting the alloseme of the root. This element is usually, but not always, the first categorizing head.

In the next chapter I argue that the same locality constraint holds at the parallel interface, that with phonology. We will derive the morphophonology of the templates and in so doing bolster the arguments for root classes active at the interface and against templates as morphemic primitives.



\section{Theories of Hebrew morphology}

This section contrasts my approach to Hebrew verbal morphology with a number of prominent proposals that share the two major assumptions: morphological structure is built in the syntax, and an abstract root (consonantal in Semitic) lies at the base of the derivation. Lexicalist analyses are reviewed in \S\ref{syn:other-stem}. 

Returning to the overarching themes guiding this dissertation, a contrast was brought up between the ``emergent'' approach to verbal templates and the morpheme-based approach. My theory treats templates as epiphenomenal, as does that of \cite{doron03} in \S\ref{syn:other-root:ed}. The analyses of \cite{arad05} in \S\ref{syn:other-root:arad} and \cite{borer13oup} in \S\ref{syn:other-root:borer} treat each template as its own morpheme.

This difference relates to a broader question: how syntax feeds the interfaces. If a template is a morpheme, then there is no strong architectural claim to be made. But if different syntactic structures can give rise to identical morphophonology, as I claim, we are forced into a strong view of the syntax as the sole generative engine. That is to say, if similar interpretations can arise from different syntactic structures---for example reflexive readings with entirely different underlying structures---we are led to a strong view of the autonomy of syntax; the existence of figure reflexives and regular reflexives does not entail one syntax for both constructions. Instead, independent structures may be interpreted similarly. The accounts surveyed below all differ from mine on these points to different extents.


	\subsection{Root-based theories}
	\cite{doron03,doron14}: discussed throughout

		\subsubsection{Templates as morphemes \citep{arad05}} \label{syn:other-root:arad}
In juxtaposition to an emergent view, \cite{arad05} treated verbal templates as distinct spell-outs of Voice. Together with \cite{arad03}, both works mounted a strong defense of the root as the base of the derivation beyond Hebrew as well.

For \citeauthor{arad05}, templates are different ``flavors'' of v, each a separate CV-skeleton, and the vocalic melody is inserted at Voice. Each template might have a feature such as [--transitive], which is shared by \tnif~and \thit. The general problem with morphemic approaches to templates has been raised a number of times so far. A given template simply does not have a deterministic syntax or semantics (save for the two passive templates). \citet[198]{arad05} actually speculates that a configurational approach (like in our theory) might be more viable than a feature-based approach. 

Regarding argument structure alternations, \cite{arad05} claims that all templates can derive roots from verbs but that two of the templates---\tnif~and \thit---can also derive verbs from other verbs. In so doing, she hopes to capture alternations such as those between \tkal~and \tnif~or between \tpie~and \thit.  When \citeauthor{arad05} tries to predict which templates can detransitivize or causativize verbs in other templates, she is forced to stipulate these relations. The basic observation is entirely accurate, in my view: in a given template, some verbs are root-derived and some are derived from another verb in a different template. In \tnif, inchoatives are root-derived and anticausatives are derived from underlying active verbs in \tkal. I submit that the theory as a whole is more constrained if functional heads have specific syntactic and semantic properties; idiosyncrasy can then be left to the root, as proposed by \citeauthor{arad05} herself.
	
		\subsubsection{The Exo-Skeletal Model \citep{borer13oup,borer15roots}} \label{syn:other-root:borer}
Another contemporary morphemic approach to templates is sketched in the Exo-Skeletal Model of \citet[Ch.~11]{borer13oup}. The bulk of \cite{borer13oup} is devoted to developing the model based on English, with the Hebrew chapter included for cross-linguistic support and as the beginning of an exploration in its own right. Given that the Exo-Skeletal analysis is still preliminary, I do not expect it to address as wide a range of phenomena as I attempt to cover in this dissertation. Nevertheless, it is worthwhile pointing out a number of potential problems as this type of analysis is developed in the future.

On the Exo-Skeletal approach, templates are once again derivational morphemes with no hierarchical structure between them. All objections I have leveled at the morphemic approach to templates are valid here too. It is not clear what the syntax and semantics of a given template should be (other than with the two passive templates) given the range of verb types that can appear in each templatic form. \cite{borer15roots} speculates that additional features like the heads from \cite{doron03} can condition the selection of certain templates, but the question then arises of how the relation between these features and these templates should be seen.

In order to derive argument structure alternations, \citet[564]{borer13oup} raises the possibility that some templates might stand in a hierarchical relationship to others, for example \tpie~and its detransitivized version \thit. The promissory discussion seems to single out these two templates but leaves open the possibility of additional hierarchical relations, rightly in my view. 

The empirical core of \citeauthor{borer13oup}'s chapter lies in patterns of nominalization and with the underspecification of \tkal. Her view of nominalization is similar to mine in \S\ref{syn:templates:nmlz} and her analysis of \tkal~is similar to mine in \S\ref{syn:templates:tkal}; see also \S\ref{sec:others-same:borer}. To a large extent the hypotheses converge, though I maintain that an analysis of Hebrew in which templates are individual morphemes cannot have the same empirical coverage or even conceptual consistency as the kind of account provided in this dissertation.


	\subsection{Stem-based approaches}\label{syn:other-stem}
Syntactic derivations with functional heads are necessary if we are to predict argument structure alternations correctly. As the next chapter shows, these heads also allow us to account for the morphophonology. But before proceeding we must discuss a different kind of alternative analysis. The other notable attempts to derive the Hebrew patterns are lexicalist, stem-based ones, which I consider next.

		\subsubsection{Morphomes and conjugation classes \citep{aronoff94,aronoff07}}
\cite{aronoff94,aronoff07} puts forward a view of morphology as an independent component of the grammar. His view is committed to treating individual stems (lexemes, ``morphomes'') as the basic lexical unit that feeds additional derivation and inflection. On such a view, each template is in essence a different conjugation class. Roots do not exist as contentful elements, only as collections of consonants over which paradigmatic, phonological generalizations can be made (though \citealt[827]{aronoff07} does suggest to treat them ``much like Latin roots'').

Theoretical differences aside, \citeauthor{aronoff94}'s framework makes no attempt to explain the argument structure alternations in the language. There is no attempt to account for how reflexive verbs only appear in \thit, or why verbs in \tpie~are agentive, or indeed any of the phenomena discussed thus far. Divorcing roots from meanings also does little to explain why meanings do persist across templates for a given root, or to delineate in what configurations meaning must be conserved (transparent argument structure alternations) and where special meaning can be tapped (alloseme of the root).
	
		\subsubsection{The Theta System \citep{reinhartsiloni05,laks11}}\label{syn:other-stem:reinhart}
\cite{reinhartsiloni05}, and following them \cite{laks11,laks13ws,laks13morpho,laks14}, present a lexicalist account of the causative alternation in Hebrew. These works argue that a process of decausativization applies in the language. Under such a view, causative verbs are the basic verbal forms in the language, from which the speaker derives reflexives, anticausatives and reciprocals in other templates. Each template thus has a prototypical role in the lexicon. For example, \tpie~houses causative verbs, from which anticausatives in \thit~are derived.

The question then arises of how to account for nonactive verbs that have no active alternation in another template. If there is no active base, how can a decausativized alternation arise? This is exactly the case of inchoative middle verbs, described in \S\ref{syn:middle:nonactive:null}. The proposed solution is that the causative verb exists in the lexicon as a \emph{frozen} entry, a verb that cannot be used in the syntactic derivation but can be used in the lexicon to derive other forms.

Two objections now arise. First, this kind of theory does not explain why a given template has whatever morphosyntactic behavior it has, e.g.~transitive or intransitive, reflexive or not. The generalizations are stipulated on a template-by-template basis: \tnif~and \thit~are anticausative, for example, because they do not receive a [cause change] feature in the lexical derivation. Yet we have seen that these ``middle'' templates are not intransitive across the board; they do house active verbs, specifically figure reflexives and in some cases reciprocals. Under a decausativization analysis, middle templates are correctly predicted to be more marked than their active bases; but it is not explained why each template has the specific morphophonological characteristics it has, and to what extent these correlate with its morphosyntax.

The second problem has to do with the notion of a ``frozen'' lexical entry. According to \citet[116]{laks14}, ``[\emph{F}]\emph{rozen entries lack phonological matrix and morphological properties} [\dots] \emph{but they are assumed to be conceptually represented in the lexicon. The frozen entry, which is not accessible for syntactic derivations, can nonetheless serve as input for lexical operations.}'' It is unclear how this notion can be falsified if a ``frozen'' entry exists to rescue the theory whenever one is necessitated. In fact, what this approach is relying on is a concept very similar to the abstract root while denying it at the same time.

%\citeauthor{laks14} goes on to describe a process of ``morphological filling'' or ``defrosting'': as mentioned earlier, the inchoative \emph{hit'alef} `fainted' does not have a standard causative correspondent *\emph{'ilef} `made faint'. Be that as it may, young innovators do use the verb \emph{'ilef}. For \citeauthor{laks14}, this is an example of the ``frozen'' entry being defrosted and becoming an active part of the lexicon. Yet now another issue arises, having to do with idiomatic meanings in the causative alternation. In their study of idioms in Hebrew and their implications for word-based and root-based approaches to morphology, \cite{horvathsiloni09} argue that any lexical entry can be associated with idiomatic meaning. Thus, looking at unaccusative-causative pairs, it may be that only one of the two alternates has idiomatic meaning, that both share an idiomatic meaning, and of course that neither does. Given that in a root-based system the meaning of the verb is a result of the interaction of the verb with different functional heads (\citealt{arad03} and the system outlined above), the two theories make the same predictions (contra the claims in \citealt{horvathsiloni09}).\footnote{I use ``idiomatic'' in a general sense to indicate non-literal meaning, as used by \cite{horvathsiloni09}. See \cite{anagnostopoulou14thli} and \cite{harley14thlib} for finer-grained distinctions.}
%
%Let us return now to \%\emph{'ilef}. The meaning of this verb can be literal, `made someone physically faint', but the verb is also used figuratively to mean `was dazzling', `knocked you off your feet'. The idiomatic meaning is only part of the causative lexical entry. Yet if this entry was frozen, it means that the idiomatic meaning was added to the verb the moment it was defrosted. This is a perplexing state of affairs. If the verb was defrosted in order to create a vehicle for an idiomatic meaning, why is the literal meaning necessary? In other words, why fill in the defrosted causative verb \emph{'ilef} rather than create an arbitrary new verb *\emph{dizel}?

%This is unexpected if the relation between \emph{'ilef} and \emph{hit'alef} involves pure valency changing, because \emph{hit'alef} can only be used in a literal sense, never in a figurative sense of `be amazed'.

%			\gll ha-kise ʃel sar ha-ocar matxil le-hitnadned\\
%			the-chair of minister.\gsc{CS} the-treasury begins to-wobble.\gsc{INTNS.MID}\\
%			\glt `The Minister of Finance is on the verge of getting fired' (lit. `The finance minister's chair has started rocking')
%			
%			\ex \gll roʃ ha-memʃala matxil le-nadned le-sar ha-ocar et ha-kise\\
%			head.\gsc{CS} the-government starts to-swing.\gsc{INTNS} to-minister.\gsc{CS} the-treasury \gsc{ACC} the-chair\\
%			\glt `The Prime Minister is starting to rock the Minister of Finance's chair.' [literal]\\
%			\# `The Prime Minister is about to fire the Minister of Finance.'

%\begin{exe}
%	\ex
%		\gll mekgayver xatax et ha-xut ve-ha-pcaca hitpoceca\\
%		MacGyver cut \gsc{ACC} the-wire and-the-bomb exploded.\gsc{INTNS.MID}\\
%		\glt `MacGyver cut the wire and the bomb blew up (as a result)
%	\ex \gll mekgayver pocec et ha-pcaca\\
%		MacGyver blew.up.\gsc{INTNS} \gsc{ACC} the-bomb\\
%		\glt `MacGyver blew up the bomb', `MacGyver detonated the bomb'
%\end{exe}

%On the root-based view taken in this paper, the problem does not arise: \emph{'ilef} would be derived by merging \va~with the root, resulting in whichever meaning the root chooses to have~\citep{doron03,arad03,arad05}. This is exactly what the locality-based approach to argument structure alternations predicts: when a verb is derived from the root, its meaning might be unpredictable, but when derived from another verb it is easier to characterize.

One recent attempt to demonstrate that the ``frozen'' idea is falsifiable is described by \cite{fadlon12}, who presents two experiments that are claimed to adjudicate between the lexicalist analysis of \cite{reinhartsiloni05} and the structural analysis of \cite{arad05} in favor of the former. However, on closer inspection the results do not provide support for one theory over the other.

		\subsubsection{Experimental arguments for ``frozen'' entries \citep{fadlon12}}  
