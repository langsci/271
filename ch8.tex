The main issue addressed in this chapter was the following: how are semantic roles distributed in the syntax and how are they reflected in the morphology. The overarching question is a familiar one from linguistic theorizing: what does the learner need to know about each and every construction (root, verb) and what can be underspecified? If the learner knows that a verb in \tnif~has one fewer external argument, for example, then she does not need to memorize that fact about each and every verb in that template.  

I have argued for a system in which neither θ-roles nor valency-reducing operations are necessary. Instead, an active syntactic structure and a nonactive syntactic structure---that is, one with an external argument and one without---can both result in the same ``middle'' morphological marking. The correct interpretation of the verb is a result of functional heads combining with an idiosyncratic root.

The analysis presented here attempted to answer a general question and a specific question relating to the morphology of Modern Hebrew. Generally, it is the case that one cannot predict the meaning of a verb from its morphophonological form (its template), nor can one predict what template a verb will have based solely on its meaning. The solution to this mapping problem was implemented in a system that builds syntactic structure and then interprets said structure at PF and LF. For a consistent system to be set up, templates must be viewed as emergent from functional heads in the structure and not as morphemes, which is the traditional view.

Once the structure is set up correctly, roots have the power to influence the interpretation at the semantics. I have sketched what a framework looks like that allows us to discuss the empirical and theoretical consequences of root semantics: the structure also needs to be interpreted under certain locality constraints. In this chapter it was argued that strict adjacency of the nearest contentful element is the deciding factor in selecting the alloseme of the root. This element is usually, but not always, the first categorizing head.

In the next chapter I argue that the same locality constraint holds at the parallel interface, that with phonology. We will derive the morphophonology of the templates and in so doing bolster the arguments for root classes active at the interface and against templates as morphemic primitives.