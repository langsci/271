\documentclass[12pt, oneside, openany]{book}
%\usepackage[left=1.7in,top=1.1in,right=1.2in,bottom=1.8in,footskip=.7in]{geometry}
\usepackage[left=1in,top=1in,right=1in,bottom=1.7in,footskip=.7in]{geometry}
 
%\usepackage[Conny]{fncychap} %Sonny
\usepackage{fncychap} %Sonny
%\makeatletter
%	\ChNumVar{\Huge}
%	\ChNameVar{\centering\LARGE}
%%	\ChNameUpperCase
%	\ChTitleVar{\centering\LARGE\bfseries}
%	\ChTitleUpperCase
%%	\ChRuleWidth{1pt}
%\makeatother

% penalize widows and orphans
\widowpenalty=100000
\clubpenalty=1000000

% for section headings (e.g. 5.2.3.1)
\setcounter{secnumdepth}{5}
\setcounter{tocdepth}{4}

\usepackage{fontspec,xunicode} %,xltxtra}
 \defaultfontfeatures{Ligatures=TeX}
\setmainfont[Renderer=ICU]{Charis SIL}
%\setmainfont[Renderer=ICU]{Brill}

%%  special font packages
%\usepackage{tipa} % load before linguex--I don't know why
%\usepackage[T1]{fontenc}

% load this early to avoid an options with pstricks (why?)
\usepackage[dvipsnames,table]{xcolor} %[usenames]

\usepackage{amsmath,amssymb}

% page format things
%\setlength{\parindent}{.5in} % geometry command - indent 1/2 inch
\usepackage{setspace} % for \doublespacing
\usepackage{fancyhdr} % though I don't actually use this
\usepackage{pstricks} % for commands like \uput
% \usepackage{pstricks-add}  % consider changing to this, but read doc first
\usepackage{comment} % to comment out big chunks

%\usepackage{xypic}
\usepackage[titletoc]{appendix}
\usepackage[titles]{tocloft}

% citations
\usepackage{natbib}
\bibpunct[:]{(}{)}{;}{a}{}{,}
%\bibpunct[:]{(}{)}{;}{a}{,}{,}

% linguistic examples
%\usepackage{linguex}
%\renewcommand{\firstrefdash}{} % refer to examples as (1a) rather than (1-a)
\usepackage{expex} %Linguistics examples
	\lingset{aboveexskip=0.5ex,belowexskip=0.5ex,aboveglftskip=0.1ex,interpartskip=0.1em,labelwidth=!6pt,belowpreambleskip=0.1ex} %, *=*?}
	\usepackage{epltxfn}

% trees
\usepackage[nocenter]{qtree}
\usepackage{tikz}
\tikzstyle{every picture}+=[remember picture]

%\usepackage{forest}
%\forestset{% 
%  % imitate qtree style
%  qtree/.style={%
%    baseline,
%    for tree={%
%      parent anchor=south,
%      child anchor=north,
%      align=center,
%      base=top,
%      inner sep=1pt,
%      % l-=4ex,
%      l=0pt,
%      before typesetting nodes={%
%        if content={}{%
%          for parent={%
%            for children={anchor=north},
%            calign=fixed edge angles,
%            calign angle=61,              
%          },
%          inner sep=0pt,
%          outer ysep=-0.55pt, %originally -0.49 individual chapters -.5
%          calign=fixed edge angles,
%          calign angle=61,       
%        }{},
%      },
%    },
%  },
%  % imitate squared tikz-qtree style
%  squared/.style={%
%    for tree={%
%      edge path={%
%        \noexpand\path[\forestoption{edge}]
%        (!u.parent anchor) -|
%        (.child anchor);
%        \forestoption{edge label};
%      },
%      parent anchor=south,
%      child anchor=north,
%      align=center,
%      base=top,
%      inner sep=1pt,
%      l sep=0.5em,
%      l=0pt,
%      before typesetting nodes={%
%        if content={}{%
%          before computing xy={l=1.5em},
%          inner sep=0pt,
%          outer ysep=-0.49pt,
%        }{},
%      },
%    },
%  },
%}


\usepackage{xytree}
\usepackage{stmaryrd}
\usepackage{wasysym} %checkbox, pointer
\usepackage{array} %tabular column formatting
	\newcolumntype{P}[1]{>{\centering\arraybackslash}p{#1}}
	
\usepackage{multicol}
\usepackage{multirow}
\usepackage{dashrule} %\hdashrule
\usepackage{arydshln}
\usepackage{dashbox} %\dbox
\usepackage{mdwlist}     %  compact lists: \begin{itemize*}

\usepackage[normalem]{ulem}
\usepackage{pifont}
\newcommand{\cmark}{\ding{51}}% 52
\newcommand{\xmark}{\ding{55}}%
 \newcommand{\hand}{\ding{43}}

% for inserting graphics; the second one allows us to insert .png files etc.
%\usepackage{graphics}
\usepackage{graphicx}

% tables
\usepackage{rotating} % rotate tables sideways: \begin{sidewaystable}...

% fonts and special characters
%\usepackage{times} % or \usepackage{palatino}
%\usepackage{fixltx2e} % for text subscripts
%\usepackage{marvosym} % MVRightarrow
%\usepackage{amssymb}


%\usepackage{textcomp} % low tilde
%\usepackage{upgreek} %  upphi
%\usepackage{verbatim} % for code

%%ITK
\usepackage{subcaption}
%\usepackage{xr}
%	\externaldocument[I-]{ch2}
%	\externaldocument[II-]{ch2}
%	\externaldocument[III-]{ch3}
%	\externaldocument[IV-]{ch4}
%	\externaldocument[V-]{ch5a}
%	\externaldocument[VV-]{ch5b}


\newcommand\trace{\rule[-0.5ex]{0.5cm}{.4pt}}
\newcommand\zero{\O{}}
%\newcommand\itp[1]{\textit{\textipa{#1}}}
\newcommand\gsc[1]{\textsc{\lowercase{#1}}} %for glossing in small caps - comment out to return to caps
\renewcommand\root[1]{$\sqrt{\text{#1}}$}
	\newcommand\olive[1]{\textcolor{olive}{#1}}
%%	\newcommand\green[1]{\textcolor{ForestGreen}{#1}}
%%	\newcommand\cyan[1]{\textcolor{cyan}{#1}}
%%	\newcommand\brown[1]{\textcolor{brown}{#1}}
%	\newcommand\gray[1]{\textcolor{gray}{#1}}
\newcommand\denote[1]{$\llbracket$#1$\rrbracket$}
\newcommand\lra{$\leftrightarrow$}

\newcommand\glemph[1]{\textbf{#1}}
 \newcommand\hammer[1]{\begin{center}\fbox{\parbox{0.9\linewidth}{\centering \textsf{#1}}}\end{center}}
 
%Color macros
%\newcommand\red[1]{\textcolor{red}{#1}}
%\newcommand\blue[1]{\textcolor{blue}{#1}}
%Voice heads
\newcommand\dgs[1]{\textsubarch{#1}}
	\newcommand\vz{\text{Voice$_{\text{[--D]}}$}}
	\newcommand\vd{\text{Voice$_{\text{[\!+\!D]}}$}}
	\newcommand\pz{\text{$p_{\text{[--D]}}$}}
	\newcommand\va{\root{\gsc{ACTION}}}
	\newcommand{\tkal}{\emph{XaYaZ}}
	\newcommand{\tpie}{\emph{Xi\dgs{Y}eZ}}
	\newcommand{\tpua}{\emph{Xu\dgs{Y}aZ}}
	\newcommand{\mpua}{\emph{meXu\dgs{Y}aZ}}
	\newcommand{\thif}{\emph{heXYiZ}}
	\newcommand{\thuf}{\emph{huXYaZ}}
	\newcommand{\mhuf}{\emph{muXYaZ}}
	\newcommand{\thit}{\emph{hitXa\dgs{Y}eZ}}
	\newcommand{\tnif}{\emph{niXYaZ}}
\newcommand{\ts}{\texttslig}
\newcommand\del[1]{$<$#1$>$}
%Morphological marking
\newcommand\unmark[1]{\textbf{#1}}
%\newcommand\caus[1]{\red{#1}}
\newcommand\caus[1]{\fbox{\unmark{#1}}}
%\newcommand\anticaus[1]{\blue{#1}}
\newcommand\anticaus[1]{\dbox{\unmark{#1}}}

\definecolor{nyu}{RGB}{87,6,140}
\definecolor{hu}{RGB}{0,51,102}
\newcommand\ruler{{\centering \color{hu} \rule[1ex]{0.6\columnwidth}{0.4pt}\par}}

% lists
\usepackage{mdwlist}     %  compact lists: \begin{itemize*}
% make second-level bullets into little circles (or \diamond)
\AtBeginDocument{
  \def\labelitemii{\(\circ\)}
  \def\labelitemiii{\(\cdot\)}
}

% for PDF markup and links
\usepackage[normalem]{ulem} % required by changebar

% ----- for revisions! comment out if not needed-----%

%\usepackage{mdframed}
%\newmdenv[leftmargin=\dimexpr-0.5em-3pt, innerleftmargin=0.5em,
%          rightmargin=\dimexpr-0.5em-3pt, innerrightmargin=0.5em,
%          linewidth=3pt,linecolor=red, topline=false, bottomline=false,
%          innertopmargin=0pt,innerbottommargin=0pt,skipbelow=0pt,skipabove=0pt,
%         ]{notex}
%\newenvironment{revisions}
% {\par\vskip\dimexpr\dp\strutbox-\prevdepth\relax\notex\strut\ignorespaces}
% {\par\xdef\notetpd{\the\prevdepth}\endnotex\vskip-\notetpd\relax}
%\newcommand{\edit}[1]{\begin{revisions}\dotuline{\textcolor{red}{#1}}\end{revisions}}

%\newcommand\mrgn[1]{\-\marginpar[\textcolor{red}{$\|$}]{$\|$}\dotuline{\textcolor{red}{#1}}}


%% Comment out for final version
%\newcommand\edit[2][]{\-\marginpar[\textcolor{red}{$\|$}]{$\|$}\sout{\textcolor{red}{#1}}\dotuline{\textcolor{red}{#2}}}
%\newcommand\edittxt[1]{\dotuline{\textcolor{red}{#1}}}

%% Comment out for tracking changes
%\newcommand\edit[2][]{#2}
%\newcommand\edittxt[1]{#1}


%\newcommand\delete[1]{\-\marginpar[\textcolor{red}{$\|$}]{}\sout{\textcolor{red}{#1}}}
%\newcommand\replace[2]{\-\marginpar[\textcolor{red}{$\|$}]{}\sout{\textcolor{red}{#1}}\dotuline{\textcolor{red}{#2}}}
% track changes: \cbstart {changed text goes here} \cbend
%\usepackage[outerbars,color]{changebar}
%\cbcolor{red}

% FINAL VERSION: comment out the next two lines and uncomment the indicated ones further below
%\newcommand{\revision}[1]{\cbstart\textcolor{red}{#1} \cbend}
%\newcommand{\edit}[1]{\cbcolor{lightgray} \cbstart{\dotuline{#1}} \cbend}

% FINAL VERSION: now uncomment the next three lines so the change-bars and the colored text disappear
%\nochangebars
%\newcommand{\revision}[1]{\cbstart{#1} \cbend}
%\newcommand{\edit}[1]{\cbstart{#1} \cbend}

% ----- end of stuff for revisions! -----%


%\usepackage{pdfcomment}
%\newcommand{\question}[1]{\pdfmargincomment[icon=Help,color={red},author=Tricia:,open=true]{#1}}
%\newcommand{\mynote}[1]{\pdfmargincomment[color={yellow},author=Tricia:,open=true]{#1}}


\usepackage{url} % links in PDF
%--------------- code from web to make URLs smaller -----------------------%
%
%% Define a new 'leo' style for the package that will use a smaller font.
% use either ttfamily or rmfamly
\makeatletter
\def\url@leostyle{%
  \@ifundefined{selectfont}{\def\UrlFont{\sf}}{\def\UrlFont{\small\rmfamily}}}
\makeatother
\urlstyle{leo}
%% Now use the newly defined style.
%-------------------------------------------------------------------------------%
%\usepackage[linkcolor=BlueViolet,citecolor=BlueViolet,colorlinks=true,urlcolor=gray,pagebackref=false]{hyperref}
\usepackage{hyperref}

% now give some commands to hyperref set up our PDF bookmarks etc.
\hypersetup{
    bookmarks=true,         % show bookmarks bar?
    unicode=true,          % non-Latin characters in Acrobat’s bookmarks
    pdftoolbar=true,        % show Acrobat’s toolbar?
    pdfmenubar=true,        % show Acrobat’s menu?
    pdffitwindow=false,     % window fit to page when opened
    pdfstartview={FitH},    % fits the width of the page to the window
    pdftitle={The Trivalency of Voice},    % title
    pdfauthor={Itamar Kastner},     % author
    pdfsubject={},   % subject of the document
    pdfcreator={},   % creator of the document
    pdfproducer={}, % producer of the document
    pdfkeywords={}, % list of keywords
    pdfnewwindow=true,      % links in new window
    colorlinks=true,       % false: boxed links; true: colored links
    linkcolor=BlueViolet,          % color of internal links; I changed from red to blue to black
    citecolor=BlueViolet,        % color of links to bibliography; I changed from green to blue, and then to black
    filecolor=blue,      % color of file links; ; I changed from magenta to blue
    urlcolor=blue           % color of external links - I go back and forth between black and blue
}



% now's when we say \doublespacing
% for revisions, put the  \doublespacing command in the "pre" file (e.g., chapter1_pre) instead!!!
%\doublespacing

% If you have a little file with your own custom shortcuts in it, include here
% \include{my_shortcuts}

%-------end of all the packages -----------%

% this must be changed at some point
% Make sure the subsubsections are numbered and included in the TOC
\setcounter{secnumdepth}{5}
\setcounter{tocdepth}{4}

% Command for use with chapters; ensures example number persistent across chapters
%\newcounter{lastex}
%\newcommand{\newchapter}[2]{\setcounter{lastex}{\value{ExNo}}%
\newcommand{\newchapter}[2]{%\setcounter{lastex}{\value{ExNo}}%
                            \chapter{#2}%
                            \label{Chapter#1}%
%                            \setcounter{ExNo}{\value{lastex}}%
                            \input{#1}} 
%                            \input{c_#1}} %TI

%\newcommand{\appendixchapter}[2]{%
%   \chapter{#2}
%   \label{Appendix#1}%
%    \input{a_#1}}
%
%\newcommand{\appendixsection}[1]{%
%    \refstepcounter{section}%
%    \section*{\protect{\thesection}\hspace{2.6ex}#1}}

% From Jen's main
 %Create list of appendices, and don't include appendices in the table of contents
%\newlistof{appendixchapter}{apx}{List of Appendices}
%\newcommand{\appendixchapter}[2]{%
%    \refstepcounter{appendixchapter}%
%    \refstepcounter{chapter}%
%    \renewcommand{\DOTIS}[1]{\DOCH \DOTI{#2}}
%    \chapter*{#2}
%    \label{Appendix#1}%
%    \addcontentsline{apx}{appendixchapter}{Appendix \protect\numberline{\theappendixchapter}#2}\par%
%    \vspace {-1.47cm}
%    \input{a_#1}}
%\renewcommand{\theappendixchapter}{\Alph{appendixchapter}}
%\newcommand{\appendixsection}[1]{%
%    \refstepcounter{section}%
%    \section*{\protect{\thesection}\hspace{2.6ex}#1}}
%\newcommand{\appendixsubsection}[1]{%
%    \refstepcounter{subsection}%
%    \subsection*{\protect{\thesubsection}\hspace{2.5ex}#1}}
% End from Jen's main

%\setlength{\parindent}{.5in} % to indent paragraphs (doesn't do it on the first one after heading)
% \setlength{\parskip}{0.125in} % to skip a line between paragraphs, for blocks of text

%===============
% this crazy code removes a bit of the spacing before and after chapter headings
\makeatletter
\renewcommand*{\@makechapterhead}[1]{%
  \vspace*{10\p@}%
  {\parindent \z@ \raggedright \normalfont
    \ifnum \c@secnumdepth >\m@ne
      \if@mainmatter%%%%% Fix for frontmatter, mainmatter, and backmatter 040920
        \DOCH
      \fi
    \fi
    \interlinepenalty\@M
    \if@mainmatter%%%%% Fix for frontmatter, mainmatter, and backmatter 060424
      \DOTI{#1}%
    \else%
      \DOTIS{#1}%
    \fi
  }}
% For the case \chapter*:
\renewcommand*{\@makeschapterhead}[1]{%
  \vspace*{10\p@}%
  {\parindent \z@ \raggedright
    \normalfont
    \interlinepenalty\@M
    \DOTIS{#1}
    \vskip 10\p@
  }}
\makeatother


%%ITK
\makeatletter %Allow superscript ^ and subscript _
\catcode`_=\active%
\gdef_#1{\ensuremath{{}\sb{#1}}}%
\catcode`^=\active%
\gdef^#1{\ensuremath{{}\sp{#1}}}%

\renewcommand*\l@section{\ifnum\c@tocdepth>\z@\vskip 1pt plus 1pt minus 1pt \fi
                         \@dottedtocline{1}{1.5em}{2.3em}}
\makeatother



\newfontfamily{\H}[Scale=1,Script=Hebrew]{David CLM} %{Frank Ruehl CLM}%{Ezra SIL}
%====================

\begin{document}

    \frontmatter

    \pagestyle{empty}

     \singlespacing

    \singlespacing
\begin{center}
\rule{165pt}{0pt} \\
\vspace{1cm}
\LARGE{The Trivalency of Voice}\\
\vspace{2cm}
\large{Itamar Kastner} \\
\vspace{0.4cm}
\normalsize{Humboldt-Universit\"at zu Berlin} \\
\vspace{1cm}
\normalsize{May, 2018} \\
\vspace{4cm}
\hfill \large{For submission to Oxford University Press}\\

\end{center} % include title page
    \pagestyle{plain}

%\doublespacing

%    \chapter*{Dedication}
%    \addcontentsline{toc}{chapter}{Dedication}
%%    \input{f-thanks}
%	\begin{flushright}
\textit{
In remembrance of\\
Asi, Edit, Greg, Irit and Maskit,\\
who showed kindness.
}
\end{flushright}

%
%    \chapter*{Acknowledgments}
%    \addcontentsline{toc}{chapter}{Acknowledgments}
%	\addchap{\lsAcknowledgementTitle} 

committee: alec, maria, stephanie, michael, edit
edit
artemis and RUESHeL, the Research Unit on (Experimental) Syntax and Heritage Languages, at the Humboldt-Universit\"at zu Berlin
audiences at a number of venues, including USC, UCSC, UoT, CamVoice and the Travis Parameters workshop
artemis, giorgos, florian, yining and odelia for discussion
susi for her support of the project, elena for being the editor
based in part on work with Vera Zu, Matthew Tucker, Odelia Ahdout
thanks to daniel harbour for general support and for suggesting the title ``Hebrew D-Voicing''
thanks to dylan bumford for suggesting the title ``Modern Morphophonemics of Hebrew'', to James Whang and to Vera Zu
yining for comments on a draft
``However, one shouldn’t forget that I wrote a very long dissertation turned book in the early 1980’s that concerned the relationship between word formation and syntax. The work is titled, “On the Nature of Grammatical Relations,” because it is, in a sense, a paean to Relational Grammar.'' And mine is a paean to Doron and Arad.
%
%    \chapter*{Abstract}
%    \addcontentsline{toc}{chapter}{Abstract}
%    \input{f-abstract}

    \tableofcontents

% added and taken from Jen's main
%    \newpage
%\phantomsection \label{listoffig}
%     \addcontentsline{toc}{chapter}{List of Figures}
%    \listoffigures
%
%       \newpage
%\phantomsection \label{listoftab}
%    \addcontentsline{toc}{chapter}{List of Tables}
%    \listoftables

% Only 1 appendix so no need to have a List of Appendices
%      \newpage
%\phantomsection \label{listofapp}
% \addcontentsline{toc}{chapter}{List of Appendices}
%  \listofappendixchapter

% end of added stuff from Jen's main

    \mainmatter
    \pagestyle{plain}

    \chapter{Introduction}
\label{chap:intro}

The aim of this monograph is to present a new theory of argument structure alternations, one which is anchored in the syntax but has systematic interfaces with the phonology and the semantics. Conceptually, my goal is to argue for a specific formal system. Empirically, my goal is to provide the most comprehensive description and analysis of Hebrew verbal morphology to date, one whose formal assumptions are as similar as possible to those made in work on non-Semitic languages. Let's first see why Hebrew is interesting.

\section{Identifying the puzzles}
	\subsection{The two problems of Semitic morphology}
In the verbal system of Modern Hebrew, verbs appear in one of seven templates. These templates are the main object of study in this book. The most important thing to know about them is that they are easy to identify based on morphophonological form (although I provide glosses just in case), and that they often carry some kind of meaning. Pinning down the essence of ``often'' and ``some kind'' is my main analytical task.

Because our main theoretical interest is in argument structure alternations, we can start with that. The following examples demonstrate three different verbs, all sharing the same root which I notate \root{ktb}. In general, it can be seen that all verbs have to do with writing in some sense. The first is a simple transitive in the template {\tkal}:
\ex	\label{ex:intro-tkal}Transitive {\tkal}\\
		\begingl
		\gla ha-talmidim \textbf{katv-u} et ha-nosim//
		\glb the-students wrote-\gsc{PL} \gsc{ACC} the-topics//
		\glft `The students wrote the topics down.'//
	\endgl
\xe

The second is a non-active variant in {\tnif}; this is how we would express the anticausative or passive version of~(\lastx).
\ex \label{ex:intro-tnif}Non-active (mediopassive) {\tnif}\\
		\begingl
		\gla ha-xiburim \textbf{nixtev-u} (al-jedej ha-talmidim)//
		\glb the-essays were.written-\gsc{PL} by the-students//
		\glft `The essays were written (by the students)'.//
	\endgl
\xe

The third is a causative version in~{\thif}.
\ex \label{ex:intro-thif}Causative {\thif}\\
		\begingl
		\gla ha-mora \textbf{hextiv-a} (la-talmidim) et reʃimat ha-nosim//
		\glb the-teacher dictated-\gsc{3SG.F} to.the-students \gsc{ACC} list.of the-topics//
		\glft `The teacher dictated the list of topics (to the students).'//
	\endgl
\xe

If this is what the language looked like, the system would be far less puzzling. The analytical issues begin to mount when we understand that verbs in~{\tkal} are not always transitive like in~(\ref{ex:intro-tkal}). Verbs in {\tnif} are not always non-active like those in~(\ref{ex:intro-tnif}). And verbs in {\thif} are not always causative like those in~(\ref{ex:intro-thif}).
\pex
	\a Unaccusative in {\tkal}:\\
		\begingl
		\gla ha-bakbuk \textbf{kafa} ba-makpi//
		\glb the-bottle froze in.the-freezer//
		\glft `The bottle froze in the freezer.'//
	\endgl	
	\a Unergative in {\tnif}:\\
		\begingl
		\gla josi \textbf{nixnas} la-xeder be-bitaxon//
		\glb Yossi entered to.the-room in-security//
		\glft `Yossi confidently entered the room.'//
	\endgl
	\a Unergative in {\thif}:\\
		\begingl
		\gla marsel \textbf{heezin} be-savlanut//
		\glb Marcel listened in-patience//
		\glft `Marcel listened patiently.'//
	\endgl
\xe

On the other hand, it is crucial that there is some method to the madness. It is not the case that any template can be associated with any syntactic or semantic construction. Certain configurations---unaccusative, transitive, reflexive, etc.---are only possible with certain templates. This is \textbf{the first problem} of Semitic morphology: what syntactic structures and semantic readings is a given template associated with, and why?

Additionally, sometimes we can find alternations like in~(\ref{ex:intro-tkal})--(\ref{ex:intro-thif}). Certain templates alternate with some but not with others. \textbf{The second problem} of Semitic morphology is thus: what templates does a given template alternate with, and why?

I believe the answers to these questions can be found once we abandon the notion of a ``template'' as some kind of morphological primitive. I propose here a decomposition of the template into functional heads in the syntax, one that is able to address both problems above. What this means is that we need to engage with what alternations are and how argument structure comes about.

	\subsection{Argument structure}
Contemporary theories of argument often take as a starting point the ``anticausative alternation'', whereby a transitive verb (\textbf{causative}) and its intransitive equivalent (\textbf{anticausative}) stand in some morphologically mediated relationship. In some languages, such as English in~(\ref{ex:intro1}), the two verbs do not differ in their morphological marking. In other languages the predominant situation is one in which a reflexive pronoun appears in the anticausative variant, as in German, (\ref{ex:intro2}). And in other languages, the anticausative variant has specific non-active morphological marking. Some verbs in Greek are like this, (\ref{ex:intro3}). Other languages fall into one or more of these typological categories.
\pex\label{ex:intro1}
	\a Meg opened the door. \hfill (causative)
 	\a The door opened.			\hfill (anticausative)
\xe

\pex\label{ex:intro2}
	\a \begingl
		\gla\rightcomment{(causative)}Florian \"offnete die T\"ur.//
		\glb Florian opened the door//
		\glft `Florian opened the door.'//
	\endgl
	\a \begingl
		\gla\rightcomment{(anticausative)}Die T\"ur \"offnete \textbf{sich}.//
		\glb the door opened \gsc{REFL}//
		\glft `The door opened.'//
	\endgl
\xe	

\pex\label{ex:intro3}
	\a \begingl
		\gla\rightcomment{(causative)}o Giorgos ekapse ti supa//
		\glb the Giorgos burned the soup//
		\glft `Giorgos burned the soup.'//
	\endgl
	\a \begingl
		\gla\rightcomment{(anticausative)}i supa \textbf{kaike}//
		\glb the soup burned.\gsc{NACT}//
		\glft `The soup burned.'//
	\endgl
\xe

Various syntactic and semantic questions arise in connection with these seemingly simple patterns, many of which have been explored in influential studies such as \cite{haspelmath93}, \cite{unaccusativity95}, \cite{schaefer08}, \cite{koontzgarboden09} and \cite{layering15}: what kind of morphological marking appears on the different variants? Is there a sense in which one is derived from the other, or do the two share a common base? Which predicates are marked as anticausative crosslinguistically? 

The degree of variation both within and across languages is substantial. However, most studies on argument structure have analyzed this aspect of the syntax-semantics interface through the lens of a language with relatively impoverished morphology. Each of these languages has contributed much to our understanding of argument structure, to be sure: the English labile alternation shines light on which predicates are likely to be marked in which way \citep{haspelmath93,unaccusativity95,koontzgarboden09}; the French, German and Spanish alternations bring in many aspects of cliticization, binding and agreement \citep{labelle08,schaefer08,cuervo14}; the Greek alternation shows consistent morphological marking for at least one class of predicates \citep{alexiadoudoron12,layering15}; and more recent work on Icelandic has identified ways in which argument structure alternations can be correlated with morphological processes and \cite{wood14nllt,wood15springer}. Yet this line of work has the drawback that these languages usually show only binary morphological distinctions, if any: either the causative variant is marked, or the anticausative one is marked (or neither is, as in the labile alternation). This problem also persists with some larger-scale typological surveys \citep{haspelmath93,arad05}.

	\subsection{Solving the two problems}
The intuition guiding my analysis is that of \cite{schaefer08}, \cite{layering15} and related work: the alternations are not alternations at all. The grammar does not derive causative forms form inchoative ones, or anticausative forms from transitive ones. Rather, what happens is that both readings are derived from one basic structure (technically a vP) with a causative component in the semantics. If we add an external argument, we get a transitive/causative verb; if we do not, we simply retain the basic event and have an anticausative verb on our hands.

This book provides a way of implementing the same idea in Hebrew. Now, I am by no means not the first to suggest that the templates be decomposed. Maya Arad and Edit Doron have both made seminal contributions to our understanding of these issues. But \cite{arad05} was torn between the need to acknowledge the idiosyncrasies of the system, on the one hand, and the need to encode the alternations, on the other hand. As a result, that theory had to implement conjugation classes in order to adequately describe which alternations exist. \cite{doron03} sidestepped the issue by providing a compositional semantics for the components making up the templates, but the result was that alternations could only be discussed in terms of their semantics, and not their morphology or syntax. What I propose is a way to get the alternations from contemporary syntactic assumptions.

The two problems are addressed as follows. By building up specific syntactic structures we are able to easily explain what syntactic configurations and semantic interpretations arise for a given structure, as well as how this structure is spelled out; that spell-out is what we call the template. Instead of figuring out the many-to-many mapping between form and meaning, I map one structure deterministically to form and to meaning, thereby solving the first problem. And by adopting the idea that a core vP carries the basic meaning of a verb, we can then layer additional heads above it, regulating the introduction of an external argument. The majority of work is carried out by the head Voice, which introduces external argument. This solves the second problem. A technical innovation lies with the syntactic feature [$\pm$D] that Voice might carry, hence the valence of Voice. But we will get to that soon enough.

Granted, there is also a third problem: how can we tell which meaning is licensed by which root? That question deserves a monograph of its own, though I will try to flag ways in which it can be approached throughout the book.

Part I of this book is comprised of case studies of the different templates, which together come to form the theory of trivalent Voice. Part II consists of two chapters situating this theory within contemporary theoretical debates.

This introductory chapter is structured as follows. I give a general overview of Hebrew morphology in Section~\ref{intro:basic}, including a brief account of what the traditional view is. Section~\ref{intro:arch} introduces the formal assumptions of my theory, which itself is outlined in Section~\ref{intro:sketch}.


\section{Traditional descriptions and basic generalizations} \label{intro:basic}
The first thing to note about Hebrew is that not \emph{all} morphology is non-concatenative. Agreement, for example, may consist of prefixes and suffixes, alongside non-concatenative changes to the stem. The future tense paradigm for the verb \emph{katav} `wrote' in {\tkal} is given in~(\nextx). The stem vowel is either /o/ or /e/, depending on whether the verb is suffixed or not, but other than that all of the agreement information is affixal.
\ex
\begin{tabular}{lll}
Person/Gender	& \gsc{SG}	& \gsc{PL}\\\hline
1				&\textbf{e-}xtov				&\textbf{ni-}xtov\\
2\gsc{M}		&\textbf{ti-}xtov				&\textbf{ti-}xtev\textbf{-u}\\
2\gsc{F}		&\textbf{ti-}xtev\textbf{-i}	&\textbf{ti-}xtev\textbf{-u}\\
3\gsc{M}		&\textbf{ji-}xtov				&\textbf{ji-}xtev\textbf{-u}\\
3\gsc{F}		&\textbf{ti-}xtov				&\textbf{ji-}xtev\textbf{-u}\\
\end{tabular}
\xe
I don't concern myself here with this distinction directly since my main interest is within the thematic domain, i.e. VoiceP. In general, it is not surprising that syntactic material from a certain height and ``upwards'' in the tree is spelled out affixally rather than non-concatenatively; see \cite{harbour08} and \cite{kastnertucker19cup} for further discussion of this cross-Semitic point.

Nevertheless, linguists and non-specialists alike often find themselves scratching their heads in an attempt to come to terms with Semitic's distinctive morphological system, built around ``roots'' and ``patterns''. Many early speakers of Modern Hebrew were such head-scratchers themselves: the language was revived in the late 19th century by individuals who, for the most part, were not native speakers of Semitic languages. The language nevertheless retained the Semitic morphology of its classical predecessor.
%On the surface, Hebrew is very different from European languages, or perhaps from any non-Semitic language. The question of how languages differ from one another is a familiar one from work in the generative tradition which often turns the question on its head, asking how languages are fundamentally similar.
Given that this book is a study of the verbal system of Hebrew, I will make repeated reference to ``roots'' and ``templates'' (the latter also called ``patterns'', ``measures'', ``forms'' and \emph{binyanim}) as the two main components of the verb. I reserve the terms ``templates'' for the systematic verbal forms and ``patterns'' for the systematic nominal and adjectival forms. These traditional terms have been used, as far as I know, for as long as the verbal systems of Hebrew and other Semitic languages have been documented. \cite{ussishkin00phd} mentions a number of works on Hebrew which use roots and templates as integral parts of the system, including \cite{gesenius}---perhaps the best-regarded grammar of Biblical Hebrew---as well as \cite{bopp1824}, \cite{ewald1827}, \cite{harris41} and \cite{chomsky51}. For Arabic, he mentions \cite{desacy1810} as one example among many of older works which make direct reference to roots and templates.

The nature of the root was already debated by the traditional Arabic grammarians of Basra and Kufa in the 8th Century, according to \citet[563ff]{borer13oup} who herself cites \cite{owens88}. Turning to more recent works, we can add foundational contributions by \cite{rosen77}, \cite{berman78}, \cite{bolozky78,bolozky99} and \cite{ravid90}, all relying on the root and the template as descriptive notions. I cannot hope to do justice here to the vast modern-day literature on Modern Hebrew, much of which has been published in Hebrew. The interested reader may want to consult the works of Yehoshua Blau, Reuven Mirkin, Uzzi Ornan and Haim Ros\'en, among others.
%; the latter two, in particular, have authored work that may be more accessible to generative linguists.
%In \S\ref{sec:jjmcc} I discuss contemporary work on Semitic morphology in the generative tradition; for now, we simply establish that roots and patterns have been invoked throughout the ages in descriptions of the Semitic system. One goal of this dissertation is to evaluate the theoretical status of these notions for Semitic and for natural language in general.

To see how the system is traditionally conceived of, let us consider first form, then meaning. The verbs in~(\lastx) are all given in the 3rd person masculine singular past tense -- the citation form. The actual conjugation of a given form across tenses and person/number/gender features is completely predictable, as~(\nextx) exemplifies for the {\tpie} template (barring the kind of lexical idiosyncrasies investigated in \citealt{kastner18nllt}). That is to say, even though the meaning of a given verb cannot be immediately guessed in its entirety, the morphophonological form is predictable. Note again how agreement material is mostly affixal.
\ex \label{table:piel}Tense and agreement marking in \tpie.\\
%\begin{small}
	\begin{tabular}{|l||l|l||l|l||l|l|} \hline
		& \multicolumn{2}{c||}{Past} & \multicolumn{2}{c||}{Present} &  \multicolumn{2}{c|}{Future} \\
		& \gsc{M} & \gsc{F} & \gsc{M} & \gsc{F} & \gsc{M} & \gsc{F} \\\hline\hline
		1\gsc{SG} & \multicolumn{2}{c||}{XiY̯aZ-ti} & me-XaY̯eZ & me-XaY̯eZ-et & \multicolumn{2}{c|}{je-XaY̯eZ}\\\hline
		1\gsc{PL} & \multicolumn{2}{c||}{XiY̯aZ-nu} & me-XaY̯Z-im & me-XaY̯Z-ot & \multicolumn{2}{c|}{ne-XaY̯eZ}  \\\hline\hline
		2\gsc{SG} & XiY̯aZ-ta & XiY̯aZ-t & me-XaY̯eZ & me-XaY̯eZ-et & te-XaY̯eZ & te-XaY̯Z-i\\\hline
%		2\gsc{PL} & XiY̯aZ-tem & XiY̯aZ-ten/m & me-XaY̯Z-im & me-XaY̯Z-ot & \multicolumn{2}{l|}{te-XaY̯Z-u}\\\hline\hline
		2\gsc{PL} & \multicolumn{2}{c||}{XiY̯aZ-tem} & me-XaY̯Z-im & me-XaY̯Z-ot & \multicolumn{2}{c|}{te-XaY̯Z-u}\\\hline\hline
		3\gsc{SG} & XiY̯eZ & XiY̯Z-a & me-XaY̯eZ & me-XaY̯eZ-et & je-XaY̯eZ & te-XaY̯eZ\\\hline
		3\gsc{PL} & \multicolumn{2}{c||}{XiY̯Z-u} & me-XaY̯Z-im & me-XaY̯Z-ot & \multicolumn{2}{c|}{je-XaY̯Z-u}\\\hline
	\end{tabular}
%\end{small}
\xe

For meaning, we may take as a starting point the essay by \cite{schwarzwald81} and its ``traditional classification'' of template meanings. I have added examples of the alternations to this classification.
%{``\dgs{Y}'' marks a non-spirantized consonant, \S\ref{sec:data:notation}.}
\ex A na\"ive classification of Hebrew templates \citep[131]{schwarzwald81}:\\
	\begin{tabular}{lccp{0.0cm}llll}
		& \textbf{Active} & \textbf{Passive} & && & & \\
	\textbf{Simple} & \tkal & \tnif & & \root{sgr} & \emph{sagar} & \emph{nisgar} & `closed'\\
	\textbf{Intensive} & \tpie & \tpua & & \root{tpl} & \emph{tipel} & \emph{tupal} & `treated'\\
	\textbf{Causative} & \thif & \thuf & & \root{kns} & \emph{hexnis} & \emph{huxnas} & `inserted' \\
	\textbf{Reflexive or reciprocal} & \multicolumn{2}{c}{\thit} & & \root{xb\dgs{k}} & \multicolumn{2}{c}{\emph{hitxabek}} & `hugged' \\
	\end{tabular}	
\xe
As \citeauthor{schwarzwald81} immediately points out herself, this classification is misleading. The relationships between the templates (the argument structure alternations) are not always predictable and most templates have additional meanings beyond those listed in~(\lastx). For example, there is little way to predict what the root \root{rʃm}, which has to do with writing down, will mean when it is instantiated in a given template. In the ``simple'' template {\tkal} we substitute the consonants in \root{rʃm} for X, Y and Z and derive \emph{raʃam} `wrote down'. In the ``middle'' template \tnif, \emph{nirʃam le-} means `signed up for', against the characterization of \tnif~as ``simple passive'' in~(\lastx). In the ``intensive middle'' \thit, \emph{hitraʃem me-} means `was impressed by', challenging the characterization of \thit~as ``reflexive or reciprocal'' in~(\lastx). 

The only cells of the table which are completely predictable are the two passive templates {\tpua} (``intensive passive'') and {\thuf} (``causative passive''). The other templates constrain the possible meaning in ways that have eluded precise specification. For example, while it is clear that many verbs in {\tnif} are passive-like, not all verbs in that template are. This gives us the two basic questions that need to be addressed, mentioned at the outset:
\begin{itemize*}
	\item What are the possible readings associated with a given template (and why)?
	\item What other templates does a given template alternate with (and why)?
\end{itemize*}

In Part I of the book we will see that the syntax and semantics of the system can nevertheless be analyzed within a constrained theory of morphosyntax. I will make precise what the unique contribution of each template is and how that contribution comes about in the syntax. We will then be able to identify the role of the root in selecting between different possible meanings for the verb in a given template.


\subsection{Transliteration and notation} \label{sec:data:notation}
I use the variables X, Y and Z for the tri-consonantal root: \root{XYZ}. This monograph contains little discussion of roots with more than three consonants, but nothing in the notation hinges on it. The list in of roots extracted from the database of \cite{ehrenfeld12} contains 311 quadrilateral roots and three quintilateral roots\footnote{\root{xntrʃ} `bullshit', \root{snxrn} `synchronize' and \root{flrtt} `flirt'.} out of 1876 roots in total.

In the Hebrew glosses, \gsc{ACC} is used for the direct object marker \emph{et} and \gsc{CS} for the head of a Construct State nominal.

As will be discussed in Chapter~\ref{voice:tpie}, Hebrew has a fairly productive process of postvocalic spirantization applying to /b/, /k/ and /p/, turning them into [v], [x] and [f] respectively. This process is blocked in certain verbal templates; to note this blocking I borrow the non-syllabicity diacritic and place it under the medial root consonant: ``\dgs{Y}''. This notation can be found in the templates \tpie~and \thit, in which this blocking holds. The same notation is used for segments which never spirantize: ``\dgs{k}''.

Transcriptions are given using the International Phonetic Alphabet with the following modifications:
\begin{itemize}
	\item ``e'' stands for /ɛ/ and /ə/.
	\item ``g'' stands for /ɡ/.
	\item ``o'' stands for /ɔ/.
	\item ``r'' stands for /ʁ/.
	\item ``x'' stands for /χ/.
	\item The apostrophe ' stands for the glottal stop. %, although I usually leave it out omitted since the glottal stop is often dropped in contemporary speech.
\end{itemize}
These changes were made purely for reasons of convenience. The syntactic literature has often used ``\v{s}'' or ``S'' for /ʃ/ and ``c'' for /ts/. In both cases I preferred to retain the IPA transcription, ``ʃ'' and ``{\ts}''. Stress is marked with an acute accent when necessary, ``\'a''. Deleted vowels are enclosed in angle brackets, ``$<>$''. \underline{Underlining} and \textbf{boldface} are used only for emphasis, never as diacritics or notation.

My notation contains various deviations from standard forms; these will probably only be of interest to readers already familiar with the language.

The template {\thif} usually appears in the literature as \emph{h\textbf{i}XYiZ}, with an /i/-/i/ vocalic pattern. Yet contemporary speakers use /ɛ/ \citep{trachtman16}, and so I transcribe ``e'' throughout. Conversely, the initial /h/ is usually dropped in speech but I retain it for two reasons. First, /h/ is still pronounced by some older speakers and certain sociolinguistic groups, often marginalized ones \citep[cf.~][]{schwarzwald81biu,gafter14phd}. And second, the initial \emph{h}- should help non-Semitist readers to distinguish this template from other ones.

Glottal stops are often dropped in speech \citep{enguehardfaust18}. I usually omit them, but at times retain an apostrophe in order to distinguish between otherwise homophonous forms, for example \emph{hefria} `he disturbed' $\sim$ \emph{hefri\underline{'}a} `she disturbed'.

When presenting verbal paradigms I include two substandard forms. The first person singular future is normally prefixed with a low vowel, e.g.~\emph{a-daber} `I will talk' (in \root{dbr}). Contemporary usage, however, syncretizes the first person singular future with the third person masculine singular future: \emph{je-daber} `I/he will talk'. I include both forms when giving paradigms. And finally, contemporary usage does not distinguish between masculine and feminine plural forms in past and future tense verbs. The traditional feminine plural endings have been discarded, syncretizing instead with the masculine plural forms.

When reproducing examples from the literature I have modified the original transcriptions for consistency. With this housekeeping out of the way, we return to the theoretical approach.


@@@
\section{Architectural assumptions} \label{intro:arch}

	\subsection{The syntax}

I assume DM

		\subsubsection{What is Voice?} \label{intro:arch:voice}
In the current neo-Davidsonian tradition, theories of argument structure have adopted a specific way of thinking about internal and external arguments in the syntax, based in large part on the interpretation asymmetries observed by \cite{marantz84} and discussed by \cite{kratzer96}. The theme or patient of the predicate is generated within the VP as the complement of V.\footnote{whether or not internal arguments end up in Spec,VP as in various approaches is immaterial here \citep{johnson91,alexiadouschaefer11wccfl}.} The agent is introduced in the specifier of a higher functional head, which takes the VP as its own complement. Since \cite{kratzer96} it has become common to call this head Voice and to associate it with accusative case licensing, thereby identifying it with causative ``little v'' of \cite{chomsky95}. The basics are given in~(\nextx), slightly modifying \citet[121]{kratzer96}. The relevant compositional functions invoked here are Functional Application and Event Identification. We will also make use of Predicate Modification later on in this book; see \cite{wood15springer} for an accessible introduction. I leave out the semantic types of the arguments.
\pex
	\a Mittie fed the dog.
	\a \Tree
	[.VoiceP\\{λe.Agent(Mittie, e) \& feed(the dog, e)}\\{\textsf{(by Functional Application})}
		[.DP\\\emph{Mittie} ]
		[.{λxλe.Agent(x,e) \& feed(the dog, e)}\\{\textsf{(by Event Identification)}}
			[.Voice\\{λxλe.Agent(x,e)} ]
			[.vP\\{λe.feed(the dog, e)}\\{\textsf{(by Functional Application)}}
				[.v\\{λxλe.feed(x,e)}
					[.\root{\gsc{FEED}} ]
					[.v ]
				]
				[.DP\\\emph{the dog} ]
			]
		]
	]
\xe

I would like to focus on two important points as a segue into the current theory. First, this original formulation does not make any claims regarding a structural difference between agents and causers (e.g.~circumstances, inanimate objects or natural forces). While there have been some attempts to draw a structural difference between the two---at least for certain psychological predicates \citep{bellettirizzi88,harleystone13}---I join the majority of work on argument structure in making no claims to that extent \citep{layering15}. For me, agents are a subset of causers, but this difference is semantic, not syntactic. What this means is that an external argument position (Spec,VoiceP) should be compatible with both agents and causers, but some additional element could force only a narrower, agentive reading. This is we will see already in Section~\ref{voice:va}. I will try to be clear about 

The second point is that as a functional head, Voice might be endowed with different features. In principle, since it licenses a DP in its specifier, it should have the EPP feature [D] \citep{chomsky95}. Once we accept that it has that feature, we can begin to ask what other features it can have, and might these features get checked in the course of the derivation. As recapped in the previous chapter, much recent work in argument structure has explored the possible values of the [$\pm$D] feature on Voice; recent approaches are discussed directly in Chapters \ref{chap:aas} and \ref{i:nie}, once the current theory has been developed in depth. To begin, I will now explain what it means for Voice to be underspecified for this feature.


		\subsubsection{Layering} \label{intro:arch:layering}
A recurring question in discussions of argument structure regards the direction of derivation in (anti-)causative alternations. For an alternation like~(\nextx), is the transitive version derived from the intransitive one via causativization or is the intransitive variant derived from the transitive one via anticausativization?
\pex
	\a John broke the vase.
	\a The vase broke.
\xe

\cite{layering15} summarize a number of reasons for thinking that neither answer is strictly speaking true. They propose that both variants have the same base: a minimal vP, (\nextx a) containing the verb (a verbalized root) and the internal argument. The difference between the two variants is that the transitive one, (\nextx b), then has the external argument added by additional functional material (Voice).
\ex
a. 
\Tree
		[.vP
			[.\emph{broke} ]
			[.\emph{the glass} ]
		]
b. \Tree
[.VoiceP
	[.\emph{John} ]
	[.
		[.Voice ]
		[.vP
			[.\emph{broke} ]
			[.\emph{the glass} ]
		]
	]
]
\xe

This view explains a range of facts about this alternation, chiefly that there is no dedicated direction of derivation which is marked by the morphology across languages. That is, while some languages mark the transitive variants, others mark the intransitive variants, and sometimes both variants are marked in the same language (as we have already seen for Hebrew). Even though there is much to say about which verbs or roots are marked in which way \citep{haspelmath93,unaccusativity95,arad05}, the grammar itself does not force derivation from one stem type to the other.

In addition to the morphological reasoning, \cite{layering15} provide a series of arguments showing that the core causative component of the vP is present even in the anticausative variants. For example, the underlined cause PPs in~(\nextx) are possible even with anticausatives, indicating that an event of causation can take place even without an external argument \citep{alexiadouetal06,alexiadouetal06nels}.
\pex
	\a The flowers wilted \underline{from the heat}.
	\a The window cracked \underline{from the pressure}.
\xe

In sum, while there is a causative base, an actual causer can only be introduced by additional structure; either in a cause-PP, or as an external argument in a higher projection; an additional layer, so to speak. Voice is the functional head enabling this layer, both in terms of licensing Spec,VoiceP in the syntax and in opening the thematic predicate agent (abstracting away here from the distinction between agents and causers, on which see Chapter~\ref{chap:intro} and \citealt[7]{layering15}). The causative alternation in English can be easily explained in these terms.


	\subsection{Interfaces}
		\subsubsection{Roots}		
		
		\subsubsection{Contextual allomorphy}

		\subsubsection{Contextual allosemy}


		


\section{Sketch of the system} \label{intro:sketch}
@@all templates can be described along two axes: the range of interpretations they are compatible with, and the canonical alternations they participate in. The first is mainly accomplished using the features on Voice. The second is accomplished using hierarchical syntactic structure@@



	\subsection{Hebrew verbal morphology}
In this book, I examine a \textbf{three-way distinction} in the marking of argument structure alternations. The empirical focus here is on the verbal system of Modern Hebrew. This morphological system is an ideal testing ground for theories of argument structure for two reasons. The first is that verb ``triplets'' can be found in which a given root clearly has three different kinds of morphological marking, as with \root{ktb} `\root{\gsc{WRITE}}' in~(\nextx). These morphological classes or \emph{templates} are traditionally described by making reference to the vowels and affixes which combine with the root consonants X, Y and Z. Hebrew has seven such templates in total, three of which can be seen in~(\nextx). In this work, ``template'' is meant purely as a descriptive term.
\pex\label{ex:general}
	\a Anticausative/mediopassive verb in {\tnif}:\\
		\begingl
		\gla ha-xiburim \anticaus{nixtev-u} (al-jedej ha-talmidim)//
		\glb the-essays were.written-\gsc{PL} by the-students//
		\glft `The essays were written (by the students)'.//
	\endgl
	
	\a Transitive verb in {\tkal}:\\
		\begingl
		\gla ha-talmidim \textbf{katv-u} et ha-nosim//
		\glb the-students wrote-\gsc{PL} \gsc{ACC} the-topics//
		\glft `The students wrote the topics down.'//
	\endgl

	\a Causative verb in {\thif}:\\
		\begingl
		\gla fabien \caus{hextiv-a} (la-talmidim) et reʃimat ha-nosim//
		\glb Fabienne dictated-\gsc{F} to.the-students \gsc{ACC} list.of the-topics//
		\glft `Fabienne dictated the list of topics (to the students).'//
	\endgl	
\xe

Importantly, while verbs in {\tnif} are non-active, (\lastx a), and those in {\thif} like~(\lastx c) are active (barring exceptions I return to below), verbs in {\tkal} are underspecified with regards to their argument structure, (\lastx b): with some roots, the verb might be transitive; with others, unergative; and with others still, unaccusative, (\nextx).
\pex\label{ex:kal}
	\a Transitive {\tkal}:\\
	\begingl
		\gla teo \textbf{axal} et ha-laxmanja//
		\glb Theo ate \gsc{ACC} the-bread.roll//
		\glft `Theo ate the bread roll.'//
	\endgl
%	\a \begingl
%		\gla ha-balʃan \textbf{katav} et ha-maamar ha-arox//
%		\glb the-linguist wrote \gsc{ACC} the-article the-long//
%		\glft `The linguist wrote the long article.'//
%	\endgl

	\a Unergative {\tkal}:\\
	\begingl
		\gla teo \textbf{rakad} ve-rakad ve-rakad (kol ha-boker)//
		\glb Theo danced and-danced and-danced all the-morning//
		\glft `Theo danced and danced and danced (all morning long).'//
	\endgl
%	\a \begingl
%		\gla teo \textbf{halax} kol ha-boker//
%		\glb Theo walked all the-morning//
%		\glft `Theo walked all morning long.'//
%	\endgl

	\a Unaccusative {\tkal}:\\
	\begingl
		\gla \textbf{nafal} le-teo ha-bakbuk//
		\glb fell to-Theo the-bottle//
		\glft `Theo's bottle fell.'//
	\endgl
	
%	\a \begingl
%		\gla ha-bakbuk \textbf{kafa} ba-makpi//
%		\glb the-bottle froze in.the-freezer//
%		\glft `The bottle froze in the freezer.'//
%	\endgl
\xe

%\ex\label{ex:alternations-heb}
%	\begin{tabular}{llllll}
%	\multicolumn{2}{c}{anticausative}	&	\multicolumn{2}{c}{transitive}	& \multicolumn{2}{c}{causative}\\
%	\multicolumn{2}{c}{\tnif}	&	\multicolumn{2}{c}{\tkal}	& \multicolumn{2}{c}{\thif}\\
%	\emph{neexal}	& `was eaten'	&	\emph{axal}	& `ate'	&	\emph{heexil}	& `fed'\\
%	\end{tabular}
%\xe

What these patterns show is that Hebrew has active marking, non-active marking and underspecified (non-)marking. As a consequence, we must move beyond purely binary distinctions of the ``causative/anticausative'' kind. The book develops the necessary trivalent system.

The second reason for investigating Hebrew is that this three-way distinction is instructive but not deterministic. A given syntactic configuration does not always entail a given template, and a given template does not always entail a given syntactic configuration. Work on Hebrew morphology, be it traditional or contemporary, has consistently found itself faced with two issues:
\begin{itemize*}
	\item What are the possible readings associated with a given template (and why)?
	\item What other templates does a given template alternate with (and why)?
\end{itemize*}

I aim to provide the most complete answers to these questions to date. We have already seen that transitive verbs exist both in {\tkal} and {\thif}, and that unaccusatives exist both in {\tkal} and in {\tnif}. Transitive verbs also exist in {\tpie}, (\nextx a), and anticausatives also exist in {\thit}, (\nextx b).
\ex\label{ex:counter1}
\raisebox{-0.6em}{
	\begin{tabular}{lllllll}
	a.& Transitive in & \tpie & \emph{biʃel} & `cooked'	& (not \tkal 		 & *\emph{baʃal})\\
%	b.& Anticausative in & \tkal & instead of \tnif	& \emph{nafal}	& `fell'		& * \emph{ninfal}\\
	b.& Anticausative in & \thit & \emph{hitparek} & `fell apart' & (not \tnif & *\emph{nifrak})\\
	\end{tabular}
}
\xe

But conversely,  a given template does not always entail a given syntactic configuration. Even {\tnif} appears on some unergatives, (\nextx a), and {\thit} instantiates not only anticausatives as in~(\lastx b) but also reflexives as in~(\nextx b).
\ex\label{ex:counter2}
\raisebox{-0.6em}{
	\begin{tabular}{llllll}
	a.& Unergative in & \tnif & \emph{nilxam} & `fought' & (not anticausative) \\
	b.&	Reflexive in & \thit & \emph{hitgaleax} & `shaved' & (not anticausative) \\
	\end{tabular}
}
\xe

These ``irregularities'' have occupied Semitist grammarians \citep[e.g.][]{gesenius} as well as modern-day authors \citep[e.g.][]{rosen77,schwarzwald81,bolozky82,doron03,arad05,aronoff07,borer13oup,kastner16phd} for as long as the language has been studied. On the one hand, it has long been acknowledged that roots show idiosyncratic behavior: broadly speaking, some roots prefer to form transitive verbs in {\tkal} and others in {\tpie}, as seen in~(\ref{ex:counter1}a). But on the other hand, it has proven much harder to accurately delimit the degree to which the templates are associated with distinct syntactic structures and semantic interpretations. When the language is viewed as a whole, the patterns in~(\ref{ex:general}) are clearly visible, but so are the exceptions.


	\subsection{The proposal}
The proposal put forward in this book showcases a novel analysis of the Hebrew patterns which captures existing work on European languages equally well. Working within Minimalist syntax \citep{chomsky95} and Distributed Morphology \citep{dm}, I rely on the idea that the external argument is introduced by the functional head Voice \citep{kratzer96,pylkkanen08,woodmarantz17}. This head might itself be endowed with syntactic features \citep{schaefer08,wood15springer}, leading me to defend the following claim:
\pex \textbf{The trivalency of Voice}
	\a Voice is associated with a [$\pm$D] feature, meaning it can be valued as [+D], [--D] or unspecified with regards to [D].\footnote{A similar view of binary features as trivalent is espoused by \cite{harbour11}.}
	\a This feature indicates whether the specifier of Voice must be filled by a DP ([+D]), cannot be filled by a DP ([--D]), or is agnostic as to whether it is filled by a DP (unspecified).
\xe

Importantly, these Voice differ in their phonological form. At its simplest, the system straightforwardly derives the alternations seen in~(\ref{ex:general})--(\ref{ex:kal}) as follows, where rows~(\nextx a--b) give examples of transitive verbs in {\tkal} and row~(\nextx c) shows an unaccusative in {\tkal}.
\ex\label{ex:alternations-heb2}
\raisebox{-3em}{
	\begin{tabular}{ll|ll|ll|ll}
	 & & \multicolumn{2}{P{4.7cm}|}{\textbf{\vd}}	&	\multicolumn{2}{P{4cm}|}{\textbf{Voice}}	& \multicolumn{2}{P{4cm}}{\textbf{\vz}}\\
%	\phantom{Semantics} & \multicolumn{2}{c|}{causative} &	\multicolumn{2}{c|}{transitive}	& \multicolumn{2}{c}{anticausative}\\\cline{2-7}
	& & \multicolumn{2}{c|}{\thif}	&	\multicolumn{2}{c|}{\tkal}	& \multicolumn{2}{c}{\tnif}\\\hline
	a.& \root{ktb} & \emph{hextiv}	& `dictated' &	\emph{katav}	& `wrote'	&	\emph{nixtav}	& `was written' \\
	b.& \root{'xl} & \emph{heexil}	& `fed' &	\emph{axal}	& `ate'	&	\emph{neexal}	& `was eaten' \\\hdashline
	c.& \root{nfl} & \emph{hepil} & `dropped' & \emph{nafal}	& `fell' & \multicolumn{2}{c}{---}\\
	\end{tabular}
}
%	\begin{tabular}{llllll}
%	\multicolumn{2}{c}{anticausative}	&	\multicolumn{2}{c}{transitive}	& \multicolumn{2}{c}{causative}\\
%	\multicolumn{2}{c}{\tnif}	&	\multicolumn{2}{c}{\tkal}	& \multicolumn{2}{c}{\thif}\\
%	\multicolumn{2}{c}{\textbf{\vz}}	&	\multicolumn{2}{c}{\textbf{Voice}}	& \multicolumn{2}{c}{\textbf{\vd}}\\
%	\emph{neexal}	& `was eaten'	&	\emph{axal}	& `ate'	&	\emph{heexil}	& `fed'\\
%	\end{tabular}
\xe

On this analysis, verbs in {\thif} are expected to be transitive or unergative because they require an external argument, verbs in {\tnif} are expected to be mediopassive (anticausative or passive) because they lack an external argument syntactically, and verbs in {\tkal} could go either way, depending on the idiosyncratic requirements of the root. The three values of Voice correspond to different morphological markings, but there is more than one way to get e.g.~an anticausative verb (namely with Voice or {\vz}). This way of looking at things dissolves the puzzle posed by examples like those in~(\ref{ex:counter1}) above: there is no reason to expect ``transitive'' to map onto specific morphology deterministically.\footnote{The two templates {\tpie} and {\thit} are derived by using an additional adverbial modifier, {\va} \citep{doron03,kastner17gjgl,sundaresanmcfadden17}, which is discussed in detail in Chapters 2 and 3.} Rather, syntax is autonomous in being able to build up structures which have different building blocks but may end up being interpreted similarly \citep{wood15springer,woodmarantz17,myler16mit}.

What still remains to be explained is the alternations. These can be shown to follow from the architectural assumptions: following \cite{kratzer96} and \cite{layering15}, it has become fairly common to assume that a core vP contains a causative component which is semantically available even in anticausatives. Voice can then add an external argument (an agent), but otherwise the vP already has a basic meaning. Accordingly, we can combine the root \root{ktb}, the verbalizer v and an internal argument. This vP gives us a basic event of writing something:
\ex
\Tree
	[.vP
		[.v
			[.\root{ktb} ]
			[.v ]
		]
		[.DP ]
	]
\xe

The combinatorics are now simple. If we merge {\vz}, no external argument is added and we have a simple anticausative. If we merge Voice, an external argument is added and we get the causative variant: an event of writing something with an agent doing the writing. And if we merge {\vd}, we will need to specify a different kind of external argument (how this happens is explored in Chapter 4).

Looking at things this way means we can abandon the notion of a ``template'' as a morphological template, and no longer need to explain why e.g.~{\tnif} alternates with {\tkal} in some cases but with {\thif} in others. Rather, it is simply a question of what the syntax can build, together with some idiosyncratic licensing properties of individual roots. This analysis is shown to be superior to that of the other major contributions in the literature \citep{doron03,arad05}.

The first part of the book (Chapters 2--5) develops this analysis based on Hebrew, with special attention to the functional heads and the features they may have. The second part of the book (Chapters 6--7) revisits existing work on argument structure alternations in light of the current system, drawing parallels and sketching the boundaries of crosslinguistic variation.




The current study takes as its empirical focus the verbal system of Modern Hebrew. This morphological system is an ideal testing ground for theories of argument for two reasons. The first is that triplets can be found in which a given root clearly has three different kinds of morphological marking, as with \root{axl} `\root{\gsc{EAT}}' in~(\nextx). As a consequence, we must move beyond binary distinctions (whether in the syntax or the morphology). I will refer to these morpho-phonological patterns as \emph{templates}. They are traditionally described by making reference to the vowels and affixes which combine with the root consonants X, Y and Z. Hebrew has seven such templates, three of which can be seen in~(\nextx).
\ex\label{ex:alternations-heb}
	\begin{tabular}{llllll}
	\multicolumn{2}{c}{anticausative}	&	\multicolumn{2}{c}{transitive}	& \multicolumn{2}{c}{causative}\\
	\multicolumn{2}{c}{\tnif}	&	\multicolumn{2}{c}{\tkal}	& \multicolumn{2}{c}{\thif}\\
	\emph{neexal}	& `was eaten'	&	\emph{axal}	& `ate'	&	\emph{heexil}	& `fed'\\
	\end{tabular}
\xe

The second reason for investigating Hebrew is that this three-way distinction is instructive but not deterministic. A given syntactic configuration does not always entail a given template, (\nextx), and a given template does not always entail a given syntactic configuration, (\anextx).
\ex\label{ex:counter1}
	\begin{tabular}{lllllll}
	a.& Transitive in & \tpie & instead of \tkal 		& \emph{biʃel} & `cooked'	& * \emph{baʃal}\\
	b.& Anticausative in & \tkal & instead of \tnif	& \emph{nafal}	& `fell'		& * \emph{ninfal}\\
	c.& Anticausative in & \thit & instead  of \tnif & \emph{hitparek} & `fell apart'	& * \emph{nifrak}\\
	\end{tabular}
\xe
\pex\label{ex:counter2}
	\begin{tabular}{lllllll}
	a.& Unergative in & \tnif & instead of & anticausative & \emph{nilxam} & `fought'\\
	b.&	Reflexive in & \thit & instead of & anticausative & \emph{hitgaleax} & `shaved'\\
	\end{tabular}
\xe

These ``irregularities'' have occupied Semitist grammarians as well as contemporary authors \citep{doron03,arad05,borer13oup,kastner16phd}. On the one hand, it has long been acknowledged that roots show idiosyncratic behavior: broadly speaking, some roots prefer to form transitive verbs in {\tkal} and others in {\tpie}, as seen above. But on the other hand, it has proven much harder to accurately delimit the degree to which the templates are associated with distinct syntactic structures and semantic interpretations. When the language as a whole is viewed, the patterns in~(\ref{ex:alternations-heb}) are clearly visible, but so are the exceptions.

The proposal put forward in this book will showcase a novel analysis of the Hebrew patterns which captures existing work on European languages equally well. Working within Minimalist syntax \citep{chomsky95} and Distributed Morphology \citep{dm}, I rely on the idea that the external argument is introduced by the functional head Voice \citep{kratzer96,pylkkanen08,woodmarantz17}. This head might itself be endowed with syntactic features \citep{schaefer08,wood15springer}, leading me to defend the following claim:
\pex \textbf{The trivalency of Voice}
	\a Voice is associated with a [$\pm$D] feature, meaning it can be valued as [+D], [--D] or underspecified with regards to [D].\footnote{A similar view of binary features as trivalent is espoused by \cite{harbour11}.}.
	\a This feature indicates whether the specifier of Voice must be filled by a DP ([+D]), cannot be filled by a DP ([--D]), or is agnostic as to whether it is filled by a DP (underspecified).
\xe

@@[D] because that's how it's often seen in EPP literature. CPs aren't subjects as such, my own view of this is in \cite{kastner15lingua}. Take no stand as to subject PPs (Slavic?). Could also be [N] but all arguments are DPs. Check Chapter~\ref{i:agree} for phi-features on Voice and whether that's equivalent.@@

Importantly, we would expect these Voice heads to differ in their phonological form. This is exactly where the Hebrew data lead us. At its simplest, the system straightforwardly derives the alternations seen in~(\ref{ex:alternations-heb}) as follows:
\ex\label{ex:alternations-heb2}
	\begin{tabular}{llllll}
	\multicolumn{2}{c}{anticausative}	&	\multicolumn{2}{c}{transitive}	& \multicolumn{2}{c}{causative}\\
	\multicolumn{2}{c}{\tnif}	&	\multicolumn{2}{c}{\tkal}	& \multicolumn{2}{c}{\thif}\\
	\multicolumn{2}{c}{\textbf{\vz}}	&	\multicolumn{2}{c}{\textbf{Voice}}	& \multicolumn{2}{c}{\textbf{\vd}}\\
	\emph{neexal}	& `was eaten'	&	\emph{axal}	& `ate'	&	\emph{heexil}	& `fed'\\
	\end{tabular}
\xe

On this analysis, verbs in {\tnif} are expected to be anticausative/unaccusative because they lack an external argument, verbs in {\thif} are expected to be transitive because they require an external argument, and verbs in {\tkal} could go either way, depending on the idiosyncratic requirements of the root. The three values of Voice correspond to different morphological markings, but there is more than one way to get e.g.~an anticausative verb (namely with Voice or {\vz}). This way of looking at things dissolves the puzzle posed by examples like those in~(\ref{ex:counter1}) above: there is no reason to expect ``transitive'' to map on to specific morphology deterministically. Rather, syntax is autonomous in being able to build up structures which have different building blocks but may end up being interpreted similarly \citep{wood15springer,woodmarantz17,kastner16phd,myler16mit}.

The first part of the book (Chapters 2--5) will develop this analysis based on Hebrew, with special attention to the functional heads and the features they may have. The second part of the book (Chapters 6--7) will revisit existing work on argument structure alternations in light of the current system, drawing parallels and sketching the boundaries of crosslinguistic variation. I will now illustrate some of the relevant questions arising in each of these parts.



We will focus on three verbal forms in Hebrew (\emph{templates}): {\tkal}, {\tnif} and {\thif}.


{\tkal} varies in its argument structure (underspecified)

\pex Transitive:
	\a \begingl
		\gla teo \textbf{axal} et ha-laxmania//
		\glb Theo ate \gsc{ACC} the-bread.roll//
		\glft `Theo are the bread roll.'//
	\endgl
	\a \begingl
		\gla ha-balʃan \textbf{katav} et ha-maamar ha-arox//
		\glb the-linguist wrote \gsc{ACC} the-article the-long//
		\glft `The linguist wrote the long article.'//
	\endgl
\xe

\pex Unergative:
	\a \begingl
		\gla teo \textbf{rakad} ve-rakad ve-rakad//
		\glb Theo danced and-danced and-danced//
		\glft `Theo danced and danced and danced.'//
	\endgl
	\a \begingl
		\gla teo \textbf{halax} kol ha-boker//
		\glb Theo walked all the-morning//
		\glft `Theo walked all morning long.'//
	\endgl
\xe

\pex Unaccusative:
	\a \begingl
		\gla \textbf{nafal} le-teo ha-bakbuk//
		\glb fell to-Theo the-bottle//
		\glft `Theo's bottle fell.'//
	\endgl
	
	\a \begingl
		\gla ha-bakbuk \textbf{kafa} ba-makpi//
		\glb the-bottle froze in.the-freezer//
		\glft `The bottle froze in the freezer.'//
	\endgl
\xe


{\tnif} is non-active
\pex
	\a \begingl
		\gla \textbf{niʃbar} l-i ha-ʃaon//
		\glb broke to-me the-watch//
		\glft `My watch broke.//
	\endgl
	\a \begingl
		\gla \textbf{neelam} l-i ha-tamsir//
		\glb disappeared to-me the-handout//
		\glft `My handout disappeared.'//
	\endgl
\xe

{\thif} is active
\pex
	\a \begingl
		\gla kevin \textbf{heexil} et ema (duvdevanim)//
		\glb Kevin fed \gsc{ACC} Emma blueberries.//
		\glft `Kevin fed Emma (blueberries).'//
	\endgl %Emma is a dog
		
	\a \begingl
		\gla ha-zamar \textbf{herkid} et ha-orxim//
		\glb the-singer made.dance \gsc{ACC} the-guests//
		\glft `The singer made the guests dance.'//
	\endgl
\xe


That's all we need
\pex Importantly, we can find triplets:
	\a Causative {\thif}\\
		\begingl
		\gla ha-more hextiv (la-talmidim) et reʃimat ha-nosim//
		\glb the-teacher dictated to.the-students \gsc{ACC} list.of the-topics//
		\glft `The teacher dictated the list of topics (to the students).'//
	\endgl
	
	\a Transitive {\tkal}\\
		\begingl
		\gla ha-talmidim \textbf{katv-u} et ha-nosim//
		\glb the-students wrote-\gsc{PL} \gsc{ACC} the-topics//
		\glft `The students wrote the topics down.'//
	\endgl
	
	\a Non-active (mediopassive) {\tnif}\\
		\begingl
		\gla ha-xiburim nixtev-u (al-jedej ha-talmidim)//
		\glb the-essays were.written-\gsc{PL} by the-students//
		\glft `The essays were written (by the students)'.//
	\endgl
\xe



	\subsection{Some specifics}
Let us flesh out a few of the functional heads relevant to the analysis. We are interested in the difference between roots and functional morphemes as a way of getting at the loci of idiosyncrasy and systematicity in the grammar. A root is an acategorial morpheme: the verb \emph{walk}, for example, consists under my assumptions of a root \root{\gsc{WALK}} and a verbalizing categorizer, little v. There are three such categorizers: a, n, and v, which serve to categorize roots as adjectives, nouns or verbs \citep{marantz01,arad05,woodmarantz15}.

The functional head v introduces an event variable and categorizes a root as a verb. A higher functional head, Voice, introduces the external argument \citep{kratzer96,pylkkanen08,marantz13lingua}. The functional head \emph{p} introduces the external argument of a preposition, also called its Figure \citep{svenonius03,svenonius07,wood14nllt}. To derive the full range of verbs in Hebrew, I use a number of overt variants of these heads. The breakdown is as follows.

\textbf{Voice and \emph{p} heads} introduce a DP in their specifier. In a regular, unmarked active clause, default (silent) Voice introduces the external argument. The head \emph{p} was proposed by \cite{svenonius03,svenonius07} to act in similar fashion to Voice or Chomskyan little \textit{v}: it merges above the PP, introducing the Figure (subject) of the \textbf{p}reposition. I will not attempt to motivate this structure but will simply assume it; it is meant to capture the predicative relationship between the two DPs, similarly to the PredP of \cite{bowers93,bowers01} and \emph{ann}-XP of \cite{mccloskey14}. In~(\ref{ex:pP}), the Figure is the DP \emph{book} and \emph{p} is circled for ease of reference.
\ex \label{ex:pP}
%\Tree
%		[\emph{p}P, qtree
%		[DP\\\emph{book}\\\gsc{figure} ]
%		[
%		[\emph{p},circle,draw ]
%		[PP
%			[P\\\emph{on} ]
%			[DP
%				[\emph{the table}\\\gsc{ground},triangle ]
%			]
%		]
%		]
%		]
\xe

To these heads I add nonactive counterparts, namely \textbf{\vz} and \textbf{\pz}. These two heads dictate that nothing may be merged in their specifiers. \vz~blocks the introduction of an external argument \citep{doron03,alexiadoudoron12,bruening13,wood15springer,spathasetal15} and \pz~blocks merger of a DP in the specifier of pP \citep{wood15springer}. The different kinds of Voice/\emph{p} only manipulate the syntax: they dictate whether a DP may or may not be merged in their specifier. As mentioned above, default Voice and \emph{p} are silent. But \vz~and \pz~are spelled out by the placeholder Vocabulary Item \gsc{MID}, which adds a prefix and triggers insertion of certain vowels.

Voice also has the strongly active counterpart \textbf{\vd}. This head requires that a DP be merged in its specifier, behaving the opposite of \vz. This definition will be refined in \S\ref{syn:templates:thif}.

Alongside these functional heads and standard lexical roots I posit \textbf{\va}. In the semantics, this {element} types the event as an Action \citep{doron03} or ``self-propelled'' \citep{folliharley08}. In the phonology, \va~is spelled out as a predictable set of vowels slotting between the root consonants. \va~also blocks intervocalic spirantization of the middle consonant as mentioned above. I assume {for now} that \va~is spelled out by the Vocabulary Item \gsc{INTNS}, which is used as a placeholder for the phonological output. The semantics of this element emerge again in \S\S\ref{syn:middle:refl}, \ref{syn:middle:recip}, \ref{syn:templates:tpie}, its phonology in \S\ref{sec:t-phi}, and its crosslinguistic equivalents in \S\ref{syn:crosslx:heads}.

\textbf{The spell-out} of these heads produces templates as an epiphenomenon and is as follows: 

Voice and v are underspecified, but when combining they result in the \tkal~template as explained in Chapter 3.

\vd~provides the prefix \emph{he}-. Recall that templates differ from each other also in the vowels that appear between the root consonants, so I assume that each of the overt functional heads used in this theory inserts the right vowels, with the full implementation as in Chapter 3. For the time being, one may think of these vocalic changes as readjustment rules \citep{embickhalle05}. These are rules that ``fix'' the phonology of a form. Informally, for English, irregular past tense verbs do not take a suffix but undergo readjustment of the stem:
\ex \emph{sang} = \emph{sing} + T[Past]
\xe
The exact processes giving rise to vowel alternations are explored in Chapter 3.

\vz~provides the prefix \emph{ni}- or its allomorph \emph{hit}- in the environment of \va:
\pex
\a \vz~\lra~ \emph{hit}- + \gsc{READJUSTMENT} / \trace~\va
\a \vz~\lra~ \emph{ni}- + \gsc{READJUSTMENT}
\xe

The root \va~triggers readjustment rules and blocks postvocalic spirantization of the medial root consonant. This phonological effect is formalized as a floating feature [--continuant] docking onto a particular consonant. See \cite{katie13} for an analysis of gemination in Akkadian and Arabic using a similar mechanism.
\ex \va~\lra~[--cont]$_{\gsc{ACT}}$ / \trace~ \{ \root{XYZ} | Y $\in$ p, b, k \}
\xe

Table~\ref{table:summary} summarizes the syntactic, semantic and morphophonological effects of these heads, deriving a subset of the verbal system of Modern Hebrew. Special Voice/\emph{p} heads affect their specifier; see for the external argument (EA) under ``Syntax'' and as a prefix under ``Phonology.'' The effects of the special root {\va} can be seen under ``Semantics'' and as de-spirantization under ``Phonology.'' Note in particular that the \thit~template is morphologically complex. It is prefixed, indicating the existence of an overt Voice/\emph{p} head, and de-spirantized, indicating the existence of \va. Each of these elements is motivated in Chapter 2, and the table expanded on there.
\begin{table}[ht] \centering
	\begin{tabular}{|lll||c|c|l|c|}\hline
		\multicolumn{3}{|c||}{Heads} & Syntax 	& Semantics & Phonology & Mnemonic\\\hline\hline
		
		Voice& &	& (underspecified) 	& (underspecified)	&  \emph{XaYaZ} & ``simple''\\
		
		Voice&\red{\va}&	& (underspecified)	& \red{Action}	 & \emph{Xi{\red{\dgs{Y}}}eZ}&  ``intensive''	\\
		
			\blue{\vd}& &		& \blue{EA}	& (underspecified)	 & \emph{{\blue{he}}-XYiZ} & ``causative'' \\
		
		\blue{\vz}& &		& \blue{No EA}	& (underspecified)	 & \emph{{\blue{ni}}-XYaZ} & ``middle'' \\
		
		\blue{\vz}&\red{\va}&	& \blue{No EA}	& \red{Action}	 & \emph{{\blue{hit}}-Xa{\red{Y̯}}eZ} &  ``intensive middle''	\\
		
		Voice& &\blue{\pz}	& \blue{EA = Figure} & (underspecified)	 & \emph{{\blue{ni}}-XYaZ} & ``middle'' \\
		
		Voice&\red{\va}&\blue{\pz}	& \blue{EA = Figure} & \red{Action}	 & \emph{{\blue{hit}}-Xa{\red{Y̯}}eZ}	& ``intensive middle''\\\hline
	\end{tabular}
	\caption{The requirements of functional heads in the Hebrew verb.\label{table:summary}}
\end{table}



	\subsection{Templates as functional heads}
Modern Hebrew makes use of seven distinct morphophonological verbal forms in which a given root may or may not be instantiated. The root \root{ktb} which has a general meaning associated with writing can give rise to the verb \emph{katav} `wrote' in the \tkal~template and \emph{hextiv} `dictated' in the \thif~template. ``Causative'' \emph{hextiv} can then be passivized to yield \emph{huxtav} `was dictated' in the passive template \thuf. We can also get a passive reading of `was written' using the \tnif~template, \emph{nixtav}. So far, the argument structure alternations seem easy enough to pin down, as the first approximation in~(\nextx) shows. We begin with a subset of four verbal templates out of the seven.
\pex\label{ex:naive-ktb} A na\"{\i}ve view of argument structure alternations, based on \root{ktb}.
  \a \tkal: unmarked/transitive.
  \a \tnif: passive of \tkal~(\ref{ex:naive-ktb}a).
  \a \thif: causative of \tkal~(\ref{ex:naive-ktb}a).
  \a \thuf: passive of \thif~(\ref{ex:naive-ktb}c).
\xe

Traditional descriptions of the templates fall along these lines. However, if the characterization in~(\lastx) were correct then this would be a very short dissertation. In reality, neither the templates nor the roots play by such simple rules. A number of counterexamples are presented in~(\nextx).
\pex Counterexamples to the generalizations in~(\lastx).
  \a \tkal: does not exist in a large number of roots (\root{kns}, \root{'lm}, \root{tsmtsm}, \root{tlfn}).
  \a \tnif: \emph{nixnas} `entered', \emph{ne'elam} `disappeared', \emph{nilxam} `fought'; are not derived from a form in \tkal~nor are they passive.
  \a \thif: \emph{hexmits} `grew sour', \emph{he'edim} `reddened'. Change-of-state verbs, not derived from a form in \tkal.
  \a \thuf: (no counterexamples -- robust generalization)
\xe

This brief rundown omits further complications introduced by two additional templates, namely \thit~and \tpie, and by \tpua~which is the passive counterpart of the latter. The seven templates, then, do not provide us with deterministic mappings from phonological form (the template) to syntax (argument structure), except in the case of the two passive templates. One goal of this chapter---and of the dissertation in general---is to argue for an analysis which treats templates not as morphosyntactic atoms (morphemes or features) but as an epiphenomenon of distinct functional heads merging in the structure. Both the regularities in~(\ref{ex:naive-ktb}) and the exceptions in~(\lastx) will need to be covered by our theory of morphosyntax.

Consider next the roots. In a ``clean'' system, placing the same root in different templates would at least result in predictable changes in meaning; the alternations of \root{ktb} mentioned above all maintain the same basic notion of writing, modified in different ways. Yet the alternations, too, are not an open-and-shut case, as can be seen for \root{pkd} in~(\nextx).
\pex\label{ex:naive-pkd}
  \a \tkal: \emph{pakad} `ordered'.
  \a \tnif: \emph{nifkad} `was absent'.
  \a \tpie: \emph{piked} `commanded' (and a passive \tpua~form).
  \a \thif: \emph{hefkid} `deposited' (and a passive \thuf~form).
  \a \thit: \emph{hitpaked} `allied himself', `conscripted'.
\xe
One could find a general semantic field of ``counting'' or ``surveying'' running through the use of this root but the alternations are in no way obvious. The problem is exacerbated when considering nominal forms as well: \emph{pakid} `clerk', \emph{mifkada} `headquarters', \emph{pikadon} `deposit'. Templates, then, do not provide us with deterministic mappings from phonological form (the template) to semantics (interpretation of a root), again with the exception of the passive templates.

We have seen that a template cannot definitively clue us in onto the argument structure or interpretation of the verb. It is equally true that a given argument structure alternation or interpretation cannot be deterministically assigned to a certain template: we have already seen transitive verbs at least in \tkal, \tpie~and \thif. Where does this lead us? This dissertation attempts to balance two angles on the morphological phenomena: an empiricist-Semitist one, in which I ask what the templates are and what they tell us; and a theoretical one, in which I take {the} conceptual issue {of locality domains in argument structure alternations} and search for relevant evidence. But it is worth pausing to consider what is at stake. To the extent that the utterances of different languages are generated by similar grammars, the task of the linguist is to identify those parts of the grammar that cannot be learned from simple exposure to input, as well as how the parts that do get learned are encoded. Semitic languages are interesting because they seem to defy a linear account of structure building, at least in one well-circumscribed domain: prosodically strict morphology \citep{jjmcc81}. Any structure that the grammar generates, and any locality conditions or other constraints that hold in said structure, must produce output that can be linearized according to phonological requirements. Theories of locality and allomorphy are especially sensitive to ordering, be it linear or hierarchical. It follows that a satisfactory account of Semitic morphology is necessary in order to test hypotheses of how morphemes are arranged, how they combine, and how they might be learned.

Before continuing there is an important question regarding whether verbs such as those in~(\ref{ex:naive-ktb}) or in~(\ref{ex:naive-pkd}) do in fact share the same root. For example, it could be argued that (\lastx a,b,c,e) as well as the noun `headquarters' share one root that has to do with military concepts, and that (\lastx d) as well as the nouns `clerk' and `deposit' stem from a homophonous root that has to do with financial concepts. There are a number of reasons to reject this claim. First, there are no ``doublets''; if we were dealing with two roots, call them \root{pkd$_1$} and \root{pkd$_2$}, then each should be able to instantiate any of the templates. But \emph{hefkid} can only mean `deposited', never something like `installed into command'. The choice of verb for that root in that template has already been made. Second, experimental {studies} have found roots to behave uniformly across their different meanings (though this conclusion has been challenged{ -- see the overview in \S\ref{proc:meg}}). We will sidestep this issue for the bulk of the discussion, returning to it in \S\ref{acq:prod:root}.

In the remainder of this dissertation I will be forced into the not-unwelcome position of claiming that templates are \emph{emergent}, arising from the combination of roots and functional heads in the syntax. Unlike much traditional work that took both roots and templates to be primitives in the system (i.e.~morphemes), and unlike some recent work which takes templates to be morphemic but denies the existence of the root, our primitives will be the root and a collection of hierarchically arranged syntactic heads.

Each of these heads will be assigned an explicit syntax, semantics and phonology. This chapter is devoted to describing the syntax and semantics of the different pieces, evaluating the model's fit to the data and testing additional predictions that are made. The goal is to identify (a) the syntactic features relevant to each morpheme and (b) which rules of interpretation operate on them. Table~\ref{table:summary-syn} summarizes the combinations of functional material that I will discuss.
%\setcounter{table}{0}
	\begin{table}[ht] \centering \small
		\begin{tabular}{|llll||c|c|l|c|}\hline
			\multicolumn{4}{|c||}{Heads} & Syntax 	& Semantics & Phonology & Section\\\hline\hline
			
			& Voice& &	& (underspecified) 	& (underspecified)	&  \emph{XaYaZ} & \S\ref{syn:templates:tkal} \\\hline
			
			& Voice&\va&	& (underspecified)	& Action	 & \emph{Xi{\dgs{Y}}eZ}&  \S\ref{syn:templates:tpie}	\\
			
			Pass & Voice&\va&	& Passive	& Action	 & \emph{Xu{\dgs{Y}}aZ}&  \S\ref{syn:templates:pass}	\\\hline
			
			& \vd& &		& EA	& (underspecified)	 & \emph{{he}-XYiZ} & \S\ref{syn:templates:thif} \\
			
			Pass & \vd& &		& Passive, EA	& (underspecified)	 & \emph{hu-XYaZ} & \S\ref{syn:templates:pass} \\\hline
			
			& \vz& &		& No EA	& (underspecified)	 & \multirow{2}{*}{\emph{ni-XYaZ}} & \S\ref{syn:middle:nonactive} \\
			
			& Voice& &\pz	& EA = Figure & (underspecified)	 &  & \S\ref{syn:middle:active} \\\hline
						
			& \vz&\va&	& No EA	& Action	 & \multirow{2}{*}{\emph{hit-Xa{Y̯}eZ} } &  \S\ref{syn:middle:nonactive} \\

			& Voice&\va&\pz	& EA = Figure & Action	 & & \S\ref{syn:middle:active} \\\hline
		\end{tabular}
		\caption{The requirements of functional heads in the Hebrew verb.\label{table:summary-syn}}
	\end{table}

Explaining briefly, Pass is a passivizing head. Voice is a functional head introducing the external argument of the verb. {\vz} is a variant which does not allow anything in its specifier and {\vd} is a variant requiring a DP in its specifier. \emph{p} introduces the subject of a preposition, and \pz~is a variant which does not allow anything in its specifier. {\va} is an agentive modifier. These elements will be introduced as the discussion progresses; the combinatorial possibilities are addressed in \S\ref{syn:crosslx:combinatorics}.

Table~\ref{table:root-summary-syn} summarizes the requirements of root classes in different configurations.
\begin{table}[ht] \centering \small
\begin{tabular}{|ll|l|l|}\hline
	 Morphology & Section & Verb type & Root type \\\hline\hline
	 \multirow{2}{*}{\thit} & \multirow{2}{*}{\S\S\ref{syn:middle:nonactive:anticaus}, \ref{syn:middle:refl}, \ref{syn:middle:roots}} &  Reflexive & Self-oriented\\
	 & & Inchoative & Other-oriented\\\hline
	 
	 \multirow{2}{*}{\thit} & \multirow{2}{*}{\S\S\ref{syn:middle:recip}, \ref{syn:middle:roots}} & Reciprocal & Naturally reciprocal\\
	 & & Figure reflexive & Naturally disjoint (Other-oriented)\\\hline
	 
	 \multirow{2}{*}{\thif} & \multirow{2}{*}{\S\ref{syn:templates:thif}} & Alternating unaccusative & Change of color\\
	 & & Alternating unergative & Emission\\\hline
	 
	 \multirow{2}{*}{\tpie} & \multirow{2}{*}{\S\ref{syn:templates:tpie}} & Pluractional object & Other-oriented\\
	 && Pluractional event & Self-oriented (activity)\\\hline
	 
	 Passive participle & \S\ref{syn:templates:adjpass} & Resultative adjective & Change of state\\\hline
	 
	 \tkal & \S\ref{syn:templates:tkal} & Underspecified & (all)\\\hline
\end{tabular}
\caption{The requirements of root classes in the Hebrew verb.\label{table:root-summary-syn}}
\end{table}



The discussion in this paper highlights how roots place requirements on the syntactic derivation. In English, for instance, it has been suggested in different ways that there is a difference between the semantics of \root{\gsc{DESTROY}}, \root{\gsc{GROW}} and \root{\gsc{BREAK}} which goes beyond pure meaning. This difference leads to an inability to take complements in nominalized form \citep{chomsky70,marantz97}.
\pex \root{\gsc{DESTROY}}: change of state, externally caused
\a The enemy's destruction of the city.
\a The city's destruction (by the enemy).
\xe

\pex \root{\gsc{GROW}}: change of state, internally caused
\a \ljudge{*} John's growth of tomatoes.
\a The tomatoes' growth (*by John).
\xe

\pex \root{\gsc{BREAK}}: result
\a \ljudge{*} John's break of the glass.
\a \ljudge{*} The glass' break.
\xe
Similar observations have been made more recently for a variety of phenomena in different languages \citep{haspelmath93,unaccusativity95,schaefer08}. The details are less important right now than the intuition that something about the lexical semantics of the root constrains what should otherwise be an identical syntactic derivation. In these cases, the underlying assumption is that the morphosyntax of the verbs \emph{destroy}, \emph{grow} and \emph{break} is identical in that they are all made up of a root and a verbalizer, with no extra syntactic material determining their argument structure.

Nevertheless, argument structure alternations can be conditioned by additional syntactic material. For instance, markers such as German \emph{sich} and Romance \gsc{SE} famously reduce the total arity of the verb, descriptively speaking \citep[e.g.][]{labelle08,schaefer08,cuervo14}. In order to account for the Hebrew facts, I will take the connection highlighted in the previous section---that between argument structure and the template---and cash it out in terms of the syntactic head Voice.

Throughout the paper I assume that morphological structure is built up in the syntax \citep{dm}, with late insertion of phonological material proceeding from the most deeply embedded element outwards \citep{bobaljik00,embick10}. The external argument is introduced by the functional head Voice \citep{kratzer96,pylkkanen08}. Acategorial roots modify one of the ``categorizing'' heads v, n and a \citep{marantz97,arad03,harley14thlia}. To see how roots affect argument structure, we begin with anticausatives.


%%%%%%%
The present study examines the division of labor between syntax, semantics, phonology and the lexicon. Generative approaches to linguistic theorizing have so far resulted in a wealth of knowledge about how an abstract syntax generates structure which is then interpreted by the semantics and by the phonology. We also have a basic vocabulary allowing us to describe how individual lexical items might have their own idiosyncrasies in the syntax (different features), in the semantics (different meanings) and in the phonology (lexical exceptionality). What this dissertation tests is the hypothesis that the syntax feeds both interfaces in the same way.

The core idea has two parts. The first is that there is a universal set of syntactic elements which can be arranged in a hierarchical way. Once the syntax generates a chunk of structure (be it a phase or an entire utterance), it must be interpreted compositionally at the interface with the semantics and linearized in order to be interpreted at the interface with the phonology. In both cases, I argue that the same kind of locality constraints hold on interpretation. The second part has to do with how individual lexical items muddy the waters. The grammar is rigid, but lexical material can influence how functional material is interpreted. The important point here is that lexical material is separated from the syntax proper: its idiosyncrasies only kick in at the two interfaces. As such, lexical material (a root) has no syntactic features, only properties which are directly related either to meaning or to pronunciation.

In support of this claim I approach a notorious empirical landscape: the non-concatenative morphology of Modern Hebrew. In Semitic languages like Hebrew, words are famously made up of various grammatical and lexical elements interleaved in a single, short (often disyllabic or trisyllabic) phonological word. Two syllables might convey a lexical root, tense information, derivational information and agreement information all at once. It is therefore not immediately obvious that hierarchical structure is there to be found. Yet that is exactly the claim put forward here. By demonstrating that the hypothesis outlined above is a valid one for Hebrew, this dissertation entails that it is valid for natural language as a whole. Functional material (in the form of structure) and lexical material (in the form of roots) can be seen to combine in systematic, predictable ways.

Taking the verbal system as the main object of inquiry, the following chapters analyze its syntactic, semantic and phonological properties in order to address a number of Hebrew-specific, Semitic-specific and language-general questions. At its analytical core, this work asks how we might apply the theories that have been developed for other languages to Semitic: how these theories might be tested and how they should be modified to accommodate a broader range of data. I frame this question in terms of the abstract syntactic structure and the way it feeds into semantic and phonological interpretation. The resulting discussion addresses three main questions: the proper description of Hebrew morphology, the proper description of the syntax and its relation to the interfaces with the semantics and the phonology, and ultimately the way such a system is learned by the child. Let us first see what the system looks like.






\section{Traditional generative treatments of the system} \label{sec:jjmcc}

@@mention jjmcc, doron, arad in general terms. I'll return to comparisons later on@@

Before we get to the meat of the dissertation, I would like to acknowledge some of the earlier work on Semitic morphology in an attempt to illustrate what we do and do not know yet. I focus here on the seminal series of works by McCarthy \citep{jjmcc79,jjmcc81,jjmcc89li,jjmccprince90} in order to bring out the inadequacies of a purely phonological account of {Semitic morphology}. Comparisons with more recent analyses will be presented in the next two chapters, where appropriate.

	\subsection{Tiers}
McCarthy's original contribution lay in dividing the Semitic (Arabic) verb into three ``planes'' or ``tiers'': the CV skeleton (C and V slots), the root (consonants) and the melody (individual vowels). For example, the verb \emph{takattab} `got written' was analyzed as follows, with a default verbal vowel -\emph{a}-.
\ex\label{ex:jjmcc-takattab}\emph{takattab} \citep[392]{jjmcc81}:\\
\xy
<0pt,1.5cm>*\asrnode{C}="C0",
<0.75em,1.5cm>*\asrnode{V}="V0",
<1.5em,1.5cm>*\asrnode{C}="C1",
<2.25em,1.5cm>*\asrnode{V}="V1",
<3em,1.5cm>*\asrnode{C}="C2",
<3.75em,1.5cm>*\asrnode{C}="C3",
<4.5em,1.5cm>*\asrnode{V}="V2",
<5.25em,1.5cm>*\asrnode{C}="C4",
<0pt,0cm>*\asrnode{t}="c0";
<0.75em,3cm>*\asrnode{a}="v0";
<1.5em,0cm>*\asrnode{k}="c1";
<2.25em,3cm>*\asrnode{a}="v1";
<3.5em,0cm>*\asrnode{t}="c2";
<4.5em,3cm>*\asrnode{a}="v2";
<5.25em,0cm>*\asrnode{b}="c4";
"C0"+D;"c0"+U**\dir{-};
"V0"+U;"v0"+D**\dir{-};
"C1"+D;"c1"+U**\dir{-};
"C2"+D;"c2"+U**\dir{-};
"C3"+D;"c2"+U**\dir{-};
"C4"+D;"c4"+U**\dir{-};
"V1"+U;"v1"+D**\dir{-};
"V2"+U;"v2"+D**\dir{-};
\endxy
%	\includegraphics[scale=0.4]{figs/jjmcc81-25}
\xe

By including the vocalism on a separate tier, \citeauthor{jjmcc81}'s theory allowed vowels to be manipulated independently of the roots or the skeleton. The melody \emph{u-a-i} is taken to derive the active participle, for {instance}:
\ex \emph{mutakaatib} \citep[401]{jjmcc81}\\
\xy
<0pt,1.5cm>*\asrnode{C}="C0a",
<0.75em,1.5cm>*\asrnode{V}="V0a",
<1.5em,1.5cm>*\asrnode{C}="C0b",
<2.25em,1.5cm>*\asrnode{V}="V0b",
<3em,1.5cm>*\asrnode{C}="C1",
<3.75em,1.5cm>*\asrnode{V}="V1",
<4.5em,1.5cm>*\asrnode{V}="V2",
<5.25em,1.5cm>*\asrnode{C}="C2",
<6em,1.5cm>*\asrnode{V}="V3",
<6.75em,1.5cm>*\asrnode{C}="C3",
<2.25em,0pt>*\asrnode{u}="u";
<3.5em,0pt>*\asrnode{a}="a";
<4.5em,0pt>*\asrnode{i}="i";
<3.5em,-1.5cm>*\asrnode{$\mu$}="mu";
"V0a"+D;"u"+U**\dir{-};
"V0b"+D;"a"+U**\dir{-};
"V1"+D;"a"+U**\dir{-};
"V2"+D;"a"+U**\dir{-};
"V3"+D;"i"+U**\dir{-};
"u"+D;"mu"+U**\dir{-};
"a"+D;"mu"+U**\dir{-};
"i"+D;"mu"+U**\dir{-};
\endxy
\xe
%	\includegraphics[scale=0.6]{figs/jjmcc81-42}

The beauty of this theory is that it allowed for a separation of three morphological elements on three phonological tiers: the root (identity of the consonants), the template (the form of the CV skeleton) and additional inflectional or derivational information (the identity of the vowels). Important extensions were proposed in \cite{jjmccprince90} to account for denominal forms, specifically plurals and diminutives.

The current work shifts the focus to the nature of the CV skeleton and the melody. \citeauthor{jjmcc81}'s approach did not attempt to model the relationships between the semantics of the different templates -- the alternations in argument structure. Yet as we have seen in \S\ref{sec:tradition}, some templates are related to others in ways that remain to be explicated. \citeauthor{jjmcc81}'s work{,} as well as work inspired by {it,} leaves us in prime position to ask the following interrelated questions:
\pex \textbf{Questions on the nature of Semitic addressed in this study}
	\a What is the syntax behind the CV skeleton?
	\a What is the syntax behind the melody?
	\a What is the relationship between different templates, that is, how are argument structure alternations derived?
\xe
My answers to these questions lead us to make different assumptions than \citeauthor{jjmcc81}. Like him, I believe that the consonantal root lies at the core of the lexicon. Unlike his theory, I do not postulate independent CV skeletons and do not accord the prosody morphemic status. The skeletons will be a by-product of how functional heads are pronounced and regulated by the general phonology of the language. There is no skeleton CVCVCCVC as in~(\ref{ex:jjmcc-takattab}) giving \emph{takattab}, for example: there would be a prefix \emph{ta}-, a number of vowels spelling out Voice, gemination spelling out an additional head, and the organization of these different segments will proceed in a way that satisfies the phonology without making reference to prosodic primitives like skeletons. Furthermore, each morpheme will have an explicit syntax and semantics associated with it. Chapter 2 develops the morphosyntactic system and Chapter 3 returns to the morphophonological side of the morphology, making a number of additional contributions to our understanding of how the syntax and the phonology interact.
	
	\subsection{Related work}
A few more pieces of research that capture generalizations important to this dissertation deserve mention. The seminal work by \cite{berman78} underscored the semi-predictable nature of the templates. \citet[Ch.~3]{berman78} made the point that the combination of root and template is neither fully regular nor completely idiosyncratic. Instead, \citeauthor{berman78} proposed a principle of \emph{lexical redundancy} to regulate the system. According to this theory, each root has a ``basic form'' in some template from which other forms are derived. Yet this theory did not formalize the relations between the templates, arbitrarily selecting one as the ``basic form'' and the others as derived from it, for each root. Nevertheless, \citeauthor{berman78}'s clear description of the regularities and irregularities in the morphology of Hebrew laid the groundwork for later works such as \cite{doron03}, \cite{arad05}, \cite{borer13oup} and the current contribution.

Alongside work that analyzed the syntactic and semantic features of roots and templates, other researchers have focused on the morphophonological properties of the system. The research program developed in a series of works by \cite{batel89,batel94} and \cite{ussishkin00phd,ussishkin05}---credited by \cite{ussishkin00phd} at least in part to \cite{horvath81}---denies the existence of the root as an independent morpheme. Instead, all verbs are derived via phonological manipulation of surface forms from each other, rather than from an underlying root. I refer to this idea as the ``stem-based approach''. We return to it in \S\ref{sec:others-diff}, discussing it more in depth there.

Even before the stem-based approach took form, other Semitists explored the idea of a Semitic system which diverged from the traditional descriptions. \cite{schwarzwald73} doubted the productivity of both the root and the templates, making an early argument for frequency effects in the interpretation of different templates. On that view, it is only the high frequency verbs of the language that show reliable alternations between templates. These verbs lead us as analysts to postulate relationships between templates, though when one looks at less frequent verbs, transparent alternations are less likely to hold. Unlike the stem-based hypothesis, which eschewed roots and relied on the template as a morphological primitive, the proposal in \cite{schwarzwald73} kept the root but relegated the template to morphophonological limbo: salient in the grammar but not operative in the syntax. While this early formulation of a template-less idea is intriguing, it cannot hold up to wug studies in which speakers generate argument structure alternations between templates using nonce words \citep{berman93jcl,moorecantwell13}.

To pick out a few studies on Arabic (as gleaned from the helpful overview in \citealt{ussishkin00phd}), \cite{darden92} offered an analysis of Egyptian Arabic that attempted to do without verbal templates; \cite{mcomber95} developed an infixation-based system similar to that of \cite{jjmcc81} which makes crucial reference to morpheme edges; and \cite{ratcliffe97,ratcliffe98} attempted to improve on \cite{jjmccprince90} by restricting the CV skeleton and treating more phenomena as cases of infixation. But let us return to the current study.





Verbs in Hebrew appear in one of seven verbal templates, a morphological phenomenon emblematic of Semitic languages. Descriptively speaking, templates are morphophonological objects made up of consonant and vowel slots; the two templates in (@) will be notated XaYaZ and niXYaZ, where X, Y and Z stand for the root consonants which combine with the template to create a verb. These template often reflect argument structure alternations \citep{doron03,arad05,kastner16phd,kastner17gjgl}, a situation which can also be seen in (@).
(@) ʃavar - niʃbar ‘broke’, kara - nikra ‘tore’, matax - nimtax ‘stretched’.

In these cases, the non-active version is a detransitivized form of the active version and shares the same root as the active verb. The derived verbs are all intransitive and their bases transitive. The English translations are labile, but Hebrew does not employ zero-derivation in the same way (cf. \cite{borer91,kastner18}).

Similar alternations hold in other templates, for example XiYeZ and hitXaYeZ \citep{kastner17gjgl}.
(@) biʃel - hitbaʃel, nipeax - hitnapeax, bitel - hitbatel

The starting point for our discussion of nominalizations is that many verbs in niXYaZ cannot be characterized as part of a causative alternation. The niXYaZ verbs in (@) either are not an anticausative version of a causative verb in (@), or do not have a causative alternant in XaYaZ to begin with.

(@)     laxam 'fought' - nilxam be- 'fought (with)' 
    axaz 'held' - neexaz be- 'held on to' 
    @@




\ex\label{ex:alternations-heb}Hebrew has \textbf{trivalent} morphological marking\citep{kastner18nllt}:\\
	\begin{tabular}{cll|ll|ll}
	& \multicolumn{2}{P{4.2cm}|}{causative} &	\multicolumn{2}{P{4cm}|}{underspecified}	& \multicolumn{2}{P{4.2cm}}{anticausative}\\\cline{2-7}
	\phantom{Semantics} & \multicolumn{2}{c|}{\thif}	&	\multicolumn{2}{c|}{\tkal}	& \multicolumn{2}{c}{\tnif}\\
	& \emph{heexil}	& `fed' &	\emph{axal}	& `ate'	&	\emph{neexal}	& `was eaten' \\
	& \emph{hextiv}	& `dictated' &	\emph{katav}	& `wrote'	&	\emph{nixtav}	& `was written' \\
	\end{tabular}
\xe

\ex\label{typo-feat}\textbf{The typology of Featural:}\\
\begin{tabular}{c|ll|ll|ll}
	& \multicolumn{2}{P{4cm}|}{\vd}	&  \multicolumn{2}{P{4cm}|}{Voice}	& \multicolumn{2}{P{4cm}}{\vz} \\\hline
%&&&&\\
Semantics	 & 		a.	&	&			b.	&& 	c. & \\
$\lambda$x 	 & 
&\Tree
[.VoiceP 
	[.DP ]
	[.
		[.{\vd} ]
		[.vP ]
	]
]
& 
&\Tree
[.VoiceP 
	[.DP ]
	[.
		[.Voice ]
		[.vP ]
	]
]
&& \phantom{Undefined.}
\\\hline
Semantics	 & 		d.		& &			e.	& &	f. & \\
\zero	 &
& \phantom{Undefined.}
&
&\Tree
	[.VoiceP
		[.Voice ]
		[.vP ]
	]
&
&\Tree
	[.VoiceP
		[.{\vz} ]
		[.vP ]
	]\\
\end{tabular}
\xe


\ex\label{ex:alternations-heb2}Featural analysis of the templates:\\
	\begin{tabular}{cll|ll|ll}
	& \multicolumn{2}{P{4cm}|}{\textbf{\vd}}	&	\multicolumn{2}{P{4cm}|}{\textbf{Voice}}	& \multicolumn{2}{P{4cm}}{\textbf{\vz}}\\
	\phantom{Semantics} & \multicolumn{2}{c|}{causative} &	\multicolumn{2}{c|}{transitive}	& \multicolumn{2}{c}{anticausative}\\\cline{2-7}
	& \multicolumn{2}{c|}{\thif}	&	\multicolumn{2}{c|}{\tkal}	& \multicolumn{2}{c}{\tnif}\\
	& \emph{heexil}	& `fed' &	\emph{axal}	& `ate'	&	\emph{neexal}	& `was eaten' \\
	& \emph{hextiv}	& `dictated' &	\emph{katav}	& `wrote'	&	\emph{nixtav}	& `was written' \\
	\end{tabular}
\xe






\part{Hebrew Argument Structure}

    \chapter{Unspecified Voice}
\label{chap:voice}

\section{Overview} \label{voice:intro}
We begin by examining the ``simple'' template {\tkal} and the ``intensive'' template {\tpie}. Section~\ref{voice:tkal} reviews the empirical picture for {\tkal} and distills a number of generalizations, followed by a formal analysis using the unspecified Voice head in Section~\ref{voice:voice}. Sections~\ref{voice:tpie} and \ref{voice:va} then do the same for {\tpie} and {\va}. A brief summary and overview of the next few chapters are provided in Section~\ref{voice:conc}.

\section{\tkal: Descriptive generalizations} \label{voice:tkal}
This chapter introduces the first part of a theory of Voice which makes room for an underspecified variant, one which neither requires nor prohibits a specifier. We will first consider morphological marking which is compatible with a variety of verbal forms, namely the template {\tkal}.

As we have already seen briefly in the previous chapter, Hebrew has dedicated active and non-active morphology. For example, verbs in {\tnif} are usually non-active and those in {\thif} are active. Verbs in {\tkal} are unique within the verbal system in that they are underspecified with regards to their argument structure. Simply knowing the morphological form (the template) is not enough to indicate what kind of verb we are dealing with. Let us examine the different possibilities, using basic diagnostics.

With some roots, the verb is transitive. The examples in~(\ref{ex:voice-intro-tr}) contain strongly transitive verbs, which require an internal argument, and assign accusative case.\footnote{There is a substantial literature on \emph{et} and what kind of syntactic element it is \citep{siloni97,danon01,borer13oup}. What is uncontroversial is that it occurs before specific accusative objects.}
\pex\label{ex:voice-intro-tr}
	\a \begingl
		\gla teo \textbf{taraf} *(et ha-laxmanja)//
		\glb Theo devoured \gsc{ACC} the-bread.roll//
		\glft `Theo devoured the bread roll.'//
	\endgl
	\a \begingl
		\gla ha-balʃan \textbf{katav} et ha-maamar ha-arox//
		\glb the-linguist wrote \gsc{ACC} the-article the-long//
		\glft `The linguist wrote the long article.'//
	\endgl
\xe

With other roots, verbs in {\tkal} are unergative. The examples in (\ref{ex:voice-intro-unerg}) show activities which can be repeated or modified with atelic adverbials.
\pex\label{ex:voice-intro-unerg}
	\a \begingl
		\gla teo \textbf{rakad} ve-rakad ve-rakad (kol ha-boker)//
		\glb Theo danced and-danced and-danced all the-morning//
		\glft `Theo danced and danced and danced (all morning long).'//
	\endgl
	\a \begingl
		\gla teo \textbf{halax} kol ha-boker//
		\glb Theo walked all the-morning//
		\glft `Theo walked all morning long.'//
	\endgl
\xe

Other roots give rise to ditransitive verbs, including strong ditransitives in which the goal cannot be omitted (\ref{ex:voice-intro-ditr}).
\pex\label{ex:voice-intro-ditr}
	\a \begingl
		\gla teo \textbf{natan} *(le-marsel) et ha-xatif//
		\glb Theo gave to-Marcel \gsc{ACC}  the-snack//
		\glft `Theo gave Marcel the treat.'//
	\endgl
	\a \begingl
		\gla teo \textbf{ʃaal} et ha-sefer me-ha-sifria//
		\glb Theo borrowed \gsc{ACC} the-book from-the-library//
		\glft `Theo borrowed the book from the library.'//
	\endgl
\xe

And finally, unaccusative verbs are also possible in this template (\ref{ex:voice-intro-unacc}). Here we can make use of Hebrew-internal diagnostics such as Verb-Subject order and the Possessor Dative~(\nextx a)---which I discuss in more depth later on, in Chapter~\ref{vz:nact:anticaus:unacc}---and typical change of state predicates~(\nextx b).
\pex\label{ex:voice-intro-unacc}
	\a \begingl
		\gla \textbf{nafal} le-teo ha-bakbuk//
		\glb fell to-Theo the-bottle//
		\glft `Theo's bottle fell.'//
	\endgl
	\a \begingl
		\gla ha-bakbuk \textbf{kafa} ba-makpi//
		\glb the-bottle froze in.the-freezer//
		\glft `The bottle froze in the freezer.'//
	\endgl
\xe

Recall that Hebrew templates can be viewed through two lenses: the constructions they are compatible with, and their canonical alternations with other templates. The generalization about verbs in {\tkal} is a negative one: there are no syntactic constraints on the kind of verb that appears in this template. For this reason, \cite{doron03} does not associate it with any specific functional heads and \cite{borer13oup,borer15roots} treats it as a verbalized root with no additional syntactic functors. Alternations will be discussed once we engage with the other templates of the language.

\hammer{
\pex \label{ex:gen-tkal}\textbf{Generalizations about {\tkal}}
	\a \textbf{Constructions:} Verbs appear in all possible argument structure configurations.
	\a \textbf{Alternations:} Participates in alternations with the other templates, as will be reviewed throughout the book.
\xe
}

I look into the patterns of {\tkal} in more depth in Section~\ref{voice:voice}, where I situate them within the current theory of Voice. I then move on to the template {\tpie} in Section~\ref{voice:tpie}, showing what it teaches us about an agentive modifier which I call {\va} in Section~\ref{voice:va}. Section~\ref{voice:conc} summarizes and outlines how the rest of the Hebrew system will inform the theory developed in the first part of this book.

\section{Unspecified Voice} \label{voice:voice}
This monograph promotes a theory of argument structure in which Voice can have one of three values: [+D], [--D] or unspecified for [$\pm$D]. As foreshadowed in the introductory chapter, the idea is that {\vd} requires an external argument and {\vz} prohibits one. We will focus on what it means for Voice not to have a preference on the matter, thereby accounting for the patterns in (\ref{ex:voice-intro-tr})--(\ref{ex:voice-intro-unacc}).

First, let me define unspecified Voice in (\nextx). All definitions of Voice heads in this book take the same form: (a) syntactic definition, (b) semantic denotation, and (c) basic spell-out rules. I give these here and expand upon them in turn.
\pex \textbf{Voice}
	\a A Voice head with no specified for a [D] feature. It has no requirements regarding whether its specifier must be filled. In transitive verbs, Voice is the locus of accusative case assignment, either itself by feature checking \citep{chomsky95} or through the calculation of dependent case \citep{marantz91}.
	\a \denote{Voice} = $\begin{cases}
		\lambda e.e & \text{/ \trace~ \{ \root{npl} `\root{\gsc{FALL}}', \root{kpa} `\root{\gsc{FREEZE}}' , \dots \} }\\
		\lambda x \lambda e.Agent(x,e) & \\
		\end{cases}$

%	\denote{Voice} = $\begin{cases}
%		\lambda x \lambda e.Agent(x,e) & \text{/ \trace~\{ \root{axl} `\root{\gsc{EAT}}', \root{ktb} `\root{\gsc{WRITE}}', \root{ntn} `\root{\gsc{GIVE}}',}\\
%			& \text{\root{ʃal} `\root{\gsc{BORROW}}', \root{\gsc{r\dgs{k}d}} `\root{\gsc{dance}}', \root{\gsc{hlx}} `\root{\gsc{WALK}}', \dots\} }\\
%		\lambda e.e & \text{/ \trace~ \{ \root{npl} `\root{\gsc{FALL}}', \root{kpa} `\root{\gsc{FREEZE}}' , \dots \} }
%		\end{cases}$
	\a Voice \lra~{\tkal} \hfill  (with the allomorph {\tpie} to follow in Section \ref{voice:va})
\xe

		\subsection{Syntax} \label{voice:voice:syn}
The view of argument and event structure adopted here (see Chapter~\ref{intro:sketch}) builds up the verbal domain in ``layers''. Taking the root \root{trf}, which has to do with devouring, we first build up a verb by adjoining the root to the verbal category head v, and then merge the DP required as an internal argument. This gives us a predicate over events of devouring that DP. Adding the traditional Voice head would do little to change the event but would add an agent role to the semantics. The current Voice head is slightly different.
\ex
\Tree
[.v
	[.\root{trf} ]
	[.v ]
]
$\Rightarrow$
\Tree
[.vP
	[.v
		[.\root{trf} ]
		[.v ]
	]
	[.DP ]
]
$\Rightarrow$
\Tree
[.VoiceP
	[.DP ]
	[.
		[.Voice ]
		[.vP
			[.v
				[.\root{trf} ]
				[.v ]
			]
			[.DP ]
		]
	]
]
\xe

In the current system, the lack of a feature on Voice means that the head is not specified for any syntactic feature constraining Spec,VoiceP. That position can be filled or left unprojected, as far as the Voice head is concerned. In this state of affairs, the expectation is that differences between verbs will result from the requirements of individual roots, rather than anything in the structure. In other words, some roots will give rise to transitive verbs, other roots to unaccusative verbs, and so on.

This is exactly what we have seen in the template {\tkal}. There are no structural restrictions on argument structure in this template: verbs in {\tkal} might be transitive, unergative, ditransitive or unaccusative. Some of the examples in (\ref{ex:voice-intro-tr})--(\ref{ex:voice-intro-unacc}) are repeated below with minimal syntactic structures (leaving out material above VoiceP, such as Tense).

In~(\nextx) we see the core transitive verb \emph{taraf}, which requires an internal argument. The accusative/DOM marker \emph{et} must also appear, indicating that this is a transitive construction.
\pex\label{ex:voice-intro-tr2}
	\a 
	\begingl
		\gla teo \textbf{taraf} *(et ha-laxmanja)//
		\glb Theo devoured \gsc{ACC} the-bread.roll//
		\glft `Theo devoured the bread roll.'//
	\endgl
	\a \Tree
	[.VoiceP
		[.\emph{teo} ]
		[.
			[.Voice ]
			[.vP
				[.v
					[.\root{trf} ]
					[.v ]
				]
				[.DP\\\emph{et ha-laxmanja} ]
			]
		]
	]
\xe

Unergative verbs are also possible, as with \emph{rakad} `danced' in~(\nextx). No internal argument is necessary, the event is an activity which can go on over a certain period of time with no concrete telos, and agent-oriented adverbs are possible.
\pex\label{ex:voice-intro-unerg2}
	\a \begingl
		\gla teo \textbf{rakad} ve-rakad ve-rakad (be-mejomanut) (kol ha-boker)//
		\glb Theo danced and-danced and-danced in-skill all the-morning//
		\glft `Theo danced and danced and danced (skillfully) (all morning long).'//
	\endgl
	\a \Tree
	[.VoiceP
		[.\emph{teo} ]
		[.
			[.Voice ]
			[.vP
				[.v
					[.\root{rkd} ]
					[.v ]
				]
			]
		]
	]
\xe

Ditransitive verbs are also possible, as in~(\nextx). We do not need to commit to any specific analysis of ditransitive verbs, so I give a general structure headed by Appl or \emph{p}, a PP-licenser \citep{koopman97,svenonius03,gehrke08phd,wood15springer} I return to in Chapter~\ref{vz:pz}.
\pex\label{ex:voice-intro-ditr2}
	\a \begingl
		\gla teo \textbf{natan} *(le-marsel) et ha-xatif //
		\glb Theo gave \gsc{ACC} to-Marcel the-snack//
		\glft `Theo gave Marcel the treat.'//
	\endgl
	\a \Tree
	[.VoiceP
		[.\emph{teo} ]
		[.
			[.Voice ]
			[.ApplP/\emph{p}P
				[.PP\\\emph{le-marsel} ]
				[.
					[.Appl/\emph{p} ]
					[.vP
						[.v
							[.\root{ntn} ]
							[.v ]
						]
						[.DP\\\emph{et ha-xatif} ]
					]
				]
			]
		]
	]
\xe

Lastly, unaccusative verbs are also possible. The two traditional diagnostics are fronting of the verb and the possibility of using a possessive dative, both evident in~(\nextx). I return to discussing these diagnostics in some more depth when we focus on unaccusative verbs in Chapter~\ref{vz:tnif:nact:unacc}. The tree in~(\anextx b) does not present the final word order, on which see \cite{preminger10}.
\pex\label{ex:voice-intro-unacc2}
	\a \begingl
		\gla \textbf{nafal} le-teo ha-bakbuk//
		\glb fell to-Theo the-bottle//
		\glft `Theo's bottle fell.'//
	\endgl
	\a \Tree
	[.VoiceP
		[.Voice ]
		[.ApplP/\emph{p}P
			[.PP\\\emph{le-teo} ]
			[.
				[.Appl/\emph{p} ]	
				[.vP
					[.v
						[.\root{nfl} ]
						[.v ]
					]
					[.DP\\\emph{he-bakbuk} ]
				]
			]
		]
	]
\xe		

This is how unspecified Voice captures the underspecified nature of the template {\tkal. Since there are no restrictions in the syntax, the root is free to require any interpretation from v and Voice (save for reflexive readings, which are discussed in Chapter~\ref{vz:va:vzva}). The question does arise of what exactly the status of Merge is in such a system, a point of discussion I postpone till Chapter~\ref{chap:i}.

%% check database for breakdown by type
		
		\subsection{Semantics} \label{voice:voice:sem}
The underspecification of this head---and of the resulting template---can be implemented in the semantics using contextual allosemy of Voice. As explained in Chapter~\ref{intro:sketch:allosemy}, the meaning of a functional head can depend on the syntactic and semantic context it appears in, a situation of conditioned allosemy. The formal mechanism allows us to state which meanings arise in which contexts.

Assuming that the causative variant is the elsewhere case, certain roots will be said to require a non-active alloseme of Voice~(\nextx a) and others will be compatible with agentive verbs~(\nextx b):\footnote{Chapter~\ref{chap:aas} contains a brief comparison of contextual allosemy with one alternative, namely postulating homophonous heads. There is little to choose between the two options.}
\pex \denote{Voice} = 
	\a $\lambda$P.P \phantom{agent(x,e)xxx} / \trace~ \{ \root{npl} `\root{\gsc{FALL}}', \root{kpa} `\root{\gsc{FREEZE}}' , \dots \}
	\a $\lambda$x$\lambda$e.Agent(x,e)
%		 & \text{/ \trace~\{ \root{trf} `\root{\gsc{DEVOUR}}', \root{ktb} `\root{\gsc{WRITE}}', \root{ntn} `\root{\gsc{GIVE}}',}\\
%			& \text{\root{ʃal} `\root{\gsc{BORROW}}', \root{\gsc{r\dgs{k}d}} `\root{\gsc{dance}}', \root{\gsc{hlx}} `\root{\gsc{WALK}}', \dots\} }\\
\xe

Other allosemes are also possible, as when \cite{kratzer96}---and in the current formalism, \cite{woodmarantz17}---suggest that Voice can introduce either the Agent or Holder role, depending on the vP it combines with.
		
		\subsection{Phonology} \label{voice:voice:phono}
While this template is unspecified in the syntax and semantics, the lack of overt heads constraining the structure means that it can be \emph{more} marked in the phonology. The intuition is that if there are no overt affixes, the root will have free reign in the phonology. For example, verbal stems are normally longer than one syllable, except for some roots in {\tkal} which flout this restriction:
\ex \emph{ba} `came', \emph{ʃav} `returned', \emph{tsats} `appeared'.
\xe

The phonological markedness of this template has been discussed in contemporary work by \cite{ussishkin05} in terms of Emergence of the Unmarked, with similar observations made by \cite{laks11} and \cite{borer13oup,borer15roots}. The basic paradigm looks as in~(\nextx); for more examples see \cite{schwarzwald08}, \cite{faust12} or \cite{kastner18nllt}.

\ex
\raisebox{-4.5em}{
\begin{small}
	\begin{tabular}{|l||l|l||l|l||l|l|} \hline
		& \multicolumn{2}{c||}{Past} & \multicolumn{2}{c||}{Present} &  \multicolumn{2}{c|}{Future} \\\hline
		& \gsc{M} & \gsc{F} & \gsc{M} & \gsc{F} & \gsc{M} & \gsc{F} \\\hline\hline
		1\gsc{SG} & \multicolumn{2}{c||}{XaYaZ-ti} & XoYeZ & XoYeZ-et & \multicolumn{2}{c|}{e-XYoZ/je-XYoZ}\\\hline
		1\gsc{PL} & \multicolumn{2}{c||}{XaYaZ-nu} & XoYZ-im & XoYZ-ot & \multicolumn{2}{c|}{ni-XYoZ}  \\\hline\hline
		2\gsc{SG} & XaYaZ-ta & XaYaZ-t & XoYeZ & XoYeZ-et & ti-XYoZ & ti-XYeZ-i\\\hline
		2\gsc{PL} & XaYaZ-tem & XaYaZ-ten/tem & XoYZ-im & XoYZ-ot & \multicolumn{2}{c|}{ti-XYeZ-u}\\\hline\hline
		3\gsc{SG} & XaYaZ & XaYZ-a & XoYeZ & XoYeZ-et & ji-XYoZ & ti-XYoZ\\\hline
		3\gsc{PL} & \multicolumn{2}{c||}{XaYZ-u} & XoYZ-im & XoYZ-ot & \multicolumn{2}{c|}{ji-XYeZ-u}\\\hline
	\end{tabular}
\end{small}
}
\xe

What I will assume throughout is that the stem vowels spell out Voice and that affixes spell out higher material (this can be seen as a Mirror Principle effect following directly from cyclic spell out; \citealt{baker85,muysken88,zukoff16nels,kastner18nllt}). The relevant Vocabulary Items for two verbs, \emph{taraf} `devoured' and \emph{katav} `wrote', are given in~(\nextx).
\pex \emph{taraf} `devoured':
	\a Voice \lra~\emph{a,a} / T[Past] \trace
	\a v \lra~(covert)
	\a \root{trf} \lra~\emph{trf}
	\a \root{ktb} \lra~\emph{ktb}
\xe

The final /b/ of \root{ktb} spirantizes to [v], a productive process in the language \citep{temkinmartinzemuellner16,kastner17gjgl,kastner18nllt}. Various other processes might apply, too. Next we will see affixation of the \gsc{3SG.F} suffix -\emph{a} as well as a process of syncope, in which a vowel is deleted (annotated \del{\emph{a}}). Recall that spell-out proceeds cyclically, first within the VoiceP domain and then within the TP domain.
\pex T[Past, 3\gsc{SG.F}]-Voice-\root{trf}, \emph{tarfa} `she wrote'
	\a \root{trf} \lra~\emph{trf}
	\a Voice \lra~\emph{a,a} / T[Past] \trace
	\a \emph{a,a}-\emph{ktb}
	\a At this point the phonology yields:\\
		/taraf/ $\Rightarrow$ \emph{taraf}.
	\a T[Past, \gsc{3SG.F}]-\emph{taraf}
	\a \gsc{3SG.F} \lra~\emph{a} / Past \trace
	\a \emph{a}-\emph{taraf}
	\a The phonology yields:\\
		/a-taraf/ $\Rightarrow$ /tar\del{a}f-a/ $\Rightarrow$ \emph{tarfa}.
\xe

\pex T[Past, 3\gsc{SG.F}]-Voice-\root{ktb}, \emph{katva} `she wrote'
	\a \root{ktb} \lra~\emph{ktb}
	\a Voice \lra~\emph{a,a} / T[Past] \trace
	\a \emph{a,a}-\emph{ktb}
	\a At this point the phonology yields:\\
		/katab/ $\Rightarrow$ /katav/ $\Rightarrow$ \emph{katav}.
	\a T[Past, \gsc{3SG.F}]-\emph{katav}
	\a \gsc{3SG.F} \lra~\emph{a} / Past \trace
	\a \emph{a}-\emph{katav}
	\a The phonology yields:\\
		/a-katav/ $\Rightarrow$ /kat\del{a}v-a/ $\Rightarrow$ \emph{katva}
\xe

How exactly these exponents are concatenated will not be derived here; in \cite{kastner18nllt} I give full derivations within an OT grammar. Importantly, the derivation proceeds modularly and cyclically: first the syntax builds up structure, then VI inserts exponents, then the phonology takes over and derives the most harmonic surface forms. But for future tense forms like \emph{ti-xtov} `she will write', we will require a different contextual allomorph for Voice such as that in~(\nextx b).
\ex \label{vi:voice} Voice \lra $\begin{cases}
		\text{a.~\emph{a},\emph{a}} & \text{/ T[Past] \trace}\\
		\text{b.~\emph{o}} & \text{/ T[Fut] \trace}\\
		\end{cases}$
\xe

Abstracting away from the spell-out of specific inflectional variants within a given template, a general schematic can be stated as in~(\nextx b). In Section~\ref{voice:va} below I introduce a modifier which constrains both the semantics and phonology of Voice, giving us the possibility of~(\nextx a).
\pex Voice {\lra}
	\a {\tpie} / {\trace} {\va}
	\a {\tkal}
\xe

Again, the spell-out rules in~(\lastx) provide a crude approximation of how Voice is handled at PF, but it is important to keep in mind that there is no one ``suffix'' {\tkal} which spells out this head. Rather, there is an intricate morphophonological system of inflectional variants which needs to be taken into account. With that in mind, my focus in this book will be more in setting up basic schemas like those in~(\lastx), whereby different syntactic configurations---mostly reflecting different values of Voice---trigger different templatic shapes. The templates themselves, then, have no independent status in the theory and serve only as useful morphophonological mnemonics.

	\subsection{Interim summary}
The template {\tkal} is unrestricted in terms of argument structure: verbs with this morphological marking might be unergative, unaccusative, monotransitive or ditransitive, all depending (idiosyncratically) on the underling root. Yet the morphophonology is consistent across all possible verbs in this template, regardless of their syntax and semantics.

In contrast to the traditional Voice head which introduces an external argument, the Voice head I use to capture this behavior is unspecified with regards to the EPP feature [D]. This head does not place any constraints on its specifier. As a result, there are no restrictions on the argument structure of verbs which are derived using unspecified Voice. Since every Hebrew verb must be instantiated in one of the seven verbal templates, the appearance of Voice can be traced in the morphology as the template {\tkal}.

In other frameworks, \cite{doron03} does not introduce any special heads in order to account for verbs in {\tkal}. \cite{borer13oup,borer15roots} takes {\tkal} to be a verbalized root, without functional material attaching to it. The two main reasons for this are the wide range of nominalizations possible in this template and the idiosyncratic phonology. I will return to nominalizations in Chapter~\ref{passn:n}, after covering the other variants of Voice, but all three frameworks are compatible in their treatment of the {\tkal}: all allow for {\tkal} to be as idiosyncratic as it needs to be, in the phonology as in the syntax.

\section{\tpie: Descriptive generalizations} \label{voice:tpie}
The next template to be examined is {\tpie}. As can be seen from the notation, there are no unique affixes to this template, but the stem vowels are different than in~{\tkal}. In addition, the middle root consonant Y blocks the process of spirantization mentioned briefly earlier. I borrow the non-syllabic diacritic \dgs{Y} to indicate this.

In this section I lay out the basics of verbs in {\tpie}, basically confirming the generalizations established by \cite{doron03}. In terms of possible constructions, verbs in this template are always active, and what's more, they are agentive in a weak sense which I will identify informally. In terms of alternations, they sometimes provide ``intensive'' alternants of verbs in {\tkal}, again in a way I will explain below. This secion provides an overview of the data; the next section gives a formal analysis, based on the head Voice we have just seen and an agentive modifier, {\va}.

First, let me clarify the terminology used here. I take \emph{causers} to be any kind of external argument. \emph{Agents} are a subset of causers, typically understood as animate and volitional causers. In the discussion below, ``agent'' will be used more or less interchangeably with ``actor'', ``direct cause'', and the other labels used in the literature. So, throughout this book, when I say \emph{agent} what I mean is a stronger type of causer, a distinction which as far as I can see is vague precisely because it is rooted in the semantics of various kinds of events rather than in the syntax. The discussion which follows should make these distinctions clear.

To understand the syntax-semantics of {\tpie}, consider the pairs in~(\ref{ex:va-piel1}). In~(\nextx a), both agents and causers are possible. In~(\ref{ex:va-piel1}b) only the agent is possible. The (a) example has the verb in {\tkal}, the (b) example in {\tpie}.

\pex \label{ex:va-piel1}
	\a \begingl
		\gla \emph{\{}\cmark~ha-jeladim / \cmark~ha-tiltulim ba-argaz\emph{\}} \glemph{ʃavr}-u et ha-kosot//
		\glb \phantom{\{\cmark~}the-children {} \phantom{\cmark~}the-shaking in.the-box \glemph{broke.\gsc{SMPL}}-\gsc{PL} \gsc{ACC} the-glasses//
		\glft `\{The children / Shaking around in the box\} broke the glasses.'//
		\endgl
	
	\a \begingl
		\gla \emph{\{}\cmark~ha-jeladim / \xmark~ha-tiltulim ba-argaz\emph{\}} \glemph{ʃibr}-u et ha-kosot//
		\glb \phantom{\{\cmark~}the-children {} \phantom{\xmark~}the-shaking in.the-box \glemph{broke.\gsc{INTNS}}-\gsc{PL} \gsc{ACC} the-glasses//
		\glft `\{The children / *Shaking around in the box\} broke the glasses to bits.' \trailingcitation{\citep[20]{doron03}}//
		\endgl
\xe

%\pex \label{ex:va-piel2}
%	\a \begingl
%		\gla ha-xelbon \glemph{jatsar} ba-guf nogdanim//
%		\glb the protein produced.\gsc{SMPL} in the body antibodies//
%		\glft `The protein produced antibodies in the body.'//
%	\endgl
%	
%	\a \begingl 
%	\gla ha-xelbon \glemphf{jitser} ba-guf nogdanim//
%	\glb the protein produced.\gsc{INTNS} in the body antibodies//
%	\glft `The protein manufactured antibodies in the body.' \trailingcitation{\citep[21]{doron03}}//
%	\endgl
%\xe

What other readings do verbs in {\tpie} have? This template is traditionally called the ``intensive'' because of alternations such as those above and in~(\nextx a--c), but it can also house pluractional verbs, (\nextx d--f), and various others, (\nextx g--i):
\ex\label{ex:voice:piel-meanings}Pretheoretical classification of some verbs in \tpie:\\
	\begin{tabular}{lll|ll|ll}
	& & & \multicolumn{2}{c|}{\tkal} &  \multicolumn{2}{c}{\tpie}\\\hline
	\multirow{3}{*}{Intensive} & a.& \root{ʃbr} & ʃavar & `broke' & ʃiber & `broke to pieces'\\
		& b.& \root{jtsr} & jatsar & `produced' & jitser & `produced'\\
	    & c.& \root{'kl} & axal & `ate' & ikel & `corroded, consumed'\\\hline

 	\multirow{3}{*}{Pluractional} & d.& \root{hlx} & halax & `walked' & hilex & `walked around'\\
 	    & e.& \root{r\dgs{k}d} & rakad & `danced' & riked & `danced around'\\
  	    & f.& \root{\dgs{k}fts} & kafats & `jumped' & kipets/kiftsets & `jumped around'\\\hline

  		\multirow{3}{*}{Non-derived} & g. & \root{tps} & \multicolumn{2}{c|}{---} & tipes & `climbed'\\
	    & h. & \root{ltf} & \multicolumn{2}{c|}{---} & litef & `petted'\\
		  & i. & \root{\dgs{k}bl} & \multicolumn{2}{c|}{---} & kibel & `received'\\
	\end{tabular}
\xe

In all cases, the verbs are active: either unergative or transitive. And in all cases, the external argument is agentive. In some examples this contrast is clear: a storm cannot ``intensively'' break a window to bits.
\pex
	\a \begingl
		\gla ha-sufa \glemph{ʃavra} et ha-xalon//
		\glb the-storm broke.\gsc{SMPL} \gsc{ACC} the-window//
		\glft `The storm broke the window.'//
		\endgl
	\a \begingl
		\gla\ljudge{*}ha-sufa \glemph{ʃibra} et ha-xalon (le-xatixot)//
		\glb the-storm broke.\gsc{INTNS} \gsc{ACC} the-window to-pieces//
		\glft (int.~`The storm broke the window to pieces')//
		\endgl
\xe

But as \citet[21]{doron03} points out, even inanimate entities can be the subjects of verbs in {\tpie}. She gives the following pair of examples. As she put it: ``\textit{The simple verb \emph{produce} in \emph{[(\nextx a)]} has a reading where the protein is the trigger for antibodies being produced. The intensive-template verb in \emph{[(\nextx b)]} can only be interpreted such that the protein actually participates in the production process itself}.''
\pex
	\a \begingl
		\gla ha-xelbon \glemph{jatsar} ba-guf nogdanim//
		\glb the-protein produced.\gsc{SMPL} in.the-body antibodies//
		\glft `The protein produced antibodies in the body.'//
		\endgl
	\a \begingl
		\gla ha-xelbon \glemph{jitser} ba-guf nogdanim//
		\glb the-protein produced.\gsc{INTNS} in.the-body antibodies//
		\glft `The protein produced antibodies in the body.'//
		\endgl
\xe

The generalizations for {\tpie}, then, are as follows:
\hammer{
\pex \label{ex:gen-tpie}\textbf{Generalizations about {\tpie}}
	\a \textbf{Constructions:} Verbs appear in active (transitive/unergative) configurations.\\
		Readings are weakly agentive.
	\a \textbf{Alternations:} When alternating with {\tkal}, provides a more ``intensive'' or agentive version.
\xe
}

Making reference to ``weak agentivity'' and ``intensive'' readings is a fine semantic line to tread. In what follows I review what I think are some similar phenomena across languages and empirical domains, before turning to the formal analysis.

	\subsection{Agentive modifiers crosslinguistically} \label{voice:tpiel:act}
%As far as the semantics is concerned, the difference in possible interpretations between~(\lastx a) and~(\lastx b) reduces to whether or not overt {\va} is there to force an agentive reading. \cite{doron03} proposed that this modifier carries the semantics of Action, which is slightly weaker than that of Agent.
A number of recent works on argument and event structure have identified a component of meaning that can be broadly described as agentive, volitional, or a ``direct cause''. In their study of animacy in English, Italian, Greek and Russian, \cite{folliharley08} considered a range of data in which the acceptability of an external argument depends on whether it is \emph{teleologically capable} of causing the event (as opposed to an agency or animacy restriction). Their review identified cases of sound emission, possession, causation, permission and consumption where the licensing conditions on external arguments cannot be understood in terms of animacy, but in terms of whether the internal properties of the external argument can bring about the relevant event.

For example, in Italian causatives, inanimate causers vary with respect to how acceptable they are. A branch is fine, but a storm is not. The explanation is that the branch is a direct causer but the storm is not a proximate enough cause; it is not teleologically capable.\footnote{See \cite{irwin18tlr} for an explication of some teleological properties in terms of body parts.}
\pex
	\a \begingl
		\gla Il ramo ha rotto la finestra//
		\glb the branch has broken the window//
		\glft `The branch broke the window.'//
		\endgl
	\a \begingl
		\gla \ljudge{?}Il vento ha rotto la finestra//
		\glb the wind has broken the window//
		\glft `The wind broke the window.'//
		\endgl
	\a \begingl
		\gla \ljudge{\#}Il temporale ha rotto la finestra//
		\glb the storm has broken the window//
		\glft (int.~`The storm broke the window')\trailingcitation{(Italian, \citealt[195]{folliharley08})}//
		\endgl
\xe		

In other cases, animacy is still the relevant factor within the teleological capability of the relevant argument. Italian \emph{fare}-causatives require the causee to be animate, as in~(\nextx). Similar considerations are familiar from control phenomena as discussed in a range of work from \cite{farkas88} to \cite{zu18phd}.
\pex
	\a \begingl
		\gla Gianni ha fatto rompere la finestra a Maria//
		\glb John has made break the window to Maria//
		\glft `John had Maria break the window.'//
		\endgl
	\a \begingl
		\gla \ljudge{\#}Gianni ha fatto rompere la finestra al ramo//
		\glb John has made break the window to.the branch//
		\glft (int.~`John had the branch break the window)'\trailingcitation{\citep[196]{folliharley08}}//
		\endgl
\xe	

A further dissociation of animacy from agentivity (in the current sense) comes from a study of manner and causation in English by \cite{beaverskoontzgarboden12}, who showed that an animate cause is still not necessarily an agent. The term they use is \emph{actor}, which is employed to discuss events in which an animate causer is or is not responsible for the consequences of its act. For them, causation is compatible with negligence but actorhood (agentivity) is not. That is why even the animate causer in~(\nextx) is not an actor (cf.~\citealt{rappaporthovav14}) :
\ex Kim broke my DVD player, but didn’t move a muscle—rather, when I let her borrow it a disc was spinning in it, and she just let it run until the rotor gave out!\trailingcitation{\citep[347]{beaverskoontzgarboden12}}
\xe

What I would like to highlight is that these kinds of reading can also be triggered by particular morphemes. Moving on to a different empirical domain, recent studies of external arguments in nominalization \citep{sichel10n,alexiadouetal13,ahdout18nom} similarly differentiate agentivity from \emph{direct causation}. The external arguments of argument-structure nominalizations are often taken to exhibit \emph{agent exclusivity}, whereby only agents are possible (see Chapter~\ref{passn:n} for additional background). Examples~(\nextx)--(\anextx) show a typical instantiation of this effect, whereby the animate agent can serve as the external argument of a nominalization, (\nextx), but an inanimate cause cannot, (\anextx).
\pex
	\a \textbf{The Allies} separated East and West Germany.
	\a \textbf{The Allies'} separation of East and West Germany.
\xe
\pex
	\a \textbf{The cold war} separated East and West Germany.
	\a \ljudge{\#}\textbf{The cold war's} separation of East and West Germany.
\xe

\cite{sichel10n} points out, however, that animacy is not always the relevant factor, as observed already in different ways by \cite{pesetsky95} and \cite{marantz97}. The core of her argument is based on natural causers, which are compatible with some nominalizations but not with others (the following judgments are hers). She takes this to mean that direct causation is insufficient if it lacks direct participation.
\pex
	\a The hurricane's \textbf{destruction} of our crops.
	\a The hurricane's \textbf{devastation} of ten coastal communities in Nicaragua
\xe
\ex \ljudge{\#} The approaching hurricane's \textbf{justification} of the abrupt evacuation of the inhabitants
\xe

\cite{alexiadouetal13,alexiadouetal13jcgl} build on \citeauthor{sichel10n}'s proposal and propose that depending on the language and construction, the restriction can depend on either agentivity or direct participation.

Syntactic environments other than nominalization can give rise to similar effects. There are cases where a specific, overt morpheme can be identified as triggering these agentivity-like effects. I mention two here, before we return to a similar phenomenon in Hebrew which I attribute to the element {\va}.

In their studies of the prefix \emph{afto-} in Greek, \cite{alexiadouafto} and \cite{spathasetal15} identified it as an \emph{anti-assistive} modifier, triggering agentive readings regardless of syntactic category, (\nextx).
\pex Agentive readings of \emph{afto-} \citep[61]{alexiadouafto}:
	\a \emph{afto-katastrefome} `self-destroy' (v.)
	\a \emph{afto-kritiki} `self-criticism' (n.)
	\a \emph{afto-didaktos} `self-educated' (a.)
\xe

Given its meaning and its similarity to an analytic paraphrase, (\nextx), \cite{spathasetal15} propose the denotation in~(\anextx).
\pex \citep[63--64]{alexiadouafto}
	\a \begingl
		\gla O Janis katigori-\textbf{te}//
		\glb the John accuses-\gsc{NACT}//
		\glft `John is accused.'//
	\endgl
	\a \begingl
		\gla O janis katigori \textbf{ton} \textbf{eafto} \textbf{tu}//
		\glb the John accuses the self his//
		\glft `John accuses himself.'//
	\endgl
	\a \begingl
		\gla O janis \textbf{afto}-katigori-\textbf{te}//
		\glb the John self-accuse-\gsc{NACT}//
		\glft `John accuses himself.'//
	\endgl
\xe
\ex \denote{\emph{afto}$_{\text{anti-assistive}}$} = $\lambda$f$\lambda$y$\lambda$e.f(y,e) \& $\forall$e'$\forall$x.(e'$\le$e \& Agent(x,e')) $\rightarrow$ x=y \hfill \citep[1335]{spathasetal15}
\xe

Additional elaboration on these complex constructions can be found in these work and the previous work they city. The technical conclusion is that \emph{afto-} is an adjunct which attaches to Voice, triggering agentive meaning.

A comparable (although still distinct) phenomenon can be found in Tamil, were the suffix -\emph{koɭ} adds ``affective semantics'' which are otherwise hard to pin down. \cite{sundaresanmcfadden17} discuss the difference in meaning between verbs with and without -\emph{koɭ} as one of ``affectedness'' in a way that can be exemplified using the examples in~(\nextx). With -\emph{koɭ}, the event affects the agent.
\pex
	\a \begingl
	\gla Mansi paal- æ uutt- in- aaɭ//
	\glb Mansi milk \gsc{ACC} pour.\gsc{TR} Past \gsc{3SG.F}//
	\glft `Mansi poured the milk.'//
	\endgl

	\a 	\begingl
	\gla Mansi paal- æ uutti- \textbf{kko-} ɳɖ- aaɭ//
	\glb Mansi milk \gsc{ACC} pour.\gsc{TR} \gsc{koɭ} Past \gsc{3SG.F}//
	\glft `Mansi poured the milk for herself.' (Reading 1)\\
		`Mansi poured the milk on herself.' (Reading 2)//
	\endgl
\xe

As \citet[165]{sundaresanmcfadden17} put it, ``\emph{the end result of some event comes back to affect one of the arguments of that same event}'', where the relevant argument is the external argument if there is one, otherwise the internal one (as with unaccusatives). In any case, the semantics of -\emph{koɭ} are such that it forces some kind of agent-oriented reading at least in clauses with external arguments.

Where does this cross-linguistic review leave us? The pretheoretical picture which emerges from these works is that natural language has a way of making a fine-grained distinction between different degrees of ``direct participation'' or agentivity. To the extent that this triggering of agentive semantics is the same phenomenon across languages---and this is an assumption I am making here---it seems highly unlikely that it has the same syntactic underpinnings in all of these cases. A more appropriate explanation would be given in semantic terms (that is, within the denotation of certain morphemes) or in pragmatic terms (world knowledge). As alluded to above, it seems clear that in at least some cases the effect is clearly grammatical, i.e.~should be encoded in the semantics of individual morphemes directly, as with agent exclusivity in nominalizations, the anti-assistive modifier in Greek and the affective modifier in Tamil. Such a proposal for Hebrew follows.

\section{Agentive modification: \va} \label{voice:va}
In this section I introduce another syntactic primitive, the agentive modifier {\va}. Strictly speaking, this modifier is not part of the theory of trivalent Voice. The reason it is introduced early on in this book is because it is necessary to capture the full empirical picture; specifically, it will return in the discussion of {\vz} in Chapter~\ref{chap:vz}. Unspecified Voice and the template {\tkal} have already been addressed, but the behavior of the template {\tpie} indicates that we need to account for additional forms.

In order to explain the behavior of verbs {\tpie} I propose to use a special root {\va}, which enforces agentive (or weakly agentive) readings.\footnote{\cite{doron03} uses a syntactic head $\iota$; see Chapter \ref{vz:others:ed} on some differences between the theories.} I assume that {\va} attaches to the verbal spine at the vP level, concretely to the verb, thereby triggering the agentive alloseme of Voice (following \citealt{doron03,doron14adj}). The morphophonology produces the templates {\tpie} and {\thit}, as I return to momentarily. Here is the basic proposal, followed by a deep dive into each part (syntax, semantics and phonology).
\pex {\va}:
	\a A modifier which attaches to vP.
	\a \denote{Voice} = $\lambda$x$\lambda$e.e \& \text{Agent}(x,e) / \trace~\va
	\a Voice {\lra} {\tpie} / {\trace} {\va}
	\a {\vz} {\lra} {\thit} / {\trace} {\va}
\xe

As a root, this element has phonological and semantic content but no syntactic features or requirements. Not much hinges on whether this element is a root or a functional head in this language; since it has no syntactic influence, but combines predictable phonology with semantics that can be difficult to characterize formally, it behaves like any other root.\footnote{For these reasons I do not consider it to be a ``flavor'' of v, for example.} The question of what other such ``underspecified'' roots might exist in natural language remains an open one for further crosslinguistic research.

	\subsection{Syntax} \label{voice:va:syn}
A transitive verb like \emph{pirek} `dismantled' has the basic structure in~(\nextx a), and an unergative verb like \emph{riked} `danced around' has the basic structure in~(\nextx b).
\pex 
	\a Transitive {\tpie}:\\
	\Tree
	[.VoiceP
		[.DP ]
		[.
			[.Voice ]
			[.vP
				[.{\va} ]
				[.vP
					[.v
						[.\root{pr\dgs{k}} ]
						[.v ]
					]
					[.DP ]
				]
			]
		]
	]

	\a Unergative {\tpie}:\\
	\Tree
	[.VoiceP
		[.DP ]
		[.
			[.Voice ]
			[.vP
				[.{\va} ]
				[.vP
					[.v
						[.\root{r\dgs{k}d} ]
						[.v ]
					]
				]
			]
		]
	]
\xe

The agentive modifier forces an agentive reading, which necessarily requires either a transitive or unergative structure. This much captures the syntactic distribution of {\tpie}.

Consider what this means in terms of alternations. Returning to the examples in~(\ref{ex:voice:piel-meanings}), we saw an ``intensive'' alternation between \emph{ʃavar} `broke' and \emph{ʃiber} `broke to pieces'. Assuming a layering view of argument structure \citep{layering15}, we first build up a minimal vP consisting of a breaking event:
\ex \Tree
[.vP
	[.v
		[.\root{ʃbr} ]
		[.v ]
	]
	[.DP ]
]
\xe

What happens next? The grammar has two options. It can either merge Voice (\nextx a), in which case we get the verb in {\tkal}, or it can merge {\va} and then Voice (\nextx b), in which case we get the verb in {\tpie}.

\pex
	\a \Tree
	[.
		[.Voice ]
		[.vP
			[.v
				[.\root{ʃbr} ]
				[.v ]
			]
			[.DP ]
		]
	]

	\a \Tree
	[.
		[.Voice ]
		[.vP
			[.{\va} ]
			[.vP
				[.v
					[.\root{ʃbr} ]
					[.v ]
				]
				[.DP ]
			]
		]
	]
\xe

As noted in all of the major works on Hebrew morphology, alternations are not always the norm: there is no guarantee that a verb in {\tkal} will alternate with one in {\tpie}, as many verbs in {\tkal} have no counterpart in {\tkal}. This property is idiosyncratic and must be listed with every root. But when {\tkal} and {\tpie} do alternate, this is how: if a given root is instantiated in both templates, then the {\tpie} version will always be an ``intensive'', agentive version of the {\tkal} verb, since {\tpie} is the spell out of adding a {\va} layer to the basic event which otherwise would be spelled out as {\tkal}. 

The derivation of verbs in {\tpie} which do not alternate with {\tkal} is identical. For (\nextx), we first build up the minimal vP, then attach {\va}, and then attach the external argument. The fact that the minimal vP cannot combine with Voice directly must be listed with the root, in whatever way regulates which functional heads can appear with which root. Now the meaning of the root is chosen by {\va}, rather than by v (since there is no verb in {\tkal}) or Voice (since {\va} is closer to the root, \citealt{arad03,marantz13,elenasamioti14}).
\pex
   \a \begingl
    \gla \underline{ha-xom} \textbf{ʃibeʃ} et ha-medidot//
    \glb the-heat disrupted.\gsc{INTNS} \gsc{ACC} the-measurements//
    \glft `The heat messed up the measurements.//
  \endgl
	\a \Tree
	[.VoiceP
		[.DP\\{\emph{ha-xom}} ]
		[.
			[.Voice\\{\emph{i,e}} ]
			[.vP
				[.{\va} ]
				[.vP
					[.v
						[.\root{ʃbʃ}\\{\emph{ʃbʃ}} ]
						[.v ]
					]
					[.DP\\{\emph{et ha-medidot}} ]
				]
			]
		]
	]
\xe

Again, what ``intensive'' means is left intentionally vague. A few options are sketched next, after a technical aside about the height of attachment for {\va}.

		\subsubsection{Height of attachment} \label{voice:va:syn:wonk}
In principle, {\va} could be argued to adjoin either to v/vP or to Voice. The benefit of adjoining it to v is that the alternations between {\tkal} and {\tpie} follow cleanly, as do those between {\tpie} and {\thit}. Here is a preview of what this looks like, to be further explore in the next chapter. Both causative \emph{pirek} `dismantled' and anticausative \emph{hitparek} are built from the basic vP in~(\nextx a). If Voice is merged, we get causative \emph{pirek} in {\tpie} (\nextx b). If {\vz} is merged, we get anticausative \emph{hitparek} in {\thit} (\nextx c).
\pex
	\a \Tree
		[.vP
			[.{\va} ]
			[.vP
				[.v
					[.\root{pr\dgs{k}} ]
					[.v ]
				]
				[.DP ]
			]
		]
	\a \emph{pirek} `dismantled'\\
		\Tree
		[.VoiceP
			[.DP ]
			[.
				[.{Voice\\\emph{i,e}} ]
				[.vP
					[.{\va} ]
					[.vP
						[.v
							[.\root{pr\dgs{k}} ]
							[.v ]
						]
						[.DP ]
					]
				]
			]
		]
		\a \emph{hitparek} `fell apart'\\
			\Tree
			[.VoiceP
				[.DP ]
				[.
					[.{\vz\\\emph{hit-,a,e}} ]
					[.vP
						[.{\va} ]
						[.vP
							[.v
								[.\root{pr\dgs{k}} ]
								[.v ]
							]
							[.DP ]
						]
					]
				]
			]
\xe

In previous work \citep{kastner16phd,kastner17gjgl,kastner18nllt} I assumed that {\va} modifies Voice, and not v as it does here. There were three reasons for this. The first was that placing {\va} between Voice and a higher element such as T correctly derives certain allomorphic patterns under the strict linear adjacency hypothesis for contextual allomorphy \citep{embick10,marantz13}, as developed in \cite{kastner18nllt}. While I am fond of this argument, much current work argues that this restriction needs to be weakened (see e.g.~\citealt{kastnermoskal18,choiharley19}). The second is that adjoining {\va} to Voice renders it similar to Greek \emph{afto}. However, it is not crucial for the theory that these two elements merge in different locations in different languages. The third is that since {\va} influences the interpretation of the external argument, adjoining it to Voice seemed most appropriate. Yet it is clear that agentive semantics can be generated low: verbs like \emph{murder} and \emph{devour} are strongly agentive \citep{haspelmath93,unaccusativity95,marantz97,layering15}, a requirement which originates within the vP (at the root). For these reasons, I now think that {\va} adjoins to v, although there are no clinching arguments either way. I thank Odelia Ahdout for discussing this shift with me; see \cite{ahdout19phd} for additional benefits of adjoining {\va} to v in the domain of nominalization, some of which I recap in Chapter~\ref{passn:n}.

	\subsection{Semantics} \label{voice:va:sem}
Given that it is a root, it is difficult to assign a semantics to {\va} without a theory of root semantics. What we can do is see its effects on the external argument, formalized as follows:
%\ex \denote{Voice} = $\lambda$x$\lambda$e.e \& \text{Agent}(x,e) / \trace~\va
%\xe
\pex \denote{Voice} = 
	\a $\lambda$P.P \phantom{agent(x,e)xxx} / \trace~ \{ \root{npl} `\root{\gsc{FALL}}', \root{kpa} `\root{\gsc{FREEZE}}' , \dots \}
	\a $\lambda$x$\lambda$e.Agent(x,e) or $\lambda$x$\lambda$e.Cause(x,e)
	\a $\lambda$x$\lambda$e.e \& \text{Agent}(x,e) / \trace~\va
%		 & \text{/ \trace~\{ \root{trf} `\root{\gsc{DEVOUR}}', \root{ktb} `\root{\gsc{WRITE}}', \root{ntn} `\root{\gsc{GIVE}}',}\\
%			& \text{\root{ʃal} `\root{\gsc{BORROW}}', \root{\gsc{r\dgs{k}d}} `\root{\gsc{dance}}', \root{\gsc{hlx}} `\root{\gsc{WALK}}', \dots\} }\\
\xe
While this formalization aims to be explicit, I have taken a few shortcuts. As already argued for by \cite{layering15}, it is the vP which provides the causative component, not Voice. The formalization in~(\lastx) is meant to indicate that both Causes and Agents are compatible with Voice, but that only Agents are possible once {\va} is in the structure. 

Let us expand the empirical view a bit more: what readings does {\va} make available? Some examples are given in~(\ref{ex:voice:piel-meanings2}), repeated from~\ref{ex:voice:piel-meanings}. While {\tpie} is traditionally called the ``intensive'' template, it can also house pluractional verbs, (\nextx d--f), and various others which do not alternate with forms in {\tkal}, (\nextx g--i):
\ex\label{ex:voice:piel-meanings2}Pretheoretical classification of some verbs in \tpie:\\
	\begin{tabular}{lll|ll|ll}
	& & & \multicolumn{2}{c|}{\tkal} &  \multicolumn{2}{c}{\tpie}\\\hline
	\multirow{3}{*}{Intensive} & a.& \root{ʃbr} & ʃavar & `broke' & ʃiber & `broke to pieces'\\
		& b.& \root{jtsr} & jatsar & `produced' & jitser & `produced'\\
	    & c.& \root{'kl} & axal & `ate' & ikel & `corroded, consumed'\\\hline

 	\multirow{3}{*}{Pluractional} & d.& \root{hlx} & halax & `walked' & hilex & `walked around'\\
 	    & e.& \root{r\dgs{k}d} & rakad & `danced' & riked & `danced around'\\
  	    & f.& \root{\dgs{k}fts} & kafats & `jumped' & kipets/kiftsets & `jumped around'\\\hline

  		\multirow{3}{*}{Non-derived} & g. & \root{tps} & \multicolumn{2}{c|}{---} & tipes & `climbed'\\
	    & h. & \root{ltf} & \multicolumn{2}{c|}{---} & litef & `petted'\\
		  & i. & \root{\dgs{k}bl} & \multicolumn{2}{c|}{---} & kibel & `received'\\
	\end{tabular}
\xe

The pluractional readings and underived verbs have potentially interesting theoretical consequences, which will be touched on here before moving on to the phonological contribution of {\va}.

		\subsubsection{Pluractionality} \label{voice:va:sem:plural}
One possible way to describe the semantics of {\va} is by extended reference to pluractionality. The intuition as is follows. Assume that {\va} is a pluractional (and perhaps also agentive) affix. Building on recent work by \cite{henderson12phd,henderson17nllt}, pluractionality can be seen as a way of pluralizing an event. This pluralization can hold spatially as well as temporally. For the ``intensive'' forms in~(\ref{ex:voice:piel-meanings}a--c), the underlying verb in {\tkal} has a direct object. The corresponding pluralized events in {\tpie} can be individuated with respect to the direct objects, e.g.~many broken pieces in (\ref{ex:voice:piel-meanings}a) or many different simultaneous corrosions of parts of the material's surface in (\ref{ex:voice:piel-meanings}c). This extension is admittedly less obvious for ``production'' in~(\ref{ex:voice:piel-meanings}b). \cite{greenberg10} makes a similar claim for verbs in {\tpie} that are derived from reduplicated roots.

For the ``pluractional'' forms in (\ref{ex:voice:piel-meanings}d)--(\ref{ex:voice:piel-meanings}f), the underlying verbs in {\tkal} are unergative. The pluralizing operation has no direct object to operate on, and so I would suggest that it pluralizes the spatio-temporal event itself in {\tpie}.

Lastly, in (\ref{ex:voice:piel-meanings}g)--(\ref{ex:voice:piel-meanings}i) there is no underlying form and hence nothing to pluralize. The resulting verbs are still agentive but not necessarily pluractional.

%The database of verbs we are employing contains over 900 forms in {\tpie}, so this line of inquiry faces a serious amount of empirical corroboration.
This way of thinking about {\tpie} is speculative at this point. A number of potential counterexamples can be conjured up fairly easily. These are cases where the alternation does not plausibly result in a plural event:
\pex
	\a \emph{lamad} `learned' $\sim$ \emph{limed} `taught'
	\a \emph{ratsa} `wanted' $\sim$ \emph{ritsa} `satisfied'
\xe

In the examples in~(\lastx) the event does not entail change of state, unlike with breaking and eating/corroding. So perhaps there is a tripartite division of roots to be made, as follows:
\pex
	\a \textbf{Other-oriented roots (change of state)} such as \root{\gsc{BREAK}} and \root{\gsc{PRODUCE}}: pluralization of the object.
	\a \textbf{Activity roots} or \textbf{self-oriented roots} such as \root{\gsc{RUN}} and \root{\gsc{JUMP}}: pluralization of the spatio-temporal aspects of the event.
	\a \textbf{Other cases:} no pluralization.
\xe

Since our current focus is not on the lexical semantics of root classes and how they integrate into the syntax, I will leave proper testing of the hypothesis in~(\lastx) for future work. Evaluating this proposal will need to proceed along the lines laid out above, testing whether each root instantiated in this template does indeed fit into one of the three cases in~(\lastx).

		\subsubsection{Underived forms} \label{voice:va:sem:underived}
A number of verbs in {\tpie} stretch the notion of ``agentivity'' to the point where even a weak definition is no longer tenable. In the examples in~(\nextx), the verb can hardly be described as agentive since the subject is inanimate, while in~(\anextx) the subject is animate but non-volitional. These verbs are compatible with agentive subjects as well, but clearly do not require them.
\pex
  \a \begingl
    \gla \underline{ha-midgam} \textbf{ʃikef} et totsot ha-emet//
    \glb the-poll reflected.\gsc{INTNS} \gsc{ACC} results.\gsc{CS} the-truth//
    \glft `The polls (correctly) reflected the results.'//
  \endgl
    
  \a \begingl
    \gla be-ritsa axat \underline{ha-ʃaon} ʃel garmin kimat \textbf{diek} kaaʃer hetsig stia kimat xasrat maʃmaut ʃel axuz ve-ktsat//
    \glb in-run one the-watch of Garmin almost was.accurate.\gsc{INTNS} when showed deviation almost devoid.of meaning of percent and-little//
    \glft `In one run, the Garmin watch was precise as it showed an almost insignificant deviation of just over one percent.'\trailingcitation{\url{www.haaretz.co.il/sport/active/.premium-1.2309128}}//
  \endgl
\xe

\ex \begingl
  \gla \underline{hu} \textbf{kibel} maka xazaka ba-regel//
  \glb he received.\gsc{INTNS} hit strong in.the-leg//
  \glft `He got hit hard in the leg.'//
  \endgl
\xe

In these examples an external argument is still required, regardless of whether it can felicitously be called an agent or not. What these examples show is that a rigid denotation of {\va} is difficult to specify, beyond some general notion of a direct cause. I believe it is significant, though, that the verbs in~(\blastx)--(\lastx) do not have correspondents in {\tkal}. That is, they are not derived by adding {\va} to an existing VoiceP or via some process of causativization: \emph{ʃikef} $\nless$ *\emph{ʃakaf}, \emph{diek} $\nless$ *\emph{dajak}, \emph{ʃibeʃ} $\nless$ *\emph{ʃabaʃ}, and \emph{kibel} $\nless$ *\emph{kabal}. They would fit with the underived group of~(\ref{ex:voice:piel-meanings2}: derived when {\va} selects the alloseme of the root directly without having to agentivize an event in vP/{\tkal}. If {\va} really is a root rather than a functional head, its partially unpredictable contributions to the meaning of the verb are not unexpected.

	\subsection{Phonology} \label{voice:va:phono}
The morphophology of {\tpie} consist of two parts that distinguish it from other templates: different stem vowels and the way it bleeds a regular phonological process of spirantization. In Modern Hebrew, /p/, /b/ and /k/ spirantize to [f], [v] and [x] postvocalically \citep{adam02,temkinmartinez08wccfl,gouskova12nllt}, a process that applies to nonce words as well \citep{temkinmartinzemuellner16}. An example of this process was seen above in~(\ref{ex:va-piel1}a--b), where /b/ spirantizes to [v] after a vowel except if {\va} is also in the structure. The inflectional paradigm for {\tpie} in three tenses is given in~(\nextx).

\ex
\raisebox{-4.5em}{
\begin{small}
	\begin{tabular}{|l||l|l||l|l||l|l|} \hline
		& \multicolumn{2}{c||}{Past} & \multicolumn{2}{c||}{Present} &  \multicolumn{2}{c|}{Future} \\\hline
		& \gsc{M} & \gsc{F} & \gsc{M} & \gsc{F} & \gsc{M} & \gsc{F} \\\hline\hline
		1\gsc{SG} & \multicolumn{2}{c||}{Xi\dgs{Y}aZ-ti} & me-Xa\dgs{Y}eZ & me-Xa\dgs{Y}eZ-et & \multicolumn{2}{c|}{a-Xa\dgs{Y}eZ/ye-Xa\dgs{Y}eZ}\\\hline
		1\gsc{PL} & \multicolumn{2}{c||}{Xi\dgs{Y}aZ-nu} & me-Xa\dgs{Y}Z-im & me-Xa\dgs{Y}Z-ot & \multicolumn{2}{c|}{ne-Xa\dgs{Y}eZ}  \\\hline\hline
		2\gsc{SG} & Xi\dgs{Y}aZ-ta & Xi\dgs{Y}aZ-t & me-Xa\dgs{Y}eZ & me-Xa\dgs{Y}eZ-et & te-Xa\dgs{Y}eZ & te-Xa\dgs{Y}Z-i\\\hline
		2\gsc{PL} & Xi\dgs{Y}aZ-tem & Xi\dgs{Y}aZ-ten/m & me-Xa\dgs{Y}Z-im & me-Xa\dgs{Y}Z-ot & \multicolumn{2}{c|}{te-Xa\dgs{Y}Z-u}\\\hline\hline
		3\gsc{SG} & Xi\dgs{Y}eZ & Xi\dgs{Y}Z-a & me-Xa\dgs{Y}eZ & me-Xa\dgs{Y}eZ-et & ye-Xa\dgs{Y}eZ & te-Xa\dgs{Y}eZ\\\hline
		3\gsc{PL} & \multicolumn{2}{c||}{Xi\dgs{Y}Z-u} & me-Xa\dgs{Y}Z-im & me-Xa\dgs{Y}Z-ot & \multicolumn{2}{c|}{ye-Xa\dgs{Y}Z-u}\\\hline
	\end{tabular}
\end{small}
}
\xe

VIs can be assigned similarly to how this was done for Voice in Section~\ref{voice:voice:phono}. The non-spirantization can be analyzed as a floating feature docking onto the medial root consonant and preventing it from acquiring a [continuant] feature. Two examples are given in~(\nextx), where the floating feature still needs a constraint to dock it onto the right segment; see \cite{kastner18nllt} for the full implementation.
\pex
	\a\label{vi:voice2}	Voice \lra~\emph{i,e} / T[Past] \trace~\va
			\a\label{vi:va}\va~\lra~[--cont]$_{\gsc{ACT}}$ / \trace~ \{ \root{XYZ} $|$ Y $\in$ p, b, k \}
			
\xe

\section{Summary and outlook} \label{voice:conc}
This chapter examined the two templates {\tkal} and {\tpie}, treating them not as morphemic atoms but as combinations of functional heads, specifically v, Voice, and (in the case of {\tpie}) {\va}. The following generalizations about the argument structure of both templates are repeated here from~(\ref{ex:gen-tkal}) and~(\ref{ex:gen-tpie}).

\hammer{
\pex \label{ex:gen-tkal2}\textbf{Generalizations about {\tkal}}
	\a \textbf{Constructions:} Verbs appear in all possible argument structure configurations.
	\a \textbf{Alternations:} Participates in alternations with the other templates, as will be reviewed throughout the book.
\xe
}

\hammer{
\pex \label{ex:gen-tpie2}\textbf{Generalizations about {\tpie}}
	\a \textbf{Constructions:} Verbs appear in active (transitive/unergative) configurations.\\
		Readings are weakly agentive.
	\a \textbf{Alternations:} When alternating with {\tkal}, provides a more ``intensive'' or agentive version.
\xe
}

To account for these patterns, I began to unfold the proposed theory of trivalent Voice. This chapter concentrated on two elements: unspecified Voice does not impose any strict constraints in the syntax but is nevertheless traceable in the morphophonology. It is compatible with whatever argument structure the root allows. The modifier {\va} enforces certain agentive or agentive-like readings which, I have argued, can be found in various other languages as well.

The next chapters in Part I examine the other templates, motivating an analysis which uses different values of Voice. In Chapter~\ref{chap:vz} we will see what happens when Voice is endowed with a [--D] feature, prohibiting the merger of DPs in its specifier. The result will be a structure that allows anticausatives and, in some cases, reflexives of different kinds. In Chapter~\ref{chap:vd} we will see the consequences of a [+D] feature appearing on Voice, requiring its specifier to be filled. And in Chapter~\ref{chap:passn} we will see how these Voice heads interact with passiviziation, nominalization and adjectivization.


    \chapter{\vz}
\label{chap:vz}

\section{Introduction} \label{vz:intro}
In the previous chapter we saw how one morphological form in Hebrew is associated with various argument structure configurations: verbs in {\tkal} might be unaccusative, unergative, \isi{transitive} or ditransitive\is{\isi{ditransitives}}, all depending on the root. The theory developed in this book attributes this freedom to the behavior of (Unspecified) Voice, which at least in Hebrew is not specified with regards to the existence of an external argument or lack of one. And we have also seen how an agentive modifier can influence possible readings of the verb. In this chapter and in the next we will consider cases in which a different value of Voice is merged, leading to specific consequences for the syntax, semantics and phonology of the resulting verb. In terms of the morphology, we will see alternations in which the same root is instantiated in different templates.

The current chapter motivates the non-active head {\vz}. Informally, {\vz} rules out the addition of an external argument. In the simplest case, this configuration leads to argument structure alternations as in Table~\ref{tab:3-1:anticaus}, where the anticausative variants are essentially marked\is{markedness} with non-active morphology. The two templates explored in this chapter are {\tnif} and {\thit}, on the right-hand side of each row in the table.
\begin{table}
\begin{tabularx}{\textwidth}{lcllll}
 \lsptoprule
Templates & Root & \multicolumn{2}{c}{Causative} & \multicolumn{2}{c}{Anticausative} \\\midrule
\multirow{3}{*}{\tkal~$\sim$ \tnif} & \root{ʃbr}& ʃavar & `broke' & niʃbar & `got broken'\\
	 & \root{\dgs{k}r}& kara & `tore' & nikra & `got torn'\\
	 & \root{mtx}& matax & `stretched' & nimtax & `got stretched'\\\tablevspace
\multirow{3}{*}{\tpie~$\sim$ \thit} & \root{pr\dgs{k}}& pirek & `dismantled' & hitparek & `fell apart' \\
	 & \root{p{\ts}{\ts}}& po{\ts}e{\ts} & `detonated' & hitpo{\ts}e{\ts} & `exploded'\\
	 & \root{bʃl} & biʃel & `cooked' & hitbaʃel & `got cooked'\\
\lspbottomrule
 \end{tabularx}
\caption{Two pairs of alternations.}
\label{tab:3-1:anticaus} 
\end{table}

The idea that non-active marking tracks intransitive morphology is certainly not new, nor is the technical innovation of a non-active variant of Voice: \cite{schaefer08} and \cite{layering15} have most notably made the case for a system contrasting Voice with non-active (``expletive'' or ``middle'') Voice, and I will return to a direct comparison with that theory in Chapter~\ref{chap:aas}. What this chapter does aim to achieve is a number of interrelated goals, as already practiced in the previous chapter: to provide a thorough description of the facts, to motivate a particular analysis, and to highlight points of divergence from existing work in preparation for the discussion in the second part of this book.

This chapter is the longest in the book, encompassing three different syntactic configurations and at least four semantic interpretation possibilities across two morphophonological templates. The names I have given these constructions are intended to be transparent and easy to compare with work on other languages. With that in mind, the richness of the system could also be confusing. What is important in terms of the big picture is that the two kinds of vPs discussed so far (one with {\va} and one without it) can merge with the non-active head {\vz}, and not just with regular Voice as in the previous chapter. In addition, there is a prepositional counterpart to this head, namely {\pz}, which derives another kind of construction -- the figure reflexive\is{figure reflexives}. And finally, pure reflexives are only possible when {\va} is in the structure. Table~\ref{tab:1-3:tnif} provides a preview.
\begin{table}
\begin{tabularx}{\textwidth}{llcc} 
 \lsptoprule
	\multicolumn{2}{c}{Construction}	& {\tnif}	& {\thit} \\\midrule
\multirow{3}{*}{Non-active} & Anticausative	& {\vz}	& {\va}, {\vz}\\
	& Inchoative & {\vz}	& {\va}, {\vz}\\
	& Passive &	{\vz}	&	---\\\tablevspace
Active & \isi{Figure} reflexive	& {\pz}	& {\va}, {\pz}\\\tablevspace
Reflexive & Reflexive	& ---	& {\va}, {\vz}\\
\lspbottomrule
 \end{tabularx}
	\caption{Verbs with [--D].}
	\label{tab:1-3:tnif}
\end{table}

These constructions are explored as follows. In Section~\ref{vz:tnif} I identify the anticausatives, inchoatives and \isi{figure reflexives} of {\tnif} (this last group underwriting a novel generalization). Section~\ref{vz:vz} analyzes the first two and Section~\ref{vz:pz} analyzes \isi{figure reflexives}. Section~\ref{vz:interim} briefly summarizes the picture for {\tnif}. I then move to the right-hand side of the table, {\thit}, in Section~\ref{vz:thit}, and its analysis in Section~\ref{vz:va}: anticausatives, inchoatives and reflexives are analyzed in Section~\ref{vz:va:vzva}; \isi{figure reflexives} are discussed in Section~\ref{vz:va:pzva}. The empirical and analytical picture is recapped in Section~\ref{vz:sum}. Section~\ref{vz:others} then compares the Trivalent approach with other treatments in the literature, at which point I take stock and preview the next chapter.

\section{\tnif: Descriptive generalizations} \label{vz:tnif}
The so-called ``middle'' template {\tnif} is traditionally viewed as a \isi{passive} one. This is a mischaracterization. While it is true that many verbs in {\tnif} have \isi{passive} readings, these verbs are often \isi{mediopassive}, compatible with a \isi{passive} or anticausative reading. Furthermore, a large group of verbs in {\tnif} have decidedly different syntactic and semantic behavior: they are active verbs, ``\isi{figure reflexives}'' in the terminology of \cite{wood14nllt}. I lay out both classes of verbs and the diagnostics used to classify them. Their uniform morphology will receive a non-uniform syntactic analysis in Sections~\ref{vz:vz} and~\ref{vz:pz}.

	\subsection{Non-active verbs} \label{vz:tnif:nact}
Most verbs in {\tnif} have \textbf{\isi{passive}} readings in that they are the \isi{passive} variant of an active verb in~{\tkal}. This is the majority group of verbs in {\tnif} and probably the reason why the template is traditionally viewed as \isi{passive}. A few examples are given on the right-hand side of Table~\ref{table:vz:tnif-pass}.
\begin{table}
\begin{tabularx}{.75\textwidth}{c>{\em}ll>{\em}ll}
 \lsptoprule
Root & \multicolumn{2}{c}{{\tkal} Causative} & \multicolumn{2}{c}{{\tnif} Anticausative} \\\midrule
\root{'mr} & amar & `said' & neemar & `was said' \\
\root{bxn} & baxan & `examined' & nivxan & `was examined' \\
\root{rtsx} & ratsax & `murdered' & nirtsax & `was murdered' \\
\root{\dgs{k}b'} & kava & `set, decided' & nikba & `was decided'\\
\lspbottomrule
 \end{tabularx}
	\caption{Examples of passives in {\tnif}.}
\label{table:vz:tnif-pass} 
\end{table}

This section also concerns verbs like those on the right-hand side of~\ref{table:vz:tnif-anticaus}, which I call \textbf{anticausative}. Intuitively, these are verbs which convey the unaccusative variant of an existing active or stative verb in {\tkal}.
\begin{table}
\begin{tabularx}{.75\textwidth}{c>{\em}ll>{\em}ll}
 \lsptoprule
Root & \multicolumn{2}{c}{{\tkal} verb} & \multicolumn{2}{c}{{\tnif} Anticausative} \\\midrule
\root{gmr} & gamar & `ended' & nigmar  & 'ended' \\
\root{dl\dgs{k}} & dalak & `was lit' & nidlak & `lit up' \\
\root{t\dgs{k}'} & taka & `jammed' & nitka & `got stuck' \\
\lspbottomrule
 \end{tabularx}
	\caption{Examples of anticausatives in {\tnif}.}
\label{table:vz:tnif-anticaus}
\end{table}

The forms in Tables \ref{table:vz:tnif-pass}--\ref{table:vz:tnif-anticaus} are unambiguous, in that \emph{nimtax} does not pass the anticausativity tests described below, only the \isi{passive} ones. However, many verbs are \textbf{ambiguous} between the two readings, like those in Table~\ref{table:vz:tnif-passanticaus}.
\begin{table}
\begin{tabularx}{\textwidth}{c>{\em}ll>{\em}ll}
 \lsptoprule
Root & \multicolumn{2}{c}{{\tkal} verb} & \multicolumn{2}{c}{{\tnif} Anticausative} \\\midrule
\root{ʃbr}	&	ʃavar & `broke' &  niʃbar  & `broke / got broken' \\
\root{sgr} & sagar & `closed' & nisgar  & `closed / got closed'\\
\root{m'k} & maax & `squished' & nimax & `squished / got squished' \\
\lspbottomrule
 \end{tabularx}
	\caption{Examples of ambiguity between anticausative and passive in {\tnif}.}
\label{table:vz:tnif-passanticaus}
\end{table}

But this section also concerns verbs like those on the right-hand side of Table~\ref{table:vz:tnif-inch}. These \textbf{inchoatives} do not alternate with a variant in {\tkal}.
\begin{table}
\begin{tabularx}{.75\textwidth}{c>{\em}ll>{\em}ll}
 \lsptoprule
Root & \multicolumn{2}{c}{{\tkal} Causative} & \multicolumn{2}{c}{{\tnif} Inchoative} \\\midrule
\root{rdm} & \multicolumn{2}{c}{---} & nirdam & `fell asleep'\\
\root{'lm} & \multicolumn{2}{c}{---} & neelam & `disappeared'\\
\root{kxd} & \multicolumn{2}{c}{---} & nikxad & `went extinct'\\
\lspbottomrule
 \end{tabularx}
	\caption{Examples of inchoatives in {\tnif}.}
\label{table:vz:tnif-inch} 
\end{table}

Out of 415 verbs in {\tnif} classified by \cite{ahdoutkastner19nels}, 275 have only \isi{passive} readings, 196 have only anticasuative or inchoative readings, and 88 are ambiguous (leading to totals above 415). I will return to the quantitative summary in Section~\ref{vz:tnif:sum}.

In what follows I apply the diagnostics introduced in Chapter~\ref{voice:tkal:nact}: compatibility with \isi{Agent}-oriented adverbs (Section~\ref{vz:tnif:nact:adv}) and the two \isi{unaccusativity tests}, VS order\is{unaccusativity tests} and the possessive dative\is{unaccusativity tests} (Section~\ref{vz:tnif:nact:unacc}). I also make use of diagnostics particular to \isi{passive} configurations. All of the tests are consistent with the claim that the verbs classified as anticausative and inchoative have no \isi{Agent}, hence are unaccusative, and that the verbs classified as passives have an implicit \isi{Agent} (or an explicit \emph{by}-phrase \isi{Agent}).

	\subsubsection{Adverbial modifiers} \label{vz:tnif:nact:adv}
\isi{Agent}-oriented adverbs are incompatible with anticausatives~(\nextx) but possible with passives in the \isi{passive} templates~(\anextx a) and in {\tnif}~(\anextx b).
 \begin{exe}
 \ex   
 \begin{xlist} 
 	\ex 	  
[*] 		{ \gll ha-{\ts}amid \glemph{niʃbar} be-mejomanut\\
 		  the-bracelet broke.\gsc{MID} in-skill\\
 		\glt (int. 'The bracelet was dismantled skillfully') } 
		
 	\ex    
[??] 		{ \gll dana \glemph{nirdem-a} be-xavana\\
 		  Dana fell.asleep.\gsc{MID-F} on-purpose\\
 		\glt (int. `Dana fell asleep on purpose') } 
		
 \z

 \ex  
 \begin{xlist} 
 	\ex 	 
[] 		{ \gll ha-ʃaon \glemph{porak} be-zehirut\\
 		  the-watch dismantled.\gsc{INTNS.PASS} in-caution\\
 		\glt `The watch was dismantled carefully.' } 
		
 	\ex  
[] 		{ \gll ha-hatsaa \glemph{nivxen-a} be-xaʃaʃ\\
 		  the-suggestion.\gsc{F} examined.\gsc{MID}-\gsc{F} in-fear\\
 		\glt `The suggestion was considered cautiously.' } 
		
 \z
\z 

Anticausatives are also incompatible with \emph{by}-phrases, which would otherwise refer to an \isi{Agent}~(\nextx). These are naturally possible with passives~(\anextx).
 \begin{exe}
 \ex  
 \begin{xlist} 
 	\ex    
[*] 		{ \gll ha-{\ts}amid \glemph{niʃbar} al-jedej ha-{\ts}oref\\
 		  the-bracelet broke.\gsc{MID} by the-jeweler\\
 		\glt (int. 'The bracelet was dismantled by the jeweler') } 
	
 	\ex    
[*] 		{ \gll dana \glemph{nirdem-a} al-jedej \{ha-xom / ha-kosem-et\}\\
 		  Dana fell.asleep.\gsc{MID-F} by the-heat {} the-magician-\gsc{F}\\
 		\glt (int. `Dana fainted/fell asleep due to the heat/due to the magician') } 
	
 \z

 \ex  
 \begin{xlist} 
 	\ex   
[] 		{ \gll ha-ʃaon \glemph{porak} al-jedej ha-{\ts}oref\\
 		  the-watch dismantled.\gsc{INTNS.PASS} by the-jeweler\\
 		\glt `The watch was dismantled by the jeweler.' } 
		
 	\ex   
[] 		{ \gll ha-mitmodedim \glemph{nivxen-u} al-jedej ha-ʃofetet\\
 		  the-contestants examined.\gsc{MID}-\gsc{PL} by the-referee\\
 		\glt `The contestants were judged by the referee.' } 
		
 \z
\z 

The `by itself\is{agentivity}' test can be assumed to diagnose the non-existence of an external argument, regardless of whether the external argument is explicit (as in \isi{transitive} verbs) or implicit (as in passives). The test is valid with anticausatives and inchoatives, (\nextx), but not with direct objects of \isi{transitive} verbs, (\anextx a), or with \isi{passive} verbs, (\anextx b).
 \begin{exe}
 \ex  
 \begin{xlist} 
 	\ex   
[] 		{ \gll ha-kise \glemph{niʃbar} me-a{\ts}mo\\
 	       the-chair broke.\gsc{MID} from-itself\\
 	     \glt  'The chair fell apart (of its own accord).' } 
    
 	\ex   
[] 		{ \gll ha-klavlav \glemph{nirdam} me-a{\ts}mo\\
 		  the-puppy fell.asleep.\gsc{MID} from-itself\\
 		\glt `The puppy fell asleep of his own accord.' } 
	
 \z

 \ex  
 \begin{xlist} 
     \ex    
[*] 	    { \gll miri \glemph{ʃavr-a} et ha-kise me-a{\ts}mo.\\
 	      Miri broke.\gsc{SMPL}-\gsc{F} \gsc{ACC} the-chair from-itself\\
 	    \glt (int. 'Miri broke the chair of its own accord') } 
    
     \ex    
[*] 	    { \gll moed ha-bxina \glemph{nikba} me-a{\ts}mo.\\
 	      date.of the-exam decided.\gsc{MID} from-itself\\
 	    \glt (int. 'The date of the exam was set of its own accord') } 
   
 \z
\z 
    
And as expected, passives allow control by the implied external argument (see \citealt{williams15} and \citealt{bhattpancheva17} for qualifications to this test):
 \begin{exe}
\ex   
[] 	{ \gll ha-delet \glemph{nisger-a} kedej limnoa me-ha-xatul lehikanes la-xeder\\
 	  the-door closed.\gsc{MID}-\gsc{F.SG} in.order to.prevent from-the-cat to.enter.\gsc{MID} to.the-room\\
 	\glt `The door was closed to prevent the cat from entering the room' } 
	
 \z 
    
The tests thus far indicate that anticausatives and inchoatives in {\tnif} do not have an external argument, while passives do.

	\subsubsection{Unaccusativity diagnostics} \label{vz:tnif:nact:unacc}
Anticausatives and inchoatives in {\tnif} allow VS order\is{unaccusativity tests}:
 \begin{exe}
 \ex   
 \begin{xlist} 
 	\ex   
[] 		{ \gll \glemph{nigmer-a} kol ha-bamba\\
 		  ended.\gsc{MID}-\gsc{F} all the-bamba\\
 		\glt `The bamba snack ran out.' } 
		
 	\ex   
[] 		{ \gll \glemph{neelm-u} me-ha-sifrija ʃloʃa kraxim ʃel britanika\\
 		  disappeared.\gsc{MID}-\gsc{3PL} from-the-library three volumes of Britannica\\
 		\glt `Three volumes of Encyclopedia Britannica disappeared from the library.'  \hfill \citep[142]{shlonsky87} } 
		
 \z
\z 

As noted by \citet[148]{shlonsky87}, VS order\is{unaccusativity tests} with passives is generally fine but less so when the \isi{Agent} is specified.
 \begin{exe}
\ex   
[] 	{ \gll \glemph{neexal} le-ruti ha-kiwi (*al-jedej ha-xatul)\\
 	  ate.\gsc{MID} to-Ruti the-kiwi by the-cat\\
 	\glt `Ruti's kiwi was eaten.' } 
	
 \z 
  
Anticausative and inchoative verbs in {\tnif} are compatible with the possessive dative\is{unaccusativity tests}, again because it presumably targets the internal argument.
 \begin{exe}
 \ex  
 \begin{xlist} 
 	\ex  	 
[] 		{ \gll \glemph{niʃbar} l-i ha-ʃaon\\
 		  broke.\gsc{MID} to-me the-watch\\
 		\glt `My watch broke.' } 
		
 	\ex  	 
[] 		{ \gll \glemph{nirdam} l-i ha-kelev al ha-regel, ma laasot?\\
 		  fell.asleep.\gsc{MID} to-me the-dog on the-leg what to.do\\
 		\glt `My dog fell asleep on my lap, what should I do?' } 
		
 \z
\z 

Taken together, these tests establish that anticausatives and inchoatives are unaccusative but the \isi{passive} verbs are not (since the latter disallow `by itself\is{agentivity}'). A common assumption in the Hebrew literature is that verbs in this template are all non-active, but we will next consider another class of verbs in {\tnif}, the \isi{figure reflexives}, which behave differently with regard to these tests.

	\subsection{Figure reflexives} \label{vz:tnif:figrefl}
It has been commonly assumed that verbs in {\tnif} are medio-\isi{passive} (non-active), but it can be shown that there is another class of verbs in this template whose properties are quite different. These verbs do have an external argument and also take an obligatory prepositional phrase\is{\isi{prepositional phrases}} as their complement. Whereas a typical prepositional phrase\is{\isi{prepositional phrases}} has a \isi{Figure} and a \isi{Ground}, roughly the subject and object of the preposition (Chapter~\ref{intro:sketch:heads}), in these verbs the \isi{Figure} is not explicitly named as a separate argument. It is, however, coreferential with the \isi{Agent} of the verb. Verbs like these are called \emph{\isi{figure reflexives}}, which is the term coined by \cite{wood14nllt} for a similar phenomenon in Icelandic. The name itself is meant to invoke the \isi{Figure}-like, reflexive-like interpretation of a \isi{Figure} in a prepositional phrase\is{\isi{prepositional phrases}} when it is the complement of certain verbs.\label{r1:3:2}

\isi{Figure} reflexives in {\tnif} include verbs such as those in Table~\ref{table:vz:figrefl}; all require a PP\is{\isi{prepositional phrases}} complement. Based on the diagnostics discussed here, \cite{ahdoutkastner19nels} have found that 74 of the 415 verbs in {\tnif} are figure reflexive\is{figure reflexives}, or ambiguous between a non-active and a figure reflexive\is{figure reflexives} reading. Some of these verbs are fairly recent (e.g.~\emph{nirʃam le-} `signed up for'), indicating that we are not dealing simply with a long list of lexicalized exceptions. Nevertheless, this class of verbs was not recognized prior to \cite{kastner16phd}, as far as I can tell.
\begin{table}
	\begin{tabularx}{.75\textwidth}{l>{\em}lll} 
 \lsptoprule
	a.& nixnas &  *(\emph{le-}) & `entered (into)'\\
	b.& nidxaf & *(\emph{derex/le-})  & `pushed his way through/into' \\
	c.& nirʃam & *(\emph{le-})  & `signed up for' \\
	d.& nilxam & *(\emph{be-}) & `fought (with)' \\
	e.& neexaz & *(\emph{be-}) & `held on to' \\
\lspbottomrule
     \end{tabularx}
	\caption{Examples of figure reflexives in {\tnif}.}
\label{table:vz:figrefl}
\end{table}

I will repeat the diagnostics from Sections~\ref{vz:tnif:nact:adv}--\ref{vz:tnif:nact:unacc}---showing that \isi{figure reflexives} pattern the opposite way from non-actives---before proceeding to discuss the complement to the verb.

		\subsubsection{Adverbial modifiers} \label{vz:tnif:figrefl:adv}
\isi{Agent}-oriented adverbs are possible with \isi{figure reflexives}:
 \begin{exe}
\ex[]{ \label{ex:vz:nixnesa} 
	 \gll dana \glemph{nixnes-a} la-kita be-bitaxon\\
 	  Dana entered.\gsc{MID}-\gsc{F} to.the-classroom in-confidence\\
 	\glt `Dana confidently entered the classroom.' } 
	
 \z 

`By itself' is not possible with figure reflexives:
 \begin{exe}
\ex   
[*] 	{ \gll dana \glemph{nixnes-a} la-xeder me-a{\ts}ma/me-a{\ts}mo\\
 	  Dana entered.\gsc{MID}-\gsc{F} to.the-room from-herself/itself\\
}
 \z 

\emph{By}-phrases are an irrelevant diagnostic when the external argument is explicit.

		\subsubsection{Unaccusativity diagnostics} \label{vz:tnif:figrefl:unacc}
\isi{Figure} reflexives fail the accepted unaccusativity diagnostics, unlike non-active verbs in {\tnif}. \textbf{VS order\is{unaccusativity tests}} is unavailable, again being grammatical but resulting in ``stylistic inversion'':
 \begin{exe}
\ex   
[\#] 	{ \gll\glemph{nixnes-u} ʃaloʃ xajal-ot la-kita\\
 	  entered.\gsc{MID}-\gsc{3PL} three soldiers-\gsc{F.PL} to.the-classroom\\
 	\glt (int. 'Three soldiers entered the classroom.') } 
	
 \z 

The \textbf{possessive dative\is{unaccusativity tests}} is likewise incompatible with \isi{figure reflexives}; example~(\nextx) is infelicitous on a reading where the cat is the speaker's.
 \begin{exe}
\ex   
[\#] 	{ \gll ha-xatul \glemph{nixnas} l-i la-xeder (kol ha-zman), ma laasot?\\
 	  the-cat enters.\gsc{MID} to-me to.the-room (all the-time) what to.do\\
 	\glt (int. 'My cat keeps coming into into my room, what should I do?') } 
	
 \z 

\citet[134]{shlonsky87} provided the pair in~(\nextx), noting in a footnote that \emph{lehikans} `to enter' is not unaccusative (an observation he credited Hagit Borer with), but he did not pursue the matter further.
 \begin{exe}
 \ex  
 \begin{xlist} 
 	\ex    
[*] 		{ \gll be-emtsa ha-seret \glemph{nixnes-u} li jeladim raaʃanim\\
 		  in-middle.of the-movie entered.\gsc{MID}-\gsc{F} to.me children noisy\\
 		\glt (int.~`In the middle of the movie (there) entered noisy children and it aggravated me') } 
		
 	\ex   
[] 		{ \gll be-emtsa ha-seret \glemph{nikre-u} li ha-mixnasaim\\
 		  in.middle.of the-movie tore.\gsc{MID}-\gsc{F} to.me the-pants.\gsc{PL}\\
 		\glt `In the middle of the movie my pants tore.' } 
		
 \z
\z 

This brief series of tests indicates that the subject of \isi{figure reflexives} is a true agent\is{\isi{Agent}}, unlike the non-actives which share the same morphology.\footnote{It is unclear to what extent the episodic plural\is{unaccusativity tests} is compatible with \isi{figure reflexives}:
 \begin{exe}
\ex  [??]
 	{ \gll \glemph{nixnas-im} pitom la-ulam!\\
 	  enter.\gsc{MID.PRES}-\gsc{PL.M} suddenly to.the-hall!\\
 	\glt (int.~`people are entering the hall all of a sudden!') } 
	
 \z }
That is one main difference. The second is the complement of these verbs, as I discuss next.

	\subsubsection{Indirect objects} \label{vz:tnif:figrefl:pp}
The novel observation is that \isi{figure reflexives} take an obligatory prepositional phrase\is{\isi{prepositional phrases}}. Table~\ref{table:vz:figrefl2} is repeated from Table~\ref{table:vz:figrefl}. Importantly, the PP\is{\isi{prepositional phrases}} complements for these verbs cannot be left out. For example, omitting the PP\is{\isi{prepositional phrases}} from~(\ref{ex:vz:nixnesa}) above results in ungrammaticality, (\ref{ex:vz:pp}a).
\begin{table}
	\begin{tabularx}{.75\textwidth}{l>{\em}lll} 
 \lsptoprule
	a.& nixnas &  *(\emph{le-}) & `entered (into)'\\
	b.& nidxaf & *(\emph{derex/le-})  & `pushed his way through/into' \\
	c.& nirʃam & *(\emph{le-})  & `signed up for' \\
	d.& nilxam & *(\emph{be-}) & `fought (with)' \\
	e.& neexaz & *(\emph{be-}) & `held on to' \\
\lspbottomrule
     \end{tabularx}
	\caption{Examples of figure reflexives in {\tnif}.}
\label{table:vz:figrefl2}
\end{table}

 \begin{exe}
 \ex  Prepositional phrase complements (indirect objects) to figure reflexives are obligatory:  \label{ex:vz:pp}
 \begin{xlist} 
 	\ex   
[] 		{ \gll dana \glemph{nixnes-a} *(la-kita)\\
 		  Dana entered.\gsc{MID}-\gsc{F} to.the-classroom\\
 		\glt `Dana confidently entered the classroom.' } 
	
 	\ex   
[] 		{ \gll ahed \glemph{nilxem-a} *(be-avlot)\\
 		  Ahed fought.\gsc{MID}-\gsc{F} in-wrongs\\
 		\glt `Ahed fought wrongdoings.' } 
	
 \z
\z 

This claim has not been made before in either the traditional grammars or contemporary work, as far as I know (the closest are \citealt[87]{berman78}, who stated that some verbs show ''ingression'', and \citealt{schwarzwald08}, who noted that some verbs in this template are active).\footnote{See \cite{neeleman97} for PP\is{\isi{prepositional phrases}} complements in Dutch and English.} Hagit Borer (p.c.) notes that~(\nextx) is fine with no overt complement, even though I claim that the PP\is{\isi{prepositional phrases}} is obligatory:
 \begin{exe}
\ex   
[] 	{ \gll \glemph{tafsik} le-hidaxef!\\
 	  stop.\gsc{CAUS} to-push.\gsc{MID}.\gsc{INF}\\
 	\glt `Stop pushing (your way in)!' } 
	
 \z 

This example has the main verb in the imperative (or rather, in the future form, which is used for the imperative reading of most verbs in Modern Hebrew; cf.~\citealt{batel02lang}). I suspect that this is a general pattern because in English, too, obligatory complements can be dropped in imperatives:
 \begin{exe}
 \ex  
 \begin{xlist} 
 	\ex  Itamar nagged *(Archie). 
 	\ex  Quit nagging! 
 \z
\z 

The resulting generalization is that external arguments in {\tnif} are possible if and only if a prepositional phrase\is{\isi{prepositional phrases}} is required. In Section~\ref{vz:pz} I show how this generalization can be derived from the structure.

	\subsection{Interim summary: \tnif} \label{vz:tnif:sum}
Verbs in {\tnif} can be classified according to their syntactic behavior and derivational relationship to other verbs. Anticausatives, inchoatives and passives are non-active; \isi{figure reflexives} are active. Passives have an implied external argument, while anticausatives and inchoatives do not. And of these two, only anticausatives stand in an alternation with a verb in {\tkal}. Looking at things structurally, anticausatives and inchoatives are unaccusative (no external argument); passives are \isi{passive} (implied external argument); and \isi{figure reflexives} are unergative (require an external argument).

Based on the diagnostics above, \cite{ahdoutkastner19nels} were able to classify 415 verbs with a high degree of certainty (out of 462 in total), with the breakdown in Table~\ref{tab:3-2:tnif}.\footnote{These findings the result of work by Odelia Ahdout as part of \citet{ahdout19phd}.} It can be seen from the first row, for example, that 91 verbs in {\tnif} have only unaccusative readings, like those in Table~\ref{table:vz:tnif-anticaus}. Since some verbs are ambiguous between a number of readings like those in Table~\ref{table:vz:tnif-passanticaus}, the total number of verbs with an unaccusative reading is 196 (first column). These numbers are not given here as part of any quantitative claim, only to demonstrate that all classes are well-attested in the language (but without factoring anything like token frequency into the equation). Additional examples can be found in \cite{ahdoutkastner19nels}.

\begin{table}
\fittable{
\begin{tabularx}{\textwidth}{lcccrr@{.}l}
 \lsptoprule
				& \multicolumn{3}{c}{Construction}	& N	& \multicolumn{2}{c}{\%} \\
				 	& Unacc	& Passive & Figure reflexive & & \multicolumn{2}{c}{} \\\midrule\hline
Only unaccusative			&	+			& ---			&	---		&	91	&	21&9\% \\
Only \isi{mediopassive}			&	+			& +				& ---		&	78	&	18&8\% \\
Only \isi{passive}					&	---			& +				&	---		&	172	& 41&4\% \\\tablevspace
Only \isi{Figure} reflexive		& ---			& ---			& +			& 32	& 7&7\% \\\tablevspace
Ambiguous unacc/unerg	& +				& ---			& +			& 17	& 4&1\% \\
Ambiguous pass/unerg	& ---			& +				& +			& 15	& 3&6\% \\
Three-way ambiguous		& +				& +				& +			& 10	& 2&4\% \\\tablevspace
Total per construction		& 196		&	275				& 74 & \multicolumn{3}{l}{} \\

\lspbottomrule
 \end{tabularx}
}
	\caption{Readings for verbs in {\tnif}.}
\label{tab:3-2:tnif}
\end{table}

Before concluding the empirical exposition of {\tnif}, a few counterexamples should be noted. As far as I could find, these are the only verbs which do not fit cleanly into the classes surveyed above. There are two verbs of emission, \emph{neenax} and \emph{neenak}, both of which mean `sighed, groaned, moaned'. Verbs of emission are generally unergative in Hebrew \citep{siloni12,gafter14li} but these verbs do not take a PP\is{\isi{prepositional phrases}} complement. The two verbs \emph{nizak} and \emph{nexpaz} `rushed, hurried' take a clausal complement, probably a TP, rather than a PP\is{\isi{prepositional phrases}}. See \citet[126]{kastner16phd} for brief discussion and speculation. And the verb \emph{nexgar} `buckled up' seems to have a purely reflexive reading, rather than non-active or figure reflexive\is{figure reflexives}.

These points for further research aside, the generalizations about {\tnif} are as follows. In terms of the constructions we see associated with this template, we have found all manner of non-active verbs as well as \isi{figure reflexives}. What we never find in this template is simple \isi{transitive} structures consisting of a subject, verb and direct object. There are also no purely reflexive verbs (this will contrast with {\thit} later in the chapter). In terms of alternations, many active (and stative) verbs in {\tkal} have a non-active alternation with {\tnif}. A summary of these points is presented in~(\nextx).

\hammer{
 \begin{exe}
 \ex  \label{ex:gen-tnif}\textbf{Generalizations about {\tnif}} 
 \begin{xlist} 
 	\ex  \textbf{Configurations:} Verbs appear in unaccusative, passive and figure reflexive structures; but never in a simple transitive configuration. 
 	\ex  \textbf{Alternations:} Some verbs are anticausative or passive versions of verbs in {\tkal}. 
 \z
\z 
}

The non-active verbs are analyzed next, in Section~\ref{vz:vz}. \isi{Figure} reflexives are analyzed in Section~\ref{vz:pz}.


\section{\vz} \label{vz:vz}
In order to explain the behavior of non-active verbs in {\tnif} I propose the head {\vz}. This non-active variant of \isi{Unspecified Voice} is defined in brief in~(\nextx). The syntax of {\vz} is similar to that of ``middle Voice'', ``non-active Voice'', ``expletive'' Voice or Voice$_{\{\}}$ of much related work in that it does not license\is{licensing} a specifier \citep{lidz01,schaefer08,alexiadoudoron12,layering15,bruening13,wood15springer,myler16mit,kastnerzu17}. Its semantics does not introduce an open \isi{Agent} role, and the Vocabulary Item spelling it out manifests as the template {\tnif}, and not as {\tkal}. The rest of this section engages more directly with the syntax, semantics and phonology of this element. In Section \ref{vz:va:vzva} I will refine the picture slightly by explaining what happens when {\va} is added to the structure.

 \begin{exe}
 \ex  \textbf{\vz} 
 \begin{xlist} 
 	\ex  A Voice head with a [--D] feature, prohibiting anything with a [D] feature from merging in its specifier. 
    As typically assumed for unaccusative little \emph{v} or unaccusative Voice, {\vz} does not assign accusative case either itself by feature checking \citep{chomsky95} or through the calculation of dependent case \citep{marantz91}.
 	\ex  \denote{\vz}\phantom{.} = $\begin{cases} 
		\text{λPλe∃x.Agent(x,e) \& P(e)} & / \text{\{\root{rtsx} `murder', \root{'mr} ‘say’, \dots\}}\\
		\text{λP}_{<s,t>}\text{.P} & \\
		\end{cases}$
 	\ex  {\vz} \lra~{\tnif} \hfill (with the allomorph {\thit} to follow in Section \ref{vz:va:vzva}) 
 \z
\z 

This basic distinction between Voice and {\vz} in the syntax thus feeds differences across the interfaces: the spell-out is different, the semantics is different and the syntax of the resulting constructions is different. 

	\subsection{Syntax} \label{vz:vz:syn}
Voice and {\vz} function in a way familiar from the work cited immediately above. External arguments are not referenced in the core vP; the position they are merged in (Spec,VoiceP) is licensed\is{licensing} by Voice in the syntax and their thematic role (\isi{Agent}) is introduced by Voice in the semantics. What this means is that a vP is a predicate of events (potentially \isi{transitive} ones) with no inherent reference to the thematic role of \isi{Agent} stemming from the syntax.

Continuing with an example from the previous chapter, we have seen that the verb \emph{ʃavar} `broke' in {\tkal} is made up of a vP, denoting a set of breaking events, and the head Voice that introduces an external argument, (\nextx).
 \begin{exe}
\ex  \emph{XaYaZ}, \emph{ʃavar} 'broke'  \\
\Tree
	[.VoiceP
		[.DP ]
		[.
			[.Voice ]
			[.vP
				[.v
					[.\root{ʃbr} ]
					[.v ]
				]
				[.DP ]
			]
		]
	]		
 \z 

Merging {\vz} instead of Voice should give us the same basic breaking event with no external argument, since {\vz} does not allow a DP to be merged in its specifier. These are precisely the \textbf{anticausatives}: verbs which differ minimally from their active alternants in that no external argument is introduced. Continuing our example, the grammar can build a core vP as above (verbalizer, root and internal argument) and merge {\vz}. This configuration gives us \emph{niʃbar} `broke' in~(\nextx). Since no external argument can be merged in the specifier of {\vz}, the structure in~(\nextx) is unaccusative. The crossed out specifier position is used as notation to make this explicit.
 \begin{exe}
\ex  {\tnif}, \emph{niʃbar} 'got broken'  \\
\Tree
	[.VoiceP
		[.{---} ]
		[.
			[.{\textbf{\vz}\\\emph{ni-}} ]
			[.vP
				[.v
					[.\root{ʃbr} ]
					[.v ]
				]
				[.DP ]
			]
		]
	]		
 \z 

The idea that verbs in this template are anticausative variants of those in {\tkal} is not new. However, the explicit morphosyntactic implementation is novel (see also \citealt{kastner17gjgl}), providing a necessary backdrop for the analyses of \isi{figure reflexives} and reflexives coming up.

The same structure derives \textbf{passives} in {\tnif}. I subscribe to the view according to which the implicit external argument of the \isi{passive} is not projected in the syntax at all (\citealt{layering15}; see \citealt{bhattpancheva17} for discussion). The analysis of {\tnif} provides support for this view, since otherwise {\vz} would need to have two distinct syntactic specifications (no specifier or implicit \isi{Agent}).

In terms of structure, \textbf{inchoatives} are identical to anticausatives and passives. The only difference is that the underlying vP does not have an interpretation with Voice, a matter of the semantic interpretation, coming up next.

Two brief points should be mentioned here. First, the relevant feature on Voice has been characterized as [$\pm$D] throughout. This raises the immediate question of whether PPs are possible in Spec,{\vz}. Hebrew does not have PP\is{\isi{prepositional phrases}} subjects of the Slavic type, so the question is moot; if it turns out that a different \isi{EPP}-like feature needs to be used, not much will change in the theory. The second point is that in a trivalent theory of Voice, {\vz} prohibits something from merging in its specifier. This is not the same as the bivalent theories mentioned above, in which Expletive Voice does not project a specifier. This conceptual difference, and the empirical differences it brings up, are addressed in Chapter~\ref{chap:aas}.

	\subsection{Semantics} \label{vz:vz:sem}
The denotations of {\vz} are as follows:
 \begin{exe}
\ex  \label{ex:vz-sem}\denote{\vz} = 
	\begin{xlist}
		\ex λPλe∃x.Agent(x,e) \& P(e) / \{\root{rtsx} `murder', \root{'mr} ‘say’, \dots\} 
		\ex λP$_{<s,t>}$.P
	\z
 \z 
Two issues need to be unpacked. The first is the difference between unaccusatives and passives. The second has to do with the composition of inchoatives.

The LF rules in~(\lastx) demonstrate a case of contextual \isi{allosemy}: a functional head has one interpretation in one context, and another in another context. Specifically, I assume that the default function of {\vz} is the identity function in~(\lastx b): it takes an event of breaking, for example, and does not modify it. Crucially, it does not add an \isi{Agent} role.

Some roots (in fact many of them) derive \isi{passive} verbs when combining with {\tnif}. This situation is similar to that of Greek, where verbs with the non-active suffix might be unaccusative or \isi{passive}. In saying this I am simplifying the empirical picture considerably but the core point remains that a non-active head is underspecified with regards to \isi{passive} and unaccusative readings. \cite{alexiadoudoron12} made this point explicit for Hebrew and Greek, and \cite{layering15} elaborated on it for Greek. The rules in~(\lastx) implement this intuition formally.\footnote{Although I do not have any formally insightful way of modeling the cases of ambiguity broached earlier. Perhaps both clauses of~(\ref{ex:vz-sem}) need to be contextualized to lists of roots.}

The second issue in the semantics of {\vz} has to do with composing inchoatives. In what follows I delve a bit deeper into inchoatives in an attempt to understand how a compositional syntax/semantics works in these cases, where there is no alternating active verb and no obvious vP for {\vz} to combine with, followed by some crosslinguistic parallels. Readers who are not troubled by the compositional details may want to skip ahead to section~\ref{vz:vz:phono}, on the morphophonology of {\vz}.

  		\subsubsection{Null allosemy in inchoatives} \label{vz:inch:analysis}
Recall the relevant semantics of {\vz}:
 \begin{exe}
\ex  \denote{\vz} = λP$_{<s,t>}$.P
 \z 
This works well when the underlying vP is an event of breaking a glass, like in our running example. In principle, we expect the vP to describe an event which might then receive an \isi{Agent} (with Voice) or not ({\vz}). But what if there is no [Voice vP] structure, i.e.~no active verb in {\tkal}, as in \emph{nirdam} `fell asleep'? It is not derived from a \isi{causative} verb *\emph{radam} because there is no such verb (nor has there been in the history of the language, as far as I know).

Two solutions come to mind, though I will not adjudicate between them. The first assumes that the vP does exist with its own semantics but cannot combine with Voice for arbitrary reasons. The second assumes that {\vz} is what selects the meaning of the root (rather than v).

\paragraph{No licensing of Voice.} One recurrent issue in the morphology of Semitic languages is that not every root can appear in every possible template. At some level a root must list which functional heads it can combine with; let us call this ``licensing'' in a way which does not commit to any specific implementation. For example, \root{ʃbr} licenses Voice (\emph{ʃavar} `broke'), {\va} (\emph{ʃiber} `broke to pieces') and {\vz} (\emph{niʃbar} `was broken'), but not {\vd} of Chapter~\ref{chap:vd} (*\emph{heʃbir}). Every root must list this kind of information; the morphological system is riddled with such arbitrary gaps.

It could be, then, that the minimal vP in~(\nextx) is a valid syntactic object, awaiting some element at the Voice layer in order to satisfy some well-formedness condition (be it morphological or phonological; recall that the Voice layer introduces the stem vowels).
 \begin{exe}
\ex  
	\Tree
	[.vP
		[.v
			[.\root{rdm} ]
			[.v ]
		]
		[.DP ]
	]
 \z 

Then, \root{rdm} simply does not license\is{licensing} \isi{Unspecified Voice}; accordingly, there is no verb *\emph{radam}. But this root can still combine with {\vz} if it does license\is{licensing} it. The rule of interpretation above does not need to be changed. The cost is acknowledging the idiosyncrasy of the system to a greater degree than before: why is it that precisely these roots do not license\is{licensing} Voice and do license\is{licensing} {\vz}? Is there some lexical-semantic generalization to be made? Can we find cases of the core vP embedded under another category head? Can \isi{Unspecified Voice} be added in innovations? I leave these questions open.

\paragraph{Weakening the Arad/Marantz Hypothesis.} Theories of Voice like the current one or that of \cite{layering15} usually adopt the so-called ``Arad/Marantz Hypothesis'' \citep{elenasamioti14}, according to which the first categorizing head merging with a root selects the meaning of the root \citep{arad03,marantz13}. For verbs, this is always v. What we could assume instead is that certain configurations allow for interpretations of the root conditioned by a high functional head (in this case {\vz}) over a lower functional head (v). The theory involved is one in which meaning is calculated over semantically contentful elements only, just as allomorphy is calculated over phonologically contentful (overt) elements (\citealt{embick10} et seq, but compare \citealt{kastnermoskal18}).

Consider anticausatives once more. In~(\ref{ex:vz:allose-decaus}a), the combination of v and \root{ʃbr} results in a contentful combination, the predicate of breaking events. The root can have various related meanings, but at this point in the derivation its meaning has been chosen. As a consequence, any higher material will in principle only be able to manipulate this meaning \citep{arad03}, not select another meaning of the root (this point will be expanded in Chapter~\ref{vd:caus}). {\vz} has a syntactic function: it blocks merger of a DP in its specifier. As a result, the VoiceP will be interpreted as a detransitivized version of the vP,~(\ref{ex:vz:allose-decaus}b).
 \begin{exe}
 \ex  Locality in interpretation: anticausatives.\label{ex:vz:allose-decaus} 
 \begin{xlist} 
     \ex  {[}v \root{ʃbr}~\!] = λxλe.\emph{break}(e) \& Theme(x,e) 
     \ex  {[}\textbf{\vz} [\emph{break}] ] = \emph{niʃbar} `got broken' 
 \z
\z 

If a given root combines with v to be verbalized, it is possible that v introduces an event variable but carries no additional semantic content when combined with this root. No verb results in this configuration,~(\ref{ex:vz:allose-incho}a). As a result, the next functional head will have a chance to select the interpretation of the root, as with {\vz} in~(\ref{ex:vz:allose-incho}b). In a sense, the root selects for a specific additional functional head.
 \begin{exe}
 \ex  Locality in interpretation: inchoatives.\label{ex:vz:allose-incho} 
 \begin{xlist} 
     \ex  {[}v \root{rdm}~\!] = undefined 
     \ex  {[}\textbf{\vz} [(v) \root{rdm}~\!] ] = `fell asleep' 
 \z
\z 

\subsubsection{Null allosemy crosslinguistically}
These are the inchoatives treated here, but similar constructions can be found in Romance languages. \cite{burzio86} observes what he calls an ``inherently reflexive'' verb which requires the nonactive clitic \emph{si} (\ili{Italian} \gsc{SE}). The glosses are his.
 \begin{exe}
 \ex  \langinfo{Italian}{}{\citealt[39]{burzio86}}
 \begin{xlist} 
 	\ex   
[] 		{ \gll Giovanni \glemph{si} sbaglia\\
 		  Giovanni himself mistakes\\
 		\glt `Giovanni is mistaken.' } 
	
	
 	\ex    
[*] 		{ \gll Giovanni sbaglia Piero\\
 		  Giovanni mistakes Piero\\
 		\glt (int. `Giovanni mistakes Piero')  }
	
 \z
\ex  	
[] 		{ \gll Giovanni \glemph{se} ne pentir\'a\\
 		  Giovanni himself of.it will.repent\\
 		\glt `Giovanni will be sorry for it.' } 
	
\ex  	
[] 		{ \gll Giovanni ci \glemph{si} \'e arrangiato\\
 		  Giovanni there himself is managed\\
 		\glt `Giovanni has managed it.' \hfill \citep[70]{burzio86} } 
 \z 

The forms *\emph{sbaglia} and *\emph{pentir\'a} are not possible without \gsc{SE}; some verbs simply require \gsc{SE} or the equivalent nonactive marker in their language, however encoded.\footnote{The facts are slightly more complicated: \emph{sbaglia} `mistake' is possible in certain contexts but I believe that the generalization about \emph{pentirsi} `repent' is robust \citep[40]{burzio86}.}

The famous case of \isi{deponents} in \ili{Latin} is similar: as discussed by various authors (e.g.~\citealt{xuetal07}), \isi{deponents} are verbs with nonactive morphology but active syntax. Although they appear with a nonactive suffix, the verbs themselves are unergative or \isi{transitive}. The deponent verb \emph{sequor} `to follow' is syntactically \isi{transitive} but has no morphologically active forms:
 \begin{exe}
 \ex  
 \begin{xlist} 
     \ex  Regular Latin alternation: 
        \emph{amo-r} `I am loved' $<$ \emph{am\=o} `I love'
     \ex  Deponent Latin verb: 
        \emph{sequo-r} `I follow' $\nless$ *\emph{sequ\=o} `I follow'
 \z
\z 

Similar patterns have been discussed for various Indo-European languages by \cite{aronoff94}, \cite{embick04}, \cite{kallulli13}, \cite{wood15springer}, \cite{kastnerzu17} and \cite{grestenberger18}, among many others. While the analyses differ, what these cases all have in common is that individual roots require nonactive morphology.

Turning to another possible crosslinguistic parallel with inchoatives, it has been pointed out that in some languages, verbalizing suffixes do not contribute eventive semantics in certain environments. That is, they are phonologically overt but semantically null, a slightly different situation than ours. \citet{elenasamioti13,elenasamioti14} document a pattern in \ili{Greek} in which certain adjectives can only be derived if a verbalizing suffix is added to the root first. Crucially, there is no eventive semantics (unlike with our inchoatives); no weaving is entailed for~(\ref{ex:elena1}) nor planting for~(\ref{ex:elena2}). {The authors suggest that -\emph{tos} requires an eventive vP as its base, which is not possible with nominal roots like `weave' and `plant'.}
 \begin{exe}
\ex  \label{ex:elena1} \emph{if-an-tos} weave-\gsc{VBLZ}-\gsc{ADJ} `woven' 
\ex  \label{ex:elena2} \emph{fit-ef-tos} plant-\gsc{VBLZ}-\gsc{ADJ} `planted' \hfill \citep[97]{elenasamioti14} 
 \z 
In fact, the part of the structure consisting of the root and verbalizer might not even result in an acceptable verb \citep[100]{elenasamioti14}:
 \begin{exe}
\ex  \emph{kamban-a} `bell' $\sim$ ??\emph{kamban-iz-o} `bell (v)' $\sim$ \emph{kamban-is-tos} `sounding like a bell' 
 \z 

In a similar vein, \cite{marantz13} argues that an \emph{atomized individual} need not have undergone atomization, and analyzes a similar phenomenon in Japanese ``continuative'' forms that must be vacuously verbalized first{ before being nominalized} \citep{volpe05}. \cite{anagnostopoulou14thli} extends this idea of a semantically null exponent to cases like -\emph{ify}- in \emph{the class\underline{ifie}ds} (but see \citealt{borer14lingua} for a dissenting view).

In sum, we have evidence that v can be active in the semantics without selecting the meaning of the root, allowing a higher {\vz} head to derive nonactive verbs directly from the root rather than from an existing verb. Crucially here, though, little v still introduces an event variable.

	
	\subsection{Phonology} \label{vz:vz:phono}
The basic Vocabulary Item for {\vz} can be given using the shorthand in~(\nextx). The remainder of this section provides some Vocabulary Items and schematic derivations which make the division of morphological labor between {\vz} and T more explicit.
 \begin{exe}
\ex  {\vz} \lra~{\tnif} 
 \z 

\label{r1:3:3}The ingredients of the template \emph{\tnif} consist of the prefix \emph{ni-} in the past tense, a person/number/gender-conditioned allomorph in the future, and certain stem vowels. A full paradigm is given in Table~\ref{tab:3-3:tnif} and similar paradigms can be found elsewhere, e.g.~\cite{schwarzwald08}. What is not often mentioned in the literature---and what I have failed to note in \cite{kastner18nllt} myself---is that a process of de-spirantization applies in {\tnif} as well, namely in the ``imperfect'' forms (future, infinitive, imperative and nominalization), whereby the first root consonant does not spirantize (\dgs{X}). I will not provide an analysis of this aspect of the system but I do note that an analysis in terms of a floating feature can be implemented, docking onto the first consonant along the lines of [--cont]$_{\text{\gsc{ACT}}}$ on \dgs{Y} for {\va} in {\tpie} and {\thit} (Chapter~\ref{voice:va:phono}).

\begin{table}
\fittable{
		\begin{tabularx}{\textwidth}{lllllll} \midrule
 \lsptoprule
			& \multicolumn{2}{c}{Past} & \multicolumn{2}{c}{Present} &  \multicolumn{2}{c}{Future} \\\midrule
			& \gsc{M} & \gsc{F} & \gsc{M} & \gsc{F} & \gsc{M} & \gsc{F} \\\midrule\hline
			1\gsc{SG} & \multicolumn{2}{c}{niXYaZ-ti} & niXYaZ & niXYeZ-et & \multicolumn{2}{c}{e-\dgs{X}aYeZ/ji-\dgs{X}aYeZ}\\
			1\gsc{PL} & \multicolumn{2}{c}{niXYaZ-nu} & niXYaZ-im & niXYaZ-ot & \multicolumn{2}{c}{ni-\dgs{X}aYeZ}  \\\tablevspace
			2\gsc{SG} & niXYaZ-ta & niXYaZ-t & niXYaZ & niXYeZ-et & ti-\dgs{X}aYeZ & ti-\dgs{X}aYZ-i\\
			2\gsc{PL} & niXYaZ-tem & niXYaZ-ten/tem & niXYaZ-im & niXYaZ-ot & \multicolumn{2}{c}{ti-\dgs{X}aYZ-u}\\\tablevspace
			3\gsc{SG} & niXYaZ & niXYeZ-a & niXYaZ & niXYeZ-et & ji-\dgs{X}aYeZ & ti-\dgs{X}aYeZ\\
			3\gsc{PL} & \multicolumn{2}{c}{niXYeZ-u} & niXYaZ-im & niXYaZ-ot & \multicolumn{2}{c}{ji-\dgs{X}aYZ-u}\\
\lspbottomrule
 		\end{tabularx}
 }
		\caption{Inflectional paradigm for {\tnif}.}
	\label{tab:3-3:tnif}
\end{table}

Generally speaking, the form of the affix is determined by the Tense and phi-features on T; see Table~\ref{tab:3-3:t}. The stem vowel can be seen as \emph{-a-}, with the \isi{allomorphy} \emph{-e-} in the future forms and in present feminine.\footnote{Naturally it is also possible to consider \emph{-e} the default form and \emph{-a-} the contextual variant. To the extent that this question is theoretically interesting, one would want to consider the status of the ``imperfect'' stems mentioned immediately above. The other \emph{-e-} stem vowels in the paradigm are likely epenthetic, as in \cite{kastner18nllt}.}
\begin{table}
\begin{tabularx}{\textwidth}{lllll}
 \lsptoprule
	a.& T[Past,& 3\gsc{SG.M}] & \textbf{ni}-gmar & `he ended' \\
	b.& T[Fut,& 3\gsc{SG.M}] & \textbf{ji}-gam\textbf{e}r & `he will end' \\
	c.& T[Past,& 2\gsc{SG.F}] & \textbf{ni}-gmar-t & `you.\gsc{F} ended'\\
	d.& T[Fut,& 2\gsc{SG.F}] & \textbf{ti}-ganr-i & `you.\gsc{F} will end'\\
	e.& T[Pres,& \gsc{F}] & nigm\textbf{e}r-et & `\{I am / you are / she is\} ending'\\
\lspbottomrule
 \end{tabularx}
	\caption{The spell-out of {\vz} is conditioned by T.}
\label{tab:3-3:t}
\end{table}

We can briefly derive \emph{jigamer} `he will end' and \emph{tigamer} `she will end' as follows. First, the Vocabulary Items.

 \begin{exe}
\ex  \root{gmr} \lra~\emph{gmr} 
\ex  v \lra~(covert) 
\ex  \label{vi:vz} {\vz} \lra $\begin{cases} 
\text{a.~\emph{i,a,e}} & \text{/ T[Fut] \trace}\\
\text{b.~\emph{i,a,e}} & \text{/ T[Pres, F] \trace}\\
\text{c.~\emph{ni,a}} & \\
\end{cases}$

 \ex  
 \begin{xlist} 
 	\ex  3\gsc{SG.M} \lra~\emph{j} / {\trace} T[Fut] 
 	\ex  3\gsc{SG.F} \lra~\emph{t} / {\trace} T[Fut] 
 \z
\z 

This last set of VIs might seem complicated, but it is necessary in order to maintain uniform VIs for certain agreement affixes across templates; see Chapter~\ref{vz:va:vzva}. This is one of a number of choice points in the phonological analysis which I will not defend here, since my focus is not on the morphophonology per se.

The prosodic well-formedness constraints discussed in \cite{kastner18nllt} ensure that the vowels are inserted into the right ``slots'': \emph{jigamer} rather than *\emph{jiaegmr} or *\emph{jigaemr}. A simplified version of the phonological derivations:
 \begin{exe}
 \ex  
 \begin{xlist} 
 	\ex  j + /i,a,e-gmr/ $\rightarrow$ j + [i.ga.mer] $\rightarrow$ [ji.ga.mer] 
 	\ex  t + /i,a,e-gmr/ $\rightarrow$ t + [i.ga.mer] $\rightarrow$ [ti.ga.mer] 
 \z
\z 

Finally, {\vz} has the allomorph {\thit} in the context of {\va}; see Section \ref{vz:va:vzva}.


\section{\pz} \label{vz:pz}
The previous section analyzed the non-active verbs of {\tnif} using the head {\vz}. This section tackles the \isi{figure reflexives}; recall that these are active (agentive) verbs which obligatorily take a prepositional phrase\is{\isi{prepositional phrases}} as the complement to the verb. I propose that the head {\pz} is to {\vz} as \textit{p} is to Voice: it fails to syntactically license\is{licensing} an external argument \emph{of a preposition}. Recall that I assume a layered theory of prepositions, according to which P introduces the ``internal argument'' of the preposition, the \isi{Ground}, and \textit{p} introduces its ``external argument'', the \isi{Figure}.

Much of the analysis here follows the analysis of similar constructions in Icelandic proposed by \cite{wood15springer}. Here are the basics:
 \begin{exe}
 \ex  \textbf{\pz} 
 \begin{xlist} 
 	\ex  A \textit{p} head with a [--D] feature, prohibiting anything with a [D] feature from merging in its specifier. 
     \ex  \denote{\pz} = \denote{\emph{p}} = λxλs.Figure(x,s) 
 	\ex  {\pz} {\lra} {\tnif} \hfill (with the allomorph {\thit} to follow in Section \ref{vz:va:pzva}) 
 \z
\z 
I discuss the syntax and semantics together in what follows.

	\subsection{Syntax and semantics} \label{vz:pz:syn}	
		\subsubsection{Ordinary prepositions}
As noted above, I adopt the idea that subjects of \isi{prepositional phrases} are introduced by a separate functional head, a suggestion which has already been made in various guises by a number of researchers interested in the structure of \isi{prepositional phrases} \citep{vanriemsdijk90,rooryck96,koopman97,gehrke08phd,dendikken03,dendikken10}. In particular, \cite{svenonius03,svenonius07,svenonius10} implements this idea using the functional head \emph{p}. Borrowing terminology from \cite{talmy78} and related work, Likening the \emph{p}P to VoiceP, \cite{wood14nllt,wood15springer} suggests a parallelism: just like the verb assigns the semantic role of \isi{Theme} to its complement, P assigns the semantic role of \textbf{\isi{Ground}}. And just like Voice assigns the semantic role of \isi{Agent} to its specifier, \emph{p} assigns the semantic role of \textbf{\isi{Figure}} to its own specifier.

The dashed arrows in~(\nextx) show the assignment of semantic (thematic) roles in this system.\footnote{I take it for given that thematic roles are semantic functions but that something like the traditional theta-role does not exist \citep{schaefer08,layering15,wood14nllt,wood15springer,woodmarantz17,myler16mit,kastner17gjgl}; see the background given in Chapter~\ref{chap:intro}.} 
 \begin{exe}
 \ex  
 \begin{xlist} 
 	\ex   
 \Tree
	 [.\emph{p}P
	 	[.DP\\\emph{the book}\\{\tikz{\node (Fig) {\textbf{\textsc{figure}}};}} ]
	 	[
	 		[.{\tikz{\node (p) {\emph{p}};}} ]
	 		[.PP
	 			[.P\\{\tikz{\node (P) {\emph{on}};}} ]
	 			[.DP\\\emph{the table}\\{\tikz{\node (Ground) {\textbf{\textsc{ground}}};}} ]
	 		]
	 	]
	 ]
	\begin{tikzpicture}[overlay]
	\draw[dashed,thick,->] (p) .. controls +(south east:1) and +(east:1) .. (Fig);
	\draw[dashed,thick,->] (P) .. controls +(south west:1) and +(west:1) .. (Ground);
	\end{tikzpicture}
 	\ex      \denote{Voice}  =  λxλe.Agent(x,e)  
 	\ex      \denote{\emph{p}}  =  λxλs.Figure(x,s)  
 \z
\z 

An ordinary prepositional phrase\is{\isi{prepositional phrases}} in Hebrew is given in~(\nextx), for a verb in {\tkal}. As seen in the previous chapter, the structure comprises the root, v and \isi{Unspecified Voice}.
 \begin{exe}
 \ex  
 \begin{xlist} 
 	\ex   
[] 		{ \gll marsel sam {ts}aa{ts}ua al ha-smixa\\
 		  Marcel put toy on the-blanket\\
 		\glt `Marcel put a toy on the blanket.' } 
		
 	\ex  \Tree 
		[.VoiceP
		   [.{DP\\\emph{marsel}\\\textsc{agent}} ]
		   [
				[.Voice ]
		        [
					[.v
						[.{\root{sjm}} ]
						[.v ]
		            ]
					[.\emph{p}P
		                  [.DP\\\emph{{ts}aa{ts}ua}\\{`toy'}\\\textsc{figure} ]
		                  [
		                      [.\emph{p} ]
		                      [.PP
			                      [.P\\\emph{al}\\{`on'} ]
			                      \qroof{\emph{ha-smixa}\\{`the blanket'}\\\textsc{ground}}.DP
		                      ]
		                  ]
		              ]
		          ]
		   ]
		]
 \z
\z 

			\subsubsection{Figure reflexives} \label{vz:pz:syn:figrefl}	
Following \cite{wood15springer}, I postulate a variant of \emph{p}, namely {\pz}, which prohibits the \isi{Merge} of a DP in Spec,\emph{p}P, (\nextx).
 \begin{exe}
 \ex  \textbf{\pz:} 
 \begin{xlist} 
 	\ex  A \emph{p} head with a [--D] feature, prohibiting anything with a [D] feature from merging in its specifier. 
     \ex  \denote{\pz} = \denote{\textit{p}} = λxλs.Figure(x,s) 
 \z
\z 

In the current system, a given head might impose a semantic requirement which is usually fulfilled immediately if the parallel syntactic requirement is met. For example, Voice might introduce an \isi{Agent} role and license\is{licensing} Spec,VoiceP, such that the argument in the latter saturates the former. But it is also possible for a semantic predicate to be satisfied later on in the derivation, in \emph{delayed saturation}. Such cases have been recently identified (sometimes as ``delayed gratification'') in work on \ili{French} \citep{schaefer12}, \ili{Icelandic} \citep{wood14nllt,wood15springer}, English, \ili{Quechua} \citep{myler16mit}, \ili{Japanese} \citep{woodmarantz17} and \ili{Choctaw} \citep{tyler19}, although the idea that a predicate may be saturated in delayed fashion is not new in and of itself \citep{higginbotham85}.

Consider first the existing analysis of \ili{Icelandic}. \isi{Figure} reflexives in this language can appear in two configurations, one with a clitic -\emph{st} which does not concern us here \citep{wood14nllt}, and the other without it, as in~(\nextx):
\begin{exe}
\ex  \langinfo{Icelandic}{}{\citealt[168]{wood15springer}} \label{ex:vz:is-figrefl} \\
 \gll Hann labbaði inn í herbergið\\
 	   he.\gsc{NOM} strolled in to room.the.\gsc{ACC}\\
 	 \glt `He strolled into the room.' 
 \z 

On \citeauthor{wood15springer}’s (\citeyear{wood15springer}) analysis, the role of \isi{Figure} is not saturated within the \emph{p}P, since no DP is possible in Spec,{\pz}P. Rather, an argument introduced later, in Spec,VoiceP, saturates this predicate. The schematic structure in~(\nextx) shows the assignment of semantic roles using dashed arrows.
 \begin{exe}
\ex  
		 \Tree
		 [.VoiceP
			 [.{DP\\\tikz{\node (Agent) {\textsc{agent}};}\\\tikz{\node (Figup) {\textsc{figure}};}} ]
			 [
				 [.\tikz{\node (Voice) {Voice};} ]
				 [
					 [.v ]
					 [.\emph{p}P
						 [.\tikz{\node (Figdown) {---};} ]
						 [
							 [.\tikz{\node (pz) {\pz};} ]
							 [.PP
								 [.\tikz{\node (P) {P};} ]
								 [.{DP\\\tikz{\node (Ground) {\textsc{ground}};}} ]
							]
						]
					]
				]
			]
		]
	  \begin{tikzpicture}[overlay]
	  \draw[dashed,thick,->] (Voice) .. controls +(north west:1) and +(north east:1) .. (Agent);
	  \draw[dashed,thick,->] (P) .. controls +(south west:1) and +(west:1) .. (Ground);
	  \draw[dashed,thick,->] (pz) .. controls +(south:1) and +(south:2) .. (Figup);
	  \draw[dashed,thick,->] (pz) .. controls +(south west:1) and +(south west:1) .. node{\LARGE $\times$}(Figdown);
	  \end{tikzpicture}		    
 \z 

The structure for~(\ref{ex:vz:is-figrefl}) is given in~(\ref{tree:vz:is-figrefl}), adapted from \citet[170]{wood15springer}. \citeauthor{wood15springer}'s insight is that there is no argument filling Spec,{\pz}P which can saturate the \isi{Figure} role of {\pz}. The next DP merged in the structure, \emph{hann} `he', will then saturate both Voice's semantic role (\isi{Agent}) and the role of \isi{Figure} introduced by {\pz}. A variety of diagnostics for Icelandic show that the verb is agentive, with the DP \emph{Hann} merged in Spec,VoiceP, just like Hebrew \isi{figure reflexives} are agentive.
 \begin{exe}
\ex  \label{tree:vz:is-figrefl} 
		\Tree
		[.VoiceP
			[.{DP\\{\emph{hann}}\\`he'\\\textsc{agent}\\\textsc{figure}} ]
			[
				[.Voice\\{(assigns Agent)} ]
				[
					[.v
						[.{\root{\gsc{STROLL}}} ]
						[.v ]
					]
					[.{\pz}
							[.\emph{p}\\{(assigns Figure)} ]
							\qroof{\emph{inn} \dots}.PP
					]
				]
			]
		]
 \z 

Returning to Hebrew, we can adopt this proposal and give the derivation in~(\nextx) for a verb like \emph{nixnas le}- `entered’ in {\tnif}, where {\pz} introduces a \isi{Figure} semantically but does not introduce an argument in the syntax.
 \begin{exe}
 \ex  
 \begin{xlist} 
 	\ex    
[] 	{ \gll oren \glemph{nixnas} la-xeder\\
 	  Oren entered.\gsc{MID} to.the-room\\
 	\glt `Oren entered the room.' } 
	
 	\ex  \hspace{-5em} 
\scalebox{0.8}{
\Tree
    [.{VoiceP\\ λe∃s.\underline{Agent(Oren,e)} \& \underline{Figure(Oren,s)} \& in(s,room) \& enter(e) \& Cause(e,s)}
       [.{DP\\\emph{oren}} ]
       [.{λxλe∃s.\underline{Agent(x,e)} \& Figure(x,s) \& in(s,room) \& enter(e) \& Cause(e,s)}
           [.{Voice\\ λxλe.Agent(x,e)} ]
           [.{vP\\ λxλe∃s.\underline{Figure(x,s)} \& \underline{in(s,room)} \& enter(e) \& Cause(e,s)}
              [.{v\\ λPλe∃s.P(s) \& enter(e) \& Cause(e,s)}
	             [.\root{kns} ]
	             [.v ]
              ]
              [.{\emph{p}P\\ λxλs.Figure(x,s) \& \underline{in(s,room)}}
                  [.{\pz\\ λxλs.Figure(x,s)\\ \emph{ni-}} ]
                  \qroof{λs.in(s,room)}.PP
              ]
          ]
       ]
    ]
}
 \z
\z 

In~(\lastx) The \emph{p}P is composed via Event Identification, the vP via Function Composition (cf.~Restrict of \citealt{chungladusaw04}), and the VoiceP again via Event Identification.

The two main consequences of this configuration are that an external argument may be merged in Spec,VoiceP and that the obligatory prepositional phrase\is{\isi{prepositional phrases}} does not have a subject of its own. The generalization on \isi{figure reflexives} can now be derived: external arguments in {\tnif} saturate the \isi{Figure} role of an otherwise subjectless preposition. While in Icelandic {\vz} has overt reflexes \citep[Ch.~3.2]{wood15springer} and {\pz} is silent, in Hebrew we find morphological support for both.

It is interesting to note that {\pz} still introduces a \isi{Figure} role despite prohibiting a specifier. In this it is similar to ``free variable'' proposals in which Voice introduces the \isi{Agent} role in the semantics but no specifier in the syntax \citep{legate14,akkus19jl}.

One would be justified in wondering whether some other argument might intervene between vP and Voice, in which case it would be able to saturate the \isi{Figure} role. High applicatives would have been relevant here, but Hebrew has been argued to have only the possessive dative\is{unaccusativity tests} as a low \isi{applicative} for internal arguments \citep[46]{pylkkanen08}, meaning that the ApplP would be too low to influence derivation of the figure reflexive\is{figure reflexives}. The affected reading of these datives, however, actually implies a different structure for unergatives \citep[59]{pylkkanen08}, the nature of which is still unclear. See \cite{barashersiegalboneh15,barashersiegalboneh16} for some ideas.
	
	\subsection{Phonology} \label{vz:pz:phono}
In Hebrew, {\vz} and {\pz} are spelled out identically: a prefix (\emph{ni}-) and the relevant stem vowels, resulting in {\tnif}. This should not be an accident. In Section~\ref{vz:interim} and in Chapter~\ref{chap:i} I return to the idea that these are one and the same head, \emph{i}*, differing only in its height of attachment.

This section concludes with an extended note on \isi{linearization} and \isi{head \isi{movement}}. I have argued that {\vz} starts off high, above v and the root, while {\pz} starts off below them. Despite their different attachment sites, {\vz} and {\pz} are pronounced identically, as a prefix to the verb and certain vocalic readjustments.
 \begin{exe}
\ex \label{tree:headmov} 
	\begin{xlist}
		\ex
	Anticausatives in {\tnif} with \vz:\\
	\Tree
 	[.TP
	 	[.T ]
	 	[.VoiceP
	 		[.{---} ]
	 		[
	 			[.{\vz\\\fbox{\emph{ni-}}} ]
	 			[
	 				[.v
	 					[.\root{root} ]
	 					[.v ]
	 				]
	 				[.DP ]
	 			]
	 		]
	 	]
	 ]

	\ex Figure reflexives in {\tnif} with \pz:\\
	\Tree
 	[.TP
	 	[.DP ]
	 	[
		 	[.T ]
		 	[.VoiceP
		 		[.\sout{DP} ]
		 		[
		 			[.Voice ]
		 			[
		 				[.v
		 				    [.\root{root} ]
		 				    [.v ]
		 				]
		 				[.\emph{p}P
			 				[.{---} ]
			 				[
				 				[.{\pz\\\fbox{\emph{ni-}}} ]
				 				[.PP
					 				[.P ]
					 				[.DP ]
					 			]
					 		]
					 	]
		 			]
		 		]
		 	]
		 ]
	]
 \z 
 \z
 
Not much needs to be said about the affixation in~(\ref{tree:headmov}a) since the structure can be linearized\is{\isi{linearization}} as is: one morphophonological cycle combines the root with Voice and associated elements, and a second cycle attaches the prefix T (Chapter~\ref{voice:voice:phono} and~\citealt{kastner18nllt}). The phonological material on T might end up as a suffix rather than prefix due to general phonological constraints of the language (for example, if T is purely vocalic).

This is a different kind of theory than that of \cite{shlonsky89} and \cite{ritter95} who assume that all affixation results from \isi{head \isi{movement}} of the verb, ``picking up'' affixes as it moves up the syntactic tree \citep{pollock89} and eventually reaching the tense affixes on T.

Not all analyses assume that V reaches T in Hebrew. According to \cite{borer95} and \cite{landau06}, Hebrew V may raise to T in cases of ellipsis and VP-fronting, but not necessarily in the general case. For \citeauthor{landau06}, this V-to-T \isi{movement} is driven by T's need to express inflectional features, which appear on T in Hebrew but may lower to V in other languages or be expressed via \emph{do}-support in English. Implementing affixation using \isi{Agree} between T and V absolves V of having to adjoin to T itself.

Returning to~(\ref{tree:headmov}b), a challenge arises if we try to linearize\is{linearization} {\pz} between the root and T. The problem is that {\pz} should be pronounced in the same position as {\vz} is in~(\ref{tree:headmov}a). The phonological consequences go beyond just one exponent which needs to be placed correctly: in {\tnif} the prefix itself is conditioned by T; see Table~\ref{tab:3-4:t}.
\begin{table}
	\begin{tabularx}{.75\textwidth}{lllll}
 \lsptoprule
	a.& T[Past,& 3\gsc{SG.M}] & \textbf{ni}-xnas & `he entered' \\
	b.& T[Fut,& 3\gsc{SG.M}] & \textbf{ji}-kan\textbf{e}s & `he will enter' \\
	c.& T[Past,& 2\gsc{SG.F}] & \textbf{ni}-xnas-\textbf{t} & `you.\gsc{F} entered'\\
	d.& T[Fut,& 2\gsc{SG.F}] & \textbf{ti}-kans-\textbf{i} & `you.\gsc{F} will enter'\\
\lspbottomrule
 	\end{tabularx}
	\caption{The spell-out of {\pz} is conditioned by T.}
\label{tab:3-4:t}
\end{table}

Under the assumptions of the current theory, {\pz} needs to be local to T in order to correctly spell out its own prefix and add vowels to the stem.

Standard \isi{head \isi{movement}} could raise {\pz} and adjoin it to v (or Voice via v), deriving the correct morpheme order. The problem is not empirical but conceptual: all other morphological derivations in the Trivalent theory proceed without \isi{head \isi{movement}}, by simply linearizing\is{linearization} structure under explicit phonological constraints. Here we would require {\pz} to raise (perhaps obligatory for \textit{p} as well). What feature drives this \isi{movement}? Any feature that accounts for solely this \isi{movement} would be suspiciously stipulative. But if \isi{head \isi{movement}} is more common, does the complex head then raise further, to Voice and then to T? A theory which allows phonological words to be read directly off the structure, but which also allows construction of phonological words by \isi{head \isi{movement}}, runs the risk of being too permissive.

Attempts to derive \isi{head \isi{movement}} effects have led to various proposals which I cannot contrast here. The operation Conflation \citep{halekeyser02,harley13oup} adjoins only the phonology of a complement onto that of its sister, similar to Local Dislocation. This operation can be thought of as purely phonological Incorporation \citep{baker85,baker88}. See \citet[Ch.~2.5]{rimell12} for an evaluation.

Another theoretical proposal is that of \isi{head \isi{movement}} as remnant \isi{movement} \citep{koopmanszabolcsi00,koopman05,koopman15u20}. On this approach all affixes are heads which take their base as a complement. Suffixes are endowed with an \isi{EPP} feature raising their complement to Spec, resulting in the affix spelling out to the right of the stem. For this proposal to work, the structure in~(\ref{tree:headmov}b) would need to be changed since \pz, as a prefix, needs to take v+\root{root} as its complement: [{\pz} [v [v \root{root}~\!] [PP\is{\isi{prepositional phrases}}]]]. But now it is not clear where the prepositional object PP\is{\isi{prepositional phrases}} appears. PP\is{\isi{prepositional phrases}} is, by hypothesis, the complement of \emph{p}; if we treated it as the complement of v, we would be abandoning the little \emph{p} hypothesis, leaving us with no morpheme to spell out the \emph{ni-} prefix in the first place.

One other kind of mechanism for exceptional tweaking of individual morphemes in the morphophonology is Local Dislocation~\citep{embicknoyer01}. This mechanism swaps the linear order of two adjacent morphemes at spell-out. Local Dislocation is assumed to apply after Vocabulary Insertion; I keep the syntactic labels in~(\nextx) for consistency of exposition.
 \begin{exe}
 \ex  
 \begin{xlist} 
 	\ex  Linearized structure: 
		T-Voice-v-\root{root}-\pz
 	\ex  Local Dislocation: 
		$\Rightarrow$ T-Voice-v-\textbf{\pz-\root{root}}
 	\ex  Pruning of silent exponents: 
		$\Rightarrow$ T-\pz-\root{root}
 \z
\z 
At the end of the day, the analysis in~(\lastx) simply formalizes the idea that {\pz} is a prefix.

Local Dislocation happens after VI, so {\pz} will not be able to be conditioned properly by T. Instead, I could assume that the actual VI for {\pz} is \emph{i}-, and the \emph{n}- prefix a partial exponent of T; but this entire setup grinds to a halt once {\va} intervenes between the two as in {\thit}:
 \begin{exe}
\ex  T-Voice-{\va}-v-\pz-\root{root} 
 \z 

None of the alternatives are particularly satisfying. I assume \isi{head \isi{movement}} and leave matters as is.


\section{Interlude: From {\tnif} to {\thit}} \label{vz:interim}
We have seen that verbal forms in {\tnif} are in principle compatible with internal and external arguments, though not within the same verb (there are no \isi{transitive} verbs in {\tnif}):
\hammer{
 \begin{exe}
 \ex  \label{ex:gen-tnif2}\textbf{Generalizations about {\tnif}} 
 \begin{xlist} 
 	\ex  \textbf{Configurations:} Verbs appear in unaccusative, passive and figure reflexive structures; but never in a simple transitive configuration. 
 	\ex  \textbf{Alternations:} Some verbs are anticausative or passive versions of verbs in {\tkal}. 
 \z
\z 
}
I proposed that two distinct verb classes exist which share the same morphology. For non-active verbs, with no external argument, it was suggested that {\vz} blocks the introduction of an external argument and triggers {\tnif} morphology. For \isi{figure reflexives}, with an agent\is{\isi{Agent}} and an obligatory PP\is{\isi{prepositional phrases}} complement, I claimed that {\pz} introduces the PP\is{\isi{prepositional phrases}} but does not supply a subject of its own for the preposition, while also triggering {\tnif} morphology. This analysis falls within a view of argument structure which distinguishes syntactic features, such as the requirement for a specifier, from semantic roles, such as the requirement for an \isi{Agent} or a \isi{Figure}.

In line with the basic root hypothesis of DM, none of the derivations go from a verb in {\tkal} to a verb in {\tnif}; to the extent that the Trivalent proposal is more explanatory than existing ones (and I believe it is, as I claim concretely in Section~\ref{vz:others}), it provides support for this assumption. In particular, {\tnif} is not one morpheme: it is a collection of identical morphophonological forms masking a variety of different structural configurations.

Importantly, the feature [--D] is used on both {\vz} and {\pz}. I have already alluded to the idea that the only difference between the two verb classes in {\tnif} is the height of attachment of the [--D] feature; in other words, that {\vz} and {\pz} are the same head, except that {\vz} is what we label it when it combines with vP and {\pz} is what we label it when it combines with a PP\is{\isi{prepositional phrases}}. Recently, \cite{woodmarantz17} have proposed that heads such as Voice, Appl\is{\isi{applicative}} and \emph{p} are indeed contextual variants of the same functional head, which they call \emph{i}*. On their view, ``Voice'' is simply the name we give to \emph{i}* which takes a vP complement, ``high Appl\is{\isi{applicative}}'' is the name we give to \emph{i}* which takes a vP complement and is in turn embedded in a higher \emph{i}* (itself being Voice), ``\emph{p}'' is the name we give to an \emph{i}* which takes a PP\is{\isi{prepositional phrases}} complement, and so on. I return to this idea in Chapter~\ref{chap:i}.

The next section re-introduces the agentive modifier {\va} from the previous chapter and explores its interaction with {\vz}. Some of these interactions are more obvious, as with \isi{figure reflexives} ({\va} + {\pz}). Others require slight tweaks to our understanding of specific elements, as with anticausatives; and others are more interesting still, as with reflexive verbs. There are no reflexive verbs in {\tnif}. The current theory will provide an answer to the ``how'' question of how these verbs appear in {\thit} as well as an answer to the ``why'' question of why {\thit} and not {\tnif}: reflexivity requires a theme (\vz) which is agentive (\va). In general, the parallels between {\tkal} and {\thif} on the one hand, and {\tpie} and {\thit} on the other hand, will reflect the layering assumption which is at the core of the current work. 


\section{\thit: Descriptive generalizations} \label{vz:thit}
The ``intensive middle'' template {\thit} is traditionally viewed as the reflexive template. Yet reflexive verbs form only a small part of it. I will first show how it houses anticausative and inchoative verbs, similarly to {\tnif}, but not passives. I then look briefly at \isi{figure reflexives}, which appear in both {\tnif} and {\thit}, and true reflexives, which only appear in this template. Section~\ref{vz:va} analyzes these patterns in terms as combinations of the modifier {\va} from Chapter~\ref{voice:va} with {\vz} or {\pz}.

This template is also considered to be a natural one for reciprocal verbs, but \cite{barashersiegal16mmm} has shown that reciprocalization is licensed\is{licensing} by strategies which do not have to do with the specific template; see also \cite{siloni12} and \cite{poortmanetal18}. Because the relationship between templates and reciprocals is indirect, I will not discuss their place in the current theory.
 

	\subsection{Non-active verbs} \label{vz:thit:nact}
A few non-active verbs in {\thit} are given in Table~\ref{tab:3-6:thit}: anticausatives in rows~a--c and inchoatives in rows~d--f.
\begin{table}
\begin{tabularx}{\textwidth}{lc>{\em}ll>{\em}ll}
 \lsptoprule
& Root & \multicolumn{2}{c}{{\tpie} active} & \multicolumn{2}{c}{{\thit} non-active} \\\midrule
a.& \root{pr\dgs{k}}& pirek & `dismantled' & hitparek & `fell apart' \\
b.& \root{p{\ts}{\ts}}& po{\ts}e{\ts} & `detonated' & hitpo{\ts}e{\ts} & `exploded'\\
c.& \root{bʃl} & biʃel & `cooked' & hitbaʃel & `got cooked'\\\tablevspace
d.& \root{'lf}& \multicolumn{2}{c}{---} & hitalef & `fainted' \\
e.& \root{'tʃ}& \multicolumn{2}{c}{---} & hitateʃ & `sneezed'\\
f.& \root{'rk} & \multicolumn{2}{c}{---} & hitarex & `grew longer'\\
\lspbottomrule
 \end{tabularx}
	\caption{Examples of non-active verbs in {\thit}.}
\label{tab:3-6:thit}
\end{table}

\textbf{Anticausatives} in {\thit} alternate with causatives in {\tpie}:
 \begin{exe}
 \ex \label{ex:vz:anticaus-va}
 \begin{xlist} 
 	\ex   
[] 		{ \gll josi \glemph{biʃel} marak.\\
 		  Yossi cooked.\gsc{INTNS} soup\\
 		\glt `Yossi cooked some soup.' } 
	
	
 	\ex   
[] 		{ \gll ha-marak \glemph{hitbaʃel} ba-ʃemeʃ.\\
 		  the-soup got.cooked.\gsc{INTNS.\gsc{MID}} in.the-sun\\
 		\glt `The soup cooked in the sun.' } 
	
 \z
\z 

As expected, they are incompatible with agent\is{\isi{Agent}}-oriented adverbs and \emph{by}-phrases:
 \begin{exe}
\ex     [*] 	{ \gll ha-{\ts}amid \glemph{hitparek} \{~al-jedej ha-{\ts}oref / be-mejomanut~\}\\
 	  the-bracelet dismantled.\gsc{INTNS.MID} by the-jeweler {} in-skill\\
 	\glt (int. `The bracelet was dismantled by the jeweler/skillfully') } 
	
 \z 

They are compatible with `by itself\is{agentivity}':
 \begin{exe}
\ex   
[] 	{ \gll ha-{\ts}amid \glemph{hitparek} \glemphu{me-a{\ts}mo}\\
 	  the-bracelet dismantled.\gsc{INTNS.MID} from-itself\\
 	\glt `The bracelet fell apart of its own accord.' } 
	
 \z 

As expected, they are also compatible with the two unaccusativity diagnostics introduced earlier,  VS order\is{unaccusativity tests} (\nextx) and the possessive dative\is{unaccusativity tests} (\anextx).
 \begin{exe}
\ex \label{ex:vs-anticaus}  
 { \gll \glemph{hitpark-u} \glemphu{ʃloʃa} \underline{galgalim} be-ʃmone ba-boker\\
 	  dismantled.\gsc{INTNS.MID}-\gsc{3PL} three wheels in-eight in.the-morning\\
 	\glt `Three wheels fell apart at 8am.' } 
	
 \ex  
 { \gll \glemph{hitparek} \glemphu{l-i} ha-ʃaon\\
   dismantled.\gsc{INTNS.\gsc{MID}} to-me the-watch\\
 \glt `My watch broke.' } 

 \z 

Note in this context that this view of anticausatives in {\thit} as alternants of an agentive \isi{transitive} verb in {\tpie} is unexpected under a certain conception which has proven popular in previous work on argument structure. The purported generalization is that decausativization can only occur if the external argument of the \isi{causative} verb is not specified with respect to its thematic role, i.e.~can be a \isi{Cause} \citep{unaccusativity95,reinhart02}. If verbs in {\tpie} are indeed agentive, but can nonetheless be decausativized into an anticausative in {\thit}, this generalization will need to be amended, but I will not do that here; see \citet[52]{layering15} for an overview of related work and ideas.

Continuing on to \textbf{inchoatives}, they pattern with anticausatives. They are incompatible with agent\is{\isi{Agent}}-oriented adverbs and \emph{by}-phrases:
 \begin{exe}
 \ex  
 \begin{xlist} 
 		\ex    
[*] 			{ \gll josi \glemph{hitalef} al-jedej ha-kosem\\
 			  Yossi passed.out.\gsc{INTNS.\gsc{MID}} by the-magician\\
 			\glt (int. `Yossi fainted by the magician') } 
		
 		\ex    
[??] 			{ \gll josi \glemph{hitalef} \glemphu{be-mejomanut}\\
 			  Yossi passed.out.\gsc{INTNS.\gsc{MID}} in-skill\\
 			\glt (int. `Yossi fainted skillfully') } 
		
 \z
 \ex   
 \begin{xlist} 
 	\ex  \label{ex:incho1}  
	{ \gll sara \glemph{hitatʃ-a} \{me-ha-avak / ??be-xavana\}\\
 	  Sarah sneezed.\gsc{INTNS.\gsc{MID}-F} \phantom{\{}from-the-dust {} \phantom{??}on-purpose\\
 	\glt `Sarah sneezed because of the dust/??on purpose' } 
	
 \z
\z 

They are compatible with `by itself\is{agentivity}', although this is less evident with animate arguments:
 \begin{exe}
 \ex  \label{ex:thit-inch-byitself} 
 \begin{xlist} 
 	\ex[]  {My current visit in Israel was supposed to last a bit longer than two weeks,\footnote{ 
		\url{https:\\www.maveze.co.il/\%d7\%9e\%d7\%95\%d7\%a8-\%d7\%9b\%d7\%94\%d7\%9f-\%d7\%91\%d7\%a7\%d7\%99\%d7\%a6\%d7\%95\%d7\%a8-\%d7\%99\%d7\%a6\%d7\%90\%d7\%aa\%d7\%99-\%d7\%a2\%d7\%9d-\%d7\%9c\%d7\%99\%d7\%9b\%d7\%95\%d7\%93\%d7\%a0\%d7\%99\%d7\%a7/}, retrieved July 2019.}\\
 \gll aval \glemph{hitarex} \glemphu{me-a{\ts}mo} od va-od\\
 			  but lengthened.\gsc{INTNS.MID} from-itself more and-more\\
 			\glt `but kept getting longer and longer.' } 
		
 	\ex   
		[??]{ \gll ha-kalb-a \glemph{hitatʃ-a} \glemphu{me-a{\ts}ma}\\
 			  the-dog-\gsc{F} sneezed.\gsc{INTNS.MID}-\gsc{F} from-herself\\
 			\glt (int.~`The dog sneezed unintentionally') } 
		
 \z
\z 

And they pass the unaccusativity diagnotics:
 \begin{exe}
\ex  
[] 	{ \gll \glemph{hitalf-u} \glemphu{ʃloʃa} \underline{xajalim} ba-hafgana\\
 	  fainted.\gsc{INTNS.\gsc{MID}}-\gsc{3PL} three soldiers in.the-protest\\
 	\glt `Three soldiers fainted during the protest.' \hfill \citep[397]{reinhartsiloni05} } 
	
\ex   
[] 	{ \gll \glemph{hitarx-u} \glemphu{l-i} kol ha-bikurim\\
 	  lengthened.\gsc{INTNS.MID}-\gsc{3PL} to-me all the-visits\\
 	\glt `All of my visits got longer.' } 
	
 \z 

Curiously, there are \textbf{no \isi{passive} verbs} in {\thit}. No verb can be used with a \emph{by}-phrase to get a \isi{passive} reading, nor can some entailment relevant to an implicit agent\is{\isi{Agent}} be obtained.\footnote{I suspect that a wug test would show this even for nonce verbs, but have not attempted such an experiment. Odelia Ahdout (p.c.) notes the following counterexamples from her comprehensive database which do seem to have \isi{passive} readings: \emph{hitstava} `was ordered', \emph{hitbatsa/hitbatsea} `was carried out', \emph{hitbakeʃ} `was asked', \emph{hitbaser} `was informed', \emph{hitkabel} `was received' and perhaps also \emph{hitbarex} `was blessed'. If these are true counterexamples then perhaps there is no structural reason for the paucity of \isi{passive} verbs in {\thit}, though this low rate should still receive some other kind of explanation.}
 \begin{exe}
\ex 	  
[*] 	{ \gll ha-{\ts}amid \glemph{hitparek} kedej lekabel pi{\ts}uj me-ha-bituax\\
 	  the-bracelet dismantled.\gsc{INTNS.MID} in.order to.receive.\gsc{INTNS} compensation from-the-insurance\\
 	\glt (int.~`The bracelet was dismantled in order to collect the insurance') } 
	
 \z 

Based on the diagnostics used throughout this mongraph, the non-active verbs in {\thit} are demonstrably unaccusative.

	\subsection{Figure reflexives} \label{vz:thit:figrefl}
\isi{Figure} reflexives in {\thit} are compatible with agent\is{\isi{Agent}}-oriented adverbs.
 \begin{exe}
 \ex \label{ex:vz:figrefl-va}
 \begin{xlist} 
 	\ex   
[] 		{ \gll bjartur \glemph{hiʃtaxel} (be-xavana) \{~derex ha-kahal / la-xeder~\}\\
 		  Bjartur squeezed.\gsc{INTNS}.\gsc{MID} in-purpose through the-crowd {} to.the-room\\
 		\glt `Bjartur squeezed (his way) on purpose through the crowd/into the room.' } 
		
 	\ex   
[] 		{ \gll ha-xatul \glemph{hitnapel} al ha-regel ʃeli (be-zaam)\\
 		  the-cat pounced.\gsc{INTNS}.\gsc{MID} on the-foot mine in-wrath\\
 		\glt `The cat angrily pounced on my foot.' } 
		
 \z
\z 

They do not pass the unaccusativity diagnostics.
 \begin{exe}
\ex    [\#] 		{ \gll \glemph{hitnapel} ha-xatul al ha-regel ʃeli\\
 		  pounced.\gsc{INTNS}.\gsc{MID} the-cat on the-foot mine\\
		\glt `Once the cat pounced on my foot, then...'\\
			(does not mean: `The cat pounced angrily on my foot.')
	}
\ex    
[*] 	{ \gll ha-xatul \glemph{hitnapel} la-mita al ha-sadin\\
 	  the-cat pounced.\gsc{INTNS.MID} to.the-bed on the-sheet\\
 	\glt (int.~`The cat pounced on the bed's bedsheet') } 
	
 \z 

As with \isi{figure reflexives} in {\tnif}, many of these verbs denote events of directed motion, (\nextx a), but there are other kind of activities as well, each with its own obligatory preposition, (\nextx b--c). It must also be acknowledged that not all have truly agentive meanings (\nextx d).\footnote{\cite{siloni08} claims that simple unergatives exist in {\thit}, but my view of the psych-verbs she presents is that they too require a PP\is{\isi{prepositional phrases}} complement, e.g.~\emph{hitbajeʃ *(me)-} `was shy (of)'.}
 \begin{exe}
 \ex  
 \begin{xlist} 
 	\ex   
[] 		{ \gll bjartur \glemph{hiʃtaxel} *(derex ha-kahal / la-xeder)\\
 		  Bjartur squeezed.\gsc{INTNS}.\gsc{MID} through the-crowd {} to.the-room\\
 		\glt `Bjartur squeezed (his way) through the crowd/into the room.' } 
		
 	\ex   
[] 		{ \gll ha-xatul \glemph{hitnapel} *(al ha-regel ʃeli) \\
 		  the-cat pounced.\gsc{INTNS}.\gsc{MID} on the-foot mine \\
 		\glt `The cat angrily pounced on my foot.' } 
		
 	\ex   
[] 		{ \gll ahed \glemph{hitmard}-a *(neged ha-avlot)\\
 		  Ahed rebelled.\gsc{INTNS}.\gsc{MID}-\gsc{F} against the-wrongs\\
 		\glt `Ahed rebelled against the wrongs.' } 
		
 	\ex   
[] 		{ \gll ha-melex \glemph{hitmaker} *(le-samim)\\
 		  the-king got.addicted.\gsc{INTNS.MID} to-drugs\\
 		\glt `The King got addicted to drugs.' } 
		
 \z
\z 

What is particularly interesting is that these \isi{figure reflexives} share morphological marking---\thit---with actual reflexives (which do not exist in {\tnif}). These are discussed next.

	\subsection{Reflexives} \label{vz:thit:refl}
By ``reflexive verbs'' I mean canonical reflexive verbs as in~(\nextx):
 \begin{exe}
\ex  \textbf{Canonical reflexive verb} 
	(i) A monovalent verb whose DP internal argument X is interpreted as both Agent and Theme, \textbf{and} (ii) where no other argument Y (implicit or explicit) can be interpreted as Agent or Theme, \textbf{and} (iii) where the structure involves no pronominal elements such as \emph{himself}.
 \z 

The definition in~(\lastx) confines our discussion to reflexives that are morphologically marked\is{markedness}, rather than construction that can use another strategy such as anaphora. As noted earlier, reflexive verbs in Hebrew are only attested in \thit. A sample is given in~(\nextx).
 \begin{exe}
\ex \label{ex:refl}\emph{hitgaleax} `shaved himself', \emph{hitraxets} `washed himself', \emph{hitnagev} `toweled himself down', \emph{hitaper} `applied makeup to himself', \emph{hitnadev} `volunteered himself'. 
 \z 

Reflexive verbs in {\thit} may~(\nextx) or may not~(\anextx) have a \isi{causative} variant in {\tpie}:
 \begin{exe}
 \ex \label{ex:vz:refl-va} 
 \begin{xlist} 
 	\ex   
[] 		{ \gll jitsxak \glemph{iper} et tomi\\
 		  Yitzhak made.up.\gsc{INTNS} \gsc{ACC} Tommy\\
 		\glt `Yitzhak applied make-up to Tommy.' } 
	
	
 	\ex   
[] 		{ \gll tomi \glemph{hitaper}\\
 		  Tommy made.up.\gsc{INTNS.\gsc{MID}}\\
 		\glt `Tommy put on make-up' (*`Tommy got make-up applied to him') } 
	
 \z

 \ex \label{ex:vz:refl-va2} 
 \begin{xlist} 
 	\ex    
[*?] 		{ \gll jitsxak \glemph{kileax} et tomi\\
 		  Yitzhak \root{\dgs{k}lx}.\gsc{INTNS}.Past \gsc{ACC} Tommy\\
 		\glt (int.~`Yitzhak showered Tommy') } 
	
 	\ex   
[] 		{ \gll tomi \glemph{hitkaleax}\\
 		  Tommy showered.\gsc{INTNS.\gsc{MID}}\\
 		\glt `Tommy showered' (*`Tommy got showered') } 
	
 \z
\z 

In Hebrew, verbs like those in~(\ref{ex:refl}) are only possible in {\thit}. Reflexive verbs often pose puzzles in various languages, since these are cases in which one argument appears to have two thematic roles, agent\is{\isi{Agent}} and patient. The degree to which this configuration is tracked by the morphology varies by language. English shows no morphological difference between (\nextx a--b), even in though the readings clearly differ.
 \begin{exe}
 \ex  
 \begin{xlist} 
 	\ex \emph{Dana kicked.} 
		$\nRightarrow$ Dana kicked herself.
 	\ex  \emph{Dana shaved.} 
		$\Rightarrow$ Dana shaved herself.
 \z
\z 

While some languages, like English, do not differentiate morphologically between verbs like \emph{shave} and verbs like \emph{kick}, many languages do express reflexivity through morphological means. I will argue in Section~\ref{vz:va:vzva:refl} that the reflexive morphology of Hebrew reflects an internal composition of \isi{agentivity} (\va) with no independent external argument (\vz), based on \cite{kastner17gjgl}.

Crosslinguistically, templates like {\tnif} and {\thit} from this chapter are reminiscent of non-active markers such as Romance \gsc{SE}, German \emph{sich}, Russian \emph{-sja} and the Greek non-active suffix \gsc{NACT}. Crosslinguistic work shows that this kind of marking is often syncretic between anticausatives, inchoatives, passives, middles, reciprocals and reflexives \citep{geniusiene87,klaiman91,alexiadoudoron12,kastnerzu17}. Yet unlike languages like French, for instance, where \gsc{se} might be ambiguous between a number of readings (reflexive, reciprocal and anticausative), {\thit} is never ambiguous in Hebrew for a given root.\footnote{See \cite{kastner17gjgl} for one possible counterexample, the verb \emph{hitnaka} `cleaned up'.} For while French \emph{se} can be used in reflexive, reciprocal and non-active contexts with a variety of predicates~(\nextx), Hebrew {\thit} is unambiguous in that a verb like \emph{hitlabeʃ} `got dressed' is only reflexive~(\anextx). It cannot be used in an anticausative context, as shown by its incompatibility with `by itself\is{agentivity}'.
 \begin{exe}
 \ex \ili{French} reflexives and reciprocals, after \citet[839]{labelle08}:
 	\begin{xlist}
 	\ex 
{ \gll Les enfants \glemph{se} sont tous soigneusement \glemph{lav\'es}.\\
 	  the children \gsc{SE} are all carefully washed.\gsc{3PL}\\
	\glt `The children all washed each other carefully' \hfill [reciprocal]\\
	`The children all washed themselves carefully' \hfill [reflexive]
}

 	\ex  French middle \citep[835]{labelle08}\\
	{ \gll Cette robe \glemph{se} \glemph{lave} facilement.\\
 	  this dress \gsc{SE} wash-\gsc{3S} easily\\
 	\glt `This dress washes easily.' } 
	
 	\ex  French anticausative \citep[835]{labelle08} \\
	{ \gll Le vase \glemph{se} \glemph{brise}.\\
 	  the vase \gsc{SE} breaks-\gsc{3S}\\
 	\glt `The vase is breaking.' } 

	\end{xlist}

\ex  Hebrew reflexives are not reciprocal:	\\
	{ \gll luk ve-pjer \glemph{hitlabʃ-u}. (*me-a{ts}mam)\\
 	  Luc and-Pierre dressed.up.\gsc{INTNS.\gsc{MID}}-\gsc{3PL} \phantom{(*}from-themselves\\
 	\glt `Luc and Pierre got dressed' \hfill [reflexive only] } 
	
\end{exe}

Implementing the rest of our diagnostics, we see that reflexives straightforwardly allow \isi{Agent}-oriented adverbs (\nextx).
 \begin{exe}
\ex  
[] 		{ \gll josi \glemph{hitgaleax} \{~be-mejomanut / likrat ha-reajon~\}\\
 		  Yossi shaved.\gsc{INTNS}.\gsc{MID} in-skill {} towards the-interview \\
 		\glt `Yossi shaved skillfully / in preparation for his interview.' } 
	
 \z 

They do not allow `by itself\is{agentivity}', which is already degraded with animate arguments as we saw in~(\ref{ex:thit-inch-byitself}b).
 \begin{exe}
\ex   
[*] 		{ \gll josi \glemph{hitgaleax} \glemphu{me-a{\ts}mo}\\
 		  Yossi shaved.\gsc{INTNS}.\gsc{MID} from-himself\\
 		\glt (int.~`Yossi's shaving happened to him') } 
	
 \z 

They also do not pass the unaccusativity diagnostics.
 \begin{exe}
\ex[\#]{  VS order:\\
	\gll \glemph{hitkalx-u} \glemphu{ʃloʃa} \glemphu{xatulim} be-arba ba-boker\\
 	  showered.\gsc{INTNS.\gsc{MID}}-\gsc{3PL} three cats in-four in.the-morning\\
 	\glt (int. `Three cats washed themselves at 4am.') } 
	
\ex[\#]{  Possessive dative:\\
	 \gll ʃloʃa xatulim \glemph{hitkalx-u} \glemphu{l-i} be-arba ba-boker\\
 	  three cats showered.\gsc{INTNS.\gsc{MID}}-\gsc{3PL} to-me in-four in.the-morning\\
	\glt `Three cats washed themselves at 4am and I was adversely affected.'\\
		(\# int. `My three cats washed themselves at 4am.')
	}

\ex[]{ Episodic plural:\\
 \gll \glemph{mitapr-im} ba-rexov, bo lirot!\\
 	  make.up.\gsc{INTNS.MID}-\gsc{PL.M} in.the-street come see\\
 	\glt `People are applying make-up in the street, come see!' } 
	
 \z 

To summarize the empirical overview of {\thit}, it is similar to {\tnif} in some respects and different in others. It, too, creates anticausatives and inchoatives (but no passives). It allows for \isi{figure reflexives} and also for canonical reflexives. What we never see---again like in {\tnif}---is a simple \isi{transitive} construction consisting of subject, verb and direct object:\footnote{One distinct counterexample is \emph{hitstarex} `needed'; see \citet[130ff16]{harveskayne12}.}

\hammer{
 \begin{exe}
 \ex  \label{ex:gen-thit}\textbf{Generalizations about {\thit}} 
 \begin{xlist} 
 	\ex  \textbf{Configurations:} Verbs appear in unaccusative, figure reflexive and reflexive structures; but not in a simple transitive configuration. 
 	\ex  \textbf{Alternations:} Some verbs are anticausative or reflexive versions of verbs in {\tpie}. 
 \z
\z 
}

This constellation of facts can be accounted for once we clarify the composition of {\va} and {\vz}. The root also plays an important part, as alluded to above, but that aspect of the data will not be discussed in depth here.


\section{Adding {\va} to [--D]} \label{vz:va}
The data above highlights the puzzle of reflexive verbs: why are they possible in {\thit} and only in {\thit}? In this section I provide analyses of the phenomena above, all based on the idea that this template is morphosyntactically (and hence morphophonologically) the most complex. Reviewing the analysis in \cite{kastner17gjgl}, I will propose that reflexives and anticausatives share an unaccusative structure, but that the root constrains the derivation in a specific way. Reflexive verbs are argued to be the result of unaccusative syntax (\vz) with an agentive modifier (\va) and particular, self-oriented lexical semantics. The crucial point for our overall purposes is that the reflexive readings fall out from the unique combinatorics of {\vz} and {\va}, a combination of elements which no other ``template'' can provide.

Section~\ref{vz:va:vzva} analyzes the combination of {\va} with {\vz}, yielding non-active verbs and reflexives. Section~\ref{vz:va:pzva} rounds off the picture with the derivation of \isi{figure reflexives}.

	\subsection{{\va} + {\vz}} \label{vz:va:vzva}
		\subsubsection{Non-active verbs} \label{vz:va:vzva:nact}
Syntactic structure building proceeds as usual. We will see this by deriving the alternation between \isi{causative} \emph{pirek} in {\tpie} and anticausative \emph{hitparek} in {\thit}. The combination of {\va} and vP predicts that an event expressed by [{\va} vP] can either receive an external argument, if we merge Voice, or not, if we merge {\vz}. This state of affairs is exactly what we find; much of the literature talks of {\tpie} and {\thit} alternating (\citealt{doron03}, \citealt{arad05}, as well as much previous work and the traditional grammars).

 \begin{exe}
 \ex  
 \begin{xlist} 
 	\ex  Core vP \\
			\Tree
   	     [.vP
                [.{\va} ]
                [.vP
                    [.v
                        [.\root{prk} ]
                        [.v ]
                    ]
                    [.DP ]
                ]
             ]

	
 	\ex  \emph{pirek} `dismantled' \\
				\Tree
		        [.VoiceP
		            [.DP ]
		            [
		                [.Voice ]
		                [.vP
			                [.{\va} ]
			                [.vP
			                    [.v
			                        [.\root{prk} ]
			                        [.v ]
			                    ]
			                    [.DP ]
			                ]
			             ]
		            ]
		        ]

 	\ex  \emph{hitparek} `fell apart' \\
			\Tree
      [.VoiceP
          [.{---} ]
          [
              [.{\vz} ]
              [.vP
	              [.{\va} ]
	              [.vP
	                  [.v
	                      [.\root{prk} ]
	                      [.v ]
	                  ]
	                  [.DP ]
	              ]
	           ]
          ]
      ]	
 \z
\z 

The semantics relevant to {\va} is repeated in~(\nextx):
 \begin{exe}
 \ex  \denote{Voice} =  
 \begin{xlist} 
 	\ex  λP.P \phantom{agent(x,e)xxx} / \trace~ \{ \root{npl} `\root{\gsc{FALL}}', \root{kpa} `\root{\gsc{FREEZE}}' , \dots \} 
 	\ex  λxλe.Agent(x,e) or λxλe.Cause(x,e) 
 	\ex  λxλe.\text{Agent}(x,e) / \trace~\va 
 \z
\z 

In this section we will see two allosemes of {\vz}, one the identity function we are familiar with (\nextx c) and one the agentive version we would expect from {\va} (\nextx a). The \isi{passive} alloseme (\nextx b) is repeated for completeness, but there is no rule invoking it in the context of {\va}.
 \begin{exe}
 \ex  \label{ex:vz-denote} 
 \begin{xlist} 
 	\ex  \denote{\vz} \lra~λxλe.Agent(x,e) / \trace~\va 
 	\ex  \denote{\vz} \lra~λPλe∃x.Agent(x,e) \& P(e) / \trace~\{\root{rtsx} `murder', \root{'mr} `say’, \dots\} 
 	\ex  \denote{\vz} \lra~λP$_{<s,t>}$.P 
 \z
\z 

When we put the pieces together, however, we find that we do not get \textbf{anticausative} (\isi{causative} but non-agentive) semantics. The translations in~(\lastx) cannot be the whole story because (\lastx a) straightforwardly entails agentive semantics for verbs in {\thit}.

\cite{kastner17gjgl} proposes that the rule of allosemy in~(\ref{ex:vz:thit-impov}) removes the agentivity requirement of {\va}~for roots such as \root{pr\dgs{k}} which give anticausatives. \cite{kastner16phd,kastner17gjgl} develops a view of roots according to which their lexical semantics determines, at least in part, whether they will trigger the rule in~(\ref{ex:vz:thit-impov}). This change renders the resulting verb \emph{hitparek} `fell apart' anticausative, rather than a potential reflexive such as `tore himself to pieces'.
 \begin{exe}
\ex \label{ex:vz:thit-impov}\denote{\va~\!} $\rightarrow$ {\zero} / {\vz} \trace~\{\root{XYZ} |  
 \root{XYZ} $\in$ 
 \\ \phantom{a} \hfill 
	\root{pr\dgs{k}} `\gsc{DISMANTLE}', \root{bʃl} `\gsc{COOK}', \root{ptsts} `\gsc{EXPLODE}', \dots\}
 \z 
The process can be likened to impoverishment \citep{bonet91,noyer98} in the semantic component (cf.~\citealt{nevins15roots}).

Another way of encoding this information would have been to build it right back into the denotations of Voice, as in~(\nextx):
 \begin{exe}
\ex  Addition to~(\blastx), to be rejected: \\
	\denote{\vz} = λP$_{<s,t>}$.P / \trace~{\va} \{\root{pr\dgs{k}} `\gsc{DISMANTLE}', \root{bʃl} `\gsc{COOK}', \root{ptsts} `\gsc{EXPLODE}', \dots\}
 \z 
The problem here is one of \isi{locality}: the root is separated from {\vz} by {\va}. Existing theories of contextual \isi{allosemy} maintain a strict linear adjacency requirement between trigger and alloseme \citep{marantz13,kastner16phd}. The kind of action-at-a-distance typical of roots \isi{licensing} a head is more similar to impoverishment, which again happens at a distance.

To summarize informally, {\va} brings in an agentive requirement, but it is also close enough to the root for certain roots to disable this requirement. It is probably no accident that these roots relate to events which are ``other-oriented'' like dismantling and cooking; see \cite{kastner17gjgl} for additional discussion of this point. But whatever the formal analysis, the current system explains why anticausatives in {\thit} look like de-transitivized versions of causatives in {\tpie}: {\vz} is added to the same structure (vP) that regular Voice would have been added to.

With anticausatives explained, not much remains to be said about \textbf{inchoatives} beyond the discussion of those in {\tnif} from Section~\ref{vz:vz:sem}. And finally, \textbf{passives} do not arise either. This behavior is captured by the rules in~(\ref{ex:vz-denote}) but is not explained by them (we could just as well have written a rule generating the \isi{passive} alloseme of {\vz} in the context of {\va}). I have no deeper explanation to propose at this point. Returning to a simple composition of {\vz} and {\va}, however, leads us to an understanding of reflexives.

		\subsubsection{Reflexives} \label{vz:va:vzva:refl}
The intuition behind the analysis of reflexives is as follows: reflexive verbs in {\thit} consist of an unaccusative structure with extra agentive semantics. This combination is only possible if the internal argument is allowed to saturate the semantic function of an external argument by delayed saturation, in the way formalized here.

The structure and semantic derivation in~(\ref{tree:vz:thit-refl}) fleshes out the derivation of the reflexive verb in~(\ref{ex:vz:thit-refl}).
 \begin{exe}
\ex  \label{ex:vz:thit-refl} 
{ \gll dani \glemph{hitraxets}\\
   Danny washed.\gsc{INTNS.\gsc{MID}}\\
 \glt `Danny washed (himself).' } 

 \z 

The argument DP, `Danny', starts off as the internal argument. No external argument is merged in the specifier of {\vz} and the structure is built up as usual. Nevertheless, the specifier of T needs to be filled because of a syntactic requirement, namely the \isi{EPP}. The internal argument then raises directly to Spec,TP in order to satisfy the \isi{EPP}, checking the syntactic feature but also satisfying the \isi{Agent} role of {\vz} in delayed saturation (Section~\ref{vz:pz:syn:figrefl}).

 \begin{exe}
\ex  \label{tree:vz:thit-refl} 
%\hspace{-7em}
\fittable{
\scalebox{0.8}{
	\Tree
	[.{TP\\λe.\emph{wash}(e) \& Theme(Danny,e) \& \underline{Agent(Danny,e}) \& Past(e)}
		[.\tikz{\node (SpecTP) {DP};}\\\emph{Dani} ]
		[.{λxλe.\emph{\emph{wash}}(e) \& Theme(Danny,e) \& Agent(x,e) \& \underline{Past(e)}}
			[.{\phantom{xx}T\phantom{xx}\\λe.Past(e)\footnotemark} ]
			[.{VoiceP\\λxλe.\emph{wash}(e) \& Theme(Danny,e) \& Agent(x,e)}
				[.--- ]
				[.{λxλe.\emph{wash}(e) \& Theme(Danny,e) \& \underline{Agent(x,e)}}
					[.{\vz\\λxλe.Agent(x,e)} ]
					[.
						[.{\va} ]
						[.{vP\\λe.\emph{wash}(e) \& Theme(\underline{Danny},e)}
							[.v\\{λxλe.\emph{wash}(e) \& Theme(x,e)}
								[.\root{rxts}\\\gsc{WASH} ]
								[.v ]
							]
						[.\tikz{\node (Obj) {DP};} ]
						]
					]
				]
			]
		]
	]

    \begin{tikzpicture}[overlay]
   	\draw[thick,->] (Obj) .. controls +(south:9) and +(south west:8) .. (SpecTP);
    \end{tikzpicture}
}
}
 \z 
\footnotetext{The exact denotation of T is immaterial here.}

\vspace{2em}

The crucial points in this derivation are the VoiceP node and Spec,TP: after the internal argument raises to Spec,TP, the derivation can converge. The resulting picture is similar to that painted by \cite{spathasetal15} for certain reflexive verbs in Greek, where the agentive modifier \emph{afto} combines with non-active Voice to derive a reflexive reading; see \cite{spathasetal15} or \cite{kastner17gjgl} for further details on the Greek.\footnote{{\va} is different than Greek \emph{afto}, and {\vz} different from Greek Non-active Voice in a number of respects I cannot treat here but list for future reference. (i) Greek non-active is \isi{passive}-like in Naturally Reflexive Verbs (\emph{wash}) and Naturally Disjoint Verbs (\emph{accuse/praise/destroy}). (ii) \emph{Afto} is only possible with Non-Active Voice, whereas {\va} can combine with \isi{Unspecified Voice}. (iii) The combination of \emph{Afto} and Non-active Voice always yields reflexives. (iv) \emph{Afto} only combines with Naturally Disjoint Verbs.}

Like with \isi{figure reflexives}, one would be justified in wondering whether other material between vP and TP could intervene, disrupting this derivation. And like with \isi{figure reflexives}, if we try to think of how applicatives fit in we see that the exact nature of the possessive dative\is{unaccusativity tests} is unclear. If we treat the construction as \isi{transitive} (since there is an internal argument), the possessive dative\is{unaccusativity tests} is a low \isi{applicative}, meaning that the ApplP would be too low to influence the derivation. In any case the possessor DP never raises out of its \isi{applicative} PP\is{\isi{prepositional phrases}} to Spec,TP, a configuration which would have disrupted this derivation. And if we were to treat this construction as unergative (one argument with an \isi{Agent} role) then the nature of the dative\is{\isi{case}} is different \citep{barashersiegalboneh15,barashersiegalboneh16}.

What about clauses smaller than TP? Embedded clauses in Hebrew are either full CPs with an overt complementizer such as \emph{ʃe-} `that' or infinitival clauses. Hebrew verbs have an infinitival prefix, \emph{le-}, which presumably spells out T, indicating that the TP layer is intact.
 \begin{exe}
\ex  
[] 	{ \gll josi ra{\ts}a \glemph{le-hitkaleax}\\
 	  Yossi wanted to-shower.\gsc{INTNS.MID}\\
 	\glt `Yossi wanted to take a shower.' } 
	
 \z 

This leaves us with \isi{nominalizations}. It is standard to assume that \isi{nominalizations} preserving the argument structure of the underlying verb are derived by merging a nominalizer with the verbal constituent, here VoiceP (as discussed in Chapter~\ref{passn:n}). In this case there really is no embedded T layer.

I can imagine two scenarios here, both promising but neither more convincing than the other at this point. The first is that if n projects a covert \emph{pro} as the external argument, then this DP will be able to take on the open \isi{Agent} role.\footnote{This is the standard assumption for \isi{nominalizations} at the moment, as recapped in Chapter~\ref{passn:n}. On a theory in which n existentially closes over the \isi{Agent}, the derivation might still be able to go through, depending on specific assumptions regarding Spec,n and the compositional semantics.} The second is simply a prediction that reflexives in {\thit} should not have a valid nominalization\is{nominalizations}. This claim has not been made before (as far as I know) and the data is unclear, judging by a few informal consultations:
 \begin{exe}
 \ex  
 \begin{xlist} 
 	\ex    
[\%] 		{ \gll \glemph{hitgalxut-o} ʃel dani lemeʃex eser dakot hergiza otanu\\
 		  shave.\gsc{INTNS.MID.NMLZ}-of of Danny during ten minutes annoyed.\gsc{CAUS} us\\
 		\glt (int.~`Danny's shaving for ten minutes annoyed us') } 
	
 	\ex    
[\%] 		{ \gll \glemph{ha-histarkut} / \glemph{ha-hitaprut} he-mejumenet ʃel ha-jeled\\
 		  the-comb.\gsc{INTNS.MID.NMLZ} {} the-makeup.\gsc{INTNS.MID.NMLZ} the-skilled of the-boy\\
 		\glt (int.~`the boy's skilled combing / application of makeup') } 
	
 \z
\z  
A much larger set of verbs would have to be tested in order to fully understand the pattern.

On another note, I have been treating reflexives as underlyingly unaccusative even though they pass \isi{agentivity} diagnostics and fail unaccusativity diagnostics. The question is what these diagnostics are actually diagnosing. Assuming that the \isi{agentivity} diagnostics are semantic in nature concords with the current analysis, since the \isi{Agent} role is saturated (this is why passives pass these tests). The unaccusativity diagnostics are more complicated: \cite{kastner17gjgl} summarizes evidence indicating that the requirement for the possessive dative\is{unaccusativity tests} might be semantic as well, and further speculates that VS order\is{unaccusativity tests} only obtains with surface unaccusatives (where the internal argument remains low; see \citealt{unaccusativity95}).

Overall, the analysis showcases how complex structure ({\vz} and {\va}~\!) correlates with complex meaning and complex morphology. On the meaning side of things, reflexives in Hebrew do not come from a dedicated functional or lexical item. There must be some conspiracy of factors in order to derive a reflexive reading. The complex structure is tracked by complex morphology: verbs in {\thit} have a number of distinguishing morphophonological properties, namely the prefix, the non-spirantized medial root consonant \dgs{Y}, and the stem vowels inherent to the template.

A verb like \emph{titnadev} `she will volunteer' is derived as follows (see \citealt{kastner18nllt}):
 \begin{exe}
\ex  
    \Tree
        	[.TP
        	[ ]
        	[
        		[.{T+Agr}
        		  [.T\\{[Fut]} ]
        		  [.\gsc{3SG.F}\\{\emph{t-}} ]
        		]
        		[.VoiceP
        		    [.{---} ]
        		    [.
        			    [.{\vz\\\emph{it-,a,e}} ]
        			    [.
        			    	[.{\va} ]
        			    	[.
	        				    [.v
	        					    [.\root{ndv} ]
	        					    [.v ]
	        					]
	        				    [.DP ]
	        				]
        			    ]
        		    ]
        		]
        	]
        	]

 \ex \label{ex:titpane2}Vocabulary Items: 
 \begin{xlist} 
     \ex  \root{ndv} \lra~\emph{ndv} 
     \ex  \va~\lra~[--cont]$_{\gsc{ACT}}$ / {\trace}~\{ \root{XYZ} $|$ Y $\in$ p, b, k \} 
     \ex  \vz~\lra~\emph{it},\emph{a,e} / T[Fut,\gsc{3SG.F}] {\trace} \va 
     \ex  3\gsc{SG.F} \lra~ \emph{t} / {\trace} T[Fut] 
 \z

\ex  Phonology: 
 	t + /it-a,e-ndv/ $\rightarrow$ t + [it.na.dev] $\rightarrow$ [tit.na.dev]
 \z 
    
	\subsection{{\va} + {\pz}} \label{vz:va:pzva}
The final piece of the jigsaw is \isi{figure reflexives} in {\thit}. At this point, it is easy to see where this piece fits. The semantics of a figure reflexive\is{figure reflexives} {\pz} is augmented by the agentive requirement of {\va}. Everything said about the semantics and phonology of these elements continues to hold; a representative derivation is given in~(\nextx).

 \begin{exe}
 \ex  
 \begin{xlist} 
 	\ex    
[] 	{ \gll bjartur \glemph{hiʃtaxel} la-xeder\\
 	  Bjartur squeezed.\gsc{INTNS.MID} to.the-room\\
 	\glt `Bjartur squeezed his way into the room.' } 
	
 	\ex  
 \z
\z 
\fittable{
\scalebox{0.8}{
\Tree
    [.{VoiceP\\ λe∃s.\underline{Agent(Bjartur,e)} \& \underline{Figure(Bjartur,s)} \& in(s,room) \& enter(e) \& Cause(e,s)}
       [.{DP\\\emph{bjartur}} ]
       [.{λxλe∃s.\underline{Agent(x,e)} \& Figure(x,s) \& in(s,room) \& enter(e) \& Cause(e,s)}
           [.{Voice\\ λxλe.Agent(x,e)} ]
			[.vP
				[.{\va} ]
	           [.{vP\\ λxλe∃s.\underline{Figure(x,s)} \& \underline{in(s,room)} \& enter(e) \& Cause(e,s)}
	              [.{v\\ λPλe∃s.P(s) \& enter(e) \& Cause(e,s)}
		             [.\root{ʃxl} ]
		             [.v ]
	              ]
	              [.{\emph{p}P\\ λxλs.Figure(x,s) \& \underline{in(s,room)}}
	                  [.{\pz\\ λxλs.Figure(x,s)\\ \emph{ni-}} ]
	                  \qroof{λs.in(s,room)}.PP
	              ]
	          ]
	        ]
       ]
    ]
}
}


Having concluded the analytical part of this chapter, I summarize the findings in Section~\ref{vz:sum}. Some alternatives are mentioned in Section~\ref{vz:others}, followed by a bigger-picture view of where this fits within the monograph.


\section{Summary of generalizations and claims} \label{vz:sum}
The generalizations about each of {\tnif} and {\thit} are repeated in~(\nextx)--(\anextx).
\hammer{
 \begin{exe}
 \ex  \label{ex:gen-tnif-sum}\textbf{Generalizations about {\tnif}} 
 \begin{xlist} 
 	\ex  \textbf{Configurations:} Verbs appear in unaccusative, passive and figure reflexive structures; but never in a simple transitive configuration. 
 	\ex  \textbf{Alternations:} Some verbs are anticausative or passive versions of verbs in {\tkal}. 
 \z
\z 
}

\hammer{
 \begin{exe}
 \ex  \label{ex:gen-thit-sum}\textbf{Generalizations about {\thit}} 
 \begin{xlist} 
 	\ex  \textbf{Configurations:} Verbs appear in unaccusative, figure reflexive and reflexive structures; but not in a simple transitive configuration. 
 	\ex  \textbf{Alternations:} Some verbs are anticausative or reflexive versions of verbs in {\tpie}. 
 \z
\z 
}

Remember, however, that ``template'' is a descriptive term for certain morphophonological forms. The traditional view is that a template is a morphological primitive with its own uniform phonology, syntax and semantics. The assumptions in this book are different: verbs are built up syntactically, and it could be that some structures end up with similar or even identical morphology. But the real distinction is between syntactic structures (and their interpretation). The anticausatives and \isi{figure reflexives} that share the template {\tnif} are no more related syntactically than the English past tense verb and past participle sharing the suffix -\emph{ed}; perhaps there is an underlying similarity there, but it would need to be argued for.

Summary Table~\ref{tab:1-8:tnif}, repeated from the introductory section, recaps:
\begin{table}
	\begin{tabularx}{\textwidth}{llcc} 
		\lsptoprule
		\multicolumn{2}{c}{Construction}	& {\tnif}	& {\thit} \\\midrule
		\multirow{3}{*}{Non-active} & Anticausative	& {\vz}	& {\va}, {\vz}\\
		& Inchoative & {\vz}	& {\va}, {\vz}\\
		& Passive &	{\vz}	&	---\\\tablevspace
		Active & Figure reflexive	& {\pz}	& {\va}, {\pz}\\\tablevspace
		Reflexive & Reflexive	& ---	& {\va}, {\vz}\\
		\lspbottomrule
	\end{tabularx}
	\caption{Verbs with [--D].}
	\label{tab:1-8:tnif}
\end{table}

It is not accurate to call {\tnif} a ``\isi{passive}'' template, nor is {\thit} the ``reflexive'' template. These constructions are possible, but what is more important is the structures giving rise to them. In addition, the existence of \isi{figure reflexives} has been documented and analyzed, providing support for a non-uniform analysis of superficially similar intransitive forms.

Reflexive verbs appear only in the template {\thit}, a fact which had not previously received any formal analysis. In a system such as the one put forward in this book, combining the \isi{agentivity} requirement of {\va} with the single-argumenthood of {\vz} derives this pattern. This analysis receives additional confirmation in the morphology, where the spell-out of both {\va} and {\vz} can be seen.

The analyses in this chapter call into question any attempt to view templates as independent morphemes as well as other decompositional accounts. Some of these vies are challenged next.


\section{Discussion and outlook} \label{vz:others}
The theory of trivalent Voice leads us to an ``emergent'' view of templates, according to which they arise from the combination of functional heads.

The traditional approach to Semitic templates has been to treat them as independent atomic elements, i.e.~morphemes. Contemporary work in this vein spans highly divergent implementations but includes \cite{arad03,arad05}, who decomposed verbal templates into flavors of v, spell-outs of Voice and conjugation\is{conjugation class} classes; \cite{borer13oup}, for whom different templates are different ``functors''; \cite{aronoff94,aronoff07}, who identifies templates with conjugation\is{conjugation class} classes; and \cite{reinhartsiloni05}, \cite{schwarzwald08} and \cite{laks11,laks14}, whose lexicalist accounts similarly grant morphemic status to verbal templates.

As far as morphemic analyses are concerned, an overarching problem is that a given template does not have a deterministic syntax nor does it have a deterministic semantics. The morphemic analysis would have to say that {\tnif} is ambiguous between a non-active and figure reflexive\is{figure reflexives} reading, or that {\thit} is three-way ambiguous between an anticausative, figure reflexive\is{figure reflexives} and canonical reflexive. Two crucial problems then arise. The first is that not all verbs in these templates are ambiguous. The second is that the existing readings are an accident; the templates could just as well have been ambiguous between a \isi{transitive} and a reflexive reading, but no Hebrew template has this property. Decompositional theories have principled explanations for what is and is not possible, as with {\tnif} where we have shown a morphological correlation between lack of \isi{Agent} and lack of \isi{Figure}. In contrast, a morphemic theory might be unnecessarily powerful and would arbitrarily list what each template, and perhaps each verb, may do. To see this, I will consider two major theories of Hebrew morphology, those of \cite{doron03,doron13voice} and \cite{arad03,arad05}. See \cite{kastnertucker19cup} for additional background and theoretical discussion.

These two alternative theories are exemplified below using the three-way alternation between a \isi{transitive} verb in {\tkal}, an ``intensive'' \isi{transitive} in {\tpie} and an anticasuative in {\thit}. The relevant data are as follows:
 \begin{exe}
 \ex  \label{ex:to-derive} 
 \begin{xlist} 
 \ex   
[] 	{ \gll ha-mar{\ts}a \glemph{kav'}-a et moed ha-bxina\\
 	  the-lecturer.\gsc{F} set.\gsc{SMPL}-\gsc{F} \gsc{ACC} date.of the-exam\\
 	\glt `The lecturer set the exam date.' } 
	
 \ex   
[] 	{ \gll eʃet roʃ ha-memʃala \glemph{kib'}-a et maamad-a ba-xevra\\
 	  wife.of head.of the-government set.\gsc{INTNS}-\gsc{F} \gsc{ACC} standing-hers in.the-society\\
 	\glt `The Prime Minister's wife cemented her place in society.' } 
	
 \ex   
[] 	{ \gll maamad eʃet roʃ ha-memʃala \glemph{hitkabea} ba-xevra\\
 	  standing.of wife.of head.of the-government set.\gsc{INTNS.MID} in.the-society\\
 	\glt `The Prime Minister's wife status in society was established.' } 
	
 \z
\z 

In the Trivalent theory, this three-way alternation is built on the core vP. Merging Voice gives the simple \isi{transitive} verb~(\ref{tree:to-derive-ik}a). Attaching {\va} to the vP modifies its semantics, (\ref{tree:to-derive-ik}b). Merging {\vz} instead of Voice gives the anticausative variant~(\ref{tree:to-derive-ik}c). I use ``EA'' for the external argument DP and ``IA'' for the internal argument DP in order to avoid ambiguity below.
 \begin{exe}
 \ex  \label{tree:to-derive-ik} 
 \begin{xlist} 
 	\ex   
		\emph{kava} `set':\\
				\Tree
				[.VoiceP
					[.EA ]
					[.
						[.Voice ]
						[.vP
							[.v
								[.\root{kb'} ]
								[.v ]
							]
							[.IA ]
						]
					]
				]			
 		\ex  \emph{kibea} `cemented': \\
				\Tree
				[.VoiceP
					[.EA ]
					[.
						[.Voice ]
						[.vP
							[.{\va} ]
							[.vP
								[.v
									[.\root{kb'} ]
									[.v ]
								]
								[.IA ]
							]
						]
					]
				]
 		\ex  \emph{hitkabea} `was cemented': \\
				\Tree
				[.VoiceP
					[.EA ]
					[.
						[.{\vz} ]
						[.vP
							[.{\va} ]
							[.vP
								[.v
									[.\root{kb'} ]
									[.v ]
								]
								[.IA ]
							]
						]
					]
				]			
 \z
\z 

	\subsection{Distributed morphosemantics \citep{doron03}} \label{vz:others:ed}
Within the decompositional theories, the most obvious alternative is the morphosemantic system of \cite{doron03}, a direct forebear to the current theory. That system was the first to identify basic non-templatic elements that combine compositionally in order to form Hebrew verbs. For example, a MIDDLE head $\mu$ was used to derive the ``middle'' template {\tnif}, where I make use of {\vz}.

		\subsubsection{The three-way alternation}
Let us see how the alternations in~(\ref{ex:to-derive}) are derived. In this theory, the root provides the basic semantics and introduces the internal argument itself. Little v introduces the external argument and the \isi{Agent} role (like our Voice). This combination yields~(\ref{tree:to-derive-doron}a). The head \textsc{intns} is the inspiration for {\va}, modifying the event and adding an \isi{Agent} role if none was there yet. This head also spells out {\tpie}, as in~(\ref{tree:to-derive-doron}b). The alternation, then, ``happens'' very low, at the level of root-attachment. Adding the non-active head \textsc{mid} instead of v removes the requirement for an \isi{Agent} and spells out {\thit} together with the \textsc{intns} head, (\ref{tree:to-derive-doron}c). Note how the internal argument now merges later.
 \begin{exe}
 \ex  \label{tree:to-derive-doron} 
 \begin{xlist} 
 	\ex  \emph{kava} `set': \\
		\Tree
		[.
			[.EA ]
			[.
				[.v ]
				[.\root{kb'}
					[.\root{kb'} ]
					[.IA ]
				]
			]
		]
 	\ex 	\emph{kibea} `cemented': \\
		\Tree
		[.
			[.EA ]
			[.
				[.v ]
				[.\textsc{intns}
					[.
						[.\textsc{intns} ]
						[.\root{kb'} ]
					]
					[.IA ]
				]
			]
		]
 	\ex  
		\emph{hitkabea} `was cemented':\\
		\Tree
		[.
			[.IA ]
			[.
				[.\textsc{mid} ]
				[.\textsc{intns}
					[.\textsc{intns} ]
					[.\root{kb'} ]
				]
			]
		]
 \z
\z 

The important conceptual difference is that my elements are syntactic whereas those in \cite{doron03} can be characterized as morphosemantic: each one had a distinct semantic role, but what regulates the syntactic \isi{licensing} of arguments remained unclear. A \citeauthor{doron03}-style system takes the semantics as its starting point, attempting to reach the templates from syntactic-semantic primitives signified by the functional heads. Such a system runs into the basic problem of Semitic morphology: one cannot map the phonology directly onto the semantics. For example, there is no way in which a \isi{causative} verb has a unique morphophonological exponent.

		\subsubsection{Additional issues}
On the empirical side more concretely, the morphosemantic theory did not engage with \isi{figure reflexives} directly but instead derived all reflexive readings using a \gsc{REFL} head. This is not a useful morphosyntactic construct since it cannot distinguish, on its own, between a figure reflexive\is{figure reflexives}, a reflexive verb such as `shave’ and a construction with an anaphor such as `shave yourself’. Yet we have seen that \isi{figure reflexives} have specific syntactic and semantic characteristics which distinguish them from intransitive reflexives like \emph{hitgaleax} `he shaved’ (the latter, for instance, does not require or even allow a prepositional phrase\is{\isi{prepositional phrases}} complement).

A similar problem arises when \citet[60]{doron03} derives reflexives in {\thit} by assuming that a head \gsc{MID} assigns the \isi{Agent} role for this root. This explains why \emph{histager} `secluded himself' is agentive, hence reflexive. However, if the only relevant elements are {\vz} and the root, then a verb in the same root in {\tnif} (where I have {\vz} and \citealt{doron03} has \gsc{MID}) is also predicted to be agentive. This expectation is incorrect: \emph{nisgar} `closed' is unaccusative. That analysis is almost a mirror image of the one presented here: while I let {\va} add \isi{agentivity} to a structure with \vz, thereby deriving reflexives, the morphosemantic account invokes added \isi{agentivity} for certain roots, bypassing the syntax in ways that lead to false predictions.

While each part of this problem could be overcome on its own, the system as a whole has little to say about the unaccusative (for anticausatives) and unergative (for reflexives) characteristics of verbs in {\thit}, since it is not based strictly in the syntax. I conclude, then, that ``templates'' are the by-product of functional heads combining in the syntax in systematic ways, in support of the general system developed in this book. Where we have made progress is by flipping one of the assumptions on its head: that the primitives have strict syntactic content and flexible semantic content, rather than strict semantic content and unclear syntactic content.
%%%
	\subsection{Templates as morphemic elements} \label{vz:others:morph}
The most explicit analysis other than \citeauthor{doron03}'s with which the Trivalent proposal can be contrasted is the foundational work by \cite{arad03,arad05}. Unlike \citeauthor{doron03}'s work and the current proposal, \cite{arad05}'s work attempted to scale back some of the structural commitments about alternations.

		\subsubsection{The three-way alternation}
Syntactically, a standard structure in \cite{arad05} is built up using a root, v and Voice. The verbalizer v additionally has four semantic ``flavors''. The template is divided phonologically into a prosodic skeleton on v and vowels on Voice. In order to fit these morphosyntactic pieces, a number of additional assumptions are required. First, roots select the templates they appear in, as a given root may idiosyncratically appear only with certain templates (as in the current theory). Second, there are four syntactic flavors of v: unmarked\is{markedness}, stative, inchoative and \isi{causative}, in order to account for the argument structural correlates of the templates. Finally, in order to specify which templates alternate with which, Arad must stipulate conjugation\is{conjugation class} classes. For example, in Conjugation Class 4, {\tpie} is the \isi{causative} variant and {\thit} is the inchoative variant \citep[220]{arad05}. It is assumed that the anticausative alternation goes from inchoative to \isi{causative}.
		
What this theory then does is specify spell-out rules using two sets of diacritics: which template a given flavor of v spells out, and which \isi{conjugation\is{conjugation class} class} this variant participates in.\footnote{\citet[227ff41]{arad05} claims that the diacritics are notationally equivalent to rules in the Encyclopedia, allowing them to interpret large segments of syntactic structure.} A subset of the spell-out rules is reproduced next, with the ones relevant to the examples in~(\ref{ex:to-derive}) highlighted \citep[230--231]{arad05}. Rules for individual templates are given first in each block, followed by rules for conjugation\is{conjugation class} classes.

 \begin{exe}
 \ex  Distributed Conjugation Diacritics in \cite{arad05}: \label{ex:arad-classes} 
 \begin{xlist} 
\begin{multicols}{2}
 	\ex   v$_{unmarked}$: \\
			\textbf{$ \alpha$ $\rightarrow$ {\tkal}} \\
			$\beta$ $\rightarrow$ {\tnif}\\
			$\gamma$ $\rightarrow$ {\tpie}\\
			$\delta$ $\rightarrow$ {\thif}\\
			$\epsilon$ $\rightarrow$ {\thit}
 	\ex  v$_{inch}$: \\
			$ \alpha$ $\rightarrow$ {\tkal} \\
			\dots \\
			\textbf{$\epsilon$ $\rightarrow$ {\thit}}\\
			\dots \\
			\textbf{Conjugation 4 $\rightarrow$ {\thit}}\\
			\dots
		\columnbreak
 	\ex  v$_{stat}$: \\
			$ \alpha$ $\rightarrow$ {\tkal} \\
			Class 3 $\rightarrow$ {\tkal}\\
			Class 5 $\rightarrow$ {\tkal}
 	\ex  v$_{caus}$: \\
			\textbf{$\gamma$ $\rightarrow$ {\tpie}}\\
			$\epsilon$ $\rightarrow$ {\thif}\\
			Conjugation 1 $\rightarrow$ {\tkal}\\
			\dots \\
			\textbf{Conjugation 4 $\rightarrow$ {\tpie}}\\
			\dots
	\end{multicols}
 \z
\z 

Causative \emph{kava} `set' is derived by applying the relevant rule from~(\ref{ex:arad-classes}a), which essentially allows a root to appear in {\tkal}. The alternation between {\tkal} and {\tpie} is not considered grammatical enough to be formalized in this theory, so we move to the alternation between \emph{kibea} `cemented' and \emph{hitkabea} `was cemented'. This is an alternation in which the former verb is \isi{causative} and the latter anticausative, and so we find the \isi{causative} template in~(\ref{ex:arad-classes}d) and the anticausative (``inchoative'') template in~(\ref{ex:arad-classes}b). The two are matched up in Conjugation Class 4. Using the correct flavors of v and the correct \isi{conjugation\is{conjugation class} class} ensures that only attested interpretations of the templates arise. There are no stative verbs in {\tpie} or {\thit}, for example, because stative v only has rules that insert {\tkal}. 

Since the goal of this work is to reduce the amount of generality encoded by the system, the technical outcome is appropriate. This does mean, however, that the theory ends up with functional structure which does not determine argument structure but is simply correlated with it, unlike in the current approach. In addition, most of the syntactic work is carried out by the flavors of v, but these have no unique spell-out, raising the question of whether there is any independent motivation for them beyond accounting for the conjugation\is{conjugation class} classes themselves. Almost by design, this theory of Hebrew cannot easily be adapted to the morphology of any non-Semitic language.

		\subsubsection{Additional issues}
Syntactic and lexicalist accounts both need to stipulate that only a subset of roots (or stems) licenses reflexive derivations. What is at issue here is the status of the template. The general problem with morphemic approaches to templates is that a given template simply does not have a deterministic syntax or semantics, as already seen time and time again in the last two chapters. \citet[197]{arad05} and \citet[564]{borer13oup} can even be read as speculating that a configurational approach (like the current theory) might be more viable than a feature-based or functor-based approach. As far as the treatment of reflexives is concerned, morphemic accounts can go no further than stipulating that {\thit} is the template for reflexive verbs.

	\subsection{Conclusion} \label{vz:others:conc}
This chapter considered a range of data and constructions in Hebrew, some familiar and some new, providing analyses based on the premise that the verbal templates are not atomic morphological elements. Instead, the trivalent theory of Voice allows us to distinguish \isi{Unspecified Voice} from {\vz}, as well as their relationship to the core vP. Thinking in terms of features on heads lets us make use of {\pz}, and the data in both Hebrew and other languages suggests a partly lexical, partly functional element {\va}.

The kinds of questions asked here were of the following type: if {\thit} were simply a morphological primitive \citep{reinhartsiloni05}, why would it be the only one to allow for reflexive verbs? And why should it have complex morphology? If {\tnif} were a morphological primitive, how can it allow for both non-active and active constructions? These facts make sense under the current decompositional view, in which functional heads build up verbs in the syntax. Certain correlations can then be explained: that {\thit} is both morphophonologically and semantically complex, for example, or that reflexives and anticausatives appear to have a shared base. The system developed here provides answers based on functional heads required elsewhere in the grammar.

In the next chapter I develop the system further, examining the ``\isi{causative}'' template {\thif} and the last value of Voice, {\vd}, in similar fashion to this chapter and the previous one.

    \chapter{\vd}
\label{chap:vd}

\section{Introduction}
When looking at the verbal system of Hebrew, it has been our goal to understand which syntactic environments a template appears in and what alternations the templatic morphology tracks. The alternations examined in this book so far have been anticausative, descriptively speaking. Taking {\tkal} or {\tpie} to be an intuitive base form, we derived anticausative versions in {\tnif} and {\thit}. Of course, I have argued that what is actually happening is quite different, conceptually: the core vP is always there, but we either add Voice or add {\vz}. The former can give us \isi{causative} verbs and the latter anticausative ones.

In this chapter we turn our attention to what can be seen as \isi{causative} alternations. The following basic observation has been guiding us throughout the book: Hebrew shows morphological marking of both \isi{causative} and anticausative forms, in addition to a ``simple'' verbal form, which itself can be either \isi{causative} or anticausative. See in particular the middle column of Table~\ref{table:vd:alternations-heb}, repeated from Chapter~\ref{chap:intro}. This three-way distinction leads to the trivalent theory of Voice defended in the monograph.
\begin{table}
	\begin{tabularx}{\textwidth}{llllll}
 \lsptoprule
	\multicolumn{2}{c}{non-active} &	\multicolumn{2}{c}{unspecified}	& \multicolumn{2}{c}{active}\\\midrule
	\multicolumn{2}{c}{\tnif}	&	\multicolumn{2}{c}{\tkal}	& \multicolumn{2}{c}{\thif}\\
	\emph{neexal}	& `was eaten' & \emph{axal}	& `ate'	&	\emph{heexil}	& `fed' \\
	\emph{nixtav}	& `was written'  & \emph{katav}	& `wrote'	&	\emph{hextiv}	& `dictated' 		\\
	\multicolumn{2}{c}{--- (idiosyncratic gap)} & \emph{nafal}	& `fell' & \emph{hepil} & `dropped' \\
\lspbottomrule
 	\end{tabularx}
	\caption{Some alternations in Hebrew.}
\label{table:vd:alternations-heb} 
	\end{table}

The formal literature on transitivity\is{\isi{transitive}} alternations has, to a large extent, focused on comparing \isi{transitive} verbs to their anticausative counterparts. It is perhaps no accident that this literature has also been based for the most part on European languages. In this chapter I examine {\thif} in depth and propose the addition of the head {\vd} to our toolbox, a functional head distinct from ordinary Voice (and as such a novel theoretical proposal). I will first describe the general properties of the template {\thif} in Section~\ref{vd:thif}.\footnote{For remarks on my notation see Chapter~\ref{sec:data:notation}.} An analysis using {\vd} follows in Section~\ref{vd:vd}. We will then look more closely at the relationship between a verb in {\tkal} and its alternants in {\tnif} and {\thif}, in Section~\ref{vd:caus}. This discussion will be followed in Section~\ref{vd:others} by a comparison with alternative approaches. Section~\ref{vd:sum} summarizes.


\section{\thif: Descriptive generalizations} \label{vd:thif}
The template {\thif} is traditionally called the ``\isi{causative}'' one: verbs instantiated in it are often \isi{causative} versions of a verb in {\tkal} (or {\tnif}; see \ref{vd:others:arad}). In practice, these verbs are active, i.e.~\isi{transitive} or unergative. I use the term ``\isi{causative}'' informally because there is no syntactic implementation of this term which is appropriate. This is, however, the traditional name, and in most \isi{transitive} uses it makes intuitive sense.

The database of \cite{ehrenfeld12} and \cite{ahdout19phd} lists between 500--600 verbs in {\thif}. Of these, more than 500 are active. They are described in Section~\ref{vd:thif:caus}. There is also a small group of anticausative verbs, which obligatorily also form zero-alternations with \isi{causative} readings in the same template. These verbs are presented in~\ref{vd:thif:inch}.

	\subsection{Causative verbs} \label{vd:thif:caus}
A few full examples are given for causatives in~(\ref{ex:4:1})--(\ref{ex:4:2}) and for an unergative in~(\ref{ex:vd:unerg}).
 \begin{exe}
 \ex  \label{ex:4:1}
 \begin{xlist} 
 	\ex   
[] 		{ \gll ha-orxim \glemph{rakd-u} ba-mesiba\\
 		  the-guests danced-\gsc{PL} in.the-party\\
 		\glt `The guests danced at the party.' } 
	
 	\ex   
[] 		{ \gll ha-zameret \glemph{herkid-a} et ha-orxim\\
 		  the-singer.\gsc{F} made.dance-\gsc{F} \gsc{ACC} the-guests\\
 		\glt `The singer made the guests dance.' } 
	
 \z
 \ex  \label{ex:4:2}
 \begin{xlist} 
 	\ex   
[] 		{ \gll ema \glemph{axl-a} uxmanjot\\
 		  Emma ate-\gsc{F} blueberries\\
 		\glt `Emma ate blueberries.' } 
	
 	\ex   
[] 		{ \gll ana \glemph{heexil-a} et ema (uxmanjot)\\
 		  Anna fed\gsc{-F} \gsc{ACC} Emma blueberries\\
 		\glt `Anna fed Emma (blueberries).' } 
	 
 \z
\ex  \label{ex:vd:unerg}  
 	{ \gll ema \glemph{hemtin-a} ad ʃe-ha-oxel haja muxan\\
 	  Emma waited-\gsc{F} until \gsc{COMP}-the-food was ready\\
 	\glt `Emma waited until the food was ready.' } 
	
 \z 

I am not sure whether there are \emph{obligatory} \isi{ditransitives} in this template (about 80 are at least non-obligatorily ditransitive\is{\isi{ditransitives}} in the database of \citealt{ahdout19phd}). Four candidates are \emph{heʃil} `lent out'~(\nextx), \emph{helva} `lent',  \emph{heskir} `rented out' and \emph{hezkir} `reminded'. Whether or not the goal argument is obligatory in contemporary speech is unclear, and in any case does not bear directly on the rest of this chapter.
 \begin{exe}
\ex   [] 		{ \gll ha-safranit \glemph{heʃil-a} (l-i) et ha-sefer\\
 		  the-librarian lent.\gsc{CAUS}-\gsc{F} to-me \gsc{ACC} the-book\\
 		\glt `The librarian lent me the book.' } 
		
 \z 
	
A few more alternations between an active verb in {\tkal} (\isi{transitive} or non-core \isi{transitive}) and a \isi{causative} in {\thif} are presented in Table~\ref{table:vd:kal-thif}.
\begin{table}
\begin{tabularx}{\textwidth}{llllll}
 \lsptoprule
	 & Root			& \multicolumn{2}{c}{Active \tkal} & \multicolumn{2}{c}{Causative \thif}\\\midrule
	a.& \root{'kl} & \emph{axal} & `ate'  & \emph{heexil (be-)} & `fed (with)'\\
	b.& \root{r'j} & \emph{raa} & `saw'  & \emph{hera (le-)} & 		 `showed (to)'\\
	c.& \root{ʃm'} & \emph{ʃama} & `heard'  & \emph{heʃmia (le-)} & `played (to)'\\
	d.& \root{nʃm} & \emph{naʃam} & `breathed'  & \emph{henʃim} & `resuscitated'\\
\lspbottomrule
 	\end{tabularx}
	\caption{Some alternations in {\thif}.}
	\label{table:vd:kal-thif} 
\end{table}

The most common way of characterizing these alternations informally is by saying that {\thif} is a \isi{causative} version of {\tkal}. Yet what we see in actuality is that verbs in {\thif} are active, regardless of whether they alternate with an unaccusative verb in {\tkal}, a \isi{transitive} verb in {\tkal}, or nothing in {\tkal}. A few examples of the \isi{causative} alternation in this template are given in Table~\ref{table:vd:alternations-heb-long} a. Many verbs are also \isi{causative} without alternating, as in~b, and others are unergative,~c.
\begin{table}
	\begin{tabularx}{\textwidth}{lllllll}
 \lsptoprule
	& \multicolumn{4}{c}{anticausative/inchoative} & \multicolumn{2}{c}{active}\\
	& \multicolumn{2}{c}{\tnif}	&	\multicolumn{2}{c}{\tkal}	& \multicolumn{2}{c}{\thif}\\\midrule
	a.& \emph{nixnas} & `entered' & && \emph{hexnis} & `inserted'\\
	 & \emph{notar} & `remained' & && \emph{hotir} & `left behind'\\
	 & \emph{nikxad} & `went extinct' & && \emph{hekxid} & `eradicated'\\
	 & \emph{ni{ts}al} & `was saved' & && \emph{he{ts}il} & `saved'\\
	 & \emph{nee{ts}av} & `was saddened' & && \emph{hee{ts}iv} & `saddened'\\
	 & \emph{nexlaʃ} & `grew weak' & && \emph{hexliʃ} & `weakened'\\\tablevspace
	 & && \emph{nafal} & `fell' & \emph{hepil} & `dropped'\\
	 & && \emph{kafa} & `froze' & \emph{hekpi} & `froze'\\
	 & && \emph{baar} & `burned' & \emph{hevir} & `lit up'\\
	 & && \emph{tava} & `drowned' & \emph{hetbia} & `drowned'\\\tablevspace
	 & && \emph{xazar} & `returned' & \emph{hexzir} & `returned'\\
	 & && \emph{jaʃav} & `sat down' & \emph{hoʃiv} & `sat down'\\
	 & && \emph{paxad} & `was afraid' & \emph{hefxid} & `scared'\\
	 & && \emph{rakad} & `danced' & \emph{herkid} & `made dance'\\
	 \tablevspace
	b.& &&&& \emph{heʃmid} & `destroyed' \\
	& &&&& \emph{heir} & `illuminated'\\
	& &&&& \emph{hevis} & `defeated'\\
	& &&&& \emph{hegdir} & `defined'\\
	& &&&& \emph{hezmin} & `invited'\\
	& &&&& \emph{heka} & `struck'\\
	& &&&& \emph{hesnif} & `sniffed'\\
	& &&&& \emph{heflil} & `incriminated'\\
	\tablevspace
	c.	& &&&&  \emph{hedrim} & `went south' \\
		& &&&&  \emph{hegzim} & `exaggerated' \\
		& &&&&  \emph{heflig} & `set sail' \\
		& &&&&  \emph{heria} & `cheered' \\
		& &&&& \emph{heezin} & `listened'\\
		& &&&& \emph{hemtin} & `waited'\\
		& &&&& \emph{heskim} & `agreed'\\
\lspbottomrule
 	\end{tabularx}
	\caption{Alternations and lack of alternations in {\thif}.}
	\label{table:vd:alternations-heb-long} 
\end{table}

The template is predominantly \emph{active}, i.e.~it has an agentive, external argument. The exact nature of what this ``causation\is{\isi{causative}}'' is will be outlined (but not decisively defined) in Section~\ref{vd:caus:mrkd}.


	\subsection{The labile alternation} \label{vd:thif:inch}
		\subsubsection{The pattern}
Hebrew does not generally have the alternation referred to as ``labile'', ``zero-derivation'' or ``conversion'' (as with English \isi{transitive} and intransitive \emph{break}$\sim$\emph{break}), with the exception of certain verbs in {\thif}. A handful of examples are attested in other templates, including \emph{a{ts}ar} `stopped' (often dispreferred to \emph{nee{ts}ar} as an inchoative), \emph{miher} `hurried' and \emph{ixer} `delayed', although the latter two are not part of my own \isi{causative} vocabulary. Over 500 of the 550--600 verbs in {\thif} are active. In this section, I explore the 33 that are non-active and undergo the labile alternation.

I will once again use the label \textbf{inchoative} as a descriptive term: an inchoative verb in {\thif} is one in which the sole argument has undergone the change of state (or changed on a scale). \textbf{Causative} is likewise a descriptive term in this section, identical in use to ``\isi{transitive}'': a structure with an external argument and an internal argument (complement to the verb). The two kinds will receive different analyses in Section~\ref{vd:vd}. The alternation is exemplified by \emph{hefʃir} `thawed' in~(\nextx). None of the inchoatives in this template have reflexive (agentive) readings.
 \begin{exe}
 \ex \label{ex:vd:thif-hefSir} 
 \begin{xlist} 
 	\ex   
[] 		{ \gll ha-jaxasim ben ʃtej ha-medinot \glemph{hefʃir-u} axarej bikur roʃ ha-memʃala\\
 		  the-relations between both the-states thawed.\gsc{CAUS}-\gsc{3PL} after visit head.of the-government\\
 		\glt `The relations between the two countries thawed after the PM's visit.' } 
		
	
 	\ex   
[] 		{ \gll bikur roʃ ha-memʃala \glemph{hefʃir} et ha-jaxasim ben ʃtej ha-medinot\\
 		  visit head.of the-government thawed.\gsc{CAUS} \gsc{ACC} the-relations between both the-states\\
 		\glt `The PM's visit thawed the relations between the two countries.' } 
		
 \z
\z 

Some examples of verbs that undergo the alternation are given in~(\nextx). Even in those cases where the inchoative is frequent, a \isi{causative} context can be set up fairly easily. Full lists are given later on in this section.
 \begin{exe}
 \ex \label{ex:vd:thif-alt}Alternating unergatives in \thif: 
 \begin{xlist} 
 	\ex  \textbf{Full alternation:} \emph{hei{ts}} `sped up', \emph{heemik} `deepened', \emph{heerix} `lengthened', \emph{hekʃiax} `stiffened', \emph{hefʃir} `thawed', \emph{heʃmin} `fattened', \emph{herza} `grew thin', \dots 

 	\ex  \textbf{Unergative preferred but causative innovation attested:} \emph{hesriax} `stank', \emph{hesmil} `went to the left',\footnote{Attested example for causative ``leften'': 
		\begin{exe}
		\ex		
		{ \gll kol ha-kavod le-barak. \glemph{hesmil} et netanjahu\\
	 		  all the-respect to-Barak. made.left \gsc{ACC} Netanyahu\\
	 		\glt `Well done to [Ehud] Barak. He made [Benjamin] Netanyahu look like a leftist.' \hfill 	\url{http:\\www.ynet.co.il/Ext/App/TalkBack/CdaViewOpenTalkBack/0,11382,L-4010352,00.html} } 	
	 	\z
	} \emph{he{ts}xin} `smelled pungent', \emph{herkiv} `rotted', \dots
	
	\ex \textbf{Unaccusative preferred but causative innovation attested:} \emph{heedim} `reddened', \emph{helbin} `whitened', \emph{heʃxir} `blackened', \emph{hevri} `got healthy',
		\emph{hexvir} `grew pale',\footnote{Attested example for causative ``palen'':
 		\begin{exe}
		\ex  ``The girl looked as though someone wrapped her up in massive metallic toilet paper. \dots \\
		{ \gll afilu ha-tseva ha-meanjen \emph{[}\dots\emph{]} \glemph{hexvir} et hofa'a-ta ʃel danst\\
 			  even the-color the-interesting {} paled \gsc{ACC} appearence-hers of Dunst\\
 			\glt `Even the interesting color \dots~made Dunst's appearance pale.' \hfill \url{http:\\www.mako.co.il/women-fashion/whats_in/Article-174f70ed642f121004.htm} } 			
 		\z 
		} \emph{her{ts}in} `became serious', \dots
 \z
 \z 

As mentioned in Chapters~\ref{vz:tnif:nact:unacc} and~\ref{vz:va:vzva:refl}, unaccusativity judgments can be fickle in Hebrew: the possessive dative\is{unaccusativity tests} has been critiqued by \cite{gafter14li} and \cite{linzen14pd} as diagnosing saliency rather than internal argumenthood, while VS (Verb-Subject order in an otherwise SVO language) is not necessarily reliable as a diagnostic of deep unaccustivity. Nevertheless, it is possible to find unaccusative verbs in \thif~which perform satisfactorily on the `by itself\is{\isi{agentivity}}' and VS diagnostics, as the examples in~(\nextx)--(\anextx) show. \citet[149]{borer91} likewise argues that inchoatives in {\thif} can be either unergative or unaccusative. Accordingly, I will assume that all three constructions (\isi{transitive}, unergative and unaccusative) are possible in this template in principle.

 \begin{exe}
 \ex  `By itself' with {\thif} inchoatives. 
 \begin{xlist} 
 	\ex   
[] 		{ \gll ha-glida \glemph{hefʃira} \glemphu{me-a{\ts}ma}\\
 		  the-ice.cream thawed.\gsc{CAUS}-\gsc{F} of-herself\\
 		\glt `The ice cream defrosted on its own.' } 
	
	
 	\ex   
[] 		{ \gll ha-tnaim le-\glemph{hafʃara} ba-jaxasim hevʃilu-u \glemphu{me-a{\ts}mam}\\
 		  the-conditions to-thawing in.the-relations ripened.\gsc{CAUS}-\gsc{3PL} of-themselves\\
 		\glt `The conditions matured enough on their own for the relations to warm.' } 
	
 \z
 \ex \label{ex:vd:vs} VS order with {\thif} inchoatives in \thif. No \emph{by}-phrase possible. 
 \begin{xlist} 
 	\ex   
[] 		{ \gll \glemph{hefʃir-a} \emph{(}l-i\emph{)} kol ha-glida \emph{(}*{al jedej} ha-xom\emph{)}\\
 		  thawed-\gsc{F} to-me all the-ice.cream \phantom{*(}by the-heat\\
 		\glt `All (my) ice cream defrosted completely (*by the heat).' } 
	
	
 	\ex   
[] 		{ \gll \glemph{hevʃil-u} ha-tnaim le-hafʃara ba-jaxasim \emph{(}*{al jedej} ha-bikur\emph{)}\\
 		  ripened-\gsc{3PL} the-conditions to-thawing in.the-relations \phantom{*(}by the-visit\\
 		\glt `The conditions matured enough for the relations to warm (*by the visit).' } 
	
 \z
\z 

But what is special about the 33 roots such as those in~(\ref{ex:vd:thif-alt}) that allows their verbs to alternate, on the one hand, and what is special about the morphological template that allows these verbs to alternate, on the other hand? A satisfying analysis of these patterns must address two questions: why these roots and why this template. A generalization about the roots is suggested next and the analysis of the template is addressed in Section~\ref{vd:vd:syn}, summarizing claims made in \cite{kastner19tlr}.

		\subsubsection{Inchoatives as degree achievements} \label{vd:thif:inch:roots}
Not many verbs take part in the labile alternation in {\thit}. A number of estimates can be found in the recent literature: \cite{arad05} counted 11 such verbs in her corpus whereas \cite{laks11} found 34. \cite{lev16} counted 81 in a survey taking into account many naturally attested, but perhaps spurious, forms. My 33 alternating verbs are broken down as follows: 15 alternating unergatives and 18 alternating unaccusatives. I have classified the alternating verbs by the alternations they participate in. Barring a judgment survey, and given that I know of no comparable lists at this level of granularity, the lists below reflect my own intuitions.

I have attempted to identify, at an informal level, which roots form verbs that participate in the labile alternation. I propose a pretheoretical classification into verb classes which is based on broad lexical semantic categories. The verbs are classified according to these categories, building towards the claim that they are all \isi{degree achievements}.

Table~\ref{tab:vd:thif-roots} lists the alternating verbs in~{\thif}. The first three rows show classes where the only inchoatives are unergative. The next row (change of color) shows a class in which the only inchoatives are unaccusative. Verbs in the other classes may be unergative or unaccusative, decided on a verb-by-verb basis. I also list whether there are \isi{transitive} verbs in this template whose lexical semantics makes them eligible to be part of the verb class.

\begin{table} \small
	\begin{tabularx}{\textwidth}{L{2cm}L{3.5cm}L{3cm}L{2.5cm}}
 \lsptoprule
		&	\textbf{Unaccusative}	&  \textbf{Unergative} & \textbf{Transitive} \\\midrule
	Emission & --- & \emph{hesriax} `stank', \emph{heviʃ} `became putrid', \emph{hetsxin} `smelled pungent'\footnotemark & --- \\
	
	Change of speed or direction & --- & \emph{heits} `accelerated', \emph{heet} `slowed down', \emph{hesmil} `went left' & \emph{heziz} `moved', \emph{hotsi} `removed', \dots \\
	
	Change of sound & --- & \emph{heriʃ} `made loud noise', \emph{hexriʃ} `quieted down' & \emph{heʃtik} `shut up' \\\tablevspace
	
	Change of color & \emph{heedim} `reddened', \emph{helbin} `whitened', \emph{hekxil} `became blue', \emph{he{ts}hiv} `yellowed', \emph{heʃxir} `blackened', \emph{hezhiv} `goldened', 
				& --- & --- \\\tablevspace
		
	Change of physical function, shape or appearance & \emph{heʃmin} `fattened', \emph{herza} `thinned', \emph{hezkin} `grew old', \emph{hekriax} `became bald', \emph{hevri} `became healthy', \emph{her{ts}in} `became serious', \emph{hexvir} `grew pale' &
		\emph{heemik} `deepened', \emph{heerix} `lengthened', \emph{he{ts}er} `narrowed', \emph{hesmik} `blushed' & \emph{hefʃit} `undressed', \emph{henmix} `lowered', \emph{hextim} `stained', \dots \\

	Change of consistency, taste or smell & \emph{hekʃiax} `stiffened', \emph{hefʃir} `thawed', \emph{hevʃil} `ripened', \emph{hekrim} `crusted'
		& \emph{hexmits} `soured', \emph{herkiv} `rotted'
		& \emph{hetsis} `fermented', \emph{heriax} `smelled', \emph{hetpil} `desalinated', \emph{heflir} `flouridated', \dots \\\tablevspace
		
	Other & \emph{hexmir} `deteriorated' & \emph{hek{ts}in} `escalated' &  \\
\lspbottomrule
 	\end{tabularx}
\caption{Lexical semantic classes for alternating verbs in {\thif} and transitive foils.\label{tab:vd:thif-roots}}
\end{table}

\footnotetext{Other verbs of emission do not entail change of state: \emph{heki} `threw up', \emph{hezia} `sweat', \emph{heflits} `farted'.}


A number of tentative generalizations can be drawn from Table~\ref{tab:vd:thif-roots}. For instance, it seems clear that change of color allows for inchoative verbs (unaccusative ones). Yet a large degree of arbitrariness exists, as when we might also have expected the forms in~(\nextx) to exist, contrary to fact. The semantic criteria alone are not enough to predict how all roots in the language will behave.
 \begin{exe}
 \ex  
 \begin{xlist} 
 	\ex  Change of speed: 
		*\emph{hemhir} ($\nless$ \emph{mahir} `quick').
 	\ex  Change of color: 
		*\emph{hesgil} ($\nless$ \emph{sagol} `purple'), *\emph{hektim}/*\emph{hextim} ($\nless$ \emph{katom} `orange').
 \z
\z 

It is also not the case that any root in the categories above necessarily derives an inchoative in {\thif}: \emph{heziz} `moved' is a change of direction, \emph{heʃtik} `shut up' is a change of sound and \emph{henmix} `lowered' is a change of physical shape, but these three verbs (and many others) are only \isi{causative}, never inchoative.

One insightful claim, made recently by \cite{lev16} and endorsed by \cite{kastner19tlr}, is that inchoatives in {\thif} are \textbf{\isi{degree achievements}} (\citealt{dowty91,hayetal99,rotsteinwinter04,kennedylevin08,bobaljik12,mcnally17}, a.m.o). These are change of state verbs such as \emph{widen} and \emph{cool} which are derived from gradable adjectives. As such, they have scalar semantics leading to a possible endpoint. \citeauthor{lev16}'s claim is that this is exactly the unifying factor for the Hebrew inchoatives in {\thif}, although it is not a bidirectional implication (not all possible \isi{degree achievements} are inchoatives in this templates), nor does this generalization drive his own analysis.

It does play a role in my own syntactic analysis insofar as inchoatives are derived from an underlying adjective (or noun). This hypothesis covers a fair bit of empirical ground and I follow \cite{lev16} in adopting it. 

	\subsection{Summary}
To summarize the empirical state of affairs, verbs in {\thif} are almost always active: either \isi{transitive} or unergative. They often form \isi{causative} versions of other verbs. And a few dozen verbs are \isi{degree achievements}, intransitive change of state verbs derived from an underlying adjective or noun.

\hammer{
 \begin{exe}
 \ex  \label{ex:gen-thif}\textbf{Generalizations about {\thif}} 
 \begin{xlist} 
 	\ex  \textbf{Configurations:} Verbs appear in transitive and unergative configurations; a small class of verbs forms unaccusative degree achievements. 
 	\ex  \textbf{Alternations:} Some verbs are causative or active versions of verbs in other templates, especially {\tkal}. A small class of verbs creates a labile alternation within {\thif}. 
 \z
\z 
}


\section{\vd: An active Voice head} \label{vd:vd}
To account for this set of data I propose {\vd}, a variant of Voice which requires that a DP be merged in its specifier, guaranteeing that an external argument appear. It introduces the \isi{Agent}/\isi{Cause} role, although unaccusatives are possible when deriving \isi{degree achievements}.\footnote{Again abstracting away from the difference between Agents and Causers, regarding which see Chapter~\ref{intro:arch}.}
 \begin{exe}
 \ex   \label{ex:vd-basics} \textbf{\vd:} 
 \begin{xlist} 
 	\ex  A Voice head with a [\!+\!D] feature, requiring that some element check the [D] feature in its specifier (usually via Merge). 
 	\ex  \label{ex:vd:sem}\denote{\vd} = $\begin{cases} 
	\text{λP.P} & / \text{\trace~(v) a} \\
	\text{λP.P} & / \text{\trace~(v) n} \\
	\text{λxλe.Agent(x,e) or λxλe.Cause(x,e)}\\
	\end{cases}$
 	\ex  {\vd} {\lra} {\thif} 
 \z
\z 

	\subsection{Syntax and semantics} \label{vd:vd:syn}

The syntax of {\vd} is as in~(\nextx), where this head obligatorily introduces an external argument. Merging that DP in Spec,{\vd} is enough to check the [D] feature, however the Spec-Head relationship is formalized. Note that I was careful to say that the feature must be checked, not that an element must be merged in the specifier; this is because of the analysis of inchoatives coming up.
 \begin{exe}
\ex \label{vd:tree:thif} 
\Tree
        [.VoiceP
            [.DP ]
            [
                [.{\vd}\\\emph{he-} ]
                [.vP
                    [.v
                        [.\root{\gsc{ROOT}} ]
                        [.v ]
                    ]
                    [.(DP) ]
                ]
            ]
        ]
     \z 

The relevant clause in the semantics is the Elsewhere case of~(\ref{ex:vd-basics}b). Since the spell-out of {\vd} is {\thif} (by hypothesis), we predict that all verbs in this template will have an external argument in the syntax and semantics.
 \begin{exe}
\ex  \denote{\vd} = λxλe.Agent(x,e) or λxλe.\text{Cause(x,e)} 
 \z 

As we have seen, this proposal is enough to describe most of the empirical landscape. It also treats causatives in {\thif} as monoclausal, ``lexical'' causatives, as expected.

But it is not enough to explain the inchoatives, where two questions in fact arise. First, how must we change our definition of {\vd}? And second, why is it this head alone that leads to labile alternations in the language?

The remainder of this section concerns itself with the first question of the two. I propose next that causatives have different structure than inchoatives, echoing claims made by \cite{borer91}. Causatives are argued to be derived from the root, whereas inchoatives are argued to be derived from an existing adjective or noun.\footnote{From a cross-Semitic perspective, \ili{Arabic} ``Form 9'' \emph{iXYaZZ} verbs show some parallels with {\thif}, though the Arabic forms are exclusively nonactive.} The more general question about labile alternations in {\thif} alone will wait until Section~\ref{vd:caus:labile}.

		\subsubsection{Inchoatives: Structure}
As a first step, I will assume that inchoatives in {\thif} are never derived directly from the root but from an underlying adjective or noun. A similar claim was already made by \cite{borer91}, who argued that causatives are derived directly from the root while these inchoatives are derived from an underlying adjective. As I point out here, inchoatives can also be derived from an underlying noun:
 \begin{exe}
 \ex  
 \begin{xlist} 
 	\ex  Underlying adjective: \emph{heedim} $<$ \emph{adom} `red', \emph{heʃmin} $<$ \emph{ʃamen} `fat'. 
 	\ex  Underlying noun: \emph{heki} $<$ \emph{ki} `vomit', \emph{he{ts}xin} $<$ \emph{{ts}axana} `stench'. 
 \z
\z 

The structure is as in~(\nextx), covering both unergatives and unaccusatives.
 \begin{exe}
\ex  \label{ex:tree:vd-inch} 
	\Tree
 [.VoiceP
     [.DP$_i$ ]
     [
         [.{\vd}\\\emph{he-} ]
         [.vP
             [.v
              [.\phantom{xx}v\phantom{xx} ]
              [.a/n
                  [.\root{\gsc{ROOT}} ]
                  [.a/n ]
              ]
             ]
             [.(DP)$_i$ ]
         ]
     ]
 ]	
 \z 

This assumption is admittedly a bit of a morphophonological stretch in certain cases.\footnote{I thank the \emph{TLR} reviewers of \cite{kastner19tlr} for emphasizing this point. I have not made progress on this issue since the publication of that paper.} For example, the verb \emph{hei{ts}} `accelerated' is arguably not derived from the noun \emph{teu{ts}a} `acceleration', whose initial /t/ is not preserved. This much indicates that perhaps the claim should be weakened such that some inchoatives are derived from adjectives/nouns and others from the root. Nevertheless, the strong assumption of crosscategorial derivation carries a few benefits. First, it allows us to talk about different constructions in terms of explicit, uniform structures. Second, it allows for the degree semantics of the underlying adjective to transfer to the verb. And third, it makes a correct prediction regarding idiomatic meaning, as I show next.

My theory of morphosemantics assumes the so-called Arad/Marantz hypothesis, according to which the first categorizing head selects the meaning of the root (see Chapter~\ref{vz:vz:sem}). If~(\lastx) is the right structure for inchoatives, then we predict that for roots which participate in the alternation, the \isi{causative} might have a meaning that the inchoative does not share. This is because in causatives {\vd} is local enough to the root to select a special meaning, whereas in inchoatives little a or little n will have already chosen the meaning of the root. This prediction is borne out by idioms involving \emph{helbin} `whitened' with the metaphorical meaning `laundered', as in~(\nextx), and \emph{heʃxir} `blackened' with the metaphorical meaning `tarnished', as in~(\anextx).
 \begin{exe}
 \ex  
 \begin{xlist} 
 	\ex  Causative, literal meaning: \\
	{ \gll ha-sid \glemph{helbin} et ha-kir.\\
 			  the-lime.plaster whitened.\gsc{CAUS} \gsc{ACC} the-wall\\
 			\glt `The lime plaster made the wall white.' } 
		
 	\ex  Causative, non-transparent meaning: \\
	{ \gll sar ha-xuts \glemph{helbin} ksafim.\\
 			  minister the-exterior whitened\gsc{CAUS} moneys\\
 			\glt `The Minister of Foreign Affairs took part in money laundering.' } 
		
 	\ex  Passive of causative, non-transparent meaning retained: 	\\	
	{ \gll nitan ʃe-ha-ksafim \glemph{hulben-u} {al jedej} sar ha-xuts.\\
 			  was.claimed \gsc{COMP}-the-moneys whitened.\gsc{CAUS.PASS}-\gsc{3PL} by minister the-exterior\\
 			\glt `It was claimed that the money was laundered by the Minister of Foreign Affairs.' } 
		
 	\ex  Inchoative, only literal meaning: \\
	{ \gll ha-ʃtarot \glemph{helbin-u}.\\
 			  the-bills whitened.\gsc{CAUS}-\gsc{3PL}\\
			\glt `The bills became white.'\\
				(not: `The bills got laundered.')
	}		
 \z
 \ex  
 \begin{xlist} 
 	\ex[]  { Causative, literal meaning: \\	
	 \gll ha-piax \glemph{heʃxir} et ha-avir.\\
 			  the-soot blackened.\gsc{CAUS} \gsc{ACC} the-air\\
 			\glt `The air grew black with soot.' } 
		
	
 	\ex[]  { Causative, non-transparent meaning: \\	
	 \gll son'e-j israel menas-im \glemph{lehaʃxir} et pane-ha ʃel medina-t israel ba-zira ha-benleumit.\\
 			  haters-\gsc{CS} Israel try.\gsc{PTCP}-\gsc{M.PL} to.blacken.\gsc{CAUS} \gsc{ACC} faces-\gsc{3F} of state-\gsc{CS} Israel in.the-arena the-international\\
 			\glt `Israel's haters are trying to make the State of Israel look bad on the international stage.' \hfill \url{http:\\www.ynet.co.il/articles/0,7340,L-4781034,00.html} } 
		
 	\ex[??]  { Inchoative, only literal meaning: \\
		 \gll pane-ha ʃel ha-medina \glemph{heʃxir-u} axarej ha-ʃaarurija ha-axrona\\
 			  faces-\gsc{3F} of the-state blackened.\gsc{CAUS}-\gsc{3PL} after the-scandal the-last\\
 			\glt (int. `The country was made to look bad after the latest scandal') } 
		
 \z
\z 

\cite{borer91} provides additional arguments for deriving the inchoative from the adjective, which I scrutinize in Section~\ref{vd:others:borer}.

The full semantics for {\vd} then looks as in~(\nextx), without introducing a causer for inchoative events in~(\nextx a--b).
 \begin{exe}
 \ex \label{ex:vd:sem-full} \denote{\vd} =
 \begin{xlist} 
 	\ex  λP.P / \trace~(v) a \hfill (v does not select the meaning) 
 	\ex  λP.P / \trace~(v) n \hfill (v does not select the meaning) 
 	\ex  λxλe.Cause(x,e) or λxλe.Agent(x,e) 
 \z
\z 

This formulation still suffers from a few potential problems, which I address in Section~\ref{vd:caus:pred}.

		\subsubsection{Inchoatives: Derivation}
Merging a DP in Spec,{\vd} will not do for the inchoatives since they are unaccusative. Allowing the internal argument to raise to the specifier and check the [D] feature there must also be ruled out because of the results of the VS diagnostic: it shows us that at least in some cases the internal argument must be allowed to remain low, (\ref{ex:vd:vs}).

To account for these cases, I assume instead that the [D] feature on {\vd} requires valuation of phi-features under \isi{Agree} \citep{nie17,schaefer17oup}. This valuation proceeds straightforwardly under Spec-Head Agreement, as we have seen, but something else needs to be said if the sole argument in the phase is the internal argument. In this case, I propose that [D] can be checked by the internal argument \emph{in situ}: {\vd} probes into its specifier upwards, finds no target, and so it probes downwards and is valued\is{Agree} by the internal argument. For more in-depth discussion of the direction of \isi{Agree}, see works such as \cite{bejarrezac09}, \cite{zeijlstra12}, \cite{preminger13tlr} and \cite{deal15nels}.

Here is what the current proposal means for an inchoative example like~(\ref{ex:vd:thif-inch}) with the structure in~(\ref{tree:vd:thif-inch}). {\vd} probes its specifier and finds nothing, (\ref{tree:vd:thif-inch}-\ding{172}), so it probes downward and checks its unvalued phi-features with the internal argument \emph{ha-xatul} `the cat' (\ref{tree:vd:thif-inch}-\ding{173}). The interpretation is as in~(\ref{ex:vd:sem-full}a): no \isi{Cause} is introduced.
 \begin{exe}
\ex \label{ex:vd:thif-inch}  
	{ \gll ha-xatul \glemph{heʃmin}.\\
 	  the-cat fattened.\gsc{CAUS}\\
 	\glt `The cat grew fat.' } 
	
 \z 
	
 	\begin{exe}
\ex \label{tree:vd:thif-inch} 
	    \Tree
	    [.VoiceP
	        [.\tikz{\node (spec) {};} ]
	        [
	            [.\tikz{\node (vd) {\vd};} ]
	            [.vP
	                [.v
		                [.\phantom{xx}v\phantom{xx} ]
		                [.a
		                    [.a ]
		                    [.\root{ʃmn} ]
		                ]
	                ]
	                [.\tikz{\node (IA) {\emph{ha-xatul}};} ]
	            ]
	        ]
	    ]
	    \begin{tikzpicture}[overlay]
		    \draw[dashed,->] (vd) .. controls +(south:1) and +(south west:1) .. node{\LARGE $\times$} node[below]{\ding{172}}(spec);
	 	    \draw[dashed,->] (vd) .. controls +(south:4) and +(south:3) .. node[below]{\ding{173}}(IA);    
	    \end{tikzpicture}
 	\z 
\bigskip

As a consequence, ungrammatical cases like~(\nextx) must now be ruled out.
 \begin{exe}
 \ex \label{ex:counterex} 
 \begin{xlist} 
 	\ex   
[*] 		{ \gll ha-xatul \glemph{hexnis}\\
 		  the-cat inserted.\gsc{CAUS}\\
 		\glt (int. `The cat got inserted') } 
	
 	\ex   
[*] 		{ \gll ha-oto \glemph{hemhir}\\
 		  the-car \gsc{FAST}.\gsc{CAUS}\\
 		\glt (int. `The car grew fast') } 
	
 	\ex   
[*] 		{ \gll ha-xatul \glemph{hekpi}\\
 		  the-cat froze.\gsc{CAUS}\\
 		\glt (int. `The cat froze') } 
	
 \z
\z 

For~(\lastx a) there is no adjective `inserted' that could be verbalized and no inchoative can be generated. In~(\lastx b) an adjective \emph{mahir} `quick' does exist, but it cannot be instantiated in {\thif} in general due to some arbitrary gap, as already mentioned in Section~\ref{vd:thif:inch:roots} (or at least, I assume that this is an arbitrary gap, in lieu of a more principled explanation).

Finally, (\ref{ex:counterex}c) is not a possible inchoative even though there exists an underlying adjective, namely \emph{kafu} `frozen'. There are a number of possible explanations which can be pursued here. One is that \emph{freeze} is not a degree achievement\is{\isi{degree achievements}} in Hebrew, and so that adjective is not a possible input to the structure. Another kind of explanation falls along the lines of extra-grammatical paradigmatic pressure, in that an inchoative (non-alternating) freezing verb already exists in another template: \emph{kafa} `froze' in {\tkal}. In this regard, I should note that speakers do steer clear of {\thif} for certain inchoatives, instantiating them in other, more canonically non-active templates: \emph{hitarex} `grew long' in {\thit} rather than \emph{heerix}, \emph{hizdaken} `grew old' in {\thit} rather than \emph{hezkin}, \emph{raza} `thinned' in {\tkal} instead of \emph{herza}, and \emph{hitadem} `reddened' in {\thit} rather than \emph{heedim} (but see \citealt[22]{doron03} for a grammatical difference between the two).

It should go without saying that the strong claim about separate derivational strategies for causatives and inchoatives awaits a more articulated semantic analysis. As a reviewer for \cite{kastner19tlr} pointed out, in~(\ref{ex:vd:thif-hefSir}) the verb \emph{hefʃir} means `thawed', i.e.~became warmer, while the underlying adjective \emph{poʃer} means `lukewarm', i.e.~not warm. Yet the inchoative does not mean ``became lukewarm''. Another incongruity between verb and adjective can be seen with \emph{heʃmin}, `grew fatter', which does not entail that its argument becomes \emph{ʃamen} `fat'. Important discussion of the relevant scales and entailments is given by~\cite{borer91}, which I turn to in the discussion of alternatives.

With the formal analysis in place, I flesh out the morphophonlogical part of the chapter before turning to more general discussion, including the question of why the labile alternation is formed with {\vd} specifically.

	
	\subsection{Phonology} \label{vd:vd:phono}
The basic VI given in~(\ref{ex:vd-basics}) was as follows:
 \begin{exe}
\ex  {\vd} {\lra} {\thif} 
 \z 

Using {\thif} as the Vocabulary Item spelling out {\vd} is shorthand for a more detailed morphophonology. A sample derivation is adapted here from~\cite{kastner18nllt}. I contrast the \gsc{3SG.M.Past} form with the \gsc{1SG.Past} form, which has an affix and a different stem vowel.
 \begin{exe}
 \ex  
 \begin{xlist} 
 	\ex  \emph{hevʃ\glemph{i}l} `he ripened' 
 	\ex  \emph{hevʃ\glemph{a}l-\textbf{ti}} `I ripened' 
 \z
\z 
The relevant Vocabulary Items:
 \begin{exe}
 \ex  
 \begin{xlist} 
 	\ex  \root{bʃl} \lra~\emph{bʃl} 
 	\ex  	\vd~\lra~ $\begin{cases} 
			\text{\emph{he,a}} & \text{/ T[1st] \trace}\\
			\text{\emph{he,i}}
			\end{cases}$\label{r1:4:4}
 	\ex  1\gsc{SG} \lra~\emph{ti} / \trace~Past 
 	\ex  \gsc{3SG} \lra~{\zero} / \trace~Past 
 \z
\z 
The cyclic derivation:
 \begin{exe}
 \ex  
 \begin{xlist} 
 	\ex  \emph{hevʃil} `he ripened': [T[Past,\gsc{3SG.M}] [{\vd} [v \root{bʃl}~\!]]] \\
	Cycle 1 (VoiceP): he-vʃil\\
	Cycle 2 (TP): {\zero}-hevʃil $\Rightarrow$ hevʃil
 	\ex  \emph{hevʃ\glemph{a}l\textbf{ti}} `I ripened': [T[Past,\gsc{1SG}] [{\vd} [v \root{bʃl}~\!]]]  \\
	Cycle 1 (VoiceP): he-vʃ\textbf{a}l\\
	Cycle 2 (TP): \textbf{ti}-hevʃal $\Rightarrow$ hevʃalti
 \z
\z 	

See the work cited for various additional cyclic and allomorphic predictions.


\section{Causation and alternation} \label{vd:caus}
This section contains a number of general points about \isi{causative} alternations which I would like to mention. Section~\ref{vd:caus:mrkd} discusses \isi{markedness} in causation\is{\isi{causative}} in general and in terms of Voice heads in particular. Section~\ref{vd:caus:product} notes how productive\is{productivity} {\vd} is, and Section~\ref{vd:caus:labile} returns to the labile alternation. A possible way of generalizing this account is surveyed in Section~\ref{vd:caus:pred}. For recent ways of conceptualizing causation\is{\isi{causative}} in Hebrew in particular, see \cite{barashersiegalboneh18wccfl}.

	\subsection{Markedness in causation} \label{vd:caus:mrkd}
Recall the basics of the anticausative alternation which we have been assuming. Both (marked\is{\isi{markedness}}) anticausatives and (unmarked\is{\isi{markedness}}) causatives share a common base, formally the vP. This phrase is a predicate over eventualities, to which Voice can add an external argument \citep{schaefer08,layering15}, (\nextx).
 \begin{exe}
 \ex   
 \begin{xlist} 
 	\ex  \emph{Mary} \textbf{Voice} [$_{\text{vP}}$ \emph{broke the glass}]. 
 	\ex  {\zero} \textbf{\vz} [$_{\text{vP}}$ \emph{The glass broke}]. 
 \z
\z 

Considering the Trivalent proposal, how and why should {\vd} differ from \isi{Unspecified Voice}? If Voice allows the grammar to add an external argument, what's left for an additional device (\vd) to do, the hypothetical (\nextx)?
 \begin{exe}
\ex  \emph{Mary} \textbf{\vd} [$_{\text{vP}}$ \emph{broke the glass}]. 
 \z 

Since the syntactic behavior of \isi{Unspecified Voice} and {\vd} is identical as far as \isi{licensing} a specifier is concerned, in this section I will discuss the semantic difference between the resulting \isi{causative} verbs. Concretely, I will suggest that Voice-causatives are more transparent than {\vd}-causatives, whereby the morphological \isi{markedness} of the latter mirrors some semantic \isi{markedness} or opacity. The {\vd} causatives are simply lexical causatives, in the sense of \cite{fodor70}, \cite{miyagawa98} and \cite{harley08}: a \isi{transitive} verb which is not derived through causativization of an existing verb. Let us explore what this means when contrasting them with ``regular'' \isi{causative} or \isi{transitive} verbs. For Hebrew, this means contrasting causatives in {\tkal} with those in {\thif}, or rather causatives derived using \isi{Unspecified Voice} with causatives derived using {\vd}.

		\subsubsection{Basic and marked alternations}
For concreteness, let us give the two alternations the names in Table~\ref{tab:4-4:alt}. The claim will be that the marked\is{\isi{markedness}} alternation is marked\is{\isi{markedness}} not only morphologically but also semantically -- a lexical \isi{causative}, i.e.~a non-transparent one.
 \begin{table}
\begin{tabularx}{\textwidth}{lccc}
 \lsptoprule
	&	Anticausative & Causative & Causative\\\midrule
Basic alternation	& {\vz} & Voice & ---  \\
Marked alternation		&	---	&  Voice & {\vd}\\
\lspbottomrule
 \end{tabularx}
 	\caption{Two basic alternation types.}
	\label{tab:4-4:alt}
\end{table}

Very little contemporary work has analyzed \isi{causative} alternations in depth within a general theory of argument structure alternations; such work normally draws on languages like Japanese \citep{jacobsen92} that are typologically distinct from Indo-European ones. \citet[62ff]{layering15} speculate that a marked\is{\isi{markedness}} \isi{causative} should entail thematic/active Voice (semantically if not syntactically), but as far as I know no formal theory has explored the implications of marked\is{\isi{markedness}} anticausatives and marked\is{\isi{markedness}} causatives existing side by side.

The first question to ask is how prevalent these two alternations are. The basic alternation was discussed at length in Chapter~\ref{vz:vz}. For the marked\is{\isi{markedness}} alternation, various examples were already given in Tables~\ref{table:vd:kal-thif} and~\ref{table:vd:alternations-heb-long}b. Out of 300+ pairs of {\tkal}--{\thif} alternations in my database, 64 show the marked\is{\isi{markedness}} alternation.

The second question is whether there is a difference between the semantics of the alternations, and here I believe the basic alternation is more transparent. The question is one of predictability: given the anticausative variant, can we predict the meaning of the \isi{causative} variant? In the basic alternation, the answer is usually affirmative, just like with the prototypical \emph{break}-\emph{break} and \emph{open}-\emph{open} examples in English. A few examples were given  above and one is elaborated on in~(\nextx):
 \begin{exe}
 \ex  
 \begin{xlist} 
 	\ex  \emph{nixtav} `was written' \\
		Writing event of a DP, no cause specified; or a passive reading with an implicit Agent.
 	\ex  \emph{katav} `wrote' \\
		Writing event of a DP, external argument specified in the syntax and interpreted as Agent.
 \z
\z 

I suggest that the \isi{causative} variant of a basic alternation introduces a direct causer \citep{bittner99,kratzer05} but that the external argument in the marked\is{\isi{markedness}} alternation is less restricted.\footnote{A similar intuition was expressed by \cite{doron03}, where the strongest claims about a template's meaning were limited to cases in which a root alternates between two templates.} For the marked\is{\isi{markedness}} \isi{causative}, the informal phrasing in~(\nextx) will do for now (I return below to the question of whether a writing event even holds in these cases):
 \begin{exe}
\ex  \emph{hextiv} `dictated' \\
	Writing event of a DP, external argument specified in the syntax and understood as an (indirect) causer.
 \z 

In terms of syntax, there appears to be no difference between the constructions. The unmarked\is{\isi{markedness}} \isi{causative} has a regular monoeventive reading, cannot be modified by conflicting temporal adverbs, and has two basic readings of `again' \citep{vonstechow96}.

 \begin{exe}
 \ex    
[] 		{ \gll ha-talmidim \glemph{katvu} et ha-nosim (\#aval lo be-a{\ts}mam).\\
 		  the-students wrote \gsc{ACC} the-topics but not in-themselves\\
 		\glt `The students wrote down the list of topics (\# but not by themselves).' } 

		\begin{xlist}
	 	\ex  Means: The students wrote the list themselves. 
 		\ex  Cannot mean: paid someone online to write the list for them. 
 		\z

\ex    
[*] 		{ \gll ha-talmidim \glemph{katvu} etmol et ha-nosim maxar\\
 		  the-students wrote yesterday \gsc{ACC} the-topics tomorrow\\
 		\glt (int.~`The students wrote something down yesterday to get the list of topics tomorrow') } 
		
 \ex   
[] 		{ \gll ha-talmidim \glemph{katvu} et ha-nosim ʃuv.\\
 		  the-students wrote \gsc{ACC} the-topics again\\
 		\glt `The students wrote down the list of topics again.' } 
	\begin{xlist}
 	\ex  Can mean: The students wrote down a list which already existed. 
 	\ex  Can mean: The students wrote down a list after having written it down once before. 
 	\ex  Cannot mean: Someone$_{i}$ had written down the list of topics, and now the students made someone$_{i/j}$ write the list of topics another time. 
 	\z
\z 

The marked\is{\isi{markedness}} \isi{causative} patterns identically: monoeventive reading, no conflicting temporal adverbs, and the same readings of `again'.
 \begin{exe}
 \ex  
[] 		{ \gll ha-more \glemph{hextiv} et ha-nosim (la-talmidim) (\#bli lomar mila).\\
 		  the-teacher dictated \gsc{ACC} the-topics to.the-students without to.say word\\
 		\glt `The teacher dictated the list of topics (to the students) (\# without saying a word). } 
	 \begin{xlist} 
	
 	\ex  Means: The teacher read the list out and the students wrote it down. 
 	\ex  Cannot mean: He stood menacingly over the students until they started writing. 
	 \z

\ex    [*] 		{ \gll ha-more \glemph{hextiv} etmol et ha-nosim (la-talmidim) maxar\\
 		  the-teacher dictated yesteday \gsc{ACC} the-topics to.the-students tomorrow\\
 		\glt (int.~`The teacher read out the list yesterday for the students to write down tomorrow') } 
	
 \ex  	
[] 		{ \gll ha-more \glemph{hextiv} et ha-nosim (la-talmidim) ʃuv.\\
 		  the-teacher dictated \gsc{ACC} the-topics to.the-students again\\
 		\glt `The teacher dictated the list of topics (to the students) again. } 
	 \begin{xlist} 
 	\ex  Can mean: The teach dictated/wrote a list which (someone else) has already dictated/written. 
 	\ex  Can mean: The teacher dictated/wrote a list having dictated/written it once before. 
 	\ex  Cannot mean: Someone$_{i}$ had dictated/written the list, and now the teacher made someone$_{i/j}$ write/dictate it another time. 
	 \z
\z 

The structures are therefore virtually identical, with the only difference being the feature on Voice (other than the identity of the external argument).

 \begin{exe}
 \ex   
 \begin{xlist} 
 	\ex   
[] 		{ \gll ha-talmidim \glemph{katvu} et ha-nosim.\\
 		  the-students wrote \gsc{ACC} topics\\
 		\glt `The students wrote down the list of topics.' } 
		
		\Tree [. [.students ] [. [.Voice ] [. [.\root{\gsc{WROTE}} ] [.topics ] ] ] ]		

 	\ex   
[] 		{ \gll ha-more \glemph{hextiv} et ha-nosim (la-talmidim).\\
 		  the-teacher dictated \gsc{ACC} topics to.the-students\\
 		\glt `The teacher dictated the list of topics (to the students). } 

		\Tree [. [.teacher ] [. [.{\vd} ] [. [.\root{\gsc{WROTE}} ] [.topics ] ] ] ]
 \z
\z 

In the basic alternation, adding a causer to writing immediately identifies the writer. But adding the marked\is{\isi{markedness}} causer changes the event slightly: the teacher does not cause writing to occur, strictly speaking. Rather, he is the causer of a dictating event, which itself brings about the writing down of the topics.

A similar pattern can be seen with \emph{neexal} `was eaten', where the basic variant \emph{axal} means `ate' and the marked\is{\isi{markedness}} variant \emph{heexil (et)} means `fed (s.o.~s.th)'. Why should this be the meaning, and not `made someone eat'? The different kind of event (feeding versus causing to eat) also implies a different position for the eater: subject in the basic variant, object in the marked\is{\isi{markedness}} variant.
 \begin{exe}
 \ex  
 \begin{xlist} 
 	\ex   
[] 		{ \gll beki \glemph{axl}-a uxmanjot.\\
 		  Becky ate-\gsc{F} blueberries\\
 		\glt `Becky ate blueberries.' } 

		\Tree [. [.\textbf{Becky} ] [. [.Voice ] [. [.\root{\gsc{ATE}} ] [.blueberries ] ] ] ]

 	\ex   
[] 		{ \gll ima \glemph{heexil}-a et beki (uxmanjot).\\
 		  mom fed\gsc{-F} \gsc{ACC} Becky blueberries\\
 		\glt `Mom fed Becky (blueberries).' } 

		\Tree [. [.mom ] [. [.{\vd} ] [. [.\root{\gsc{ATE}} ] [.\textbf{Becky} ] ] ] ]

 \z
\z 

This contrast illustrates the limits of syntax within the current system: it can rigidly dictate which elements go where, but the structure itself is not driven by semantic or lexical-semantic considerations.

A few more examples of the marked\is{\isi{markedness}} alternation are given in Table~\ref{table:vd:triplets-caus}, showing that the exact type of ``\isi{causative}'' relation for the marked\is{\isi{markedness}} variant is not uniform.\footnote{The lexical semantics of the root could have something to do with the type of causation\is{\isi{causative}} in the marked\is{\isi{markedness}} alternation, a question I leave open. See \citet[44]{doron03} for a proposed explanation in terms of whether the {\tkal} form is a verb of consumption or a psych-verb, building on \cite{colesridhar77}.} We have already seen the different meanings of rows~a and~c.
\begin{sidewaystable}
	\begin{tabularx}{\textwidth}{lllllllcc}
 \lsptoprule
		\multicolumn{7}{c}{}		& Make O V	& Make O be V-ed\\\midrule
		 a.& \emph{neexal}	& `was eaten'	& \emph{axal} & `ate'		& \emph{heexil} & `fed'			& \cmark	& \xmark\\
		 b.& \emph{nilbaʃ}	& `was worn'	& \emph{lavaʃ} & `wore' 	& \emph{helbiʃ}	&	`dressed up' 	& \cmark	& \xmark\\\tablevspace
		 c.& \emph{nixtav} & `was written' & \emph{katav} & `wrote' & \emph{hextiv} & `dictated' & \xmark	& \cmark\\
		d.& \emph{nikra}	& `was read'	& \emph{kara} & `read'		& \emph{hekri}	& `read out'	& \xmark	& \cmark \\
		e.&	\emph{nidxak}	& `was pushed aside'	& \emph{daxak}	& `shoved'	& \emph{hedxik}	& `suppressed'\footnotemark	& \xmark	& \cmark\\
		f.& \emph{nilxats}	& `was pressed' &  \emph{laxats} & `pressed'	& \emph{helxits} & `pressured'	& \xmark	& \cmark \\\tablevspace
		 g.& \emph{nixtam}	& `was signed'	& \emph{xatam} & `signed'	& \emph{hextim}	& `made sign'	& \cmark	& \cmark\\\tablevspace
		h. & \emph{nidgam} & `was sampled'	& \emph{dagam} & `sampled'	& \emph{hedgim}		& `demonstrated'	& \xmark	& \xmark\\
		i. & \emph{niklat} & `was received' & \emph{kalat} & `received' & \emph{heklit} & `recorded' & \xmark & \cmark?\\
		j.& \emph{nisgar}	& `closed'	& \emph{sagar} & `closed'		& \emph{hesgir} & `extradited'	& \xmark	& \cmark?\\
\lspbottomrule
 		\end{tabularx}
	\caption{Alternation types in {\thif}.}
	\label{table:vd:triplets-caus} 
\end{sidewaystable}

The last examples, in rows~h--j, are particularly revealing: to extradite someone is in no way an obvious semantic extension of ``closing'' them. More examples like this can be found, in which the basic alternant has predictable semantics but the marked\is{\isi{markedness}} one does not.

Before we go on to discuss the consequences of these differences, it is important to note a possible objection. The fact that marked\is{\isi{markedness}} causatives can vary so widely in their interpretation from the basic variants could be taken as an argument against treating these verbs as sharing the same abstract root. That is to say, why should we even think that closing and extraditing share the same root? Would that not be stretching its assumed shared semantics too thin? I believe the overall picture emerging from this book and from work treating roots more directly is that we do want to assume abstract roots, but be more specific in what their shared meaning is and under which circumstances it can vary. See also \cite{kastnertucker19cup} for related discussion of root meaning.

More concretely, however, we can make an argument from the lack of doublets. There are no \emph{additional} verbs in {\tnif} that alternate with {\thif}. That is to say, suppose that \emph{nikra} `was read' and \emph{kara} `read' are derived from \root{krj1} and that \emph{hekri} `read out' was derived from a homophonous root \root{krj2}. Assume similarly that \emph{nisgar} `closed' and \emph{sagar} `closed' were derived from \root{sgr1} but that \emph{hesgir} `extradited' was derived from \root{sgr2}, and so on for all cases of non-predictable \isi{causative} variants. If this were the case, we would suppose that \root{krj2} and \root{sgr2} could also be instantiated with other functional heads, for instance with {\vz} to create an anticausative in {\tnif}. But this is not the case: there is no \#\emph{nikra} `was read out' to alternate with \emph{hekri} `read out', and no \#\emph{nisgar} `got extradited' to alternate with \emph{hesgir} `extradited'. In other words, there are no \emph{doublets} (compare the discussion of root suppletion in \citealt{harley14thlia,harley14thlib,harley15roots} and \citealt{borer14thli}).

I conclude tentatively that {\vd} invokes some indirect notion of causation\is{\isi{causative}}, in those cases where regular (direct) causation\is{\isi{causative}} is already triggered by Voice. But why might this be the case?

\footnotetext{E.g.~memories or emotions.}

			\subsubsection{Markedness in Voice heads} \label{vd:caus:markvoice}
The observations made so far bring us to the proposed generalization for causativity marking in~(\nextx).
 \begin{exe}
 \ex \label{ex:vd:causgen}\textbf{The causative generalization for transitivity alternations} \\
 	If a language has both anticausative and causative marking:
 \begin{xlist} 
 	\ex  The anticausative alternation is transparent. 
 	\ex  The causative alternation is not (indirect, root-specific). 
 \z
\z 

I would not be surprised if closer examination of other languages reveals similar patterns. In \ili{French}, for instance, the prefixes \emph{a-} and \emph{en-} are often described as having a general ``transitivizing'' or ``\isi{causative}'' function \citep{junker87}. In some cases, especially denominal and de-adjectival ones, the \isi{causative} alternation between an unprefixed anticausative verb and a prefixed \isi{causative} verb is transparent (\nextx a). But this is not always the case: in (\nextx b) the unprefixed verb is an activity and the prefixed verb is \isi{transitive} (it has the obligatory reflexive marker) but does not strictly speaking mean `make yourself fly' or `fly yourself'. Even more strikingly -- and certainly reminiscent of the Hebrew datapoints -- is~(\nextx c), where the prefixed version has different meaning than the unprefixed one.
 \begin{exe}
\ex  \langinfo{French}{}{\citealt{junker87}}
	\begin{xlist}
	\ex \emph{faiblir}	`grow weak' $\sim$ 	\emph{\textbf{a}ffaiblir} `weaken s.o'
	\ex \emph{voler} `fly' $\sim$ \emph{s'\textbf{en}voler} `take off'
	\ex \emph{fermer}	`close' $\sim$ \emph{\textbf{en}fermer} `imprison, lock up'
\z
 \z 
It is tempting to analyze these prefixes as Voice heads, perhaps even {\vd}, but that idea goes beyond the scope of the current work. What we see is that there is no transparent relationship of ``causation\is{\isi{causative}}'', however defined, between a marked\is{\isi{markedness}} form and an unmarked\is{\isi{markedness}} one.

The discussion of \isi{causative} marking will conclude by examining whether~(\ref{ex:vd:causgen}) can be derived directly from our theoretical assumptions. I believe that it can. Concretely, this generalization follows directly from the general layering approach to transitivity\is{\isi{transitive}} alternations. If the core vP already has a \isi{causative} component, then it is clear what \emph{not} adding an external argument would mean: that is the anticausative alternant. Adding an external argument, as noted above, amounts to introducing the most direct causer. This much derives~(\ref{ex:vd:causgen}a).

Say we have an event of causation\is{\isi{causative}}, (\nextx):
 \begin{exe}
\ex  
\Tree
	[.vP
		[.v ]
		[.DP ]
	]
 \z 


Not adding a causer is easy, (\nextx):
 \begin{exe}
\ex  
\Tree
[.VoiceP
	[.{\vz} ]
	[.vP
		[.v ]
		[.DP ]
	]
]
 \z 

For~(\ref{ex:vd:causgen}b) We will need to assume that structures derived with different Voice heads will have different meanings, perhaps by some principle of economy.
Then, various kinds of external arguments can go with different causation\is{\isi{causative}} events:
 \begin{exe}
\ex  a. 
\Tree
[.VoiceP
	[.DP_1 ]
	[.
		[.Voice ]
		[.vP
			[.v ]
			[.DP ]
		]
	]
]
b.
\Tree
[.VoiceP
	[.DP_2 ]
	[.
		[.{\vd} ]
		[.vP
			[.v ]
			[.DP ]
		]
	]
]
 \z 

This result makes sense if {\vd} is a marked\is{\isi{markedness}} head which only appears in the inventory of a language once this language already has \isi{Unspecified Voice} (see also Chapter~\ref{chap:aas}). In other words, the two heads stand in an implicational relationship and we do not expect to find a language with {\vd} (and {\vz}) but without Voice.

On the other hand, it is not possible to have various kinds of \emph{lack} of external arguments. This point brings us to a novel prediction, namely that a specific kind of argument structure triplet should be highly rare, if not impossible.
 \begin{exe}
 \ex  \label{ex:vd:causpred}\textbf{The triplet prediction for argument structure alternations} \\
 	If a language has both anticausative and causative marking:
 \begin{xlist} 
 	\ex  Triplets of the form 
		{[}marked \underline{unaccusative} $\sim$ unmarked \textbf{causative} $\sim$ marked causative] \\
		may be possible.
 	\ex  Triplets of the form 
	 	{[}marked \underline{unaccusative} $\sim$ unmarked \textbf{\underline{unaccusative}} $\sim$ marked causative] \\
	 	will not be possible.
 \z
\z 	

We have already seen examples of triplets such as those predicted by~(\lastx a) to exist in Table~\ref{table:vd:triplets-caus}. Those like~(\lastx b) are much more difficult to find. The two triplets in Table~\ref{tab:4-4:trip} could be argued to exist in Hebrew.
\begin{sidewaystable}
	\begin{tabularx}{\textwidth}{lllllll}
 \lsptoprule
	Root	& {\tnif} & & {\tkal} & & {\thif} & \\\midrule
	\root{xrv}  & \emph{nexrav} & `turned to ruins' & ??\emph{xarav} & `turned to ruins' & \emph{hexriv} & `demolished, turned to ruins'\\
	\root{ʃlm} & \emph{niʃlam} & `reached conclusion'	& ??\emph{ʃalam} & `became whole' & \emph{heʃlim} & `made up with someone'\\
\lspbottomrule
 	\end{tabularx}
	\caption{Potential alternation triplets in Hebrew.}
	\label{tab:4-4:trip}
\end{sidewaystable}

In both cases the {\tkal} form is archaic and exists in contemporary speech only in a few set idioms, if at all. Speakers seem to prefer the {\tnif} form for the anticausative, in accordance with~(\lastx).

Outside of Hebrew, \ili{Acehnese} has been reported to have an unmarked\is{\isi{markedness}} anticausative and an additional marked\is{\isi{markedness}} anticausative \citep{ko09afla}. Interestingly, this latter marked\is{\isi{markedness}} anticausative looks like it is derived from the marked\is{\isi{markedness}} \isi{causative}. I therefore submit the generalization in~(\ref{ex:vd:causgen}) and the prediction in~(\ref{ex:vd:causpred}) as claims to be tested in more careful crosslinguistic work.

	\subsection{Productivity} \label{vd:caus:product}
Another point about the semantic flexibility of {\vd} concerns its \isi{productivity}. The template {\thif} is a productive\is{productivity} \isi{causative} template in which speakers may innovate new forms on the fly \citep{lev16}. The verb \emph{taka} `stuck', for instance, is an ordinary \isi{transitive} verb in {\tkal}, but the online comment in~(\nextx) innovates \emph{hetkia} in {\thif} (presumably for literary or comic effect). The article concerns a roller coaster which became stuck mid-ride on a Saturday, the prescribed day of rest, stranding those riding it for the better part of an hour.
 \begin{exe}
\ex  `Why don't you understand that the roller coaster also wanted to observe the Sabbath and rested 40 minutes {\dots}'  \\
	{ \gll {...} elokim hevi la-xem siman ʃe-lo taalu al ha-mitkan be-ʃabat ve-hine hu \glemph{hetkia} et-xem le-40 dakot be-jom ʃabat kodeʃ\\
 	  {} G-d brought to-you sign that-\gsc{NEG} you.will.rise on the-device in-Saturday and-here he stuck \gsc{ACC-2PL} for-40 minutes in-day Saturday holy\\
 	\glt `{...} G-d gave you a sign not to go on the ride on Saturday, and there you go, he made you get stuck for 40 minutes on the holy Sabbath.' \hfill \url{https:\\www.ynet.co.il/Ext/App/TalkBack/CdaViewOpenTalkBack/0,11382,L-3441716-7,00.html} } 		
 \z 
Additional examples can be found in \cite{lev16}.

In such cases, there is no strong prediction with regards to the kind of causation\is{\isi{causative}} event; my expectation would be that different kinds of causation\is{\isi{causative}} would be possible, as was the case for the examples in Table~\ref{table:vd:triplets-caus}. This much seems to be correct. The verb \emph{hexʃid}, from \emph{xaʃad} `suspected', is attested in both readings: `be suspected', `be made into a suspect' in~(\nextx) and `make X suspect s.th' in~(\anextx).

 \begin{exe}
 \ex  Make O V-ed (turn into a súspect) 
 \begin{xlist} 
 	\ex   
[] 		{ \gll be-tviat-am toanim ha-ʃnaim ki gilboa \glemph{hexʃid} et deri be-re{ts}ixat-a ʃel ester verderber\\
 		  in-lawsuit-theirs claim the-two that Gilboa suspect.\gsc{CAUS} \gsc{ACC} Deri in-murder-hers of Esther Verderber\\
 		\glt `The two claim in their lawsuit that Gilboa turned Deri into a suspect in the murder of Esther Verderber.' \hfill \url{https:\\www.ynet.co.il/articles/0,7340,L-2443354,00.html} } 
	
 	\ex   
[] 		{ \gll ha-seruv \glemph{hexʃid} et netanjahu ve-sar-av ha-krovim ki re{ts}on-am litol le-a{ts}mam samxut-al jexudit, ve-lo linhog be-ʃkifut.\\
 		  the-refusal suspect.\gsc{CAUS} \gsc{ACC} Netanyahu and-ministers-his the-close that will-theirs to.take to-themselves authority-superior unique and-\gsc{NEG} to.behave in-transparency\\
 		\glt This refusal makes one suspect that Netanyahu and his closest ministers wish to avail themselves of unique authority, rather than conduct themselves transparently. \hfill \url{https:\\www.israelhayom.co.il/opinion/294269} } 
	
 \z

 \ex  Make X verb (make someone suspéct s.th, make someone suspicious) 
 \begin{xlist} 
 	\ex   
[] 		{ \gll ʃalom lifnej beerex 5 elef hexlafti galgalʃ kolel ʃarʃeret (z750 2010) ve-ha-mexir ʃe-kibalti k{ts}at \glemph{hexʃid} ot-i\\
 		  hello before about 5 thousand I.changed sprocket including chain (z750 2010) and-the-price that-I.got a.little suspect.\gsc{CAUS} \gsc{ACC}-me\\
 		\glt `Hello, I changed my gear and chain (z750 2010) about five years ago and the price I got made me a little suspicious.'\\ \hfill \url{http://fullgaz.co.il/forums/archive/index.php/t-793.html} } 
	
 	\ex   
[] 		{ \gll galaj ha-mataxot lo hetria u-{vexol zot} ha-falestini \glemph{hexʃid} et loxamej {miʃmar ha-gvul}\\
 		  detector the-metal \gsc{NEG} warn and-nevertheless the-Palestinian suspect.\gsc{CAUS} \gsc{ACC} warriors.of {the Border Patrol}\\
 		\glt `The metal detector did not give any warning but nevertheless, the Palestinian aroused the suspicion of the Border Patrol soldiers.' \hfill \url{http:\\www.93fm.co.il/radio/445111/} } 
	
 \z
\z 

These examples confirm that there are clear compositional differences with the marked\is{\isi{markedness}} \isi{causative} alternation: forms built from Voice/{\vz} are transparent, while those built from {\vd} are marked\is{\isi{markedness}}.

	\subsection{The labile alternation} \label{vd:caus:labile}
The main characteristic of {\vd} is that it is supposed to guarantee the availability of an external argument; in other words, a \isi{transitive} construction is possible if the event has change-of-state semantics, i.e.~an internal argument. Looking back at the labile alternation, I have not yet found any alternations in which the \isi{causative} is preferred and the inchoative is a recent innovation; or inchoatives in {\thif} which have no \isi{causative} counterpart. I take these findings to be emblematic of the \isi{causative} meaning inherent in {\vd}: even if inchoative verbs have arisen, contemporary usage overwhelmingly tends to coin causatives in this template rather than another kind of verb \citep{laks14}.

Let us continue to assume that the process of inchoative formation in {\thif} is productive\is{productivity}, as argued for by \cite{lev16}, and is not merely a list of exceptions, as implicitly assumed in most of the literature (with the exception of \citealt{doron03}, to be discussed in Section~\ref{vd:others:ed}). Then, when the speaker is faced with the choice of a construction for their de-adjectival or denominal verb, they might choose {\vd} because this structure guarantees that an external causer can be added (we should keep in mind that {\thit} is the more productive\is{productivity} template for novel derived forms, e.g.~\citealt{laks11}). That is perhaps why this is the only head which is compatible with labile alternations.

One consequence of the overall analysis is that it allows us to state in formal terms the difference in argument distribution between causatives and inchoatives. 
I have proposed that once the structure contains a more deeply embedded a/n node, v is too far away from the root for particular selectional requirements to be stated. This idea receives potential corroboration from the behavior of -\emph{en} in English.\footnote{Thanks to Jim Wood for pointing out this phenomenon.} As noted by \cite{harley09n}, English verbalizers such as -\emph{ify}, \emph{-ize} and \emph{-ate} can derive verbs that are uniformly unaccusative (e.g.~\emph{oscillate}), uniformly unergative (e.g.~\emph{detoriorate}) or labile (e.g.~\emph{activate}), but -\emph{en} verbs are always labile. An examination of the list in \citet[245]{levin93} confirms this claim. If we assume that these latter verbs contain additional structure, for instance [v [CMPR [a \root{Root}~\!]]] \citep{bobaljik12}, we arrive at a similar analysis to that of {\thif} inchoatives: they cannot impose selectional restrictions and are ``stuck'' with the argument structure imposed by the syntax. But I will not develop this idea or the crosslinguistic ramifications any further.

	\subsection{Generalizing to Pred/\textit{i}*} \label{vd:caus:pred}
Before turning to alternative accounts of the patterns above, I would like to briefly consider a variant of the account given above (suggested by Jim Wood, p.c.~November 2019). The intuition here is that the head deriving inchoatives in {\thif} is not {\vd} itself but a variant of the predicative head Pred, which itself is another label for \textit{\isi{i*}} (a generalized argument introducer I discuss later on, in Chapter~\ref{i:i}).

The formal analysis builds on the notion of a predicative head Pred, which has been invoked in various ways in the literature. Since the specifics are not important for this short discussion, I will simply point out \cite{bowers93,bowers01} as one influential account and \cite{matushansky19} for a recent reply.

According to the Pred alternative, denominal and de-adjectival verbs in this template at least are derived using the head {\predd}. Causatives would then have the structure in~(\nextx) and unaccusatives the structure in~(\anextx): the internal argument is introduced in Spec,PredP, which is then embedded under Voice.
 \begin{exe}
\ex  
	\Tree 
	[.VoiceP
		[.DP ]
		[.
			[.{\vd} ]
			[.vP
				[.v ]
				[.PredP
					[.DP ]
					[.
						[.{\predd} ]
						\qroof{red}.\emph{aP}
					]
				]
			]
		]
	]

\ex  
	\Tree
	[.VoiceP
		[.Voice ]
		[.vP
			[.v ]
			[.PredP 
				[.DP ]
				[.
					[.{\predd} ]
					\qroof{red}.\emph{aP}
				]
			]
		]
	] 
 \z 

If we assume this analysis, we can then replace {\predd} with the generalized \emph{\isi{i*}}$_{\text{[\!+\!D]}}$, which should have the same spell-out as {\vd}; again, see Chapter~\ref{i:i}.

Are there strong reasons to adopt or reject this proposal? On the one hand, we now have an explanation for why both embedded \emph{n} and embedded \emph{a} give {\thif} non-active semantics; the formulation in~(\ref{ex:vd:sem}) and~(\ref{ex:vd:sem-full}) makes this seem like an accident. The Pred analysis would also mean that {\vd} no longer needs to see both v and the embedded a/n in~(\ref{ex:tree:vd-inch}), a problematic situation in terms of \isi{locality} constraints.

On the other hand, a main cause for concern would be the increased combinatorics associated with an additional head, in this case Pred: what about Unspecified Pred and {\predz}? Evidence for {\predz} is hard to establish, although Jim Wood (p.c.) suggests that constructions such as \emph{The wizard turned invisible to avoid being detected} or \emph{The fish turned red to impress its mate} could be the adjectival counterparts of \isi{figure reflexives} (Chapter~\ref{vz:tnif:figrefl}), at least in English.

In addition, {\predd} would need to be morphologically conditioned by T over the intervening Voice head. But this issue does already arise for {\pz}, as discussed in Chapter~\ref{vz:pz:phono}.


\section{Alternative accounts} \label{vd:others}
This section focuses on a number of competing analyses aiming to explain the behavior of verbs in {\thif}, concentrating mostly on the inchoative alternation. Apart from these, \cite{lev16} sketched a theory in which labile verbs are less agentive than others, a claim that could explain why \emph{heet} `slowed down' is possible as opposed to *\emph{hemhir} (from `quick'). However, that idea cannot be extended to explain the existence of minimally different \emph{hei{ts}} `accelerated' so it will be set aside.

I start with a general question of how alternations should be treated, one which in a way ties together the threads of the last three chapters, before turning to more specific points about {\thif}.
	
	\subsection{Where do alternations live?} \label{vd:others:arad}
As I have tried to make clear, under the current proposal there is no formal way in which an anticausative verb is derived from a \isi{causative} verb, or a \isi{causative} verb from an inchoative verb \citep{schaefer08}. There are only different Voice heads which can be merged with a core vP. In contrast to this approach, the traditional view of argument structure alternations for Semitic (and beyond) assumes that verbs in one template are derived, or might be derived, from verbs in another template.

\cite{arad05} is unique in making such a theory formally explicit and internally consistent. The precise formulation enables us to see exactly what the strengths and weaknesses of such an approach are. Some of these were already mentioned in Section~\ref{vz:others:morph}; for others see \cite{kastnertucker19cup}. I will invoke her analysis once more in order to further explain how alternations -- or perceived alternations -- work in different theories..

As noted in Section~\ref{vz:others:morph}, \cite{arad05} assumes that alternations hold between specific templates, as in the following conjugation\is{\isi{conjugation class}} classes \cite[226]{arad05}. The ones relevant to {\thif} are highlighted.
 \begin{exe}
 \ex  
 \begin{xlist} 
 	\ex  Conjugation 1: {\tnif} $\rightarrow$ {\tkal} 
 	\ex  \textbf{Conjugation 2: {\tkal} $\rightarrow$ {\thif}} 
 	\ex  \textbf{Conjugation 3: {\tnif} $\rightarrow$ {\thif}} 
 	\ex  Conjugation 4: {\thit} $\rightarrow$ {\tpie} 
 	\ex  Conjugation 5: {\thit} $\rightarrow$ {\tpie} 
 	\ex  \textbf{Conjugation 6: {\thif} $\rightarrow$ {\thif}} 
 \z
\z 
Three conjugation\is{\isi{conjugation class}} classes are needed because {\thif} ostensibly alternates with three other templates: {\tkal}, {\tnif} and {\thif} itself. For example, \emph{nirdam} `fell asleep' alternates with \emph{herdim} `put someone to sleep; applied anesthesia', instantiating~(\lastx c).

The spell-out rules relevant to {\thif} are highlighted in~(\nextx). In prose: If Class 2, then the inchoative is {\tkal} and the \isi{causative} is {\thif}. If Class 3, then the inchoative is {\tnif} and the \isi{causative} is {\thif}. If Class 6, then the inchoative is {\thif} and the \isi{causative} is also {\thif}. If the root does not take part in an alternation, then a verb in {\thif} can spell out unmarked\is{\isi{markedness}} v, inchoative v or \isi{causative} v (but not stative v).
 \begin{exe}
 \ex  Distributed Conjugation Diacritics in \citet[230]{arad05}: \label{ex:arad-classes2} 
 \begin{xlist} 
\begin{multicols}{2}
 	\ex   v$_{unmarked}$: \\
			$ \alpha$ $\rightarrow$ {\tkal} \\
			$\beta$ $\rightarrow$ {\tnif}\\
			$\gamma$ $\rightarrow$ {\tpie}\\
			\textbf{$\delta$ $\rightarrow$ {\thif}}\\
			$\epsilon$ $\rightarrow$ {\thit}
 	\ex  v$_{inch}$: \\
			$ \alpha$ $\rightarrow$ {\tkal} \\
			\textbf{$\beta$ $\rightarrow$ {\tnif}} \\
			\dots \\
			\textbf{$\delta$ $\rightarrow$ {\thif}}\\
			$\epsilon$ $\rightarrow$ {\thit}\\
			\dots \\
			\textbf{Conjugation 2 $\rightarrow$ {\tkal}}\\
			\textbf{Conjugation 3 $\rightarrow$ {\tnif}}\\
			Conjugation 4 $\rightarrow$ {\thit}\\
			\textbf{Conjugation 6 $\rightarrow$ {\thif}}\\
			\dots
		\columnbreak
 	\ex  v$_{stat}$: \\
			$ \alpha$ $\rightarrow$ {\tkal} \\
			Class 3 $\rightarrow$ {\tkal}\\
			Class 5 $\rightarrow$ {\tkal}
 	\ex  v$_{caus}$: \\
			$\gamma$ $\rightarrow$ {\tpie}\\
			\textbf{$\epsilon$ $\rightarrow$ {\thif}}\\
			Conjugation 1 $\rightarrow$ {\tkal}\\
			\textbf{Conjugation 2 $\rightarrow$ {\thif}}\footnote{\citet[231]{arad05} has this as {\tnif}, which as far as I can tell is a typo.}\\
			\textbf{Conjugation 3 $\rightarrow$ {\thif}}\\
			Conjugation 4 $\rightarrow$ {\tpie}\\
			\textbf{Conjugation 5 $\rightarrow$ {\thif}}\\
			\textbf{Conjugation 6 $\rightarrow$ {\thif}}\\
			\dots
	\end{multicols}
 \z
\z 

In the Trivalent theory none of this machinery is required. Alternations are an intuitive way of describing what happens when a given core vP combines with \isi{Unspecified Voice} compared to {\vz} and compared to {\vd}.

Importantly, I am not claiming that the current theory does away with all of the idiosyncratic listing that \citeauthor{arad05}'s had. As emphasized throughout this book, all theories need to list at some level which roots can combine with which functional heads/templates. But I hope it is clear why the Trivalent theory is to be preferred: beyond the empirical arguments I adduce, the overall view of the grammar is more streamlined, less stipulative, and much more in line with our theories of non-Semitic languages.

A theory using conjugation\is{\isi{conjugation class}} classes needs to make reference to silent flavors of v (whereas the elements I have proposed are all overt). It can also encode virtually any alternation. That may well be too powerful but it does absolve the theory of the need to worry about the combinatorics of individual heads, which is a potential advantage over the Trivalent theory insofar as the behavior of {\vd} is concerned. Concretely, my account does not contain a principled answer to the question why {\vd} cannot attach to a [{\va} vP] -- this is possible, but I am not claiming that any template instantiates this combination. One possible answer is that a structure such as [{\vd} [{\va} vP]] would just entail an agentive reading, something that {\vd} is already compatible with, or a specific \isi{causative} reading from among the kinds in Table~\ref{table:vd:triplets-caus}. Furthermore, the phonology of {\va} could in principle be impoverished under {\vd}. Under a functionalist view, {\va} and {\vd} do similar work. This kind of issue, however, is much more pressing when pointed right back at the conjugation\is{\isi{conjugation class}} class and morphemic accounts: why are these specifically the conjugation\is{\isi{conjugation class}} classes that exist and the templates that exist?

	\subsection{Added structure} \label{vd:others:struct}
A different alternative view might counter that {\vd} does not exist. On this view, verbs in {\thif} are not derived using {\vd} but by additional structure atop of regular, active Voice.

Let me briefly outline what such an analysis would look like. This structure would presumably consist of some higher \isi{causative} head, perhaps another layer of Voice.\footnote{For variants of the idea that an analytic \isi{causative} can be deriving using Voice-over-Voice, see \cite{tubinoblanco11}, \cite{harley13lingua,harley17oup} and \cite{nie20}.} While this idea holds theoretical promise, there are two reasons I do not think it can account for the Hebrew data, having to do with incorrect predictions in the phonology and in the syntax-semantics.

First, {\vd} seems as integrated into the morphophonological system as the other Voice heads, namely \isi{Unspecified Voice} and {\vz}. In particular, the analysis in \cite{kastner18nllt} shows that the spell-out of {\vd} is subject to the same \isi{locality} constraints as that of Voice and {\vz}. Splitting {\vd} into two layers of Voice will disrupt these \isi{locality} relations, thereby making the wrong predictions in the phonological component.

Second, this approach would treat {\thif} as a straightforward ``make''-\isi{causative} (analytical \isi{causative}). This much is clearly wrong for the unergative and unaccusative verbs in this template as it misses the fact that causatives in this template are ``lexical'', as discussed in Section~\ref{vd:caus:mrkd}. 

Third, assuming that another Voice head can be added to VoiceP makes the false prediction that a \isi{transitive} in {\tkal} and a \isi{causative} in {\thif} will have the same internal argument. This is once again incorrect, as we saw in Section~\ref{vd:caus:mrkd}.

And fourth, in Chapter~\ref{passn:pass} I discuss how passives are derived by use of an additional head Pass. As explained there, Pass can only attach to {\vd} or to Voice+{\va}, but not to Voice on its own. If we were to assume that {\vd} is actually Voice+Voice, we would need to make an additional stipulation regarding configurations that can be passivized. This last problem is not insurmountable but it would complicate the theory.

	
	\subsection{CAUS and existential closure} \label{vd:others:ed}
A different kind of alternative analysis concerns itself mostly with the inchoatives of Section~\ref{vd:thif:inch}. This analysis would posit a silent, generic \isi{Cause} in Spec,{\vd}. The analysis in \citet[61]{doron03} -- which in many ways is a precursor to the theory presented in this work -- assumes that a \isi{causative} head \gsc{CAUS} gives rise to {\thif}. The problem for the system in \cite{doron03} is that if these verbs are derived using \gsc{CAUS} rather than the middle head \gsc{MID}, we have no explanation for their unaccusativity.

As a result, \citet[62]{doron03} must conclude that ``\emph{x reddened} is equivalent to \emph{Something caused x to redden}'', with \gsc{CAUS} introducing a Causer that is existentially quantified over. This kind of account is more in line with a \isi{passive} analysis than a \isi{causative} one.\footnote{Although Doron's morphosemantic head formed the direct inspiration for the current {\vd}, just as her \gsc{MID} and \gsc{INTNS} heads paved the way for {\vz} and {\va}.}

Assume for the sake of the argument that a silent element fills Spec,{\vd} in inchoatives. One would need to specify the exact featural makeup of this element, for example a null subject \emph{pro}. The result would be a \isi{transitive} structure where \emph{pro} should be assigned Nominative \isi{case} and the internal argument should be assigned Accusative \isi{case}. Definite accusative\is{case} objects in Hebrew take the direct object marker \emph{et}, so we would predict that \emph{et} appears before inchoatives in \thif. But this is incorrect: the generic \isi{Cause} cannot be a silent pronoun in a \isi{transitive} relationship with the internal argument.
 \begin{exe}
 \ex  
 \begin{xlist} 
 	\ex    
[*] 		{ \gll heʃmin \glemph{et} ha-xatul\\
 		  fattened.\gsc{CAUS} \gsc{ACC} the-cat\\
	}
	
 	\ex    
[*] 		{ \gll \glemph{et} ha-xatul heʃmin\\
 		  \gsc{ACC} the-cat fattened.\gsc{CAUS}\\
 		\glt (int. `The cat grew fat') } 
	
 \z
\z 

Another tack would be to say that instead of \emph{pro}, the silent external argument is a Weak Implicit Argument in the sense of \cite{landau10}: a bundle of phi-features with no [D] feature (cf.~\citealt{legate14,bhattpancheva17}), distinguishing it from a Strong Implicit Argument such as \emph{pro}. If there is no [D] feature on the weak EA, it does not participate in the calculus of \isi{case} and the IA will receive unmarked\is{\isi{markedness}} \isi{case}, i.e.~Nominative. This kind of analysis ends up being very similar to the one in the current proposal: the external argument is not taking part in any relevant syntactic process, and whatever requirements {\vd} has still need to be satisfied. Furthermore, This bundle of phi-features would then have no detectable effects in the syntax or phonology, rendering it purely stipulative. In the absence of a convincing account for implied causers, I reject this analysis.

	\subsection{Contextual allomorphy}
Another possible analysis of inchoatives is strictly morphological in nature. Under this account, unaccusative inchoatives are true unaccusatives derived with {\vz}, except that the allomorphic rule in~(\ref{ex:vd:allo-unacc}) causes {\vz} to be pronounced like {\thif} rather than like {\tnif}.
 \begin{exe}
 \ex  
 \begin{xlist} 
   \ex  \vz~\lra~{\thif} / \trace~\{\root{lbn}, \root{'dm}, \root{xl\dgs{k}}, \root{xvr}, \root{ʃmn}, \dots \}\label{ex:vd:allo-unacc} 
   \ex  \vz~\lra~{\tnif} 
 \z
\z 

One question which arises is whether we would like to postulate this rule for a list of just over 30 roots. Furthermore, the mystery would remain of why it is specifically {\thif} that houses inchoatives: why doesn't the rule in~(\lastx a) insert the form of any other template, such as {\tkal}, {\tnif} or {\tpie}? This solution is technically possible but conceptually unenlightening, and it postulates two homophonous VIs {\thif} (one spelling out {\vd} and one spelling out {\vz}).

That being said, it does correctly predict that the roots to which this rule applies cannot surface in {\tnif}, only in {\thit}, (\nextx). The reason is that forms in {\tnif} are generated using {\vz}, but {\vz} is pronounced as {\thif}.
 \begin{exe}
\ex \label{ex:vd:allo-pred} 
	\begin{xlist}
		\ex  \emph{helbin} $\sim$ *\emph{nilban} `whitened' 
		\ex \emph{heedim} $\sim$ *\emph{nidam} `reddened'
		\ex \emph{heʃmin} $\sim$ \emph{*niʃman} `fattened'
	\z
 \z 

	\subsection{Verbalizing affix} \label{vd:others:borer}
In the last alternative to be considered, \cite{borer91} presents an analysis of {\thif} alternations couched in Parallel Morphology (which I have translated into comparable terms in the current theory). Her account consists of two main parts. In the first, she argues that inchoative forms are derived from adjectives while \isi{causative} forms are derived from a root/verb. In the second, she presents an analysis showing why it must be the case -- given certain assumptions -- that causatives are formed in the lexicon and inchoatives in the syntax. The Trivalent analysis is similar to hers in adopting separate structures for causatives and inchoatives, albeit using different argumentation. The content of the analysis is different, though, since for \cite{borer91} {\thif} is a single verbalizing morpheme which subcategorizes for an adjective.

This approach takes Hebrew {\thif} and English -\emph{en} to be verbalizers subcategorizing for an adjectival stem, be it a property root or an adjective \citep[136]{borer91}. 
When this is done in the ``lexicon'' by verbalizing a root, the result is a \isi{causative} verb:
 \begin{exe}
\ex \label{ex:vd:thif-borer-caus}{[}$_{\text{v}}$ \root{\gsc{wide}} -\emph{en}] 
 \z 
When this is done in the syntax by verbalizing an adjective, the result is an inchoative verb:
 \begin{exe}
\ex \label{ex:vd:thif-borer-inch}{[}$_{\text{v}}$ [$_{\text{a}}$ \root{\gsc{wide}} a] -\emph{en}] 
 \z 

Crucially for us, the analysis does not answer the questions posed earlier on in the discussion of inchoatives: why this template and why these roots. Here {\thif} is assumed to be a de-adjectival verbalizer, just like -\emph{en}, without discussion of this template's role in the overall morphosyntax of the language. While it is stipulated that \thif~as a verbalizer subcategorizes for an adjective, this is not always the case: as shown above, many inchoatives are derived from underlying nouns. More importantly, even run-of-the-mill causatives such as \emph{hexnis} `inserted', \emph{heexil} `fed' and \emph{helbiʃ} `dressed' are not derived from underlying adjectives.

The \isi{causative} \emph{hexnis} `inserted' is derived from \root{kns}, but without a simple adjective *[$_{a}$ \root{kns} a]. One could posit an abstract adjective that is never lexicalized, but it is unclear what this non-existent adjective would be like or what its phonological form would have been (*\emph{kanus}?).
 \begin{exe}
\ex   
[] 	{ \gll ha-malka \glemph{hexnis-a} et ha-sefer la-tik.\\
 	  the-queen inserted.\gsc{CAUS-F} \gsc{ACC} the-book to.the-bag\\
 	\glt `The queen put the book in the bag.' } 
	
 \z 

The \isi{causative} \emph{heexil} `fed' is derived from \root{'kl}, but probably not from \emph{axul} `consumed', a rare adjectival \isi{passive} of \emph{axal} `ate'.
 \begin{exe}
 \ex  
 \begin{xlist} 
 	\ex   
[] 		{ \gll ha-nasix \glemph{heexil} et ha-kivsa.\\
 		  the-prince fed.\gsc{CAUS} \gsc{ACC} the-sheep\\
 		\glt `The prince fed the sheep.' } 
		
		
 	\ex    
[$\ne$] 		{ \gll ha-nasix garam la-kivsa lihiot \glemph{axula}.\\
 		  the-prince caused to.the-sheep to.be consumed\\
 		\glt `The prince caused the sheep to be consumed.' } 
		
 \z
\z 

And the \isi{causative} \emph{helbiʃ} `dressed' is derived from \root{lbʃ}, but probably not from \emph{lavuʃ} `dressed up', the adjectival \isi{passive} of \emph{lavaʃ} `wore', which seems to be reserved for descriptions of a full costume.
 \begin{exe}
 \ex  
 \begin{xlist} 
 	\ex   
 		{ \gll ha-ima \glemph{helbiʃ-a} et ha-jeled (be-)xalifa jafa.\\
 		  the-mom dressed.\gsc{CAUS}-\gsc{F.SG} \gsc{ACC} the-boy in-suit pretty\\
 		\glt `The mother put the boy's pretty suit on (him). } 
		
	
 	\ex  ``On making his discovery, the astronomer had presented it to the International Astronomical Congress, in a great demonstration, \dots 	
 		{ \gll aval iʃ lo he'ezin le-dvara-v, miʃum ʃe-haja \glemph{lavuʃ} be-tilobʃet turkit. ka'ele hem ha-mevugarim.\\
 		  but nobody \gsc{NEG} listened to-words-his, since \gsc{COMP}-was dressed.up in-outfit Turkish such \gsc{3PL} the-grown.ups\\
 		\glt But he was in Turkish costume, and so nobody would believe what he said. Grown-ups are like that.''\hfill {(Antoine de Saint-Exup\'ery, \emph{The Little Prince}, Chapter 4. Hebrew by Jude Shva\footnotemark)} } 
	
\footnotetext{\url{http:\\www.oocities.org/sant\_exupery/c4.htm}}
 \z
\z 

The analysis in \cite{borer91} did not attempt to find an underlying reason for why {\thif} is used for both causatives and inchoatives, as well as for general causativization in the rest of the system. Nevertheless, it remains the only in-depth study of this alternation that I know of. Recall, for the last part of this discussion, that this analysis also postulates a structural difference between {\thif} causatives, (\ref{ex:vd:thif-borer-caus}), and inchoatives, (\ref{ex:vd:thif-borer-inch}). I review this distinction next.

The logic works as follows: if an adjective passes certain diagnostics, and the inchoative does but the \isi{causative} does not, then the adjective must be embedded in the inchoative \citep[130]{borer91}. Starting with an English example, the adjective \emph{wide} is said to license\is{\isi{licensing}} comparisons with \emph{as}/\emph{like} and comparative forms, whereas the inchoative \emph{widen} does not. \citeauthor{borer91}'s claim is that comparison adverbials and the comparative must be licensed\is{\isi{licensing}} by an adjective (judgments hers).
 \begin{exe}
 \ex  
 \begin{xlist} 
   \ex  The canal is \{as \glemphu{wide} as a river / \underline{wider} than a river.\} 
   \ex  The canal \glemphu{widened} \{like a river / more than a river\}. 
	  (int. `The canal became as wide as a river is wide / became more wide than a river is wide')
   \ex   The flood \glemphu{widened} the canal \{like a river / more than a river\}. 
	  (int. `The flood made the canal as wide as a river is wide / made the canal wider than a river is wide')
 \z
\z 
I suspect that there is much more variation in acceptability for the utterances in~(\lastx), and that an adverbial reading normally overpowers the scalar one (`The flood widened the canal like a river widens it'). Three native speaker linguists I have consulted do not share these contrasts but I am more interested in the argumentation involved with this approach.

Taking the adjective \emph{ʃmen-a} `fat-\gsc{F.SG}', it is claimed to license\is{\isi{licensing}} comparatives,~(\nextx a). Inchoatives license\is{\isi{licensing}} comparatives too,~(\nextx b), but causatives do not,~(\nextx c). Judgments are as in \cite{borer91}; example (\nextx c) does not sound as degraded to me, but it does to another speaker I consulted informally.
 \begin{exe}
 \ex  
 \begin{xlist} 
   \ex[] { Adjective:\\  
  \gll ha-xatula \glemph{ʃmena} \{~kmo xazir / joter mi-xazir~\}.\\
       the-cat fat like pig {} more than-pig\\
     \glt `The cat is fat as a pig / fatter than a pig.' }
  
   \ex[] {  Inchoative:\\  
	 \gll \glemphu{ha-xatula} \glemph{heʃmina} \{~kmo xazir / joter mi-xazir~\}.\\
       the-cat fattened like pig {} more than-pig\\
     \glt `The cat grew as fat as a pig / fatter than a pig.' } 
  
   \ex[*] {  Causative:\\  
	 \gll \glemphu{ha-zrika} \glemph{heʃmina} et ha-xatula \{~*kmo xazir / *joter me-xazir~\}\\
       the-injection fattened \gsc{OM} the-cat.\gsc{F} like pig {} more than-pig\\
     \glt (int. `The injection made the cat fat as a pig / more than a pig is fat.') } 
  
 \z
\z 

Similarly, some adverbs (\emph{haxi ʃe-efʃar} `as much as possible') must be licensed\is{\isi{licensing}} by an adjective and accordingly only appear with inchoatives, not causatives.

It seems to me that the success of this diagnostic depends to a large extent on the lexical items chosen. For example, using the antonym \emph{herza} `grew thin', my judgments are slightly different:
 \begin{exe}
 \ex  
 \begin{xlist} 
   \ex[] {  Adjective:\\  
	 \gll ha-xatula \glemph{raza} \{~kmo makel / ?joter mi-makel~\}.\\
       the-cat thin like stick {} more than-stick\\
     \glt `The cat is as thin as a rail / skinnier than a rail.' } 
  
   \ex[?] {  Inchoative:\\  
	 \gll \glemphu{ha-xatula} \glemph{herzeta} \{~kmo makel / ??joter mi-makel~\}.\\
       the-cat thinned like stick {} more than-stick\\
     \glt (int. `The cat became as thin as a rail / skinnier than a rail.') } 
  
   \ex[??] {  Causative:\\  
	\gll \glemphu{ha-zrika} \glemph{herzeta} et ha-xatula \{~??kmo makel / ??joter me-makel~\}\\
       the-injection thinned \gsc{OM} the-cat like stick {} more than-stick\\
     \glt (int. `The injection made the cat as thin as a rail / skinnier than a rail.') } 
 \z
\z 

With \emph{heet} `slowed down' I judge inchoatives unacceptable and causatives slightly better though still degraded. These judgments are meant to highlight the variance, not to be taken as categorical for all alternations or all speakers.
 \begin{exe}
 \ex  
 \begin{xlist} 
 	\ex[] {  Adjective:\\  
	 \gll ha-mexonit ha-zo \glemph{itit} \{~kmo {ts}av / joter mi-{ts}av~\}.\\
 		  the-car the-this slow like turtle {} more than-turtle\\
 		\glt `This car is as slow as a turtle / slower than a turtle.' } 
	
 	\ex[*] {  Inchoative:\\  
	\gll ha-mexonit ha-zo \glemph{heeta} \{~*kmo {ts}av / *joter mi-{ts}av~\}\\
 		  the-car the-this slowed like turtle {} more than-turtle\\
		\glt (int. `This car slowed down to turtle speed / to sub-turtle speed.')\\
			(More acceptable on a reading of `The car slowed down like a turtle slowed down'.)
		}
	
 	\ex[??] {  Causative:\\  
	\gll \glemphu{ha-baaja} \glemphu{ba-hiluxim} \glemph{heeta} et ha-mexonit \{~??kmo {ts}av / ??joter mi-{ts}av~\}\\
 		  the-problem in.the-gears slowed \gsc{ACC} the-car like turtle {} more than-turtle\\
 		\glt (int. `The problem with the gear box slowed the car down to turtle speed / to sub-turtle speed.') } 
 \z
\z 

It is also left vague what precisely this diagnostic is probing. In~(\nextx), for instance, there is no underlying adjective `beloved' but the utterance is completely acceptable:\footnote{Thanks to Idan Landau for pointing this out to me.}
 \begin{exe}
\ex  
	{ \gll ani \glemph{ohev} otxa \glemphu{kmo} \glemphu{ax}.\\
 	  I love.\gsc{SMPL}.\gsc{PTCP} you.\gsc{M} like brother\\
 	\glt `I love you like a brother.' } 
 \z 

Since I am not sure that the argument from comparatives generalizes, and given that no explicit syntax or semantics for this modification was put forward, I do not endorse the arguments for distinct structures put forward in \cite{borer91}. Nevertheless, the current proposal has recast that intuition in contemporary terms and supported it using different arguments. Hopefully these were explicit enough to be similarly challenged in future work.


\section{Conclusion} \label{vd:sum}
This chapter developed the theory of [\!+\!D] in Voice based on an in-depth analysis of various verb types in {\thif}.
%\footnote{I do not posit a [\!+\!D] variant of \textit{p} since there is no evidence for such a head, be it syntactic, semantic, or phonological.}
This template predominantly instantiates active verbs, usually causatives. It is also reasonably productive\is{productivity}. Yet a number of roots derive inchoative verbs in this template.

\hammer{
 \begin{exe}
 \ex  \label{ex:gen-thif-sum}\textbf{Generalizations about {\thif}} 
 \begin{xlist} 
 	\ex  \textbf{Configurations:} Verbs appear in transitive and unergative configurations; a small class of verbs forms unaccusative degree achievements. 
 	\ex  \textbf{Alternations:} Some verbs are causative or active versions of verbs in other templates, especially {\tkal}. A small class of verbs creates a labile alternation within {\thif}. 
 \z
\z 
}

The analysis proposed here showed how the influence of a certain class of roots can be accommodated in the grammar, while keeping constant the overall behavior of the functional head. The existence of unmarked\is{\isi{markedness}} and marked\is{\isi{markedness}} causatives was discussed with respect to the leeway different roots have within similar structures. The feature [+D] must have some semantic content beyond the unmarked\is{\isi{markedness}} \isi{causative}. 

The two factors conspiring to create a labile alternation in a language that otherwise does not allow such an alternation are the root and the syntactic structure. The roots fall under various lexical semantic classes but all appear to derive \isi{degree achievements} from underlying nouns or adjectives, as suggested by \cite{lev16}. The syntax which facilitates this derivation is one in which a noun or adjective is first formed before it is verbalized, and then combined with a specific \isi{causative} head {\vd}. This theoretical approach allows us to ask more specific questions about how the idiosyncratic information associated with roots interacts with the syntactic structure in which they are embedded.

Taken together, these last three chapters cashed out the trivalent theory of Voice which is at the core of this book. The next chapter rounds off the empirical picture by examining cases in which these heads interact with additional structure: passivization, adjectival passives and nominalization\is{nominalizations}.


    \chapter{Passives and nominalizations}
\label{chap:passn}

The three preceding chapters introduced a system of Voice heads which, at present, has only been proposed for argument structure alternations in Hebrew (the next two chapters consider whether other languages should be analyzed similarly). Recapping Chapters~\ref{chap:voice}--\ref{chap:vd}, there are seven distinct possibilities for verbal forms (five non-\isi{passive} templates), summarized in Table~\ref{tab:5-1:overview}.

\begin{table}
\begin{tabular}{llll}
 \lsptoprule
Voice? & {\va}? & {\pz}? & Template \\ \midrule
Voice	&		&	&	\tkal\\
Voice	&	\va	&	&	\tpie\\
{\vz}	&		&		&	\tnif\\
{\vz}	&	\va &	&	\thit\\
{\vd}	&		&		&	\thif\\
Voice	&		&	{\pz} & \tnif\\
Voice	&	\va	&	\pz	& \thit\\
\lspbottomrule
 \end{tabular}
	\caption{Seven combinations of functional heads so far\label{tab:5-1:overview}}
\end{table}

But argument structure exists beyond just the verbal domain: some nominals and adjectives have arguments too, and in Hebrew in particular, the morphological history of some derived forms is clearly verbal. The question to be examined here is, how well does the Trivalent Theory predict behavior in derived forms?

In this chapter I look into three constructions whose behavior is generally well understood: \textsc{verbal passives}, \textsc{adjectival passives} and \textsc{\isi{nominalizations}}. I say ``well understood'' but of course implementations differ, as do some theoretical views. The scope of the current chapter is limited: I take the theory developed thus far and essentially see what happens when the different VoicePs are embedded in additional structure. Where I believe the results do bear on current debates, I highlight this, but otherwise the focus is on showing how, once the ``exotic'' VoiceP of a non-concatenative language has been built up, higher material combines in a fairly transparent fashion syntactically, semantically and phonologically.

In other words, I am taking the structures underlying the constructions of this chapter to be universal. Accordingly, the different sections of this chapter will look a little different than the previous ones. Section~\ref{passn:pass} looks at \isi{passive} verbs (the head Pass), Section~\ref{passn:adjpass} at adjectival passives (the head \emph{a}), Section~\ref{passn:n} at \isi{nominalizations} (the head n), and a conclusion with some discussion of denominal verbs follows in Section~\ref{passn:conc}. In each case I begin with some general background on the state of the art. Then the Hebrew data is introduced, followed by the formal analysis (combining the general consensus with the Hebrew data). Importantly, the different templates interact with these embedding heads in different ways, so we will spend some time analyzing these interactions as well.


\section{Passivization} \label{passn:pass}
	\subsection{Background}
My definition of a \textsc{\isi{passive}} verb is given in~(\ref{ex:5:1}). It is not meant to be controversial in any way; see \cite{williams15} for an overview of various related issues.

 \begin{exe}
 \ex  \label{ex:5:1}
 \begin{xlist} 
 	\ex  A passive verb is an intransitive verb which does not have an overt external argument in the regular subject position but does have an Agent which is either (a) implicit and existentially closed over or (b) made overt in a \emph{by}-phrase. 
 	\ex  Formally, there is no external argument in Spec,VoiceP (or Spec,TP, for that matter) but there is an Agent role in the semantics. 
 \z
\z 

I would like to clarify right from the start that what is often descriptively called ``the \isi{passive}'' is not necessarily what I mean by my formal definition. One often reads about the ``\isi{passive}'' in Romance languages (as with \emph{se} in \ili{French}), in \ili{Greek} or in various other languages and language families. But this term is used pretheoretically and as a matter of convenience: the element tracked by French \emph{se}, Greek non-active morphology, the \ili{Kannada} non-active suffix \citep{lidz01} and so on is a non-active Voice head. As argued by \cite{alexiadoudoron12} and \citet[123]{layering15}, and emphasized again by \cite{spathasetal15} and \cite{kastnerzu17}, there are two structures which can give rise to \isi{passive} \emph{readings}. One is a VoiceP with a non-active Voice head, as in Greek, Romance and many other languages. The other is what we obtain when we use a dedicated \isi{passive} head, Pass. This is the case in English, German and a few other languages.\footnote{The list is not very long, consisting also of \ili{Classical Greek}, some Semitic languages and \ili{Fula}. See \cite{klaiman91} and \cite{alexiadoudoron12}.} Hebrew, as we have seen, has both options \citep{alexiadoudoron12}: existential closure as an alloseme of {\vz} (Section~\ref{vz:vz}) and the head Pass which I implement next.

The literature suggests a number of characteristics of passives which can be used as diagnostics \citep{bakeretal89,sichel09,spathasetal15}, a few of which were already used in Section~\ref{vz:tnif:nact}. Passive verbs/clauses can take \emph{by}-phrases specifying the agent\is{Agent}~(\ref{ex:pass-by-en}), allow agent\is{Agent}-oriented adverbs~(\ref{ex:pass-adv-en}), allow control into adjunct clauses~(\ref{ex:pass-pro-en}) and show disjoint reference effects~(\ref{ex:pass-dre-en}), i.e.~no coreference of agent\is{Agent} and theme. Existential binding of the implicit agent\is{Agent} means that it itself cannot be controlled or bound~(\ref{ex:pass-bind-en}).

 \begin{exe}
\ex  \label{ex:pass-by-en}The ship was sunk (by Bill). 
\ex  \label{ex:pass-adv-en}The ship was sunk deliberately. 
\ex  \label{ex:pass-pro-en}EA$_i$ The ship was sunk to PRO$_i$ collect the insurance. 
\ex  \label{ex:pass-dre-en}EA$_i$ The child$_{*i/j}$ was combed. 
\ex  \label{ex:pass-bind-en} 
 \begin{xlist} 
 	\ex  Mary$_i$ wants John to be seen (*by Mary$_i$). 
 	\ex \sloppy Every journalist$_i$ wants the President to be interviewed (by someone$_{*i/j}$). 
 \z
\z 

Synthesizing the existing literature on \isi{passive} heads \citep{bruening13,layering15}, I formalize Pass as follows. In the \textit{syntax}, Pass merges above Voice. It is incompatible with merger of a DP in Spec,VoiceP immediately below it. \cite{bruening13} implements this constraint as a selectional requirement on the size of the VoiceP combining with the \isi{passive} head.

In the \textit{semantics}, Pass likewise brings about existential closure over an implicit external argument, (\ref{ex:5n7}). There are two ways of formalizing this idea. The one I adopt is that of \cite{bruening13}, where Pass takes the VoiceP as its argument and closes off the \isi{Agent} role:

 \begin{exe}
\ex  \label{ex:5n7}\denote{Pass} = λPλe∃x.Agent(x,e) \& P(e) 
 \z 
This denotation is identical to what I suggested for the \isi{passive} alloseme of\linebreak\relax {\vz} in Section~\ref{vz:vz:sem}, (\ref{ex:pass-sem}).

 \begin{exe}
	\ex  \label{ex:pass-sem}\denote{\vz} = 
	\begin{xlist}
		\ex λPλe∃x.Agent(x,e) \& P(e) / \{\root{rtsx} `murder', \root{'mr} ‘say’, {\dots} \} 
		\ex λP$_{<s,t>}$.P
		\z
	\z 

An alternative semantics is to force Voice to choose an agent\is{Agent}-less alloseme in the context of Pass, and then let Pass introduce an (existentially closed off) agent\is{Agent} itself. I note this possibility for completeness.

 \begin{exe}
\ex  \denote{Voice} = λP.P / Pass \trace 

\ex  \denote{Pass} = λPλe∃x.P(e) \& Agent(x,e) 
 \z 

The \textit{phonology} of Pass is language-specific. In English it spells out the auxiliary \emph{be}, in \ili{German} it spells out \emph{werden}, and in Hebrew I will claim below than it overwrites the vowels of the stem VoiceP it merges with.


	\subsection{Descriptive generalizations} \label{passn:pass:tpua}
There are two exclusively \isi{passive} templates in Hebrew: {\tpua} and {\thuf}. I have not described these templates yet. An active-\isi{passive} example pair with {\tpua} is given in~(\ref{ex:5n11}).

 \begin{exe}
 \ex  \label{ex:5n11}
 \begin{xlist} 
 	\ex   
[] 		{ \gll ha-jeled \glemph{sider} et ha-xeder.\\
 		  the-boy organized.\gsc{INTNS} \gsc{ACC} the-room\\
 		\glt `The boy tidied up his room.' } 
	
 	\ex   
[] 		{ \gll ha-xeder \glemph{sudar} (al-jedej ha-jeled).\\
 		  the-room organized.\gsc{INTNS.PASS} by the-boy\\
 		\glt `The room was tidied up (by the boy).' } 
	
 \z
\z 

I use {\tpua} and {\thuf} interchangeably here since there does not seem to be any difference between them, beyond the fact that they are derived from different templates.

Hebrew passives pass the standard tests above. The \emph{by}-phrase can be seen in~(\ref{ex:5n11}) and the rest are given below: agent\is{Agent}-oriented adverbs~(\ref{ex:pass-adv-he}), control into adjunct clauses~(\ref{ex:pass-pro-he}), disjoint reference effects~(\ref{ex:pass-dre-he}) and existential binding of the implicit agent\is{Agent}~(\ref{ex:pass-bind-he}).

 \begin{exe}
\ex  \label{ex:pass-adv-he}  
	{ \gll ha-sfina \glemph{hutbea} \glemphu{be-jodin}.\\
 	  the-ship sank.\gsc{CAUS.PASS} in-cognizance\\
 	\glt `The ship was sunk deliberately.' } 
	
\ex  \label{ex:pass-pro-he}  
 	{ \gll EA$_i$ ha-sfina \glemph{hutbea} kedej PRO$_i$ lekabel et ha-bituax.\\
 	  {} the-ship sank.\gsc{CAUS.PASS} in.order.to {} to.receive \gsc{ACC} the-insurance\\
 	\glt `The ship was sunk to collect the insurance.' } 
	
\ex  \label{ex:pass-dre-he} 
		{ \gll EA$_i$ ha-jeled$_{*i/j}$ \glemph{sorak}.\\
 	  {} the-boy combed.\gsc{INTNS.PASS}\\
 	\glt `The boy was combed.' } 
	
 \ex  \label{ex:pass-bind-he} 
 \begin{xlist} 
 	\ex   
		{ \gll dana$_i$ ro{\ts}a ʃe-ha-jeled \glemph{jesorak} (*al-jedej dana$_i$).\\
 		  Dana wants that-the-boy will.comb.\gsc{INTNS.PASS} by Dana\\
 		\glt `Dana wants the boy to be combed (*by Dana$_i$)'. } 
		
 	\ex   
		{ \gll kol hore$_i$ ro{\ts}e ʃe-roʃ ha-memʃala \glemph{jesorak} (al-jedej miʃeu$_{*i/j}$).\\
 		  every parent wants that-head.of the-government will.comb.\gsc{INTNS.PASS} by someone\\
 		\glt `Every parent wants the Prime Minister to be combed (by someone else).' } 
		
 \z
\z 


It is generally accepted that verbal passives in Hebrew are derived from an active counterpart via some operation of passivization in the syntax, be the framework syntactic \citep{doron03,alexiadoudoron12,borer13oup} or lexicalist \citep{reinhartsiloni05,ussishkin05,laks11}. The meaning of a verb in the \isi{passive} is compositional and transparent in a way that non-\isi{passive} templates are not. For example, verbs in the ``\isi{passive} intensive'' {\tpua} are always the passivized version of an active verb in ``intensive'' {\tpie}, Table~\ref{tab:5-1:alt}a, and verbs in ``\isi{passive} \isi{causative}'' {\thuf} are always the passivized version of an active verb in ``\isi{causative}'' {\thif}, row~b.

\begin{table}
	\begin{tabularx}{\textwidth}{llllll}
 \lsptoprule
	& & \multicolumn{2}{c}{Active} & \multicolumn{2}{c}{Passive} \\\midrule
	a. & {\tpie} $\sim$ {\tpua} & \emph{biʃel} & `cooked' & \emph{buʃal} & `was cooked'\\
	b. & {\thif} $\sim$ {\thuf} & \emph{heʃmid} & `destroyed' & \emph{huʃmad} & `was destroyed'\\
\lspbottomrule
 	\end{tabularx}
	\caption{Predictable alternations in the passive templates}
	\label{tab:5-1:alt}
\end{table}

A derivational view ``in the syntax'', according to which an existing active verb is passivized into a \isi{passive} verb, accounts for two important facts about passives in Hebrew: first, there do not exist any \isi{passive} verbs (that is, verbs in {\tpua} and {\thuf}) without an active base from which they are derived; and second, that \isi{passive} verbs cannot mean anything other than passivization of the active form, where \textsc{passivization} means suppression of the external argument as defined above.

Morphophonologically, verbs in the two \isi{passive} templates have two important characteristics. The first, as mentioned above, is that they form predictable alternations. Verbs in {\tpua} are derived from active verbs in {\tpie}, while verbs in {\thuf} are derived from active verbs in {\thif}. The second characteristic had not been discussed explicitly before \cite{kastner18nllt}, although some aspects of it were noticed in a number of works \citep{ussishkin05,borer13oup}: the form of the stem uniformly has the vowels \emph{u-a}, regardless of underlying active template, root, tense or any other variable. This can be seen in the full paradigms in Tables~\ref{table:pass-vowels-past}--\ref{table:pass-vowels-fut} and stands in stark contrast to the active forms seen throughout this book.

\begin{table}
	\begin{tabular}{lllll}
	 \lsptoprule
	 & \multicolumn{2}{c}{\tpua~\root{gdl}}	& \multicolumn{2}{c}{\thuf~\root{gdl}}\\\cmidrule(lr){2-3}\cmidrule(lr){4-5}
	 & \gsc{SG} & \gsc{PL}	& \gsc{SG} & \gsc{PL}\\\midrule
	1 & g\textbf{u}d\textbf{a}l-ti & g\textbf{u}d\textbf{a}l-nu		& h\textbf{u}gd\textbf{a}l-ti & h\textbf{u}gd\textbf{a}l-nu\\
	2M & g\textbf{u}d\textbf{a}l-ta & g\textbf{u}d\textbf{a}l-tem	& h\textbf{u}gd\textbf{a}l-ta & h\textbf{u}gd\textbf{a}l-tem\\
	2F & g\textbf{u}d\textbf{a}l-t & g\textbf{u}d\textbf{a}l-tem	& h\textbf{u}gd\textbf{a}l-t & h\textbf{u}gd\textbf{a}l-tem\\
	3M & g\textbf{u}d\textbf{a}l & g\textbf{u}d\del{\textbf{a}}l-{u}	& h\textbf{u}gd\textbf{a}l & h\textbf{u}gd\del{\textbf{a}}el-{u}\\
	3F & g\textbf{u}d\del{\textbf{a}}l-{a} & g\textbf{u}d\del{\textbf{a}}l-{u}	& h\textbf{u}gd\del{\textbf{a}}el-{a} & h\textbf{u}gd\del{\textbf{a}}el-{u} \\
	\lspbottomrule
	 \end{tabular}
	\caption{Past of passive \emph{gudal} `was raised' and \emph{hugdal} `was enlarged'\label{table:pass-vowels-past}}
\end{table}

\begin{table}
	\begin{tabular}{lllll}
	 \lsptoprule
	 & \multicolumn{2}{c}{\tpua~\root{gdl}}	& \multicolumn{2}{c}{\thuf~\root{gdl}}\\\cmidrule(lr){2-3}\cmidrule(lr){4-5}
	 & \gsc{SG} & \gsc{PL}	& \gsc{SG} & \gsc{PL}\\\midrule
	1 & j-e-g\textbf{u}d\textbf{a}l & n-e-g\textbf{u}d\textbf{a}l		& j-\textbf{u}gd\textbf{a}l & n-\textbf{u}gd\textbf{a}l\\
	2M & t-e-g\textbf{u}d\textbf{a}l & t-e-g\textbf{u}d\del{\textbf{a}}l-{u}	& t-\textbf{u}gd\textbf{a}l & t-\textbf{u}gd\del{\textbf{a}}el-{u}\\
	2F & t-e-g\textbf{u}d\del{\textbf{a}}l-{i} & t-e-g\textbf{u}d\del{\textbf{a}}l-{u}	& t-\textbf{u}gd\del{\textbf{a}}el-{i} & t-\textbf{u}gd\del{\textbf{a}}el-{u}\\
	3M & j-e-g\textbf{u}d\textbf{a}l & j-e-g\textbf{u}d\del{\textbf{a}}l-{u}	& j-\textbf{u}gd\textbf{a}l & j-\textbf{u}gd\del{\textbf{a}}el-{u}\\
	3F & t-e-g\textbf{u}d\textbf{a}l & j-e-g\textbf{u}d\del{\textbf{a}}l-{u}	& t-\textbf{u}gd\textbf{a}l & j-\textbf{u}gd\del{\textbf{a}}el-{u} \\
	\lspbottomrule
	 \end{tabular}
	\caption{Future of passive \emph{jegudal} `will be raised' and \emph{jugdal} `will be enlarged'\label{table:pass-vowels-fut}}
\end{table}

These are the last two verbal templates we will address as such, so here is a summary of their (identical) behavior.

% \hammer{
 \begin{exe}
 \ex  \label{ex:gen-pass}Generalizations about {\tpua} and {\thuf}
 \begin{xlist} 
 	\ex  \textit{Configurations:} Verbs appear in passive configurations only. 
 	\ex  \textit{Alternations:} Verbs in {\tpua} are always the passive version of an active verb in {\tpie}. Verbs in {\thuf} are always the passive version of an active verb in {\thif}. 
 \z
\z 
% }


	\subsection{The head Pass in Hebrew} \label{passn:pass:pass}
Following the argument for independent Pass in \cite{doron03} and \cite{alexiadoudoron12}, I have argued that Pass combines with VoiceP in fairly uninteresting ways in Hebrew, although some aspects of the results are informative \citep{kastnerzu17,kastner18nllt}. I summarize the findings here.

An existing VoiceP can be passivized by Pass. To make things precise, the structure for active \emph{hegdil} `enlarged' is given in~(\ref{ex:5n17}a) and for \isi{passive} \emph{hugdal} `was enlarged' in~(\ref{ex:5n17}b). 

 \begin{exe}
 \ex  \label{ex:5n17}
 \begin{xlist} 
 	\ex   
	\Tree
	[.TP
		[.DP ]
		[.
			[.T ]
			[.VoiceP
				[.\sout{DP} ]
				[.
					[.{\vd}\\\emph{he-,i} ]
					[
						[.v
							[.\root{gdl} ]
							[.v ]
						]
						[.DP ]
					]
				]
			]
		]
	]
 	\ex   
		\Tree
		[.TP
			[.DP ]
			[.
				[.T ]
				[.PassP				
					[.Pass\\{\emph{-u-}} ]
					[.
						[.{\vd}\\\emph{he-,a} ]
						[
							[.v
								[.\root{gdl} ]
								[.v ]
							]
							[.\sout{DP} ]
						]
					]
				]
			]
		]
 \z
\z 

These structures derive the syntactic and semantic generalizations, in that the \isi{passive} verbs are derived directly from active verbs in two specific templates.

\cite{kastner18nllt} shows in detail how this structural configuration also predicts the right allomorphic interactions. For the example above, {\vd} is structurally adjacent to T and so its stem vowels can be conditioned by the value of Tense or the phi-features on T. Such contextual conditioning of {\vd} is not possible once Pass intervenes, leading to the uniform \emph{u-a} pattern. The same holds for {\tpua}.

 \begin{exe}
\ex \label{tree:loc3} 
\Tree
    [.TP
        [.\tikz{\node (TAgr) {T+Agr};} ]
        [
	        [.\textbf{Pass}\\{\tikz{\node (Pass) {\textbf{\emph{u}}};}} ]
	        [
	            [.{\vd}\\{\tikz{\node (Voice) {\emph{he,a}};}} ]
	            [.vP ]
	         ]
	     ]
	 ]
    \begin{tikzpicture}[overlay]
    \draw[dotted,thick,->] (Pass) to[bend right] (Voice.west);
    \draw[dotted,thick,->] (TAgr) .. controls +(south west:3) and +(south west:2) .. node{\LARGE $\times$}(Voice);
    \draw[dotted,thick,->] (Voice) .. controls +(south:1) and +(south:2) .. node{\LARGE $\times$}(TAgr);
    \end{tikzpicture}
\bigskip \bigskip

 \ex \label{ex:pass-vi} 
 \begin{xlist} 
 	\ex  \root{gdl} \lra~\emph{gdl} 
 	\ex  {\vd} \lra 
	$\begin{cases}
	\text{\emph{he,a}} & / \text{Pass \trace}\\
	\text{\emph{he,i}} & / \text{\trace}\\
	\end{cases}$
 	\ex  Pass \lra~[+high +round]$_{\text{Pass}}$ 
 \z

 \ex  
 \begin{xlist} 
 	\ex  Cycle 1 (VoiceP): /he,a/ + /gdl/ $\Rightarrow$ hegdal 
 	\ex  Cycle 2 (PassP): /u/ + hegdal $\Rightarrow$ hugdal 
 \z
\z 

Consider next why it is not possible to posit an additional, \isi{passive} variant of Voice for Hebrew (an additional non-active Voice head). If {\thif} is derived using \vd, as assumed, then a transparent passivization cannot be accomplished by changing the Voice head: we would end up with an entirely different construction, one that loses all connection (semantic and phonological) to \vd/{\thif}. I conclude that \isi{passive} verbs really are derived by use of a Pass head above Voice and below T, and that combining the Pass analysis of passives with the system presented in this book correctly predicts the syntactic, semantic and phonological behavior of \isi{passive} verbs in Hebrew.
	
What remains to be discussed is the combinatorics of Pass with the different VoicePs. The combinations in Table~\ref{tab:5-1:comb} should be considered.

\begin{table}
\begin{tabular}{llllc} 
 \lsptoprule
	& & & & Attested? \\\midrule
	a.& Pass	&	Voice	& 		& \xmark \\
	b.& Pass	& Voice		& \va	& {\tpua} \\
	c.& Pass	& {\vz}		& 		& \xmark\\
	d.& Pass	& {\vz}		&	\va	& \xmark\\
	e.& Pass	& {\vd}		&		& {\thuf} \\
\lspbottomrule
 	\end{tabular}
	\caption{Combinations of Pass and VoiceP\label{tab:5-1:comb}}
\end{table}

There is no overt morphological evidence for Pass combining with \isi{Unspecified Voice}. As far as I can tell, this is a historical accident: classical Hebrew had a template encoding ``the \isi{passive} of {\tkal}'', i.e. [Pass VoiceP]. \citet[120]{kastner16phd} speculates that in Modern Hebrew, Pass can only combine with structures which ``guarantee'' an external argument; these are rows~b and~e of Table~\ref{tab:5-1:comb}, but not the others.

It could also be suggested that what I have called the \isi{passive} alloseme of {\vz} is in fact the spell-out of [Pass Voice]. But \cite{ahdoutkastner19nels} marshal a number of arguments against this possibility. First, Pass-passives (in {\tpua} and {\thif}) cannot ever undergo nominalization\is{nominalizations} or form infinitives and imperatives \citep{kastnerzu17}, but {\vz}-passives can. And second, Pass has the predictable morphological properties mentioned above, while {\vz} is morphologically unrelated to other forms.\label{r1:5:4}


\section{Adjectival passives} \label{passn:adjpass}

	\subsection{Background}\largerpage[2]

The distinction between \textsc{verbal passives} and \textsc{adjectival passives} is well-known in the literature, although accounts differ on where the line should be drawn \citep{wasow77,levinrappaport86,borerwexler87,embick04li,alexiadouetal14,layering15,bruening14nllt}. However diagnosed and analyzed, verbal passives are taken to be part of the verbal system and adjectival passives to pattern distributionally with adjectives.

What I wish to highlight is the place of Voice in adjectival passives, a point for which we will need a bit more background on the different readings associated with these constructions. It has by now become standard to assume that adjectival passives which entail prior events are compatible with at least some agents of these events. The main insights are as follows.

Adjectives can be distinguished according to whether they describe a stative characteristic of an entity or a state that has come about as the result of some previous event; this is the stative/resultative distinction from \cite{embick04li}, who presented the following diagnostics to distinguish between stative \emph{open} and resultative \emph{opened} by way of example.

 \begin{exe}
 \ex  Event-oriented adverbs: resultatives only for agent-oriented adverbs as in (a), disambiguated readings for other adverbs as in (b). 
	 \begin{xlist} 
 	\ex  \emph{The package remained \glemphu{carefully} \xmark open/\cmark opened.} 
 	\ex  \emph{The \glemphu{recently} open door.} [it was open recently] \\
		\emph{The \glemphu{recently} opened door.} [ambiguous: door was open recently or door was being opened recently]
	 \z

\ex  Verbs of creation (statives only). \\
	\emph{The door was \{\glemphu{built}/\glemphu{created}/\glemphu{made}\} \cmark open/\xmark opened}.

\ex  Secondary predicates (statives only). \\
	\emph{Mary \glemphu{kicked} the door \cmark open/ \xmark opened}.

\ex  Prefixation of -\emph{un} (mostly resultatives). \\
	\xmark \emph{\glemphu{un}open} / \cmark \emph{\glemphu{un}opened}
 \z 

In some cases the morphology indicates whether a certain form is stative or resultative: \emph{open} and \emph{molten} are stative, whereas \emph{opened} and \emph{melted} are resultative. In many cases, however, the form is ambiguous: \emph{closed}, \emph{fractured} and so on. The tests above distinguish ``simple'' adjectives from adjectives embedding an event. In \citeauthor{embick04li}'s analysis, the former are derived by adjectivizing a root, and the latter by adjectivizing an event (vP/VoiceP). \citeauthor{embick04li}'s resultatives thus fold in both ``target state'' and ``result state'' adjectival passives, a semantic distinction which can diagnosed by whether the adjectival \isi{passive} can be modified by `still' \citep{kratzer00bls,alexiadouetal14}.

Work since has investigated the kind of modifiers that can be attached to an adjectival \isi{passive} \citep{meltzerasscher11,mcintyre13,alexiadouetal14,bruening14nllt,gehrkemarco14}. At least the following constructions are available for (resultative) adjectival passives in English, \ili{German}, Hebrew and \ili{Spanish}. \isi{Agent} implication is not possible:

 \begin{exe}
 \ex  
 \begin{xlist} 
 	\ex[]{ The door is \glemph{opened}, but no one has opened it. }
 	\ex[*] { \langinfo{German}{}{\citealt[124]{alexiadouetal14}}\\    
	\gll Die M\"unze ist schon lange \glemph{versunken} *aber keiner hat sie je versenkt\\
 		  the coin is already long sunk.Adj but nobody has she ever sunk.\gsc{PASS.PTCP}\\
 		\glt `The coin has been sunk for a while, but nobody has sunk it.'  } 	
 \z
\z 

\emph{By}-phrases are possible only if their modification of the agent, and therefore of the event, is discernible by examining the end state. One can tell that an editor did good work but not that the editor was bored:

 \begin{exe}
\ex  
[] 		{ \gll ha-sefer \glemph{arux} {al-jedej} orex {\cmark}metsujan / {\xmark}meʃoamam.\\
 		  the-book edited.Adj by editor \phantom{\cmark}excellent {} \phantom{\xmark}bored\\
 		\glt `The book was edited by an excellent/*bored editor.'  \hfill \citep[823]{meltzerasscher11} } 
	
 \z 

Similarly, instruments are possible only if their modification of the event can be discerned by examining the end state. The writing of a blue pencil is distinguished from that of other pencils but the writing of a pretty pencil is not (though cf.~\citealt{mcintyre13} and \citealt{bruening14nllt}):

 \begin{exe}
\ex  
[] 		{ \gll ha-mixtav \glemph{katuv} be-iparon {\cmark}kaxol / {\xmark}jafe.\\
 		  the-letter written.Adj in-pencil \phantom{\cmark}blue {} \phantom{\xmark}pretty\\
 		\glt `The letter was written with a blue/*pretty pencil.'  \hfill (\citealt[825]{meltzerasscher11}, attributed to Julia Horvath) } 
	
 \z 

The exact syntactic structure is a matter discussed from a crosslinguistic perspective by \cite{alexiadouetal14,layering15} based on fine differences between English and Greek. What I take from their discussion and the existing literature are the basic structures in~(\ref{ex:5n28}), which are intended to be uncontroversial:

 \begin{exe}
 \ex  \label{ex:5n28}
 \begin{xlist} 
 	\ex  Adjective (stative): {[}\root{Root} a] 
 	\ex  Adjectival passive (resultative): {[}[a [Voice [\root{Root} v]]] 
 \z
\z 

I will not commit to a specific semantics for adjectives or adjectival passives, on any reading, but one could begin from the semantics of a resultant state adjective proposed by \cite{kratzer00bls}:
 \begin{exe}
\ex  \denote{Adj} = λRλt∃e,y.R(e)(y) \& $\tau$(e) $\le$ t 
 \z 

	\subsection{Descriptive generalizations} \label{passn:adjpass:mpua}
This section goes through a few of the established diagnostics in Hebrew. Adjectival passives appear in one of the two \isi{passive} participial forms {\mpua} and {\mhuf} (participles of {\tpua} and {\thuf} respectively), or in the \emph{XaYuZ} form associated with \tkal.

Hebrew participles serve as present tense verbal forms and as Romance-style participles, by which I mean a mixed nominal-adjectival category. The Hebrew participle is, in general, ambiguous in form between a verb and an adjective or noun \citep{boneh13tense,doron13ehll}. In {\tkal} the active participle can be either a verb or a noun. In other templates (and in the \emph{XaYuZ}~\isi{passive} participle) an adjectival reading is also available, as with \emph{metsujan} `excellent' in~(\ref{ex:5n30}b).

 \begin{exe}
 \ex  \label{ex:5n30}
 \begin{xlist} 
 	\ex   
[] 		{ \gll ha-ʃelet \glemph{more} al ha-derex la-park.\\
 		  the-sign indicates.\gsc{PTCP.SMPL} on the-road to.the-park\\
 		\glt `The sign is indicating the way to the park.' } 
	
	
 	\ex   
[] 		{ \gll josi \glemph{more} \glemphu{metsujan}.\\
 		  Yossi teacher.\gsc{PTCP.SMPL} excellent.\gsc{INTNS.PASS.Prs}\\
 		\glt `Yossi is an excellent teacher.' } 
	
 \z
\z 

The forms {\mpua} and {\mhuf} are ambiguous between a verbal form and an adjectival form, just like English \emph{closed} is. \cite{doron00} establishes ten diagnostics distinguishing verbal passives from adjectival passives (in fact, many of them distinguish verbs from adjectives in general). Here I give a few examples of what these differences look like. Importantly, only bounded events (change-of-state and inchoatives) can serve as input to adjectival passives (which are resultative).

In active forms, the finite verb often contrasts with a combination of copula and participle. Consider future verbs~(\ref{ex:pres-act}a) and future participles~(\ref{ex:pres-act}b).

 \begin{exe}\judgewidth{\#}
 \ex  \label{ex:pres-act} 
 \begin{xlist} 
  \ex[]{  Future verb: \\
	\gll maxar ani \{\glemph{oxal} \emph{/} \glemph{aklit}\}.\\
        tomorrow I will.eat.\gsc{SMPL} {} will.record.\gsc{CAUS}\\
      \glt `Tomorrow I'll eat/record something.' } 
 
  \ex[\#]{ Future copula with a participle: \\
	\gll maxar ani \glemphu{eheje} \{\glemph{ox\'el} \emph{/} \glemph{maklit}\}.\\
        tomorrow I will.be.\gsc{SMPL} eat.\gsc{SMPL.PRS} {} record.\gsc{SMPL.CAUS}\\
      \glt (int. `Tomorrow I will be eating/recording.') } 
 
 \z
\z 
\cite{doron00} shows that \textit{verbs are not allowed after a copula}, so the forms in (\ref{ex:pres-act}b) must be adjectives or nominals. They can be used when the participle is used in a generic context as a noun, as in ``eater of vermin'' (\ref{ex:pres-act2}a) or ``recorder of things'' (\ref{ex:pres-act2}b). This is to be expected if the complement of the copula in~(\ref{ex:pres-act2}) is a participle.

 \begin{exe}
 \ex  \label{ex:pres-act2} 
 \begin{xlist} 
  \ex  Participle of {\tkal}: \\
	{ \gll az tagidi, ʃe-rak ani \glemphu{eheje} \glemph{ox\'el} ʃratsim ve-ʃ'ar mini basar ha-'asurin al jehudim? ;-)\\
        so say.\gsc{2SG.F.FUT} \gsc{COMP}-only I will.be eat.\gsc{SMPL.PRS} vermin and-rest kinds.\gsc{CS} meat the-proscribed on Jews\\
      \glt `So say so! What, you want me to be the only one here who eats vermin and other kinds of meat that are proscribed for Jews? ;-)'\footnotemark } 
 
\footnotetext{\url{http://www.tapuz.co.il/forums2008/archive.aspx?ForumId=1277&MessageId=96791273} (retrieved November 2014). The example appears in a forum conversation in which participants discuss their experiences eating shrimp in Norway. \emph{ʃratsim} `vermin' is a common term for non-Kosher foods such as seafood. The adjective \emph{asurin} `proscribed' is written in an intentionally jocular/archaic way, with a final -\emph{n} that has changed to -\emph{m} in the modern language.}

  \ex  Participle of {\thif}: \\
 	{ \gll kanir'e ʃe-ani \glemphu{eheje} \glemph{maklit} kavua ʃel ze.\\
        probably \gsc{COMP}-I will.be record.\gsc{CAUS.Prs} constant of this\\
      \glt `Looks like I'll be the one recording this', `Looks like I'll be a constant recorder of this.' \url{http://www.forumtvnetil.com/index.php?showtopic=18312} } 
 
 \z
\z 

It is thus possible to distinguish verbal passives from adjectival passives in Hebrew, and to tease apart different readings of the participle. Recall that for English, \cite{embick04li} demonstrated that if the door is \emph{closed}, it could have been built closed (adjectival \isi{passive}, stative) or been closed from an open state (verbal \isi{passive}, eventive). The same logic holds for verbs like \emph{record} and \emph{cover} in Hebrew. The implied present tense in~(\ref{ex:pres-ambig}a) is ambiguous between a verbal (progressive) reading and an adjectival (stative) reading. However, in Hebrew the future copula diagnoses an adjectival \isi{passive} form \citep{doron00}. Accordingly, the future tense in~(\ref{ex:pres-ambig}b) is unambiguously adjectival \citep{doron00,horvathsiloni08,meltzerasscher11}.

 \begin{exe}
 \ex  \label{ex:pres-ambig} 
 \begin{xlist} 
     \ex   
[]         { \gll ha-kontsert \glemph{muklat}.\\
           the-concert record.\gsc{CAUS.PASS.PRS}\\
         \glt `The concert is being recorded.'\\`The concert has been recorded.' } 
    
        
     \ex   
[]         { \gll ha-kontsert \glemph{jihie} \glemphu{muklat}.\\
           the-concert will.be.\gsc{SMPL} record.\gsc{CAUS.PASS.PRS}\\
         \glt `The concert will have (already) been recorded.' } 
    
 \z

 \ex  \label{ex:pres-ambig2} 
 \begin{xlist} 
     \ex   
[]         { \gll ha-sir \glemph{mexuse}.\\
           the-pot cover.\gsc{INTNS.PASS.PRS}\\
         \glt `Someone is covering the pot.' (verbal)\\`The pot is covered.' (adjectival) } 
    
        
     \ex   
[]         { \gll ha-sir \glemph{jihie} \glemphu{mexuse}.\\
           the-pot will.be.\gsc{SMPL} cover.\gsc{INTNS.PASS.PRS}\\
         \glt `The pot will be covered.' (adjectival only) } 
    
 \z
\z 

Two additional differences between verbal and adjectival passives have been mentioned in the literature \citep{horvathsiloni08,horvathsiloni09,meltzerasscher11,kastnerzu17}. First, whereas the adjectival forms may have an \textit{idiomatic reading}~(\ref{ex:idiom}a), \isi{passive} verbs~(\ref{ex:idiom}b) are always compositional.

 \begin{exe}\judgewidth{\#}
 \ex  \label{ex:idiom} 
 \begin{xlist} 
     \ex    
[]         { \gll ze \glemph{jihie} \glemphu{muvan} \glemphu{me-elav}.\\
           this will.be.\gsc{SMPL} understand.\gsc{CAUS}.\gsc{PASS.PRS} from-to.him\\
         \glt `It will be self-evident.' } 
       
     \ex    
[\#]         { \gll ze \glemph{juvan} \glemphu{me-elav}\\
           this understand.\gsc{CAUS}.\gsc{PASS}.Fut from-to.him\\
         \glt (no immediate clear meaning) } 
        
 \z
\z 

Passive participles, being adjectival passives, can take on idiomatic readings regardless of their template. The \isi{passive} participle of ``simple'' \emph{matsats} `sucked' can have an idiomatic reading~(\ref{ex:5n36}a), but \isi{mediopassive} ``middle'' \emph{nimtsats} is understood literally~(\ref{ex:5n36}b).

 \begin{exe}
 \ex  \label{ex:5n36}
 \begin{xlist} 
     \ex  
[]         { \gll ze \glemph{haja} \glemphu{matsuts} \glemphu{me-ha-etsba}.\\
           this was.\gsc{SMPL} sucked.\gsc{SMPL} from-the-finger\\
         \glt `It was entirely made up.' } 
        
    
     \ex  
[]         { \gll ze \glemph{nimtsats} \glemphu{me-ha-etsba}.\\
           this sucked.\gsc{MID} from-the-finger\\
         \glt `This was sucked from the finger.' (no idiomatic reading) } 
        
 \z
\z 

Second, synthetic passives force \textit{disjoint readings} in which the external argument and the internal argument cannot refer to the same entity \citep{bakeretal89}. The adjectival form~(\ref{ex:disjoint}a), with the participle, allows coreference whereas the verbal form~(\ref{ex:disjoint}b) does not \citep[720]{sichel09}:

 \begin{exe}
 \ex  \label{ex:disjoint} 
 \begin{xlist} 
     \ex   
[]         { \gll ha-jalda \glemph{hajta} \glemphu{mesorek-et}.\\
         the-girl was comb.\gsc{INTNS.PASS.PRS}-\gsc{F}\\ \jambox{(agent =/$\neq$ theme)}
         \glt `The girl was combed.' } 
        
     \ex  
[]         { \gll ha-jalda \glemph{sork-a}.\\
         the-girl comb.\gsc{INTNS}.\gsc{PASS}.Past-\gsc{F}\\ \jambox{(agent $\neq$ theme)}
         \glt `The girl got combed.' } 
        
 \z
\z 

And finally, there is clear reason to think that the split between adjectival passives and verbal passives really is the result of a difference between verbs and adjectives. The Hebrew direct object marker \emph{et} is licensed\is{licensing} by verbs, (\ref{ex:5n38}a), but it never appears in analytic forms in Hebrew when they have a stative reading, (\ref{ex:5n38}b), shown here with active forms (which license\is{licensing} Accusative). 

 \begin{exe}\judgewidth{??}
 \ex  \label{ex:5n38}
 \begin{xlist} 
 	\ex   
[] 		{ \gll ha-arafel texef \glemph{jexase} et kol ha-rexov.\\
 		  the-fog soon will.cover.\gsc{INTNS} \gsc{ACC} all the-street\\
 		\glt `The fog is about to cover the entire street.' } 
			
 	\ex   
[??] 		{ \gll ha-arafel ha-kaved \glemph{jihie} \glemphu{mexase} et kol ha-rexov\\
 		  the-fog the-heavy will.be.\gsc{SMPL} cover.\gsc{INTNS.PRS} \gsc{ACC} all the-street\\
 		\glt (int.~`The heavy fog will be covering the entire street) } 
		
 \z
\z 
\cite{horvathsiloni08} give additional reasons for assigning the two forms to these two lexical categories.

The picture for Hebrew is thus fairly similar to that in the Romance and Germanic languages discussed in the literature. Where Hebrew differs is in the differences between templates, which I will get to in Section~\ref{passn:adjpass:a:temp}.

	\subsection{The adjectivizer \emph{a} in Hebrew} \label{passn:adjpass:a}
Within DM, it has become standard to assume that simple (stative) adjectives are derived by merging an adjectivizing \emph{a} head with the root. I assume that the same head derives all kinds of adjectives, be they stative or \isi{passive} -- the only thing that matters is the structure embedded under this head. But this means that I need to first say a few words about the morphology of adjectives in Hebrew. What I will end up postulating is phonologically different \emph{a} heads for stative adjectives, whereas adjectival passives are the result of merging the \emph{a} head with a VoiceP. The same kind of story will be proposed for \isi{nominalizations} in Section~\ref{passn:n:n}.

		\subsubsection{Stative adjectives}
Stative (ordinary) adjectives have no event implications or internal structure. I assume that in Hebrew, like in most baseline analyses in other languages, adjectives are derived by merging an adjectivizing head (here little \textit{a}) with the root \citep{embick04li}:

 \begin{exe}
\ex \label{ex:adj-en} 
	\begin{minipage}[t]{0.3\textwidth}
		a. \emph{open}\\
		\Tree
			[.a
				[.{\root{\gsc{open}}} ]
				[.a ]
			]
	\end{minipage}
	\begin{minipage}[t]{0.5\textwidth}
		b. \emph{closed} (stative reading)\\
		\Tree
			[.a
				[.{\root{\gsc{close}}} ]
				[.a\\\emph{-ed} ]
			]
	\end{minipage}
 \z 

Adjectives can appear in various morphophonological patterns, each listed as a possible exponent of little \textit{a}.\footnote{As noted in Chapter~\ref{chap:intro}, I use the term \emph{pattern} when referring to one of the morphophonological forms in the adjectival or nominal domains. There are, in principle, an unlimited number of distinct patterns, but only seven verbal \emph{templates}.}

 \begin{exe}
\ex  \label{ex:5n40}
	\begin{minipage}[t]{0.32\textwidth}
		a. \emph{barur} `clear' (\emph{XaYuZ})\\
			\Tree
			[.a
				[.{\root{brr}} ]
				[.a$_{\text{XaYuZ}}$ ]
			]
	\end{minipage}
	\begin{minipage}[t]{0.32\textwidth}
		b. \emph{katan} `small' (\emph{XaYaZ})\\
			\Tree
			[.a
				[.{\root{\dgs{k}tn}} ]
				[.a$_{\text{XaYaZ}}$ ]
			]
	\end{minipage}
	\begin{minipage}[t]{0.32\textwidth}
		c. \emph{ʃamen} `fat' (\emph{XaYeZ})\\
			\Tree
			[.a
				[.{\root{ʃmn}} ]
				[.a$_{\text{XaYeZ}}$ ]
			]
	\end{minipage}
 \z 

Two of these patterns are homopohonous with the present-tense (participle) verbal passives {\mpua} and {\mhuf}. Therefore, I am forced to assume the existence of two separate adjectival heads, namely a$_{\text{\gsc{INTNS}}}$ and a$_{\text{\gsc{CAUS}}}$, alongside any other possible patterns, just like English shows evidence of adjectival -\emph{ed} (\emph{wingéd}, \emph{learnéd}) alongside verbal -\emph{ed}. A given root typically has one basic adjectival form like the ones in~(\ref{ex:5n40}). So an adjective might appear in this form or in the participial-like forms, with either subtle~(\ref{ex:adj-tpie}a--b) or substantial~(\ref{ex:adj-thif}a--b) differences in meaning.

 \begin{exe}
\ex \label{ex:adj-tpie} 
	\begin{minipage}[t]{0.32\textwidth}
		a. \emph{kaur} `ugly' (\emph{XaYuZ})\\
			\Tree
			[.a
				[.{\root{k'r}} ]
				[.a$_{\text{XaYuZ}}$ ]
			]
	\end{minipage}
	\begin{minipage}[t]{0.32\textwidth}
		b. \emph{mexoar} `ugly'\\
			\Tree
			[.a$_{\text{\gsc{INTNS}}}$
				[.{\root{k'r}} ]
				[.a$_{\text{\gsc{INTNS}}}$ ]
			]
	\end{minipage}
 \z 

 \begin{exe}
\ex \label{ex:adj-thif} 
	\begin{minipage}[t]{0.32\textwidth}
		a. \emph{parua} `wild' (\emph{XaYuZ})\\
			\Tree
			[.a
				[.{\root{pr'}} ]
				[.a$_{\text{XaYuZ}}$ ]
			]
	\end{minipage}
	\begin{minipage}[t]{0.32\textwidth}
		b. \emph{mufra} `deranged'\\
			\Tree
			[.a$_{\text{\gsc{CAUS}}}$
				[.{\root{pr'}} ]
				[.a$_{\text{\gsc{CAUS}}}$ ]
			]
	\end{minipage}
 \z 

An alternative would be to assume that even these stative adjectives have underlying verbal structure, except that this structure is not interpreted. This approach is reminiscent of the inchoatives and the Greek facts mentioned in Section~\ref{vz:vz:sem}. Perhaps in the Hebrew cases above there is only one adjectivizing head \emph{a}, which takes a verbal structure that is not interpreted. I do not have particular reason to support one view or the other, and so I stick to the analyses in~(\ref{ex:adj-tpie}--\ref{ex:adj-thif}) simply because they involve less structure. The same point can be made for complex event nominals in the next section. Note, however, that this alternative should then extend to English cases such as (\ref{ex:adj-en}b): what is to stop us from assuming underlying verbal structure in \emph{closed} which is simply not interpreted before being adjectivized by -\emph{ed}?

		\subsubsection{Adjectival passives}
The difference between stative adjectives and adjectival passives is that the latter embed VoiceP. The internal argument of adjectival passives has been argued by \citet[386]{bruening14nllt} to be an Operator, bound by the noun interpreted as the argument. Implementing this for Hebrew gives us structures like the following (where the exact nature of the copula is irrelevant). Like with stem vowels in verbs, I assume that the stem vowels originate on Voice and are conditioned by the embedding \emph{a} head.

 \begin{exe}
 \ex \label{ex:adjpass-heb1-tree}Adjectival passive in \emph{XaYuZ} (from {\tkal}): 
 \begin{xlist} 
     \ex   
          \gll ha-sefer \glemph{jihie} \glemphu{katuv} be-et kaxol.\\
           the-book will.be.\gsc{SMPL} written in-pen blue\\
         \glt `The book will be (will have been) written in blue ink.'  
    
     \ex \resizebox{\linewidth}{!}{%
    	\Tree
        [.TP
            [.{DP$_i$}\\\emph{ha-sefer}\\\emph{the book} ]
            [
                [.T$_{\textrm{[Fut]}}$\\\emph{ji-} ]
                [.
                	[.Voice ]	
	                [.vP
	                    [.v\\\emph{-hie} ]
	                    [.aP
	                        [.\phantom{xxxx}{Op$_i$}\phantom{xxxx} ]
	                        [.aP
	                            [.a ]
	                            [.VoiceP
		                            [.VoiceP
		                                [.Voice\\\emph{a,u} ]
		                                [.vP
			                                [.v
			                                    [.{\root{ktv}} ]
			                                    [.v ]
											]
		                                    [.\sout{Op$_i$} ]
		                                ]
		                            ]
		                            [.{pP\\\emph{be-et kaxol}\\\emph{in blue ink}} ]
		                        ]
	                        ]
	                    ]
	                ]
	            ]
            ]
        ]
  }
 \z\bigskip

 \ex \label{ex:adjpass-heb2-tree}Adjectival passive in {\mpua} (from {\tpie}): 
 \begin{xlist} 
     \ex   
         \gll dani \glemph{jihie} \glemphu{mesorak}.\\
           Danny will.be.\gsc{SMPL} combed.\gsc{INTNS.PASS.PRS}\\
         \glt `Danny will be combed (already).' 
    
     \pagebreak \ex  \resizebox{\linewidth}{!}{%
     \Tree 
        [.TP
            [.{DP$_i$}\\\emph{dani} ]
            [
                [.T$_{\textrm{[Fut]}}$\\\emph{ji-} ]
                [.
                	[.Voice ]
	                [.vP
	                    [.v\\\emph{hie} ]
	                    [.DP$_i$
		                    [.D ]
		                    [.NP
			                    [.N\\\sout{\emph{dani}} ]
		                        [.aP
		                            [.a ]
		                            [.VoiceP
		                                [.Voice\\\emph{me-,o,a} ]
		                                [.vP
		                                	[.{\va} ]
			                                [.vP
			                                    [.v
			                                        [.{\root{sr\dgs{k}}} ]
			                                        [.v ]
			                                    ]
			                                    [.{Op$_i$} ]
			                                ]
			                            ]
		                            ]
		                        ]
		                    ]
	                    ]
	                ]
            	]
            ]
        ]
    }
 \z
\z 

The derivations in this section are similar to the ones in \cite{doron14adj}, though I depart from her specific implementation for a number of reasons. First -- as already discussed in previous chapters -- the functional heads used by \citeauthor{doron14adj} are semantic primitives which drive the semantics but do not translate straightforwardly into the morphophonology as syntactic heads usually do, nor is their exact syntactic job clear. Additionally, and more specifically to adjectival passives, \cite{doron14adj} utilizes an active Voice head introducing the EA-related head v, which in turn introduces the external argument. In order to produce a verb in active voice, then, her system needs a lower head that requires Active Voice -- this is \gsc{CAUS} -- so that \gsc{CAUS} introduces Active Voice, Active Voice introduces v, and v introduces the external argument. Some of these heads split up the semantic work that can be done by one head (Voice and v in particular), and not all of them have overt spell-out. There are consequently more syntactic elements than seems necessary. 


\subsubsection{Templates} \label{passn:adjpass:a:temp}
In terms of the combinatorics involved with different VoicePs in Hebrew, the picture is similar to that of verbal passives except that row~a in Table~\ref{tab:5-2:comb} is possible.
\begin{table}
\begin{tabular}{llllc} 
 \lsptoprule
	& & & & Attested? \\\midrule
	a.& a	&	Voice	& 		& \emph{XaYuZ} \\
	b.& a	& Voice		& \va	& {\mpua} \\
	c.& a	& {\vz}		& 		& \xmark\\
	d.& a	& {\vz}		&	\va	& \xmark\\
	e.& a	& {\vd}		&		& {\mhuf} \\
\lspbottomrule
 	\end{tabular}
	\caption{Combinations of little \emph{a} and VoiceP\label{tab:5-2:comb}}
\end{table}

{\vz} is incompatible with adjectival passives. Informally, adjectival passives denote the result of an event without explicitly naming the cause\is{Causer}, though one is assumed; in this sense they are similar to verbal passives. \cite{alexiadouetal14} and \cite{bruening14nllt} implement this by allowing Adj (and Pass) to only select for Voice that needs to fill its specifier. {\vz} is not such a Voice head (although \citealt{embick04li} does allow his non-active Voice to derive unaccusative adjectival passives in English): since there is no expectation of an external argument, there is no adjectival passive.

What remains is to see what special interactions arise from the combination of other [D] values (or {\va}) with little \textit{a}. First, recall the claim in \cite{doron00} that change-of-state roots are better inputs to adjectival passives than atelic events. All three templatic forms are compatible with both stative adjectives and adjectival passives, as already mentioned. \citet[170]{doron14adj} shows that \textsc{stative adjectives} are incompatible with event modifications or event readings. Some of them even have no corresponding underlying verb:

 \begin{exe}
 \ex  
 \begin{xlist} 
 	\ex   
[] 		{ \gll ti'un \glemphu{barur} (*bekfida)\\
 		  argument clear \phantom{(*}carefully\\
 		\glt `A clear argument' } 
	

 	\ex   
[] 		{ \gll beged \glemphu{mexoar} (*beriʃul)\\
 		  garment ugly.\gsc{INTNS} \phantom{(*}carelessly\\
 		\glt `An ugly garment' } 
	

 	\ex   
[] 		{ \gll pirxax \glemphu{mufra} (*bexipazon)\\
 		  brat deranged.\gsc{CAUS} \phantom{(*}hastily\\
 		\glt `A deranged brat' } 
	
 \z
\z 

And while all three \textit{adjectival \isi{passive}} forms are \emph{compatible} with external arguments, \citet[175]{doron14adj} observes that (resultative) adjectival passives in ``\isi{causative}'' {\mhuf} \emph{require} an implied \isi{Causer} to be interpreted, even if it is implicit and not overtly represented. So, while an adjectival \isi{passive} in {\mpua} does not entail the existence of a \isi{Causer}, (\ref{ex:sportaim}a), every adjectival \isi{passive} in {\mhuf} does, (\ref{ex:sportaim}b). In a telling near-minimal pair, the athletes in~(\ref{ex:sportaim}a) might have trained on their own, but the athletes in~(\ref{ex:sportaim}b) must have been trained through some kind of organized program.

 \begin{exe}
 \ex \label{ex:sportaim} 
 \begin{xlist} 
 	\ex   
[] 		{ \gll sportaim \glemphu{meuman-im} bekfida\\
		 athletes trained.\gsc{INTNS.PASS}-\gsc{PL} carefully\\ \jambox{(\mpua)}
 		\glt `Carefully trained athletes' } 
	
	
 	\ex   
[] 		{ \gll sportaim \glemphu{muxʃar-im} bekfida\\
		 athletes prepared.\gsc{CAUS.PASS}-\gsc{PL} carefully\\ \jambox{(\mhuf)}
 		\glt `Carefully trained athletes' } 
	
 \z
\z 

\cite{doron14adj} attributes this difference to the behavior of the causative head \gsc{CAUS} which for her underlies {\thif}. My analysis, using {\vd}, follows in the same vein. Note that the implied EA is not syntactically represented; it cannot, for example, create a new discourse referent.

 \begin{exe}
\ex[*] { \label{ex:sportait}  
	 \gll nadia komanetʃi \glemph{haj-ta} sportait \glemphu{(EA$_i$)} \glemph{muxʃer-et} bekfida. \glemphu{hu$_i$} asa avoda tova aval safag harbe bikoret\\
 		  Nadia Com\u{a}neci was.\gsc{SMPL}-\gsc{F} athlete.\gsc{F} {} prepared.\gsc{CAUS.PASS}-\gsc{F} carefully he did.\gsc{SMPL} job good but absorbed.\gsc{SMPL} much criticism\\
 		\glt (int. `Nadia Com\u{a}neci was a carefully trained athlete. He (=B\'ela K\'arolyi) did a good job but was heavily criticized.') } 
 \z 

I conclude with additional observations by template, collected here for completeness, and drawing heavily on \cite{doron00} as well as \cite{doron14adj} and \cite{meltzerasscher11}.

\subsubsubsection{\tkal~(adjectival form \emph{XaYuZ})} 

No verbal passive exists for {\tkal} but stative and resultative adjectives are both possible.

\cite{doron00} argues that only change of state roots are possible input to adjectives in this template. For example, the unattested form *\emph{karu}/*\emph{karuj} (int. `read') does not exist as a stative adjective or as an adjectival passive:

 \begin{exe}
\ex  
[]     { \gll ha-mixtav \glemphu{katuv} \emph{/} *\glemphu{karuj}.\\
       the-letter written {} read\\
     \glt `The letter is written (*is read).' } 
  
 \z 

For those roots that can form adjectives, the main difference is between roots that derive intransitive verbs in \tkal~and those that derive \isi{transitive} verbs. The former lead to stative adjectives and the latter to adjectival passives (see \citealt{meltzerasscher11} for a lexicalist account).

 \begin{exe}
 \ex   \label{ex:5n49}
 \begin{xlist} 
   \ex  Stative adjectives from intransitives: \emph{kafu} `frozen' $<$ \emph{kafa} `froze'; \emph{davuk} `glued' $<$ \emph{davak} `stuck to'.  
   \ex  Adjectival passives are possible with change of state roots: \emph{ʃavur} `broken' $<$ \emph{ʃavar} `broke'; \emph{sagur} `closed' $<$ \emph{sagar} `closed'; \emph{saruf} `burnt' $<$ \emph{saraf} `burned'.  
   \ex  Stative adjectives with no corresponding verb in \tkal: \emph{paʃut} `simple', \emph{savux} `complex', \emph{pazur} `scatterd', \emph{ʃaluv} `intertwined', \emph{akum} `crooked', \emph{tarud} `preoccupied'. 
   \z
\z 
The roots underlying~(\ref{ex:5n49}c) do not appear as verbs in \tkal, meaning that they cannot combine with v and Voice. If this is the case, they cannot form the underlying VoiceP necessary for an adjectival \isi{passive} and are only possible as input to stative adjectives. For the roots in~(\ref{ex:5n49}a), their corresponding \tkal~verbs are intransitive. This means that the interpretation of [Voice [v \root{db\dgs{k}}]], for example, is unaccusative. If this is the case, then an implicit \isi{Agent} cannot be licensed\is{licensing}.


\subsubsubsection{\tpie~(adjectival form \mpua)}

This template can serve as input to both verbal and adjectival passives. \cite{lakscohen16} argue (and provide experimental evidence for the claim) that the middle stem vowel might be pronounced differently for verbs and adjectives, further supporting the split between the two (one that can be encoded regardless of theoretical framework).

Among the adjectives, there are two kinds of stative adjectives: those that do not have a corresponding verb, (\ref{ex:5n50}a), and those that are homophonous with an adjectival \isi{passive} like English \emph{closed} is, as it can be stative or resultative, (\ref{ex:5n50}b).

 \begin{exe}
 \ex  \label{ex:5n50}
 \begin{xlist} 
   \ex  No corresponding verb: \emph{meguʃam} `clumsy' ($\nless$ *\emph{giʃem}), \emph{meunax} `vertical' ($\nless$ *\emph{inex}), \emph{memuʃma} `disciplined' ($\nless$ *\emph{miʃmea}), \emph{metupaʃ} `silly' ($\nless$ *\emph{tipeʃ}). 
   \ex \sloppy Ambiguous between resultative and stative: \emph{megune} `obscene', \emph{mekubal} `accepted', \emph{mefuzar} `scattered', \emph{meluxlax} `dirty', \emph{megulgal} `rolled up', \emph{mekulkal} `out of order'.  
 \z
\z 


The verbs underlying~(\ref{ex:5n50}b), and any which do not fall under~(\ref{ex:5n50}a), can form adjectival passives. For the forms in~(\ref{ex:5n50}b), the stative reading is more salient and is often different than the compositional adjectival \isi{passive} reading. For instance, the adjectival \isi{passive} \emph{megune} `obscene' literally means `that which has been censured'.

\subsubsubsection{\thif~(adjectival form \mhuf)}

This template can serve as input to both verbal and adjectival passives.

Stative adjectives are only possible from roots that do not have a corresponding verb in \thif, (\ref{ex:5n51}a). A form ambiguous with a resultative might also exist, in which case its meaning is different, as with \emph{muʃlam} `perfect (stative adj.)'/`that which has been completed (adj. pass)'.
 \begin{exe}
 \ex  \label{ex:5n51}
 \begin{xlist} 
   \ex  No corresponding verb: \emph{muda} `aware', \emph{muʃlag} `snowy', \emph{mugaz} `carbonated'. 
   \ex  Ambiguous between resultative and stative: \emph{muʃlam} `perfect', \emph{mufʃat} `abstract'. 
 \z
\z 
As an innovation, a verb might be back-formed based on adjectives like those in~(\ref{ex:5n51}a) or derived from the related noun. For example, the substandard verb \emph{heʃlig} `snowed' is attested in the poet Bialik's work and can be found in use online.

Adjectival passives are available for all roots that have verbs in \thif. As discussed above, these constructions entail an implied EA.

	\subsection{Summary of adjectival passives}
We have now accounted for the existing generalizations regarding what kind of \isi{passive} (verbal or adjectival) and what kind of adjective (stative or resultative) can appear with what kind of root in each of the templates. The summary in Table~\ref{tab:5-2:a} concludes this section.

\begin{sidewaystable}
\begin{tabularx}{\textwidth}{cclcll}
 \lsptoprule
	& Interpretation & Heads/structure & EA? & Form & (template) \\\midrule
\multirow{3}{*}{Adjectives} & \multirow{3}{*}{stative} & [\root{root} a$_{\text{\gsc{SMPL}}}$] & \xmark & \emph{XaYuZ} & (\tkal)\\
& & [\root{root} a$_{\text{\gsc{INTNS}}}$] & \xmark & \mpua & (\tpie) \\
& & [\root{root} a$_{\text{\gsc{CAUS}}}$] & \xmark & \mhuf & (\thif) \\\tablevspace
\multirow{3}{*}{Adjectival passives} & \multirow{3}{*}{resultative} & [a [Voice [\root{root} v]]] & \cmark/\xmark & \emph{XaYuZ} & (\tkal)\\
& & [a [Voice [{\va} [\root{root} v]]]] & \cmark/\xmark & \mpua & (\tpie)\\
& & [a [{\vd} [\root{root} v]]] & \cmark & \mhuf & (\thif)\\
\lspbottomrule
 \end{tabularx}
	\caption{The head little \emph{a} in different configurations}
	\label{tab:5-2:a}
\end{sidewaystable}
The analysis of Hebrew provides further evidence for an eventive layer in adjectival passives. Hebrew also supports the claim that the same morphophonological form can spell out both stative and adjectival passives.

Finally, it is worth pointing out that the adjectival \isi{passive} is still productive\is{productivity}, especially since passives have been characterized as no longer productive\is{productivity} in Hebrew, a claim that seems too strong for adjectival passives novel forms such as \emph{meturgat} `targeted':
 \begin{exe}
\ex  ``For whatever reason, after years of complete openness with Google, and full access to all of the data and information that I produce, it looks like the only thing they know about [me] is that I'm a man. Enough already! I'm tired of ads for shaving, cars, insurance and cologne! \dots '' \\
	 \gll ex ani \glemph{jaxol} ligrom le-gugel latet l-i pirsom-ot ʃe-beemet \glemphu{meturgat-ot} el-aj?!\\
 		  how I can.\gsc{PTCP.SMPL} to.cause to-Google to.give to-me ad-\gsc{F.PL} \gsc{COMP}-really targeted.\gsc{INTNS.PASS.Prs}-\gsc{F.PL} to-me\\
 		\glt `How can I get Google to give me ads that are really targeted at me?' \hfill \url{http://www.facebook.com/elad.lerner/posts/1207164259295353} 
 \z 


\section{Nominalization} \label{passn:n}
This section addresses the deverbal nominalization\is{nominalizations} in Hebrew, also known as gerund, gerundive, action noun and \emph{masdar}. The claim I am building up to will be similar to that made for adjectival passives in Section~\ref{passn:adjpass}: nominal forms can arise in two ways. One is by nominalizing a root using a nominalizer with specific morphophonological form, which may or may not be similar to that of eventive \isi{nominalizations}. The other is by nominalization\is{nominalizations} of an existing verbal form (a VoiceP), in which case the nominalizer is little n and the result is a nominal with internal verbal structure.

	\subsection{Background}
To start, we need to recap some basic observations and proposals from the general literature. It has famously been proposed \citep{grimshaw90} that three different kinds of \textsc{\isi{nominalizations}} exist, (\ref{ex:5n53}). Much of the literature is devoted to discussing whether these classes really are mutually exclusive and what the best way to diagnose membership in one class or the other is \citep{alexiadou01,alexiadou09,alexiadou10b,alexiadou17,borer13oup,borer14lingua}. This question is inherently tied to formal proposals for how these classes might differ \citep{chomsky70,marantz97,harley09n,bruening13,wood19lsa}.

 \begin{exe}
 \ex  \label{ex:5n53}
 \begin{xlist}\sloppy
 	\ex  \textsc{simple nominals} appear monomorphemic. 
 	\ex  \textsc{result nominals} are nominalizations without argument structure whose semantics does carry the implication -- if not entailment -- of a completed event. They usually appear polymorphemic and are often homophonous with a CEN (seen next). 
 	\ex  \textsc{complex event nominalizations} are nominalizations of verbal forms. They have internal argument structure. 
 \z
\z 

Whether or not result nominals are a distinct class has been debated. I will not enter that debate here, referring the reader instead to \cite{ahdout19glow,ahdout19phd} for an in-depth investigation of \isi{nominalizations} in Hebrew (including some striking findings for result nominals, such as their varying acceptability with different templates). I focus instead on the two simplest cases: uncontroversially simple nominals and uncontroversial CENs, treating purported result nominals as simple nominals for present purposes.

Simple nominals like \emph{book }have no internal structure: there are no arguments to bookhood.

 \begin{exe}
\ex  The enemy's book (*of the city) (*in less than a day) 
 \z 	

CENs can be diagnosed in various ways, all converging on the conclusion that the noun contains a verb and its internal argument, together with possible modifiers.

 \begin{exe}
\ex  The \glemph{destruction} *(of the city) (in less than a day) 
 \z 
The meaning of a CEN\is{nominalizations} is always transparently related to that of the underlying verb.

Two additional points of contention are the status of the implied external argument, and the internal structure of polymorphemic simple nominals. For the implicit external argument, views are converging on the conclusion that it is projected in the syntax (as \textit{pro}) in the specifier of little n, although implementations still differ \citep{bruening13,layering15}. For nouns like \emph{nominal-\textbf{iz}-ation}, \emph{exam-\textbf{in}-ation} and perhaps even \emph{trans-\textbf{miss}-ion}, some overt verbalizer seems to be embedded in a simple nominal. The issue is whether the noun does involve internal structure which is somehow defused, or whether these nouns are still derived directly from the root with some complex suffix \citep{alexiadou01,alexiadou08,alexiadou09,alexiadou17,borer14lingua,moulton14,wood19lsa}. To the extent that the Hebrew data is relevant, I lean towards a complex suffixation analysis, but cannot provide novel evidence that tells the two possibilities apart.

	\subsection{Descriptive generalizations} \label{passn:n:data}
Most templates have dedicated nominal forms, given in Table~\ref{tab:5-3:noms}. The exceptions are {\tkal}, which has a number (varying by root; \citealt{borer13oup,ahdout19phd}), and the \isi{passive} templates {\tpua} and {\thuf}, which have no \isi{nominalizations} of their own \citep{kastnerzu17}.

\begin{table}
\begin{tabularx}{.5\textwidth}{ll} 
 \lsptoprule
	Verbal form	& Derived nominal form\\\midrule
	{\tpie}	& \emph{Xi\dgs{Y}uZ}\\
	{\thif} & \emph{haXYaZa}\\
	{\tnif} & \emph{hi\dgs{X}aYZut}\\
	{\thit} & \emph{hitXa\dgs{Y}Zut}\\
\lspbottomrule
 	\end{tabularx}
	\caption{Deverbal nouns in Hebrew}
	\label{tab:5-3:noms}
\end{table}

Hebrew CENs behave as would be expected, patterning like their English counterparts with regard to possible arguments and adverbs. 

 \begin{exe}
 \ex \label{ex:nom-destruct} 
 \begin{xlist} 
 	\ex   
[] 		{ \gll ha-ojev \glemph{heʃmid} et ha-ir (tox jom).\\
 		  the-enemy destroyed.\gsc{CAUS} \gsc{ACC} the-city within day\\
 		\glt `The enemy destroyed the city (in a day).' } 
		
 	\ex   
[] 		{ \gll \glemph{haʃmada-t} ha-ojev et ha-ir (tox jom)\\
 		  destruction.\gsc{CAUS}-of the-enemy \gsc{ACC} the-city within day\\
 		\glt `The enemy's destruction of the city (in a day)' } 
	
 \z

 \ex  \label{ex:nom-restore} 
 \begin{xlist} 
 	\ex   
[] 		{ \gll ha-mitnadvim \glemph{ʃikmu} et ha-jaar (be-zrizut).\\
 		  the-volunteers rehabilitated.\gsc{INTNS} \gsc{ACC} the-forest in-quickness\\
 		\glt `The volunteers rehabilitated the forest (quickly).' } 
		
 	\ex   
[] 		{ \gll \glemph{ha-ʃikum} (ha-zariz) ʃel ha-jaar (al-jedej ha-mitnadvim)\\
 		  the-rehabilitation.\gsc{INTNS} the-quick of the-forst by the-volunteers\\
 		\glt `The (quick) rehabilitation of the forest (by the volunteers)' } 
		
 \z
\z 

	\subsection{The head n in Hebrew} \label{passn:n:n}
Simple nouns exist in various patterns in Hebrew. I assume that these patterns spell out variants of the nominalizer little n; there are potentially dozens of these. Assuming a decomposition into root and nominalizer, example structures for simple nouns are as follows:

 \begin{exe}
\ex  
	\begin{minipage}[t]{0.32\textwidth}
		a. \emph{kelev} `dog'\\
		\Tree
			[.n
				[.{\root{klb}} ]
				[.n_{\text{XeYeZ}} ]
			]
	\end{minipage}
	\begin{minipage}[t]{0.32\textwidth}
		b. \emph{telefon} `phone'\\
		\Tree
			[.n
				[.{\root{tlfn}} ]
				[.n_{\text{XeYeZoW}} ]
			]
	\end{minipage}
 \z 	

To derive a CEN\is{nominalizations}, in Hebrew as in other languages, we simply add little n above an existing VoiceP structure \citep{hazout95,engelhardt00}. Since my main interest is in the morphology and how it reflects the syntax and semantics, I do not engage with questions such as where the arguments are generated (as full DPs within vP, or base-generated as an operator with the full DP adjoined to the noun, like for internal arguments of adjectival passives).\footnote{The nominalizer n might attach even higher for some constructions e.g.~[n TP] \citep{alexiadou17}. I do not explore larger structures like that one here; see \cite{wood20oup} for a recent review and synthesis of the question, \cite{kastner15lingua} on nominalization\is{nominalizations} of entire clauses in Hebrew, and~\cite{kastnerzu17} on the incompability of n with Pass in Hebrew.}\label{r1:5:6}

 \begin{exe}
\ex  {\emph{haʃmada} `destruction'} \\
	\Tree
	[.n
		[.n\\\emph{-a} ]
		[
			[.{\vd}\\\emph{ha-,a} ]
			[
				[.v
					[.\root{ʃmd} ]
					[.v ]
				]
				[.DP ]
			]
		]
	]
 \z 
It seems reasonable to assume that the nominalizer is spelled out as (the feminine singular) -\emph{a} while conditioning \isi{allomorphy} of the vowels on {\vd}, but I do not develop a detailed morphophonological analysis here. Most of what has been said about verbs should carry over to nouns as well.

It is well-known that some forms are ambiguous between a simple and a CEN\is{nominalizations} reading; English \emph{transmission} and \emph{examination} are famous examples, or Hebrew \emph{kibu{\ts}}, which can mean either `gathering' (CEN\is{nominalizations}) or a kibbutz (simple noun). The analysis I have sketched here ends up saying that in Hebrew, the form is ambiguous between an action nominalization\is{nominalizations} of the verb \emph{kibets} `gathered', (\ref{ex:5n60}a), and a noun derived directly from the root, (\ref{ex:5n60}b).

 \begin{exe}
 \ex  \label{ex:5n60}
 \begin{xlist} 
 	\ex 
[] 		{ \gll medina-t israel \glemph{tihie} ptuxa le-alia jehudit ve-le-\glemphu{kibuts} galujot\\
 		  state-of Israel will.be.\gsc{SMPL} open to-immigration Jewish and-to-gathering.\gsc{INTNS}.of diasporas\\
 		\glt `The State of Israel will be open for Jewish immigration and for the Ingathering of the Exiles.' \hfill (Israeli Declaration of Independence) } 
	

		\Tree
		[.n
			[.n ]
			[.VoiceP
				[.Voice ]
				[.
					[.{\va} ]
					[.vP
						[.v
							[.\root{\dgs{k}bts} ]
							[.v ]
						]
						[.DP ]
					]
				]
			]
		]

 	\ex   
		``According to his testimony, in the early 60s, before he began his political career in the USA, \dots\\
		 \gll \glemph{ʃaha} sanderz kama xodaʃim be-israel ve-\glemph{hitnadev} be-\glemphu{kibuts}\\
 		  stayed.\gsc{SMPL} Sanders a.few months in-Israel and-volunteered.\gsc{INTNS.MID} in-kibbutz\\
 		\glt \dots Sanders stayed in Israel for a few months and volunteered in a Kibbutz.''\footnotemark 
	

		\Tree
		[.n
			[.\root{\dgs{k}bts} ]
			[.n$_{\text{\gsc{INTNS}}}$ ]
		]
 \z
\z 
\footnotetext{\url{https://goo.gl/GzqQUQ} (retrieved April 2016).}

This view fits with the original argument for roots within a syntactic approach to Semitic morphology as put forward by \cite{arad03}, who showed how nouns may be derived either from roots or from existing nouns. In the context of the Trivalent proposal, verbal templates are special: each functional head in the verbal domain has deterministic spell-out, modulo contextual \isi{allomorphy}. In contrast, nouns and adjectives can be derived using a range of nominal patterns, perhaps because there is nothing to signal about their argument structure. This much seems to be supported by the data: while there are five active verbal templates, there are dozens of nominal patterns (especially if we wish to assume that a loanword like \emph{en{\ts}iklopedja} `encyclopedia' instantiates the one-off pattern \emph{CeCCiCCoCeCCa}).

Another consideration points towards this conclusion \citep{kastner18nllt}: a simple noun might not even have any corresponding action nominal if there is no underlying verb. The noun \emph{kibuʃ} `occupation' is not derived from an underlying verb in {\tpie}, meaning that the morphophonological nominal pattern \gsc{INTNS} must exist independently of a verb with similar morphology.
 \begin{exe}
 \ex  \label{ex:5n61}
 \begin{xlist} 
 	\ex   
[] 		{ \gll daj la-\glemph{kibuʃ}!\\
 		  enough to.the-occupation\\
 		\glt `Down with the occupation!' } 
		
 	\ex[*]{  \emph{kibeʃ} }
 \z
\z 

What the Hebrew data do show, however, is that CENs contain a Voice layer because the Voice-level morphology is overt. This point is consistent with existing analyses of embedded Voice in CENs, many of which are agnostic regarding whether Voice is or is not embedded under n in languages where Voice is covert \citep{alexiadou17,wood19lsa}.

Few differences between templates have been noticed, having to do with variation within {\tkal} \citep{borer13oup} and a gap in {\tnif} \citep{silonipreminger09,ahdoutkastner19nels}; again, see \cite{ahdout19glow,ahdout19phd} for some proposed distinctions. Further discussion of the nominal system is beyond the scope of this book, but see for instance \cite{fausthever10} and \cite{laks15ws}. \citet[534fn13, 555]{borer13oup} outlines a theory in which template-specific nominalizers merge above templatic verbalizers. The meanings of these nominal forms are similar to those of the underlying verbs, but they need not be. In that system a noun derived from a verb can still have different meaning than the verb; the Trivalent system adheres to a stricter view of \isi{locality} which forces a proliferation of morphophonological patterns. But if every noun that looks like a potential (verbal) CEN\is{nominalizations} is derived from a verbal form, the Exo-Skeletal model needs to admit an underlying verb-like piece which might not otherwise exist, like~(\ref{ex:5n61}b).


\section{Conclusion} \label{passn:conc}
This chapter analyzed cases in which the existing structures presented in the previous three chapters are embedded under the \isi{passive}, adjectival and nominal heads that have been proposed elsewhere in the literature. The architectural bottom line is that a VoiceP can serve as the input to further derivation. If Pass, little \textit{a} or little n are merged above it, the result is entirely predictable: a \isi{passive} verb, an adjectival \isi{passive} or a CEN\is{nominalizations}. Adjectives and \isi{nominalizations} have forced us to make the theory slightly weaker in that there exist independent adjectivizers and nominalizers which look like the existing templates. For example, we have seen evidence for a nominalizer n$_{\gsc{INTNS}}$ which has the same output as nominalizing an existing verb, [n \tpie]. All templates have such analogous simple \isi{nominalizations}. This result seems to be a necessary evil on the morphophonological side, leading to predictable results on the syntactic-semantic side. But \isi{nominalizations} and adjectivizations of the root do not have internal verbal structure (contra \citealt{borer13oup,borer14lingua}) and might carry different meaning than that of the homophonous complex form derived from the verb.\largerpage

This all seems like a welcome result, since we would not expect Hebrew to be radically different in terms of architecture than any other language. The fact that different functional heads can be merged with predictable results once the basic verb has been built, and that these derivational processes appear to be essentially identical in Hebrew and in other languages, is a strong argument in favor of the general approach. See \cite{ahdout19phd} for related explorations of the nominal system.\label{r1:5:7}

Before concluding, one last word ought to be said about \emph{denominal verbs}, i.e. verbs derived from underlying nouns. In these cases a verb seems to be derived from another word instead of the root \citep{batel94,ussishkin99,ussishkin05,arad03}. In Hebrew, this phenomenon is evident in that the verb carries along affixal material that was attached to the ``base'' noun before it was verbalized, as in~(\ref{ex:denominal-verbs-intro}). The boldfaced affixes arguably attach only to nouns, making their appearance in the corresponding verb inexplicable unless the verb is itself denominal.

 \begin{exe}
 \ex  \label{ex:denominal-verbs-intro}Denominal verbs contain nominal affixes: 
 \begin{xlist} 
 	\ex   {\textit{kam{\ts}\textbf{-an}}, `stingy person' $\longrightarrow$ \textit{hitkam{\ts}e\textbf{n}}, `was stingy'} 
 	\ex  {\textit{ki{\ts}\textbf{-on-i}}, `extreme' $\longrightarrow$ \textit{hek{\ts}i\textbf{n}}, `brought to extremity'} 
 	\ex  {\textit{\textbf{ta-}xzuk\textbf{-a}}, `maintenance' $\longrightarrow$ \textit{\textbf{t}ixzek}, `maintained'} 
 	\ex  {\textit{\textbf{mi-}spar}, `number' $\longrightarrow$ \textit{\textbf{m}isper}, `enumerated'} 
 \z
\z 
As \cite{arad03} points out, these denominal verbs have compositional semantics insofar as they have predictable meanings when compared to their underlying nouns. \cite{kastnertucker19cup} point out how \citeauthor{arad03} noticed that these facts indicate a derivational ``point of no return'' for non-concatenative morphology, based on cyclic spell-out.\footnote{Omer Preminger (p.c.) points out the pair \emph{izker} `commemorated, mentioned' and \emph{azkara} `memorial service', where the noun seems to have a meaning that the underlying verb does not. Setting aside the question of whether a memorial service counts as a commemoration, this case is especially interesting because the verb is quadrilateral \citep{schwarzwald16}. Could it be that a quadrilateral root \root{'zkr} was innovated and then used for the other form, i.e.~the noun and verb have different derivational histories? Checking the database, I found 24 verbs in \emph{iXYeZ} that are plausibly derived from nouns. Perhaps back-formation from nouns is deriving new quadrilateral roots, or vice versa:

 \begin{exe}
 \ex  
 \begin{xlist} 
 	\ex  \emph{ixles} `populated' $<$ \emph{uxlusija} `population', \emph{ifjen} `characterized' $<$ \emph{ofen} `facet', \emph{irgen} `organized' $<$ \emph{irgun} `organization' 
 	\ex  \emph{iʃpez} `hospitalized' $\overset{?}{>}$ \emph{iʃpuz} `hospitalization', \emph{iʃrer} `ratified' $\overset{?}{>}$ \emph{iʃrur} `ratification', \emph{itxel} `rebooted' $\overset{?}{>}$ \emph{itxul} `reboot'. 
 \z
\z 
In all of these cases except for \emph{izker} the \citeauthor{arad03}-style generalization holds. Whether these \emph{a}-nouns and \emph{i}-verbs have different structure than in the original \emph{m}-nouns and regular verbs discussed by \cite{arad03} is an interesting question for future work. See also \cite{ouhalla16} for a different line of argumentation and \cite{brice16} for experimental evidence supporting root embedding.}

A substantial body of work considers denominal verbs to pose a problem for root-centric views of Semitic morphology such as the current one. \cite{batel94,batel03,batel17} suggests a denominal derivation for these verbs, one which is not possible in theories in which verbs can only be derived from roots. The argument is that some word-formation processes in Hebrew need to be analyzed in terms which allow word-like inputs to subsequent word formation, and hence all word formation is based on words (or stems) and not roots. Yet there are two problems with this view. First, no word-based analysis has shown that a root-based analysis is unable to capture the same patterns when allowing word-based derivation as well, whereas root-based analyses have been able to show the inadequacy of word-based analyses \citep{kastner18nllt}. The second issue is even more general. As much work has already argued \citep{arad03,arad05,doron03}, and as we have just seen in this chapter, this kind of objection is fundamentally misdirected. The syntactic approach inherent in DM accounts of Hebrew (not just the Trivalent one) obviates much of the debate on whether word-formation in the language is ``root-based'' or ``word-based'' because DM accounts have as a matter of theoretical hypothesis the notion that word-formation takes as input whatever the syntax can generate \citep{kastnertucker19cup}. We have seen in this chapter that the syntax can passivize, nominalize or adjectivize complex constituents regardless of the language. Taking this as a given, it is not surprising that denominal verbs show morphophonological and morphosyntactic properties that suggest that they are derived from an underlying verb -- these are precisely the sorts of effects that one would expect in a syntactic approach to word-building.\largerpage

With this conclusion, according to which Hebrew is not all that different from other languages after all, we turn to Part~\ref{part:2} of the book: general crosslinguistic considerations.


\part{Crosslinguistic Consequences}

    \chapter{Syntactic vs semantic transitivity}
\label{chap:aas}

\section{Introduction} \label{sec:intro}
The first part of this book developed a theory of Voice which recognizes three possible values: {\vd}, {\vz} and Unspecified Voice. We have seen that the syntactic features of Voice are correlated with semantic interpretation in ways which themselves are informative: {\vd} introduces a thematic external argument, although this requirement can be voided in de-adjectival and de-verbal inchoatives, indicating that the semantics is still sensitive to syntactic structure (Chapter \ref{chap:vd}); {\vz} does not introduce a syntactic external argument, but it can trigger existential closure over an Agent, and {\pz} does introduce a Figure role, indicating that the thematic interpretation is sensitive to the extended verbal projection (Chapter \ref{chap:vz}); and Voice places no requirements either in the syntax or in the semantics, although the two are once again correlated (Chapter \ref{chap:voice}).

The current theory assumes that every verbal projection contains Voice. In this chapter I would like to highlight some differences between this theory and the one most closely related to it, which I will call for simplicity the \textbf{Layering} approach \citep{schaefer08,layering15}. I discuss the two basic premises behind Layering in Section~\ref{aas:layering} and show how they are manifested in the current theory and how they are different in Section~\ref{aas:compare}. The differences will pattern as follows: everything that can be expressed using Layering can be expressed in the current approach, but causative alternations are beyond the purview of Layering and require a trivalent system. In addition, a number of necessary stipulations are arguably less stipulative in the current theory. Sections~\ref{aas:hebrew} and~\ref{aas:jim} verify that Layering cannot be applied to the Hebrew data. This comparison between theories sets up expectations for additional crosslinguistic study, to which I turn in the Conclusion, Section~\ref{aas:conc}, and in the next chapter.


\section{Layering} \label{aas:layering}
Many recent syntax-based theories of argument structure adopt two core assumptions which have been most notably defended in the work of \cite{schaefer08} and colleagues \citep{alexiadouetal06,layering15}. These are the shared core for causative and anticausative alternants (reviewed in Section~\ref{aas:layering:base}) and the dissociation of syntactic and semantic transitivity (reviewed in Section~\ref{aas:layering:features}).

	\subsection{Causative core} \label{aas:layering:base}
This component is crucial for the current approach and was reviewed in Chapter~\ref{intro:arch:layering}. Its main parts are summarized here.

In argument structure alternations such as~(\nextx), it is not accurate to think that the transitive variant is derived from the intransitive one, nor is it accurate to think that the intransitive variant is derived from the transitive one. 
\pex
	\a John broke the vase.
	\a The vase broke.
\xe

\cite{layering15} propose that both variants have the same base: a core vP~(\nextx a) containing the verb (a verbalized root) and the internal argument. The difference between the two variants is that the transitive one, (\nextx b), then has the external argument added by additional functional material, namely Voice.
\ex
a. 
\Tree
		[.vP
			[.\emph{broke} ]
			[.\emph{the glass} ]
		]
b. \Tree
[.VoiceP
	[.\emph{John} ]
	[.
		[.Voice ]
		[.vP
			[.\emph{broke} ]
			[.\emph{the glass} ]
		]
	]
]
\xe

There is therefore no dedicated direction of derivation which is marked by the morphology across languages: some languages mark the transitive variants, others mark the intransitive variants, and sometimes both variants are marked in the same language (as we have already seen for Hebrew).

In addition to the morphological reasoning, \cite{layering15} provide a series of arguments showing that the core causative component of the vP is present even in the anticausative variants, like the anticausative examples in~(\nextx) which nevertheless have cause PPs.
\pex
	\a The flowers wilted \{from the heat / *from the gardener\}.
	\a The window cracked \{from the pressure / *from the worker\}.
\xe

The causative component is thereby dissociated from the external argument, the latter being introduced in an additional structural layer. Voice is the functional head enabling this layer, both in terms of licensing Spec,VoiceP in the syntax and in opening the thematic predicate agent. This much suffices to account for the English alternation.

	\subsection{The transitivity of Voice} \label{aas:layering:features}
The second tenet of Layering is informed by alternations in additional languages. The existence of marked anticausatives as in~(\nextx) raises the question of what their morphology is tracking in the syntax.
\ex \begingl
	\gla Die T\"ur hat \underline{sich} ge\"offnet.//
	\glb the door has \gsc{SICH} opened//
	\glft `The door opened.'\trailingcitation{(German)}//
	\endgl
\xe

\cite{layering15} propose a system in which Voice can be ``transitive'' both in syntactic and semantic terms. In the syntax, Voice might be either associated with a specifier or not; in the semantics, it might introduce a thematic agent role or not. This conceptual innovation is implemented by using a syntactic feature [D], an EPP feature on Voice.\footnote{Unlike the current system---in which [D] is a feature that must simply be checked, or intuitively some kind of filter---the [D] feature in Layering is inherent to Voice and is structure-building, being the only thing that can license/project the specifier position; see Section~\ref{aas:jim}.}

The important consequence is that there are five possible configurations. Four are derived using Voice, depending on whether it is syntactically active and whether it is semantically thematic. The fifth is the complete lack of Voice, as suggested for unmarked anticausatives. With this basic assumption, Layering is able to handle cases in which syntactic elements like expletives do not involve thematic roles as well as cases in which the anticausative variant is marked morphologically, as we will see below. Taken together, these two components combine to provide substantial empirical coverage and theoretical insight.

Let us get to the details. Assuming that Voice may or may not have a [D] feature, and that it may or may not introduce an agent, the four-celled typology of \citet[109]{layering15} emerges in~(\nextx):
\ex\label{ex:typo-layer}
\raisebox{-9em}{
\begin{tabular}{c|ll|ll}
	& \multicolumn{2}{c|}{Syntax D}	& 	\multicolumn{2}{c}{Syntax {\zero}} \\\hline
%&&&&\\
Semantics & 	a.&	Thematic active 	&	b.&	Thematic non-active\\
$\lambda$x 	 & &
\Tree
[.VoiceP 
	[.DP ]
	[.VoiceP
		[.{Voice\{$\lambda$x, D\}} ]
		[.{\dots~vP \dots} ]
	]
]
& &
\Tree
[.VoiceP 
		[.{Voice\{$\lambda$x, \zero\}\\\gsc{NACT}} ]
		[.{\dots~vP \dots} ]
]
\\
%&&&&\\
%& & \ding{228} Transitive verb.	& & \ding{228} Passives in Greek.\\
&&&&\\\hline
Semantics & 	c.&	Expletive active 	&	d.&	Expletive non-active\\
{\zero}	 & &
\Tree
[.VoiceP 
	[.DP\\\gsc{SE} ]
	[.VoiceP
		[.{Voice\{\zero, D\}} ]
		[.{\dots~vP \dots} ]
	]
]
& &
\Tree
[.VoiceP 
		[.{Voice\{\zero, \zero\}\\\gsc{NACT}} ]
		[.{\dots~vP \dots} ]
]
\\
%&&&&\\
%& & \ding{228} Marked anticausatives (German).	& & \ding{228} Anticausatives in Greek.\\
\end{tabular}
}
\xe

This typology can be augmented by including the Voiceless unmarked anticausative, vP, giving us the full table in~(\nextx).
\ex\label{typo-layer-all}The typology of Voice under Layering:\\
\begin{tabular}{c|ll|ll|ll}
	& \multicolumn{2}{P{4cm}|}{Syntax D}	&  \multicolumn{2}{P{4cm}|}{vP}	& \multicolumn{2}{P{4cm}}{Syntax {\zero}} \\\hline
%&&&&\\
Semantics	 & 		a.	&	&			b.	&& 	c. & \\
$\lambda$x 	 & 
&\Tree
[.VoiceP 
	[.DP ]
	[.
		[.{Voice\{$\lambda$x, D\}} ]
		[.vP ]
	]
]
& 
& \phantom{Undefined.}
&& \Tree
[.VoiceP 
		[.{Voice\{$\lambda$x, \zero\}\\\gsc{NACT}} ]
		[.vP ]
]
\\\hline
Semantics	 & 		d.		& &			e.	& &	f. & \\
\zero	 &
& \Tree
[.VoiceP 
	[.DP\\\gsc{SE} ]
	[.VoiceP
		[.{Voice\{\zero, D\}} ]
		[.vP ]
	]
]
&
&\Tree
		[.vP ]
&
&\Tree
[.VoiceP 
		[.{Voice\{\zero, \zero\}\\\gsc{NACT}} ]
		[.vP ]
]
\\
\end{tabular}
\xe



I examine the cells one by one. The structure in~(\lastx a) is a straightforward transitive derivation, at least since \cite{kratzer96} and \citep{pylkkanen08}. The [D] feature on Voice licenses a DP in its specifier, and the agent role is introduced in the semantics (notated here simply as $\lambda$x).

The structure in~(\lastx f) derives a marked anticausative, similar to {\tnif} from Chapter \ref{vz:vz}. There is no [D] feature, so no DP can be merged in Spec,Voice. The lack of a specifier is spelled out as non-active morphology, \gsc{NACT} in short, via a rule we return to in~(\ref{ex:aas:vi-nact}). No agent is introduced in the semantics. The result is a marked anticausative (unaccusative) construction. The unmarked anticausative is derived in~(\lastx e), by not merging any Voice head at all. Since no Voice head exists in the structure, there is no agentive semantics either, so~(\lastx b) is undefined.

The particularly interesting cases are those in which we find ``mismatches'' between the values of the syntactic feature and semantic specification, namely~(\lastx c) and~(\lastx d). Starting with~(\lastx d), we have a situation in which no agentive semantics is introduced but Voice still requires a specifier. The Layering analysis proposes that this is the situation for the Romance expletive \gsc{SE} and the German \emph{sich}, which appear in marked anticausatives but contribute nothing to the semantics. Similar analyses have been proposed for Icelandic by \cite{wood14nllt,wood15springer} and for various phenomena in English and Quechua by \cite{myler16mit}.

Finally, the configuration in~(\lastx c) is also possible. Here, Voice does not have a [D] feature and does not project a specifier. However, it does introduce a thematic role. \cite{layering15} propose that this is the correct analysis of passive verbs in Greek, which are identical morphologically to anticausatives; the analysis captures the fact that the morphology of~(\lastx c) and~(\lastx f) is identical, since in neither case is a specifier projected. The open predicate must presumably be closed off by existential closure later in the derivation. It is worth keeping in mind that regardless of combination with thematic non-active Voice, every root must still need to state whether the \gsc{NACT} variant will be anticausative, passive, or compatible with both \citep[88]{alexiadouanagnostopoulou04,layering15}.


\section{Comparison} \label{aas:compare}
The current theory differs from Layering in two concrete ways. First, Layering assumes that no Voice layer is projected for unmarked non-active constructions. I assume that a VoiceP is always projected, except that its specifier might not be filled. We have seen this in the difference between active and non-active verbs in {\tkal} (Chapter \ref{chap:voice}), and in the difference between {\vz} and {\vd} (Chapters \ref{chap:vz}--\ref{chap:vd}). The morphological reflexes of this difference are briefly highlighted in Section~\ref{aas:compare:vi-nact}. The second difference  is more substantial, building on the first: there are three possible values of the [D] features, closely associated with semantic interpretation (Section~\ref{aas:compare:features}).

	\subsection{Non-active layers} \label{aas:compare:vi-nact}
Marked anticausatives show consistent morphological marking on the anticausative member of an alternation. The Latin example in~(\nextx) is from \citet[662]{kastnerzu17}:
\ex \begingl
	\gla vulnus \textbf{claudi-t-ur}//
	\glb wound.\gsc{NOM} close-\gsc{3SG}-\gsc{NACT}//
	\glft `The wound heals.'//
	\endgl
\xe

To account for the appearance of non-active morphology in marked anticausatives, Layering proposes the rule in~(\nextx), following \cite{embick04}:
\ex\label{ex:aas:vi-nact}Voice \lra~\gsc{NACT} / \trace~No Spec
\xe
%What if the verb moves to T and then there's a specifier?

A theory of Vocabulary Insertion which allows~(\ref{ex:aas:vi-nact}) must then able to make reference to syntactic contexts such as ``lack of a specifier''. While this is not impossible, it does complicate the theory somewhat.\label{r1:6:1}

I make a different claim: non-active morphology such as \gsc{NACT} and {\tnif} is the spell-out of {\vz}. In other words, it is the flavor of Voice which is spelled out as non-active morphology, not Voice when it has no specifier. Recall the reason for this preference: Unspecified Voice in Hebrew is spelled out as {\tkal} regardless of whether it has a specifier or not. The spell-out rule in~(\nextx) is thus more consistent crosslinguistically; the cases in~(\ref{typo-layer-all}c) and~(\ref{typo-layer-all}f) can both be accounted for using the rule in~(\lastx), if we assume that {\vz} is in fact the Voice head in those structures.
\ex {\vz} \lra~\gsc{NACT} \hfill (always No Spec)
\xe

This technical difference aside, In what follows I address more substantive differences between the theories.

	
	\subsection{The trivalency of transitivity} \label{aas:compare:features}
The two systems ended up looking as follows:
\ex\label{ex:aas:typo-layer-all2}The typology of Voice heads in Layering:\\
\begin{tabular}{c|ll|ll|ll}
	& \multicolumn{2}{P{5.05cm}|}{Syntax D}	&  \multicolumn{2}{P{4cm}|}{vP}	& \multicolumn{2}{P{4cm}}{Syntax {\zero}} \\\hline
%&&&&\\
Semantics	 & 		a.	&	&			b.	&& 	c. & \\
$\lambda$x 	 & 
&\Tree
[.VoiceP 
	[.DP ]
	[.
		[.{Voice\{$\lambda$x, D\}} ]
		[.vP ]
	]
]
& 
& \phantom{Undefined.}
&& \Tree
[.VoiceP 
		[.{Voice\{$\lambda$x, \zero\}\\\gsc{NACT}} ]
		[.vP ]
]
\\\hline
Semantics	 & 		d.		& &			e.	& &	f. & \\
\zero	 &
& \Tree
[.VoiceP 
	[.DP\\\gsc{SE} ]
	[.VoiceP
		[.{Voice\{\zero, D\}} ]
		[.vP ]
	]
]
&
&\Tree
		[.vP ]
&
&\Tree
[.VoiceP 
		[.{Voice\{\zero, \zero\}\\\gsc{NACT}} ]
		[.vP ]
]
\\
\end{tabular}
\xe

\ex\label{ex:aas:typo-feat}The current typology:\\
\begin{tabular}{c|ll|ll|ll}
	& \multicolumn{2}{P{5.05cm}|}{\vd}	&  \multicolumn{2}{P{4cm}|}{Voice}	& \multicolumn{2}{P{4cm}}{\vz} \\\hline
%&&&&\\
Semantics	 & 		a.	&	&			b.	&& 	c. & \\
$\lambda$x 	 & 
&\Tree
[.VoiceP 
	[.DP ]
	[.
		[.{\vd} ]
		[.vP ]
	]
]
& 
&\Tree
[.VoiceP 
	[.DP ]
	[.
		[.Voice ]
		[.vP ]
	]
]
&& (Figure reflexives) %\phantom{A-ha!}
\\\hline
Semantics	 & 		d.		& &			e.	& &	f. & \\
\zero/$\exists$x	 &
& ({\vd} inchoatives) %\phantom{A-ha!}
&
&\Tree
[.VoiceP
	[.(\gsc{SE}) ]
	[.
		[.Voice ]
		[.vP ]
	]
]
&
&\Tree
	[.VoiceP
		[.{\vz} ]
		[.vP ]
	]
\\
%\hline
%Semantics	 & 		g.		& &			h.	& &	i. & \\
%$\exists$x	 &
%& 
%&
%&
%&
%&\Tree
%	[.VoiceP
%		[.{\vz} ]
%		[.vP ]
%	]\\
%
\end{tabular}
\xe

While the semantics of the cells in Layering is deterministic (modulo existential closure), the current theory relies on contextual allosemy of the Voice head.\footnote{Layering can also be formalized using allosemy, as in~\cite{schaefer17oup}.} Its semantics looks broadly as in~(\nextx), summarizing what we have seen in previous chapters:
\pex Semantics (abstracting away from Agent $\neq$ Cause):
	\a \denote{\vd} = $\lambda x \lambda e$.Agent($x,e$)
	\a \denote{Voice}\phantom{.......} = $\begin{cases}
		\lambda x \lambda e.\text{Agent}(x,e) & \text{/ \trace \{\root{\gsc{eat}}, \dots\} }\\
		\lambda e.e & \text{/ \trace \{\root{\gsc{fall}}, \dots\} }\\
	\end{cases}$
	\a \denote{\vz}\phantom{.} = $\begin{cases}
		\lambda e \exists x.\text{Agent}(x,e) & \text{/ \trace \{\root{\gsc{write}}, \dots\} }\\
		\lambda e.e & \\
	\end{cases}$
\xe

There are two points to be made about how powerful the current approach is: first, that it has all the empirical coverage necessary for the Layering patterns. Very little has to be said in order to maintain the coverage of Layering as applied to English, German and Greek. For instance, expletive constructions in Germanic and Romance are derived by simply adding the expletive in Spec,VoiceP, (\ref{ex:aas:typo-feat}e).

\label{r1:6:2}@The second is that this power actually comes from a system that is just as constrained as Layering. While in Layering all features may combine freely, in the current theory semantic interpretation tracks the syntactic feature on the functional heads: barring exceptional cases, the active head {\vd} has an agentive reading and the non-active head {\vz} has either a non-active reading or a passive reading. So {\vd} head does not have a straightforward non-active alloseme, and {\vz} head is the only one with a passive alloseme, (\lastx c). In this sense, at least, the interpretation of these heads is natural. The exceptional cases have already been discussed: .

I take this correlation to be a welcome result, though I do not attempt to derive how the syntax feeds the semantics in this way. With that in mind, however, one might still wonder whether~(\ref{ex:aas:typo-feat}c) and~(\ref{ex:aas:typo-feat}d) could be possible. In fact, we have already seen that these configurations are possible, but only when additional \emph{syntactic} constraints are at play.

Chapter \ref{vz:pz} analyzed figure reflexives in {\tnif}. This was a situation in which a [--D] head introduced an external argument, specifically the Figure role of {\pz}. As noted in Chapter~\ref{vz:interim}, {\pz} and {\vz} may be considered contextual variants of each other and of the generalized head \emph{i*} (I return to this point in Chapter~\ref{chap:i}). Still, why can {\pz} introduce a thematic role? One answer can be found in the work of \cite{wood15springer}. He suggests that the allosemic sensitivity of a head depends on its place in the extended projection of the verb. Metaphorically speaking, since {\pz} ``knows'' that it is not the last head in the VoiceP, it can introduce a Figure role knowing that this role can be saturated later on. Importantly, this is not a case of lookahead: the derivation will crash if no DP is merged to saturate that role. So a generalized version of~(\ref{ex:aas:typo-feat}c) is possible after all, if the syntactic configuration is just right (as expected).\footnote{As mentioned earlier, \cite{legate14} and \cite{akkus19jl} suggest that the Agent role can be introduced and then closed off. Perhaps deponent verbs can also be treated in similar fashion.}

The second case is the cell in~(\ref{ex:aas:typo-feat}d): a situation in which a causative-marked verb turns out to be inchoative. As have seen in Chapter \ref{vd:vd}, this is no hypothetical. Inchoatives do exist in {\thif}, but only in specific syntactic configurations (when the verb is de-adjectival or de-nominal). The allosemic rule in~(\ref{ex:vd:sem-full}) stated this explicitly.\footnote{One crosslinguistic correlate might be the ``adversity causative'' of Japanese \citep{pylkkanen08,woodmarantz17}, where the Voice head itself is potentially {\vd} (see Chapter \ref{i:i:jap}) but does not have its own agentive semantics, instead taking a possessor role passed up from lower in the tree.}

As a final point of comparison before returning to Hebrew, it is important to consider how Layering allows languages to pick and choose between features to be combined. For example, Greek passives are derived as in~(\ref{ex:aas:typo-layer-all2}), where existential closure applies to the open Agent role. \cite{schaefer17oup} later adopts a position similar to the one here, whereby $\exists$x is another possible semantic value for Voice heads, thus removing the need for existential closure of $\lambda$x.

Either way, given that existential closure can apply at some level as in Greek, the question arises of why it does not apply in situations where an overt DP appears. Specifically, there is nothing to prevent an expletive such as German \emph{sich} from being the DP in~(\ref{ex:aas:typo-layer-all2}a). The expletive would not have a semantic role to saturate, but an Agent would still be entailed. The result should be a construction with an expletive whose reading is not anticausative but passive. Yet this is impossible in German:
\ex \begingl
	\gla Die T\"ur hat sich (*absichtlich) ge\"offnet (*um das Zimmer zu l\"uften)//
	\glb the door has \gsc{SICH} on.purpose opened in.order the room to air.out//
	\glft (int. `The door opened on purpose and/or in order to air out the room')//
	\endgl
\xe

What is relevant in this regard is that Greek and German might avail themselves of different cells of the typology. Specifically, German can be argued not to have $\exists$x Voice heads (passivization applies above the VoiceP in German; cf.~Chapter~\ref{passn:pass}). \cite{schaefer17oup} discusses similar cases of passivization in depth, concluding that $\exists$x is a necessary semantic possibility for Voice heads (as already mentioned above) and providing an analysis explaining why French and other languages do allow passives as in~(\lastx), albeit without \emph{by}-phrases.

The problem is, then, that e.g.~French might have Voice\{$\exists$x, D\} for these passives but does not have Voice\{$\exists$x, \zero\}; the selection of features from the universal pool appears arbitrary. A similar problem arises for Layering when turning to causative marking: we would expect that a language with causative marking could combine it with an expletive. This does not seem to be correct, although I have not conducted enough crosslinguistic work to make this assertion conclusively.

In the current theory, these issues do not arise because the dichotomy of thematic/expletive Voice is abandoned, as is the idea that languages pick only a subset of features to instantiate across cells. Instead, Voice is allosemic in ways which are constrained both by the root and by the feature [D].


%Passives with anticausative marking in Hebrew and Greek do not belong in the $\lambda$x row on my theory, but arise when the passive alloseme of {\vd} is invoked, (\ref{typo-layer-all2}c)--(\ref{typo-feat}f). The semantics is thus constrained by the syntax: {\vz} cannot have an agentive alloseme, unlike {\vd} and Voice, but the latter two cannot have a passive (existential) alloseme, unlike {\vz}. %While the number of cells in~(\lastx) is larger, this is only because I have placed the existential alloseme of Voice in its own row. The exact same thing could be done for~(\ref{typo-layer-all2}).


\section{Hebrew with Layering} \label{aas:hebrew}
The last issue to be tackled here is whether the current theory is a necessary development. Could we tweak Layering to account for the patterns analyzed in this book?

Recall that Hebrew has trivalent morphological marking, and crucially that verbs in {\tkal} might be unaccusative or transitive (Chapter~\ref{chap:voice}).
%\ex\label{ex:alternations-heb}
%	\begin{tabular}{cll|ll|ll}
%	& \multicolumn{2}{P{4.2cm}|}{causative} &	\multicolumn{2}{P{4cm}|}{underspecified}	& \multicolumn{2}{P{4.2cm}}{anticausative}\\\cline{2-7}
%	\phantom{Semantics} & \multicolumn{2}{c|}{\thif}	&	\multicolumn{2}{c|}{\tkal}	& \multicolumn{2}{c}{\tnif}\\
%	& \emph{heexil}	& `fed' &	\emph{axal}	& `ate'	&	\emph{neexal}	& `was eaten' \\
%	& \emph{hextiv}	& `dictated' &	\emph{katav}	& `wrote'	&	\emph{nixtav}	& `was written' \\\cdashline{4-5}
%	& \emph{\textbf{he}p\textbf{i}l} & `dropped' & \emph{nafal}	& `fell' & \multicolumn{2}{c}{---}\\
%	\end{tabular}
%\xe

\ex\label{ex:aas:alternations-heb2}Basic analysis of the templates as proposed in this book:\\
	\begin{tabular}{ll|ll|ll}
	 \multicolumn{2}{P{4.7cm}|}{\textbf{\vd}}	&	\multicolumn{2}{P{4cm}|}{\textbf{Voice}}	& \multicolumn{2}{P{4cm}}{\textbf{\vz}}\\\hline
%	\phantom{Semantics} & \multicolumn{2}{c|}{causative} &	\multicolumn{2}{c|}{transitive}	& \multicolumn{2}{c}{anticausative}\\\cline{2-7}
	\multicolumn{2}{c|}{\thif}	&	\multicolumn{2}{c|}{\tkal}	& \multicolumn{2}{c}{\tnif}\\
	\emph{heexil}	& `fed' &	\emph{axal}	& `ate'	&	\emph{neexal}	& `was eaten' \\
	\emph{hextiv}	& `dictated' &	\emph{katav}	& `wrote'	&	\emph{nixtav}	& `was written' \\ %\cdashline{4-5}
	\emph{hepil} & `dropped' & \emph{nafal}	& `fell' & \multicolumn{2}{c}{---}\\
	\end{tabular}
\xe

Trying to analyze Hebrew using the machinery of Layering will require us to take {\tkal} as the spell out of v, not Voice as in Chapter~\ref{chap:voice}. Then, the distinction between active and non-active Voice would derive the distinction between active verbs in {\tkal} and verbs in {\tnif}. To derive the active verbs in {\thif}, additional functional structure would be necessary (since there are only two Voice heads under Layering, regular/transitive and non-active). This alternative approach to Hebrew is summarized in~(\nextx).

\ex Layering-style analysis of Hebrew (to be rejected):
\xe
\begin{small}
\hspace{-2em}\begin{tabular}{l||c|c|c|c}
			&	unmarked anticausative	&	unmarked transitive &	marked anticausative	& marked transitive\\\hline
		Derivation					& \Tree [.vP ] 		&	\Tree [.VoiceP [.DP ] [ [.Voice ] [.vP ] ] ]	&	\Tree [.VoiceP [.{Voice\{\zero, \zero\}} ] [.vP ] ] 	& \Tree [.\gsc{CAUS}P [.\gsc{CAUS} ] [. [.DP ] [ [.Voice ] [.vP ] ] ] ] \\
		Spell-out					& \multicolumn{1}{c}{\tkal}	&	{\tkal}					& {\tnif}	& \thif\\
	\end{tabular}
\end{small}

I can identify a number of problems with this approach.

First, it is not possible to treat {\tkal} akin to English or Greek unmarked alternations because {\tkal} does not have the zero-alternation. If \emph{kafa} `froze' is an unaccusative verb derived without Voice, adding Voice should simply give us transitive `froze' with identical pronunciation, contrary to fact. While it is true that various constraints dictate whether zero-derivation is possible in a given language, it is striking that the alternation is not possible in Hebrew (setting aside the discussion in Chapter \ref{vd:caus:labile}). This version of Layering predicts that zero-alternation should be fairly prevalent. Note that this point is crucial to maintaining the current view. If I am wrong about this, the way is paved for a theory of Hebrew-as-Greek consisting of vP, VoiceP and {\vz}P (without {\vd}).

Second, there is no convincing argument for positing extra structure in {\thif} (``lexical'' causatives, Chapter~\ref{vd:caus:marked}). This template seems to be as integrated into the morphological system as any other, meaning that {\vd} is as integrated into the system as the other heads. In \cite{kastner18nllt} I explained how the current system derives a number of allomorphic interactions correctly. The behavior of {\vd} with regards to constraints such as locality in allomorphy is qualitatively identical to that of {\vz} and Voice. Adding structure for {\thif} would lose a number of morphophonological generalizations regarding the interplay of roots and functional structure.

Third, it is unclear what the relevant function of an additional head would be. As a  distinct causative head, it would be an odd type of causativizer, since it would not necessarily add any argument; it would take a transitive structure and turn it into a different transitive structure. As discussed in Chapter~\ref{vd:caus}, transitive verbs in {\thif} are lexical causatives, not analytic causatives.

And fourth, nominalizations clearly contain the morphology of the underlying verbal template (Chapter~\ref{passn:n}). But then why should deverbal nouns be derived from Voice when they can all be derived from v?

We could also imagine a mixed view, under which what I have called {\vz} is not a syntactic requirement on Spec,Voice but a semantic one: {\vz} simply has only the non-active allosemes, but does not ban DPs in its specifier as such. This would mean, for example, that in the reflexive derivation of Chapter~\ref{vz:va:vzva:refl} the Theme raises to Spec,{\vz} and not to Spec,TP. Such a view also opens the door for derivations in which an internal argument raises through Spec,VoiceP, as has been proposed for some ergative languages \citep{deal19li}. The problem with such a system is conceptual, in that {\vz} now has no syntactic feature distinguishing it from Unspecified Voice. Given that Unspecified Voice has a non-active alloseme, it is not clear what this semantic {\vz} would be signaling to the learner. This view also severs the similarity between modified {\vz} (which has no syntactic requirements and does not introduce a thematic role) and {\pz} or a modified {\pz} (which has a syntactic requirement and does introduce a semantic role).\footnote{Thanks to Yining Nie for noting this possibility and its implications.}

One final alternative would maintain the basics of Layering while placing an emphasis on processes of Impoverishment. This possibility is discussed next.

%\paragraph*{Non-active nominalizations.} Greek nominalizations cannot have a marked anticausative as their base. In other words, there is no morphological sequence *\root{root}-\gsc{ACT}-\gsc{NMLZ} in Greek. One could assume that this is because nominalizations are incompatible with Voice, but this cannot be true from a crosslinguistic perspective (see Chapter @ on non-active marking in Hebrew). It is however predicted if specifically {\vz} cannot be nominalized, as in Chapter @.


\section{An alternative with Impoverishment} \label{aas:jim}

\label{r1:g:2c2}As mentioned already, in the current system [D] acts a feature than needs to be checked or a sort of ``filter'', rather than a structure-building feature. Work in the Layering tradition---most explicitly that of \cite{schaefer08} and \cite{wood15springer}---therefore differentiates between Voice with \{D\}, which necessarily projects a specifier, and Voice with empty \{\}, which does not. Recent discussions with Jim Wood (p.c.~Sep-Nov 2019) help clarify how this specific view of [D] and Merge could be maintained in the face of the Hebrew data. I present this alternative first in Section~\ref{aas:jim:pros}, and then list my reasons for rejecting it in Section~\ref{aas:jim:cons}.

Since much of the discussion will have to do with ``triplets'', let me recall the empirical picture. The most complicated cases are those where a given root occurs in the three templates {\tkal}, {\tnif} and {\thif}. These are the ones I take to be simplest structurally, as they do not involve {\va} or Pass (and the associated syntactic-semantic complexities). As far as I know, there is no curated list of all such triplets in Hebrew. Searching the database of \cite{ehrenfeld12}, I found 147 roots that are instantiated in all three templates out of 1,875 roots in total. Not all make for clear triplets, and so I searched for good examples by hand. The table in~(\nextx) lists the ten clearest cases I have found, in which a semantic relationship holds between all three forms and at least two of these are transparently related.
\ex \label{aas:ex:triplets}
	\begin{tabular}{ll|>{\em}ll|>{\em}ll|>{\em}ll}
& Root & \multicolumn{2}{c|}{\tkal} &	\multicolumn{2}{c|}{\tnif} & \multicolumn{2}{c}{\thif}\\\hline
a.& \root{axl} & axal 	& ate 	& neexal 	& was eaten 	& heexil 	& fed\\
b.& \root{xʃb} & xaʃav 	& thought 	& nixʃav 	& was considered 	& hexʃiv 	& considered\\
c.& \root{jda} & jada 	& knew 	& noda 	& was known 	& hodia 	&announced\\
d.& \root{ktb} & katav 	& wrote 	& nixtav 	& was written 	& hextiv 	& dictated\\
e.& \root{m{\ts}a} & matsa 	& found 	& nimtsa 	& was found 	& hemtsi 	& invented\\
f.& \root{sgr} & sagar 	& closed 	& nisgar 	& was closed 	& hesgir 	& extradited\\
g.& \root{ark} & arax 	& edited 	& neerax 	& was edited 	& heerix 	& estimated\\
h.& \root{pnj} & pana 	& faced 	& nifna 	& turned towards 	& hifna 	& directed towards\\
i.& \root{krj} & kara 	& read 	& nikra 	& was read 	& hekri 	& read out\\
j.& \root{raj} & raa 	& saw 	& nira 	& was seen 	& hera 	&showed\\
	\end{tabular}
\xe

	\subsection{An Impoverished Layering theory of Hebrew} \label{aas:jim:pros}
		\subsubsection{Basics}
This alternative attempts to maintain the structure-building view of [D], whereby there is no Unspecified Voice, only {\vds} which projects a specifier and {\vzs} which does not. The [D] feature, like any other feature, can undergo Impoverishment.

The most basic spell-out rules, to be revised immediately, are in~(\nextx). The transitive Voice head is spelled out as the ``causative'' template {\thif}, and the non-active Voice head as the non-active template (\tnif).
\pex Initial VIs (to be revised)
	\a {\vds} \lra~{\thif}
	\a {\vzs} \lra~{\tnif}
\xe

We also know that some roots simply need to appear in a certain templates. In particular, some agentive verbs do not appear in {\thif} but in {\tkal} (Chapter~\ref{chap:vd}), and some unaccusative verbs do not appear in {\tnif} but in {\tkal} (Chapter~\ref{chap:vz}). Calling these simply \root{Root1}, \root{Root2} and so on for time being, we have the revised VIs in~(\nextx).\footnote{The listed roots could appear under either the marked or unmarked template in each case; I give them in the marked cases here.}

\pex Revised VIs (to be revised further)
	\a {\vds} \lra~$\begin{cases}
		\text{\thif} & \text{/ \trace~\root{Root1}, \root{Root2}, \dots}\\
		\text{\tkal} & \\
		\end{cases}$
	\a {\vzs} \lra~$\begin{cases}
		\text{\tnif} & \text{/ \trace~\root{Root3}, \root{Root4}, \dots}\\
		\text{\tkal} & \\
		\end{cases}$
\xe


		\subsubsection{Appl}
It is now that the challenge posed by triplets can be re-introduced. Here the theory makes use of the applicative head Appl. The intuition is that when an alternation holds between {\tkal} and {\thif}, the latter form can be derived by using an Appl head.

Since I have spent some time discussing the relationship between {\tkal} and {\thif} in Chapter~\ref{chap:vd}, I will not repeat the details here; see the list in~(\ref{vd:ex:triplets-caus}) for some examples. This idea does receive empirical support from pairs like those in~(\ref{aas:ex:triplets}a) and~(\ref{aas:ex:triplets}d), where `feed' and `dictate' arguably take an additional argument when compared to `eat' and `write'. But it is a bit more of a stretch with cases like~(\ref{vd:ex:triplets-caus}e), \emph{daxak} `shoved' $\sim$ \emph{hedxik} `suppressed (emotions)'. In this case, the latter would have to involve some kind of low Appl-\gsc{INTO}, and it would be far less clear what it means to be an applied argument.

In any case, if we were to accept this premise, we would have the revised VIs in~(\nextx), again not a final proposal.

\pex Revised VIs (pre-final version)
	\a {\vds} \lra~$\begin{cases}
		\text{\thif} & \text{/ \trace~\textbf{Appl}, \root{Root1}, \root{Root2}, \dots}\\
		\text{\tkal} & \\
		\end{cases}$
	\a {\vzs} \lra~$\begin{cases}
		\text{\tnif} & \text{/ \trace~\root{Root3}, \root{Root4}, \dots}\\
		\text{\tkal} & \\
		\end{cases}$
\xe

		\subsubsection{Additional diacritic}
Finally, we need to account for the triplets that cannot be handled with Appl. Consider a triplet like~(\ref{aas:ex:triplets}e), where it is highly doubtful whether `invent' is an applicative version of `find'. Since this root appears with both templates, they cannot be differentiated by listing a \root{Root5} in both cases for {\vds}. Some additional diacritic (or feature) would be necessary, call it simply F like in~(\nextx).

\pex \label{aas:ex:jim-vis}VIs in the Impoverishment alternative (final version)
	\a {\vds} \lra~$\begin{cases}
		\text{\thif} & \text{/ \trace~\textbf{F}, Appl, \root{Root1}, \root{Root2}, \dots}\\
		\text{\tkal} & \\
		\end{cases}$
	\a {\vzs} \lra~$\begin{cases}
		\text{\tnif} & \text{/ \trace~\root{Root3}, \root{Root4}, \dots}\\
		\text{\tkal} & \\
		\end{cases}$
\xe

	
	\subsection{Discussion} \label{aas:jim:cons}
I have tried to lay out this alternative as explicitly as possible so that its strengths and weaknesses may be evaluated. The main gain would be a theory-internal one: as mentioned at the outset, a theory with only {\vds} and {\vzs} preserves a specific conceptualization of Merge, one which is no longer available once Unspecified Voice enters the picture in a trivalent system. An additional benefit is a closer connection with extant theories insofar as Appl can be used in similar fashion.

These strengths are outweighed by the weaknesses, in my eyes, and these are of conceptual as well as empirical nature. Starting with the use of Impoverishment, the following points arise. First, since the choice of template for a given root has syntactic an semantic effects, this means that Impoverishment would have to apply in the syntax proper and not early in Spell-Out, as commonly assumed, where it does not have semantic effects (e.g.~\citealt{harbour03}). Second, Impoverishment would need to be triggered by particular roots and not by marked features or feature combinations. Third, this would only happen some of the time, because many roots can appear in more than one template, e.g.~in both {\thif} and {\tkal}. This last point could be discounted due to the use of Appl, which is the next point of discussion.

As already mentioned, the definition of semantics of Appl might be stretched fairly thin, depending on which specific cases it should be applied to. In addition, even in the cases in which an applicative \emph{semantics} is easier to motivate, an applicative \emph{syntax} is only optional. That is to say, even in~(\ref{aas:ex:triplets}i) \emph{hekri} `read out', a Goal argument does not have to be expressed. This issue could be addressed by assuming that an expletive Appl$_{\text{\{\}}}$ is possible (\citealt{wood15springer}, p.c.) with no argument introduced in its specifier.

As a last conceptual point, the system as a whole makes reference to an additional mechanism sorting out triplets, namely the diacritic/feature [F]. Taken together, this approach is increasingly reminiscent of \cite{arad05}, where the grammar lists conjugation classes a given root participates in. It is also noteworthy that Appl and [F] form a natural class somehow. 

Finally, there is also one important empirical point: this alternative is able to derive labile alternations in {\tkal} as a general rule. Imagine a root \root{Root5} which is not on either of the lists for {\vds} and {\vzs} in~(\ref{aas:ex:jim-vis}). Then it would be spelled out as {\tkal} for {\vds}, but also as {\tkal} for {\vzs}. As noted in Chapter~\ref{vd:thif:inch}, this is not the case with any verb except for perhaps \emph{a{\ts}ar} `stopped'.

In conclusion, even though I have identified various reasons to doubt a Layering approach to Hebrew (whether implemented as in this section or as in Section~\ref{aas:hebrew}), it is important to acknowledge that not all of the explanations given here are particularly deep. For instance, I have implicitly assumed that all Hebrew verbs need Voice, in contrast to existing assumptions for certain verbs in English, German and Greek. This assumption raises the question of whether Voice should be obligatory for all verbs in all languages, a point leading us to the concluding remarks for this comparison.


\section{Conclusion} \label{aas:conc}
This chapter presented a direct comparison of the theory developed in this book with what I have called the Layering approach, the prevalent syntax-based theory of transitivity alternations as implemented by \cite{schaefer08,schaefer17oup} and \cite{layering15}. I have identified a number of weaknesses with the Layering approach and illustrated how its considerable explanatory power can be mirrored in the current approach. Furthermore, I have identified cases which require a concrete departure from the features of Layering.

Aside from the specific weaknesses discussed here, the main empirical difference underlying the most substantial need for a revised theory is that the Layering theories were based on an exploration of anticausative marking, not of causative marking (see the discussion in Chapter \ref{chap:vd}). The languages on which work in this approach is based are languages that show anticausative marking, including English \citep{myler16mit}, German \citep{schaefer17oup} and Greek \citep{spathasetal15}, but also Albanian \citep{kallulli13}, Icelandic \citep{wood15springer}, Latin \citep{embick04,kastnerzu17} and Spanish \citep{schaefervivanco16}.

The theory developed here on the basis of Hebrew makes explicit room to accommodate causative marking. The trivalent view of Voice is most useful when considering languages that show reflexes of this marking, including recent work on Japanese \citep{oseki17nyu} as well as a number of Austronesian and Polynesian languages \citep{nie17}. In other words, it becomes clear that causative marking has much to tell us about argument structure alternations, alongside anticausative marking and ideally in a joint theory as the one presented thus far.

%In \citet[99--100]{layering15}, the authors give three reasons motivating the analytical distinction between unmarked causatives (just vP) and marked causatives (vP and expletive Voice). These three can be applied directly to the study of marked causatives.
%
%\paragraph*{Morphology.} In the languages under consideration, all non-active constructions share the same morphology (anticausatives, passives and reflexives all have \gsc{NACT}/\gsc{SE}). The Layering analysis provides a useful way for all non-active constructions share morphology related to Voice (anticausatives, passives and reflexives all have \gsc{NACT}/\gsc{SE}).\\
%	The current way of looking at things overlaps with this claim to a large extent. In this book, all transitivity-related morphology is related to Voice, including umarked anticausatives and causatives (Chapter \ref{chap:voice}). Marked anticausatives still share the same morphology with other marked non-active structures, but marked causatives are added to the mix.
%
%\paragraph*{Transitive syntax.} Marked anticausatives in German and Romance have transitive syntax, so the expletive can be generated in Spec,VoiceP.\\
%	This conclusion is also shared by the current approach.
%	
%\paragraph*{Markedness.} Marked anticausatives appear on verbs ``which express changes which are conceived of as occurring less likely spontaneously''. Since unmarked anticausatives do not contain Voice, they are less marked.\\
%	This point relates to a topic which has come up throughout the book but cannot receive in-depth treatment, that of the interaction of individual roots with the functional material. But in any case, for this generalization to hold, all that is necessary is a difference in \emph{markedness} between marked and unmarked anticausatives. This much can be achieved by assuming that {\vz} is more marked than Voice. If this third reason is on track, we also predict a similar difference to arise between marked and unmarked causatives. It is too early to tell whether this is the case, but barring more extensive crosslinguistic work, the initial treatment in Chapter \ref{vd:caus} certainly points in this direction.

To conclude, let us put the pieces together and speculate on what the Voice inventory of a given language might be. I see three possibilities.

On the one hand, it is possible that all languages have the trivalent system of~(\ref{ex:aas:typo-feat}). We would then assume that in English, German and so on {\vd} and Voice are syncretic. On the other hand, it might be the case that only Voice heads that are morphophonologically distinct can be argued to exist in a given language. This is essentially the view of \cite{layering15}. That work proposed that learners of English do not hypothesize the existence of expletive Voice because there is no morphological evidence for it. If this is the case, then languages with marked anticausatives and marked causatives are trivalent languages, whereas languages with only marked anticausatives are Layering languages. Finally, one could also come up with a hybrid view, in which all languages are at least active/non-active Layering languages, even when there is no morphological evidence (as in English), following from the basic active/non-active distinction that came up in the context of causative marking (Chapter \ref{vd:caus}).

I will not argue for any of these views explicitly, although I do maintain that the trivalent theory is the most flexible and most constrained simultaneously (whether this flexibility means that trivalent Voice should be hard-coded into the grammar is debatable). In addition, treating transitivity alternations in terms of various features on Voice---extending the original Layering view---paves the way for a more nuanced view of what these features might be. In Chapter~\ref{vd:caus:markvoice} I speculated that French could be treated as a trivalent language, if certain prefixes spell out {\vd}. I do not know if that view can be maintained (it probably cannot) but it does highlight what new perspectives can be gained by looking at different argument structure phenomena in terms of layering with a certain feature set. In the next chapter I consider recent proposals extending the coverage of feature-based approaches beyond transitivity marking, as they interact with case and agreement.



%		\subsection{Specific advantages of the Featural approach}
%What do you do with unmarked anticausatives in Greek?
%\pex
%	\a \begingl
%		\gla ta ruxa \textbf{stegnosan} apo/me ton ilio//
%		\glb the clothes dried.\gsc{ACT} from/with the sun//
%		\glft `The clothes dried from the sun.'\trailingcitation{\citet[34]{layering15}}//
%	\endgl
%	\a \begingl
%		\gla i porta \textbf{anikse} apo moni tis//
%		\glb the door opened.\gsc{ACT} by alone hers//
%		\glft `The door opened by itself.'\trailingcitation{\citet[35]{layering15}}//
%	\endgl
%	\a \begingl
%		\gla i sakula adiase//
%		\glb the bag.\gsc{NOM} emptied.\gsc{ACT}//
%		\glft `The bag emptied.'\trailingcitation{\citet[64]{layering15}}//
%	\endgl
%\xe
%
%\gsc{NACT} marking allows a passive (e.g.~a by-phrase to be added), just like in Hebrew:
%\ex \begingl
%	\gla o tixod aspristike/*asprise apo ton petro.//
%	\glb the wall whitened.\gsc{ACT}/\gsc{NACT} by the Peter//
%	\glft `The wall was whitened by Peter.'\trailingcitation{\citet[37]{layering15}}//
%	\endgl
%\xe
%

%Next: if I understand correctly, Greek by-phrases are only possible with marked anticausatives, not with unmarked anticausatives. Where does this difference come from? Sounds like it comes from the semantics \cite[88]{alexiadouanagnostopoulou04,layering15}. It's an extra bit of lambda that expletive Voice can have.


%\addtocontents{toc}{\protect\setcounter{tocdepth}{0}}
    \chapter{The features of Voice}
\label{chap:i}


summary of generalizations

summary of proposal

summary of claim

being honest about the data/generalizations - didn't check every single verb

acquisition?


\section{i*}
In a recent account of the way argument structure is derived and interpreted, \cite{woodmarantz15} propose to reduce the overall inventory of functional heads. Working within a similar framework, in which much of the burden of deriving properties of the event lies in the semantic component, \citeauthor{woodmarantz15} suggest that non-internal arguments (external and applied arguments introduced by Voice, Appl, \emph{p} and P) are in fact variants of the same predicational head. This head is called \emph{i*}.

If \cite{woodmarantz15} are correct, the difference between \emph{p}, Appl and Voice is an illusion: they are all the same predicational head underlyingly, albeit in different contexts. Voice is but \emph{i*} that merges with a vP. Little \emph{p} is but \emph{i*} that merges with a PP. And P itself is \emph{i*} modified by a (prepositional) root.

Our goal here is not to evaluate their proposal, which is supported by conceptual considerations as well as empirical study of figure reflexives in Icelandic, the Adversity Causative in Japanese and possession in Quechua and other languages. Instead, I want to highlight one welcome point of convergence between the \emph{i*} hypothesis and my proposal for Hebrew. In the inventory of functional heads I have laid out, \vz~and \pz~are conspicuously similar: they do similar work in the syntax and have the same spell-out. If we follow the \emph{i*} hypothesis, the two \emph{should} be similar: they are the same functional head, only in different contexts.

To be clear, I do not believe that the \emph{i*} hypothesis must be true for my account to go through. But if this hypothesis is on the right track, a strong version can be formulated under which all exponents of \emph{i*} (as well as its variants \emph{i*}$_{\text{\zero}}$ and \emph{i*}$_{\{\text{D}\}}$) should be identical to each other. Such a hypothesis would immediately predict the similarity between Voice and \emph{p}---both default and silent---and that between \vz~and \pz. {Adopting the \emph{i*} hypothesis we may modify~(\ref{ex:heads-langs}) as~(\ref{ex:heads-langs2}).}
\ex\label{ex:heads-langs2}Syntactic elements in a number of unrelated languages, adopting \cite{woodmarantz15}:\\
	\begin{tabular}{l|lll}
	Head 		& Hebrew 		& Greek  		& Japanese\\\hline
	\emph{i*}   	& \tkal     & (silent)      & -\emph{e}-\\
	\emph{i*}$_{\text{\zero}}$ 	& \tnif 	& \emph{-thike}	& -\emph{r}-\\
	\emph{i*}$_{\text{\{D\}}}$	& \thif		& ?		& -\emph{s}-\\
	{\va}	& \tpie		& \emph{afto-}	& -\emph{ak}-?\\
	\end{tabular}
\xe

The \emph{i*} hypothesis would also lead us to expect similar correlations crosslinguistically, a prediction which remains to be tested as the tools employed in this dissertation and in related works such as \cite{schaefer08}, \cite{spathasetal15} and \cite{wood15springer} are extended to additional languages.


i*: CAUS=APPL in Kinyarwanda https://doi.org/10.1017/S0022226717000044
p. 64 onwards: http://www.conormquinn.com/CQuinn2006Referential-AccessDependencyInPenobscot.pdf
Mithun on Iroquian

\section{Japanese}
Oseki
The framework developed in \S\S\ref{syn:middle}--\ref{syn:templates} makes use of a number of different syntactic heads. In this section I show how these fit into a crosslinguistic theory of argument structure.

Two questions ought to be distinguished when we ask about the crosslinguistic validity of this theory:
\begin{enumerate}
	\item Does the syntactic inventory of every language always contain these heads (Voice, v and \emph{p})?
	\item Does every language have the kinds of features on these heads that Hebrew does, i.e. {\vd}, \pz, \va, etc?
\end{enumerate}
The answer to the first question is yes. I assume that Voice, v and \emph{p} are an inherent part of the syntactic system of every language. I am less certain about the applicative head Appl, but the recent proposal by \cite{woodmarantz15} which I discuss in \S\ref{syn:crosslx:woodmarantz} below allows us to reconceptualize Appl as a variant of Voice.

The answer to the second question is less clear cut and potentially more interesting. First, we need to ask what features are possible on different heads, for example on Voice.

In Hebrew, I have made the case that Voice can be [+D] as in \vd, [--D] as in {\vz} or underspecified as in default Voice. But the architecture allows any syntactic feature to appear on Voice. Nothing in the setup prohibits Voice$_{\text{[wh]}}$, for instance, which would require a \emph{wh}-phrase in Spec,VoiceP. Now granted, any theory of syntax must stipulate in one way or another which features are possible on which functional elements. One way to restrict our theory is to require only \emph{uninterpretable} features \citep{chomsky95}, being purely syntactic features, to exist on Voice. The EPP feature [D] is one such feature. This kind of solution would rely on a certain view of which features are interpretable and which are not; the notion of whether uninterpretable features are necessary has itself been questioned in recent work \citep{preminger14mit}. Ideally, our theory of features on argument-introducing heads would be part of a general theory of argument structure, feeding processes such as case assignment and specifying the triggers for A-movement.

Expanding the crosslinguistic envelope, then, is every language predicted to have the same features on the same heads as Hebrew does? Not necessarily. It is certainly possible that English, for instance, has only one Voice head, so that argument structure alternations such as those in~(\nextx) arise through the general underspecification of Voice. \pex {[}Voice [v \root{\gsc{BREAK}}~\!]]
	\a John broke the vase.
	\a The vase broke.
\xe
Another possibility is pursued by \cite{schaefer08} for German and \cite{wood15springer} for Icelandic, where similar variants of Voice are used as in our system but at times without any phonological indication. I remain neutral with regards to specific claims about these languages, though my preference is to only postulate a variant of a head (meaning a head with a marked feature on it) when there is morphophonological reason to do so.

In some languages it is possible to find overt evidence for these heads. Icelandic \vz~fits the bill \citep{wood15springer} and the Greek data discussed by \cite{spathasetal15}, reviewed in \S\ref{syn:middle:refl}, led to a theory making use of \vz~(pronounced -\emph{thike}) and \va~(pronounced \emph{afto-}). In recent {unpublished }work following similar lines, \cite{oseki16nyu} proposes that Japanese makes use of Voice, {\vz} and \vd. The following table summarizes Hebrew, Greek and Japanese. The table is simplified: \tnif~is technically epiphenomenal rather than an exponent of \vz, to name one example{ (\S\ref{syn:middle:nonactive})}.
\ex\label{ex:heads-langs}Exponents of syntactic elements in a number of unrelated languages:\\
	\begin{tabular}{l|lll}
	Head 	& Hebrew 		& Greek  		& Japanese\\\hline
	Voice   & \tkal     & (silent)      & -\emph{e}-\\
	{\vz} 	& \tnif 	& \emph{-thike}	& -\emph{r}-\\
	{\vd}	& \thif		& ?		& -\emph{s}-\\
	{\va}	& \tpie		& \emph{afto-}	& -\emph{ak}-?\\
	{\pz}	& \tnif		& ?		& ?\\
	\end{tabular}
\xe
It thus appears to be potentially useful to adopt this framework for additional languages and map out which heads and features are instantiated in which language.

This framework not only allows us to describe different languages with similar tools, it allows us to ask more fine-grained questions about crosslinguistic variation. Consider the following differences between Hebrew and Greek: The first can lead to new discoveries about Greek. The second can form part of a general theory of lexical semantics and how it interacts with the syntax.
\pex
	\a Hebrew uses \thit~for both reflexives and reciprocals (\S\S\ref{syn:middle:refl}--\ref{syn:middle:recip}). Greek uses \emph{afto}- with a nonactive base for reflexives (\S\ref{syn:middle:refl}) and \emph{alilo}- with a nonactive base for reciprocals \citep{alexiadouafto}.
	\a Hebrew Other-Oriented roots are not compatible with \va; their semantics is impoverished, (\ref{sem:thit-incho}). In Greek, only Other-Oriented roots are compatible with \emph{afto-}.
\xe

The first difference opens up the following line of questioning: are Greek reflexives and reciprocals different in ways that can inform our understanding of unaccusativity, in the way that Hebrew reflexives, reciprocals and anticausatives in \thit~mask structural differences? To the best of my knowledge, this question has not been explored yet. It is possible to ask this question now, though, since we have a vocabulary with which to describe the system and make predictions. For example, if there is no structural difference between Greek reflexives and reciprocals save for the form of the prefix, it should be the case that the same root can instantiate both constructions, though only if its lexical semantics is compatible with them. See \cite{alexiadouborerschaefer14} for related discussion of crosslinguistic interactions between the lexical semantics of roots and syntactic structure.

The second difference shows that even if we find similar classes of roots crosslinguistically (in this case Other-Oriented roots such as \root{\gsc{BREAK}} or \root{\gsc{HIT}}), they might not interact with similar elements in the same way. Again, we can develop this theory now that we have a framework for roots of different semantic classes and the syntactic elements they combine with.

%AM: We've been re-visiting the account of Japanese +D causative -s marking, with the possible correlation with obligatory -o marking (and thus no -o drop), precisely because Yining's use of +D on voice is designed to force obligatory phi agreement with the/an object (which is straightforward for ergative languages, in which the phi agreement goes along with the absolute, but less intuitive for nom/acc language, in which the spec of the +D voice will generally be able to provide phi features for T; nevertheless, the prediction is that +D voice should go along with object marking of the specific/definite/animate sort, and a type of transitivity that elevates the direct object to discourse representation status as a potential topic or focus).


	\subsubsection{Logical possibilities for combinations} \label{syn:crosslx:combinatorics}
Sections \S\S\ref{syn:middle}--\ref{syn:templates} attempted to account for a range of data by exploring the combinations of different heads in the structure. Table~\ref{table:summary-syn-rep} above provided a summary of how different heads combine in the structure. I will address two points here: what combinations are not attested within Hebrew and what combinations should be attested crosslinguistically.

The unattested combinations are listed in Table~\ref{table:unattested-syn}. The final column notes whether the \emph{non-existence} of each form is predicted under our theory (check marks show a good match of theory to data).
\begin{table}[hb] \centering
\begin{tabular}{|lllll||c|c|c|}\hline
	\multicolumn{5}{|c||}{Heads} & Syntax 	& Semantics & Predicted? \\\hline\hline
	
	a. & & \blue{\vd} &\red{\va}&	& \blue{EA}	& \red{Action} & \xmark \\

	b. & & \blue{\vd} &\red{\va}& \blue{\pz}	& \blue{EA}, \blue{EA = Figure}	& \red{Action} & \xmark \\\hline
			
	c. & & \blue{\vd} & &\blue{\pz}	& \blue{EA}, \blue{EA = Figure} & (underspecified)  & \xmark \\\hline
	
	d.& \olive{Pass} & Voice& &		& \olive{Passive}	& (underspecified) & \cmark/\xmark \\
		
	e.& \olive{Pass} & Voice& &	\blue{\pz}	& \olive{Passive}, \blue{EA = Figure}	& (underspecified) & \cmark/\xmark \\
	
	f.& \olive{Pass} & \blue{\vz}& &		& \olive{Passive}, \blue{No EA}	& (underspecified) & \cmark \\\hline	
	
	g. & & \blue{\vz} & &\blue{\pz}	& \blue{No EA}, \blue{EA = Figure} & (underspecified) & \cmark \\\hline
\end{tabular}
\caption{Unattested combinations of syntactic elements in Hebrew.\label{table:unattested-syn}}
\end{table}

The combinations in \textbf{(a)} and \textbf{(b)} predict a strongly agentive template: \vd~requires an external argument and \va~ensures that this argument has agentive semantics. But we have no evidence for such a template, which would have a prefix from \vd~and non-spirantization from \va. This gap has no principled explanation: perhaps both heads perform similar enough work functionally that such a template is not necessary. There already exists a head that requires an external argument (\vd), and there already exists a head that requires agentive semantics (\va~\!), so the system is in no need of a redundant combination of the two.

It is also possible for (a) and (b), at least technically, that when \vd~and \va~combine a null allomorph of \va~is selected: thus when a verb in \thif~is agentive, there is a silent \va~in the structure. This proposal is unfalsifiable and I will not attempt to defend it.

The structure in \textbf{(c)} would consist of an external argument saturating the Figure role passed up by \pz. In effect, this is what happens in ordinary figure reflexives. Regular Voice carries out this role since if it did not, the derivation would not converge. But I do not think that the combination in (c) is blocked for any principled reason.

It is worth noting at this point that the combinations in (a)--(c) all involve an overt variant of Voice and an overt variant of \emph{p}. Time will tell if this is a spurious correlation or whether this pattern indicates an issue with multiple overt argument-introducing heads.

The combinations in \textbf{(d)--(f)} all pertain to the structures Pass can combine with. Pass can only merge with structures that have a guaranteed external argument: {\vd} and {\va} are fine, but as seen in the table, the other heads are not. The intuition is that Pass must ``know'' that it has an external argument to quantify over. This intuition is at least partly compatible with the structures in (d)--(f), and in fact Biblical Hebrew did have a passive counterpart to the ``simple'' (Voice+v) template. That template has since been lost.

The combination in (f) is correctly predicted not to be possible since Pass has no external argument to quantify over.

Finally, the combination in \textbf{(g)} is predicted not to exist, and correctly so. If \vz~and \pz~were to combine, the Figure role of \pz~would be passed up but no external argument would be able to saturate it. The derivation would crash at interpretation; this gap is predicted.

On the crosslinguistic angle now, the theory did predict that certain heads could combine. Since \va~is by definition a modifier of Voice, it may combine with Voice, yielding \tpie, or with \vz/\pz, yielding \thit. Because the individual elements all have their own syntax and semantics, the result of combining them is to a large extent predictable.

The combination of \vz~and \va~has already been explored for Greek in \S\ref{syn:middle:refl}, where it was shown that the two bring about reflexive readings. This is exactly the kind of pattern that is predicted to arise when these heads are in the structure, though as discussed, the rest of the grammar (including the root) needs to be compatible with the construction. At some point along the line the theory should be able to predict how heads combine in a given language, depending on other aspects of the grammar.

\cite{oseki16nyu} does suggest that the Japanese causative \emph{-sase} should be decomposed into an Applicative head \emph{-s-} and the \vd~morpheme -\emph{s}-{, in a system which allows for various combinations of Voice heads}. Other combinations thus do seem to be attested in one language where another cannot combine them; in Hebrew this {specific }combination cannot be examined because Hebrew does not have a ``morphological'' Appl affix; benefactive and malefactive arguments are introduced using the preposition \emph{le-} `to'.
\ex \begingl
    \gla ha-arje biʃel ʃu'it (\underline{la}-jeladim)//
    \glb the-lion cooked beans (\underline{to.the}-children)//
    \glft `The lion cooked (the children) beans'//
    \endgl
\xe

Standard Arabic and some Arabic dialects might be more informative. In these languages an ``Applicative template'' can introduce an applied argument (an extra indirect object) without need for a preposition \citep{alkaabi12}. It remains to be seen how the Hebrew framework can accommodate patterns in Arabic.

Applicatives bring up a more general point which relates to the kinds of argument-introducing heads in the syntax: the framework as a whole now allows for Voice heads, \emph{p} heads and Appl heads. All are empirically necessary, but recent work suggests that they need not be distinguished theoretically.

\section{Valuing features} \label{i:nie}
Nie

Legate

Wurmbrand

\section{Notes}
Triplets.
Legate conversation at CamVoice: she's against underspecified Voice. Instead, have active Voice, nonactive Voice, and then choose the zero allomorph based on some other structural difference between the transitive constructions. So one practical question is, is there some generalization about the internal arguments of thif vs transitive tkal, or of tnif vs unaccusative tkal.
	I don't think so, but it's worth looking at systematically. And worst case, I'll have a foil to argue against.
	The relevant question for me is point where vP is one event and then Voice does something predictable. Because triplets show that this is usually not the case: sagar-nisgar-hesgir. So it's not really the case that there's a "closing" vP and then you either block an external argument (tnif) or not (tkal), and also can require one (thif).
	So really what I'm forced to say is that the connection between sagar and nisgar is just as tight as between nisgar and hesgir.
	There aren't that many unaccusative doublets with tkal-tnif either (if any).
	I think this really requires figuring out the empirical state of affairs, quantitatively even (causativization isn't the main function of thif anyway).
	Now have a few examples in slides.

AM:  In short, it's not clear to me that you're going in a different direction than the agreement/valuation one -- you just have not yet emphasized the interplay of factors associated with ergativity, obligatory transitive verbs, and the featural EPP.
In filling out your typology of structures, with possible allosemes for the different voice heads, you might include the +D -s suffix of Japanese as one that, in the case of adversity causatives, does not have the agentive/causer semantics (but still has a thematic external argument, since the possessor role is "passed up" the tree).
And of course there are the deponent verbs --  -D voice that nevertheless has agentive semantics.

%    \chapter{Conclusion}
\label{chap:conc}

summary of generalizations

summary of proposal

summary of claim

being honest about the data/generalizations - didn't check every single verb

acquisition?
%\addtocontents{toc}{\protect\setcounter{tocdepth}{4}}
%    \chapter*{Appendix: Some Data}
%     \addcontentsline{toc}{chapter}{Appendix: Some Data Here}
%    \input{f_appendixA}


%    \begin{appendices}

    % The following commands redefine example numbering for appendices.
    % Warning - once these commands are given, examples in footnotes will no longer work properly, and need to be given
    % labels manually.
%    \renewcommand{\NormalEx}{\refstepcounter{ExNo}\Exformat[(\thechapter\arabic{ExNo})]}
%    \renewcommand{\theExNo}{(\thechapter\arabic{ExNo})}
%    \setcounter{tocdepth}{2}

%    \appendixchapter{appendixA}{Corpus Experiment Data}

%    \end{appendices}

    \backmatter

    \addcontentsline{toc}{chapter}{Bibliography}
    \bibliographystyle{linquiry2}
    \bibliography{lingxbib}  % put the name of your bibliography here; make sure LaTeX can find it!

\end{document}
