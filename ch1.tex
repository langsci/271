\label{chap:intro}
\section{Introduction}
The aim of this monograph is to present a new theory of argument structure alternations, one which is anchored in the syntax but has systematic interfaces with the phonology and the semantics. Virtually every theory of argument structure takes as its starting point alternations such as those in~(\nextx)--(\anextx). In English~(\nextx) the causative and anticausative variants do not differ in their morphology. In German the predominant situation is one in which a reflexive pronoun appears in the anticausative variant, (\anextx).
\pex
	\a Meg opened the door. \hfill (causative)
 	\a The door opened.			\hfill (anticausative)
\xe
\pex
	\a \begingl
		\gla\rightcomment{(causative)}Florian \"offnete die T\"ur.//
		\glb Florian opened the door//
		\glft `Florian opened the door.'//
	\endgl
	\a \begingl
		\gla\rightcomment{(anticausative)}Die T\"ur \"offnete \textbf{sich}.//
		\glb the door opened \gsc{REFL}//
		\glft `The door opened.'//
	\endgl
\xe	

Various questions arise in connection with these seemingly simply patterns, many of which have been explored in much recent work \cite[e.g.][]{unaccusativity95,schaefer08,layering15}: what kind of morphological marking appears on the different variants? Is there a sense in which one is derived from the other, or do the two share a common base? Which predicates are marked as causative and which as unaccusative crosslinguistically?

The degree of variation both within and across languages is substantial. However, most studies on argument structure have analyzed this aspect of the syntax-semantics interface through the lens of a language with relatively impoverished morphology. Each of these languages has contributed much to our understanding of argument structure, to be sure: the English labile alternation shines light on which predicates are likely to be marked in which way \citep{haspelmath93,unaccusativity95,koontzgarboden09}; the French, German and Spanish alternations bring in many aspects of cliticization, binding and agreement \citep{labelle08,schaefer08,cuervo14}; the Greek alternation shows consistent morphological marking for at least one class of predicates \citep{alexiadoudoron12,layering15}; and more recent work on Icelandic has identified ways in which argument structure alternations can be correlated with morphological processes and \cite{wood14nllt,wood15roots,wood15springer}. Yet this line of work has the drawback that these languages usually show only binary morphological distinctions, if any: either the causative variant is marked, or the anticausative one is marked (or neither is, as in the labile alternation). This problem also persists with some larger-scale typological surveys \citep{haspelmath93,arad05}.



This template usually appears in the literature as \emph{h\textbf{i}XYiZ}, with an /i/-/i/ vocalic pattern. Yet contemporary speakers use /\textipa{E}/ \citep{trachtman16}, and so I transcribe ``e'' throughout. Conversely, the initial /h/ is usually dropped in speech but I retain it for two reasons. First, /h/ is still pronounced by some older speakers and certain sociolinguistic groups, often marginalized ones \citep[cf.~][]{schwarzwald81biu,gafter14phd}. And second, the initial \emph{h}- should help non-Semitist readers to distinguish this template from other ones.

\section{Warm-up: Some data}
We will focus on three verbal forms in Hebrew (\emph{templates}): {\tkal}, {\tnif} and {\thif}.

	\subsection{{\tkal} varies in its argument structure (underspecified)}
\pex Transitive:
	\a \begingl
		\gla teo \textbf{axal} et ha-laxmania//
		\glb Theo ate \gsc{ACC} the-bread.roll//
		\glft `Theo are the bread roll.'//
	\endgl
	\a \begingl
		\gla ha-balʃan \textbf{katav} et ha-maamar ha-arox//
		\glb the-linguist wrote \gsc{ACC} the-article the-long//
		\glft `The linguist wrote the long article.'//
	\endgl
\xe

\pex Unergative:
	\a \begingl
		\gla teo \textbf{rakad} ve-rakad ve-rakad//
		\glb Theo danced and-danced and-danced//
		\glft `Theo danced and danced and danced.'//
	\endgl
	\a \begingl
		\gla teo \textbf{halax} kol ha-boker//
		\glb Theo walked all the-morning//
		\glft `Theo walked all morning long.'//
	\endgl
\xe

\pex Unaccusative:
	\a \begingl
		\gla \textbf{nafal} le-teo ha-bakbuk//
		\glb fell to-Theo the-bottle//
		\glft `Theo's bottle fell.'//
	\endgl
	
	\a \begingl
		\gla ha-bakbuk \textbf{kafa} ba-makpi//
		\glb the-bottle froze in.the-freezer//
		\glft `The bottle froze in the freezer.'//
	\endgl
\xe


	\subsection{{\tnif} is non-active}
\pex
	\a \begingl
		\gla \textbf{niʃbar} l-i ha-ʃaon//
		\glb broke to-me the-watch//
		\glft `My watch broke.//
	\endgl
	\a \begingl
		\gla \textbf{neelam} l-i ha-tamsir//
		\glb disappeared to-me the-handout//
		\glft `My handout disappeared.'//
	\endgl
\xe

	\subsection{{\thif} is active}
\pex
	\a \begingl
		\gla kevin \textbf{heexil} et ema (duvdevanim)//
		\glb Kevin fed \gsc{ACC} Emma blueberries.//
		\glft `Kevin fed Emma (blueberries).'//
	\endgl %Emma is a dog
		
	\a \begingl
		\gla ha-zamar \textbf{herkid} et ha-orxim//
		\glb the-singer made.dance \gsc{ACC} the-guests//
		\glft `The singer made the guests dance.'//
	\endgl
\xe

\newpage
	\subsection{That's all we need}
\pex Importantly, we can find triplets:
	\a Causative {\thif}\\
		\begingl
		\gla ha-more \caus{hextiv} (la-talmidim) et reʃimat ha-nosim//
		\glb the-teacher dictated to.the-students \gsc{ACC} list.of the-topics//
		\glft `The teacher dictated the list of topics (to the students).'//
	\endgl
	
	\a Transitive {\tkal}\\
		\begingl
		\gla ha-talmidim \textbf{katv-u} et ha-nosim//
		\glb the-students wrote-\gsc{PL} \gsc{ACC} the-topics//
		\glft `The students wrote the topics down.'//
	\endgl
	
	\a Non-active (mediopassive) {\tnif}\\
		\begingl
		\gla ha-xiburim \anticaus{nixtev-u} (al-jedej ha-talmidim)//
		\glb the-essays were.written-\gsc{PL} by the-students//
		\glft `The essays were written (by the students)'.//
	\endgl
\xe



\section{Traditional descriptions of the Hebrew verbal system} \label{sec:tradition}
Linguists and non-specialists who encounter a Semitic language like Hebrew for the first time often find themselves scratching their heads in an attempt to come to terms with the language's distinctive morphological system, built around ``roots'' and ``patterns''. Many early speakers of Modern Hebrew were such head-scratchers themselves: the language was revived in the late 19th century by individuals who, for the most part, were not native speakers of Semitic languages. The language nevertheless retained the Semitic morphology of its classical predecessor. On the surface, Hebrew is very different from European languages, or perhaps from any non-Semitic language. The question of how languages differ from one another is a familiar one from work in the generative tradition which often turns the question on its head, asking how languages are fundamentally similar.

As this dissertation is a study of the verbal system of Hebrew, I will make repeated reference to ``roots'' and ``templates'' (the latter also called ``patterns'', ``measures'', ``forms'' and \emph{binyanim}) as the two main components of the verb. I reserve the terms ``templates'' for the systematic verbal forms and ``patterns'' for the systematic nominal and adjectival forms. These traditional terms have been used, as far as I know, for as long as the verbal systems of Hebrew and other Semitic languages have been documented. \cite{ussishkin00phd} mentions a number of works on Hebrew which use roots and templates as integral parts of the system, including \cite{gesenius}---perhaps the best-regarded grammar of Biblical Hebrew---as well as \cite{bopp1824}, \cite{ewald1827}, \cite{harris41} and \cite{chomsky51}. For Arabic, he mentions \cite{desacy1810} as one example among many of older works which make direct reference to roots and templates.

The nature of the root was already debated by the traditional Arabic grammarians of Basra and Kufa in the 8th Century, according to \citet[563ff]{borer13oup} who herself cites \cite{owens88}. Turning to more recent works, \cite{borer13oup} also cites foundational contributions by \cite{berman78}, \cite{bolozky78,bolozky99} and \cite{ravid90}, all relying on the root and the template as at least descriptive notions. To this list I hasten to add \cite{rosen77}. I cannot hope to do justice here to the vast modern-day literature on Modern Hebrew, much of which has been published in Hebrew. The interested reader may want to consult the works of Yehoshua Blau, Reuven Mirkin, Uzzi Ornan and Haim Ros\'en, among many others; the latter two, in particular, have authored work that may be more accessible to generative linguists. In \S\ref{sec:jjmcc} I discuss contemporary work on Semitic morphology in the generative tradition; for now, we simply establish that roots and patterns have been invoked throughout the ages in descriptions of the Semitic system. One goal of this dissertation is to evaluate the theoretical status of these notions for Semitic and for natural language in general.

To see how the system is traditionally conceived of, let us take as a starting point the essay by \cite{schwarzwald81} which begins with ``the traditional classification'' of template meanings. I have added examples of the alternations to this classification. {``\dgs{Y}'' marks a non-spirantized consonant, \S\ref{sec:data:notation}.}
\ex A na\"ive classification of Hebrew templates \citep[131]{schwarzwald81}:\\
	\begin{tabular}{lccp{0.0cm}llll}
		& \textbf{Active} & \textbf{Passive} & && & & \\
	\textbf{Simple} & \tkal & \tnif & & \root{sgr} & \emph{sagar} & \emph{nisgar} & `closed'\\
	\textbf{Intensive} & \tpie & \tpua & & \root{tpl} & \emph{tipel} & \emph{tupal} & `treated'\\
	\textbf{Causative} & \thif & \thuf & & \root{kns} & \emph{hexnis} & \emph{huxnas} & `inserted' \\
	\textbf{Reflexive or reciprocal} & \multicolumn{2}{c}{\thit} & & \root{xb\dgs{k}} & \multicolumn{2}{c}{\emph{hitxabek}} & `hugged' \\
	\end{tabular}	
\xe
As \citeauthor{schwarzwald81} immediately points out herself, this classification is misleading. The relationships between the templates (the argument structure alternations) are not always predictable and most templates have additional meanings beyond those listed in~(\lastx). For example, there is little way to predict what the root \root{rʃm}, which has to do with writing down, will mean when it is instantiated in a given template. In the ``simple'' template {\tkal} we substitute the consonants in \root{rʃm} for X, Y and Z and derive \emph{raʃam} `wrote down'. In the ``middle'' template \tnif, \emph{nirʃam le-} means `signed up for', against the characterization of \tnif~as ``simple passive'' in~(\lastx). In the ``intensive middle'' \thit, \emph{hitraʃem me-} means `was impressed by', challenging the characterization of \thit~as ``reflexive or reciprocal'' in~(\lastx).

The only cells of the table which are completely predictable are the two passive templates {\tpua} (``intensive passive'') and {\thuf} (``causative passive''). The other templates constrain the possible meaning in ways that have eluded precise specification. For example, while it is clear that many verbs in \tnif~are passive-like, not all verbs in that template are. In Chapter 2 we will see that the syntax and semantics of the system can nevertheless be analyzed within a constrained theory of morphosyntax. I will make precise what the unique contribution of each template is and how that contribution comes about in the syntax. We will then be able to identify the {}role of the root in selecting between different possible meanings for the verb in a given template.

The verbs in~(\lastx) are all given in the 3rd person masculine singular past tense -- the citation form. Pending minor complications that will be addressed in Chapter 3, the actual conjugation of a given form across tenses and person/number/gender features is completely predictable, as~(\nextx) exemplifies for the \tpie~template. That is to say, even though the meaning of a given verb cannot be immediately guessed in its entirety, the morphophonological form is predictable.
\ex \label{table:piel}Tense and agreement marking in \tpie.\\
%\begin{small}
	\begin{tabular}{|l||l|l||l|l||l|l|} \hline
		& \multicolumn{2}{c||}{Past} & \multicolumn{2}{c||}{Present} &  \multicolumn{2}{c|}{Future} \\
		& \gsc{M} & \gsc{F} & \gsc{M} & \gsc{F} & \gsc{M} & \gsc{F} \\\hline\hline
		1\gsc{SG} & \multicolumn{2}{l||}{XiY̯aZ-ti} & me-XaY̯eZ & me-XaY̯eZ-et & \multicolumn{2}{l|}{je-XaY̯eZ}\\\hline
		1\gsc{PL} & \multicolumn{2}{l||}{XiY̯aZ-nu} & me-XaY̯Z-im & me-XaY̯Z-ot & \multicolumn{2}{l|}{ne-XaY̯eZ}  \\\hline\hline
		2\gsc{SG} & XiY̯aZ-ta & XiY̯aZ-t & me-XaY̯eZ & me-XaY̯eZ-et & te-XaY̯eZ & te-XaY̯Z-i\\\hline
%		2\gsc{PL} & XiY̯aZ-tem & XiY̯aZ-ten/m & me-XaY̯Z-im & me-XaY̯Z-ot & \multicolumn{2}{l|}{te-XaY̯Z-u}\\\hline\hline
		2\gsc{PL} & \multicolumn{2}{l|}{XiY̯aZ-tem} & me-XaY̯Z-im & me-XaY̯Z-ot & \multicolumn{2}{l|}{te-XaY̯Z-u}\\\hline\hline
		3\gsc{SG} & XiY̯eZ & XiY̯Z-a & me-XaY̯eZ & me-XaY̯eZ-et & je-XaY̯eZ & te-XaY̯eZ\\\hline
		3\gsc{PL} & \multicolumn{2}{l||}{XiY̯Z-u} & me-XaY̯Z-im & me-XaY̯Z-ot & \multicolumn{2}{l|}{je-XaY̯Z-u}\\\hline
	\end{tabular}
%\end{small}
\xe
The system appears to be at once very regular and riddled with idiosyncrasies. Is there a method to the madness? In \S\ref{intro:problems} I present three research questions that guide our investigation. These questions are crosscut by three themes introduced in \S\ref{intro:themes}. Section \S\ref{sec:data} moves on to discuss the sources of data, \S\ref{sec:theory} provides a brief rundown of the theoretical assumptions and \S\ref{sec:jjmcc} touches on a number of prominent analyses from the past. Section \S\ref{intro:summary} provides an overview of each of the following chapters.


\section{Transliteration and notation} \label{sec:data:notation}
I use the variables X, Y and Z for the tri-consonantal root: \root{XYZ}. This dissertation contains little discussion of roots with more than three consonants, but nothing in the notation hinges on it. The list in \cite{ehrenfeld12} contains 311 quadrilateral roots and three quintilateral roots\footnote{\root{xntrʃ} `bullshit', \root{snxrn} `synchronize' and \root{flrtt} `flirt'.} out of 1876 roots in total.

In the Hebrew glosses, \gsc{ACC} is used for the direct object marker \emph{et} and \gsc{CS} for the head of a Construct State nominal.

As will be discussed in Chapter 3, Hebrew has a fairly productive process of postvocalic spirantization applying to /b/, /k/ and /p/, turning them into [v], [x] and [f] respectively. This process is blocked in certain verbal templates; to note this blocking I borrow the non-syllabicity diacritic and place it under the medial root consonant: ``\dgs{Y}''. This notation can be found in the templates \tpie~and \thit, in which this blocking holds. The same notation is used for segments which never spirantize: ``\dgs{k}''. Again, see Chapter 3.

Transcriptions are given using the International Phonetic Alphabet with the following modifications:
\begin{itemize*}
	\item ``e'' stands for /ɛ/ and /ə/.
	\item ``g'' stands for /ɡ/.
	\item ``o'' stands for /ɔ/.
	\item ``r'' stands for /ʁ/.
	\item ``x'' stands for /χ/.
	\item The apostrophe ' stands for the glottal stop /ʔ/ in the example sentences in Chapter 2.
\end{itemize*}
These changes were made purely for reasons of convenience. The syntactic literature has often used ``\v{s}'' or ``S'' for /ʃ/ and ``c'' for /\texttslig/. In both cases I preferred to retain the IPA transcription, ``ʃ'' and ``ts''. Stress is marked with an acute accent, ``\'a''. Deleted vowels are enclosed in angle brackets, ``$<>$''.

\underline{Underlining} and \textbf{boldface} are used only for emphasis, never as diacritics or notation.

  \subsection{Deviations from the standard forms}
The template \thif~usually appears in the literature as \emph{h\underline{i}XYiZ}, with the first vowel an /i/. In my experience, speakers of my generation and at least one generation older use the /e/ form, and so I use /e/ throughout.

In contrast, the initial /h/ in \emph{\underline{h}iXYiZ}~is usually dropped in speech. Nevertheless, I retain the segment in my transcription for two reasons. First, the /h/ is still pronounced by some older speakers and certain sociolinguistic groups, especially marginalized ones (\citealt{schwarzwald81biu}; see \citealt{gafter14phd} for related discussion). Second, the initial \emph{h}- should help the non-Semitist reader distinguish this template from others.

Glottal stops are often dropped in speech \citep{faust05,faust15iatl}. I usually omit them, but at times retain an apostrophe in order to distinguish between otherwise homophonous forms, for example \emph{hefria} `he disturbed' $\sim$ \emph{hefri\underline{'}a} `she disturbed'.

When presenting verbal paradigms I use two substandard forms. The first person singular future is normally prefixed with a low vowel, e.g.~\emph{a-daber} `I will talk' (in \root{dbr}). Contemporary usage, however, syncretizes the first person singular future with the third person masculine singular future: \emph{je-daber} `I/he will talk'. As far as I can tell, nothing important in the analysis depends on this distinction (or lack thereof). I do not use a dedicated \gsc{1SG} form in my transcription.

Finally, Modern Hebrew does not make a distinction between masculine and feminine plural forms in past and future tense verbs. The traditional feminine plural endings have been discarded, syncretizing instead with the masculine plural forms.

When building on existing work I modify the original transcriptions for consistency. With this housekeeping out of the way, we return to the theoretical approach.



\section{Sketch of the system}
The current study takes as its empirical focus the verbal system of Modern Hebrew. This morphological system is an ideal testing ground for theories of argument for two reasons. The first is that triplets can be found in which a given root clearly has three different kinds of morphological marking, as with \root{axl} `\root{\gsc{EAT}}' in~(\nextx). As a consequence, we must move beyond binary distinctions (whether in the syntax or the morphology). I will refer to these morpho-phonological patterns as \emph{templates}. They are traditionally described by making reference to the vowels and affixes which combine with the root consonants X, Y and Z. Hebrew has seven such templates, three of which can be seen in~(\nextx).
\ex\label{ex:alternations-heb}
	\begin{tabular}{llllll}
	\multicolumn{2}{c}{anticausative}	&	\multicolumn{2}{c}{transitive}	& \multicolumn{2}{c}{causative}\\
	\multicolumn{2}{c}{\tnif}	&	\multicolumn{2}{c}{\tkal}	& \multicolumn{2}{c}{\thif}\\
	\emph{neexal}	& `was eaten'	&	\emph{axal}	& `ate'	&	\emph{heexil}	& `fed'\\
	\end{tabular}
\xe

The second reason for investigating Hebrew is that this three-way distinction is instructive but not deterministic. A given syntactic configuration does not always entail a given template, (\nextx), and a given template does not always entail a given syntactic configuration, (\anextx).
\ex\label{ex:counter1}
	\begin{tabular}{lllllll}
	a.& Transitive in & \tpie & instead of \tkal 		& \emph{biʃel} & `cooked'	& * \emph{baʃal}\\
	b.& Anticausative in & \tkal & instead of \tnif	& \emph{nafal}	& `fell'		& * \emph{ninfal}\\
	c.& Anticausative in & \thit & instead  of \tnif & \emph{hitparek} & `fell apart'	& * \emph{nifrak}\\
	\end{tabular}
\xe
\pex\label{ex:counter2}
	\begin{tabular}{lllllll}
	a.& Unergative in & \tnif & instead of & anticausative & \emph{nilxam} & `fought'\\
	b.&	Reflexive in & \thit & instead of & anticausative & \emph{hitgaleax} & `shaved'\\
	\end{tabular}
\xe

These ``irregularities'' have occupied Semitist grammarians as well as contemporary authors \citep{doron03,arad05,borer13oup,kastner16phd}. On the one hand, it has long been acknowledged that roots show idiosyncratic behavior: broadly speaking, some roots prefer to form transitive verbs in {\tkal} and others in {\tpie}, as seen above. But on the other hand, it has proven much harder to accurately delimit the degree to which the templates are associated with distinct syntactic structures and semantic interpretations. When the language as a whole is viewed, the patterns in~(\ref{ex:alternations-heb}) are clearly visible, but so are the exceptions.

The proposal put forward in this book will showcase a novel analysis of the Hebrew patterns which captures existing work on European languages equally well. Working within Minimalist syntax \citep{chomsky95} and Distributed Morphology \citep{dm}, I rely on the idea that the external argument is introduced by the functional head Voice \citep{kratzer96,pylkkanen08,woodmarantz17}. This head might itself be endowed with syntactic features \citep{schaefer08,wood15springer}, leading me to defend the following claim:
\pex \textbf{The trivalency of Voice}
	\a Voice is associated with a [$\pm$D] feature, meaning it can be valued as [+D], [--D] or underspecified with regards to [D].\footnote{A similar view of binary features as trivalent is espoused by \cite{harbour11}.}.
	\a This feature indicates whether the specifier of Voice must be filled by a DP ([+D]), cannot be filled by a DP ([--D]), or is agnostic as to whether it is filled by a DP (underspecified).
\xe

@@[D] because that's how it's often seen in EPP literature. CPs aren't subjects as such, my own view of this is in \cite{kastner15lingua}. Take no stand as to subject PPs (Slavic?)@@

Importantly, we would expect these Voice heads to differ in their phonological form. This is exactly where the Hebrew data lead us. At its simplest, the system straightforwardly derives the alternations seen in~(\ref{ex:alternations-heb}) as follows:
\ex\label{ex:alternations-heb2}
	\begin{tabular}{llllll}
	\multicolumn{2}{c}{anticausative}	&	\multicolumn{2}{c}{transitive}	& \multicolumn{2}{c}{causative}\\
	\multicolumn{2}{c}{\tnif}	&	\multicolumn{2}{c}{\tkal}	& \multicolumn{2}{c}{\thif}\\
	\multicolumn{2}{c}{\textbf{\vz}}	&	\multicolumn{2}{c}{\textbf{Voice}}	& \multicolumn{2}{c}{\textbf{\vd}}\\
	\emph{neexal}	& `was eaten'	&	\emph{axal}	& `ate'	&	\emph{heexil}	& `fed'\\
	\end{tabular}
\xe

On this analysis, verbs in {\tnif} are expected to be anticausative/unaccusative because they lack an external argument, verbs in {\thif} are expected to be transitive because they require an external argument, and verbs in {\tkal} could go either way, depending on the idiosyncratic requirements of the root. The three values of Voice correspond to different morphological markings, but there is more than one way to get e.g.~an anticausative verb (namely with Voice or {\vz}). This way of looking at things dissolves the puzzle posed by examples like those in~(\ref{ex:counter1}) above: there is no reason to expect ``transitive'' to map on to specific morphology deterministically. Rather, syntax is autonomous in being able to build up structures which have different building blocks but may end up being interpreted similarly \citep{wood15springer,woodmarantz17,kastner16phd,myler16mit}.

The first part of the book (Chapters 2--5) will develop this analysis based on Hebrew, with special attention to the functional heads and the features they may have. The second part of the book (Chapters 6--7) will revisit existing work on argument structure alternations in light of the current system, drawing parallels and sketching the boundaries of crosslinguistic variation. I will now illustrate some of the relevant questions arising in each of these parts.




	\subsection{Some specifics}
Let us flesh out a few of the functional heads relevant to the analysis. We are interested in the difference between roots and functional morphemes as a way of getting at the loci of idiosyncrasy and systematicity in the grammar. A root is an acategorial morpheme: the verb \emph{walk}, for example, consists under my assumptions of a root \root{\gsc{WALK}} and a verbalizing categorizer, little v. There are three such categorizers: a, n, and v, which serve to categorize roots as adjectives, nouns or verbs \citep{marantz01,arad05,woodmarantz15}.

The functional head v introduces an event variable and categorizes a root as a verb. A higher functional head, Voice, introduces the external argument \citep{kratzer96,pylkkanen08,marantz13lingua}. The functional head \emph{p} introduces the external argument of a preposition, also called its Figure \citep{svenonius03,svenonius07,wood14nllt}. To derive the full range of verbs in Hebrew, I use a number of overt variants of these heads. The breakdown is as follows.

\textbf{Voice and \emph{p} heads} introduce a DP in their specifier. In a regular, unmarked active clause, default (silent) Voice introduces the external argument. The head \emph{p} was proposed by \cite{svenonius03,svenonius07} to act in similar fashion to Voice or Chomskyan little \textit{v}: it merges above the PP, introducing the Figure (subject) of the \textbf{p}reposition. I will not attempt to motivate this structure but will simply assume it; it is meant to capture the predicative relationship between the two DPs, similarly to the PredP of \cite{bowers93,bowers01} and \emph{ann}-XP of \cite{mccloskey14}. In~(\ref{ex:pP}), the Figure is the DP \emph{book} and \emph{p} is circled for ease of reference.
\ex \label{ex:pP}
%\Tree
%		[\emph{p}P, qtree
%		[DP\\\emph{book}\\\gsc{figure} ]
%		[
%		[\emph{p},circle,draw ]
%		[PP
%			[P\\\emph{on} ]
%			[DP
%				[\emph{the table}\\\gsc{ground},triangle ]
%			]
%		]
%		]
%		]
\xe

To these heads I add nonactive counterparts, namely \textbf{\vz} and \textbf{\pz}. These two heads dictate that nothing may be merged in their specifiers. \vz~blocks the introduction of an external argument \citep{doron03,alexiadoudoron12,bruening13,wood15springer,spathasetal15} and \pz~blocks merger of a DP in the specifier of pP \citep{wood15springer}. The different kinds of Voice/\emph{p} only manipulate the syntax: they dictate whether a DP may or may not be merged in their specifier. As mentioned above, default Voice and \emph{p} are silent. But \vz~and \pz~are spelled out by the placeholder Vocabulary Item \gsc{MID}, which adds a prefix and triggers insertion of certain vowels.

Voice also has the strongly active counterpart \textbf{\vd}. This head requires that a DP be merged in its specifier, behaving the opposite of \vz. This definition will be refined in \S\ref{syn:templates:thif}.

Alongside these functional heads and standard lexical roots I posit \textbf{\va}. In the semantics, this {element} types the event as an Action \citep{doron03} or ``self-propelled'' \citep{folliharley08}. In the phonology, \va~is spelled out as a predictable set of vowels slotting between the root consonants. \va~also blocks intervocalic spirantization of the middle consonant as mentioned above. I assume {for now} that \va~is spelled out by the Vocabulary Item \gsc{INTNS}, which is used as a placeholder for the phonological output. The semantics of this element emerge again in \S\S\ref{syn:middle:refl}, \ref{syn:middle:recip}, \ref{syn:templates:tpie}, its phonology in \S\ref{sec:t-phi}, and its crosslinguistic equivalents in \S\ref{syn:crosslx:heads}.

\textbf{The spell-out} of these heads produces templates as an epiphenomenon and is as follows: 

Voice and v are underspecified, but when combining they result in the \tkal~template as explained in Chapter 3.

\vd~provides the prefix \emph{he}-. Recall that templates differ from each other also in the vowels that appear between the root consonants, so I assume that each of the overt functional heads used in this theory inserts the right vowels, with the full implementation as in Chapter 3. For the time being, one may think of these vocalic changes as readjustment rules \citep{embickhalle05}. These are rules that ``fix'' the phonology of a form. Informally, for English, irregular past tense verbs do not take a suffix but undergo readjustment of the stem:
\ex \emph{sang} = \emph{sing} + T[Past]
\xe
The exact processes giving rise to vowel alternations are explored in Chapter 3.

\vz~provides the prefix \emph{ni}- or its allomorph \emph{hit}- in the environment of \va:
\pex
\a \vz~\lra~ \emph{hit}- + \gsc{READJUSTMENT} / \trace~\va
\a \vz~\lra~ \emph{ni}- + \gsc{READJUSTMENT}
\xe

The root \va~triggers readjustment rules and blocks postvocalic spirantization of the medial root consonant. This phonological effect is formalized as a floating feature [--continuant] docking onto a particular consonant. See \cite{katie13} for an analysis of gemination in Akkadian and Arabic using a similar mechanism.
\ex \va~\lra~[--cont]_{\gsc{ACT}} / \trace~ \{ \root{XYZ} | Y $\in$ p, b, k \}
\xe

Table~\ref{table:summary} summarizes the syntactic, semantic and morphophonological effects of these heads, deriving a subset of the verbal system of Modern Hebrew. Special Voice/\emph{p} heads affect their specifier; see for the external argument (EA) under ``Syntax'' and as a prefix under ``Phonology.'' The effects of the special root {\va} can be seen under ``Semantics'' and as de-spirantization under ``Phonology.'' Note in particular that the \thit~template is morphologically complex. It is prefixed, indicating the existence of an overt Voice/\emph{p} head, and de-spirantized, indicating the existence of \va. Each of these elements is motivated in Chapter 2, and the table expanded on there.
\begin{table}[ht] \centering
	\begin{tabular}{|lll||c|c|l|c|}\hline
		\multicolumn{3}{|c||}{Heads} & Syntax 	& Semantics & Phonology & Mnemonic\\\hline\hline
		
		Voice& &	& (underspecified) 	& (underspecified)	&  \emph{XaYaZ} & ``simple''\\
		
		Voice&\red{\va}&	& (underspecified)	& \red{Action}	 & \emph{Xi{\red{\dgs{Y}}}eZ}&  ``intensive''	\\
		
			\blue{\vd}& &		& \blue{EA}	& (underspecified)	 & \emph{{\blue{he}}-XYiZ} & ``causative'' \\
		
		\blue{\vz}& &		& \blue{No EA}	& (underspecified)	 & \emph{{\blue{ni}}-XYaZ} & ``middle'' \\
		
		\blue{\vz}&\red{\va}&	& \blue{No EA}	& \red{Action}	 & \emph{{\blue{hit}}-Xa{\red{Y̯}}eZ} &  ``intensive middle''	\\
		
		Voice& &\blue{\pz}	& \blue{EA = Figure} & (underspecified)	 & \emph{{\blue{ni}}-XYaZ} & ``middle'' \\
		
		Voice&\red{\va}&\blue{\pz}	& \blue{EA = Figure} & \red{Action}	 & \emph{{\blue{hit}}-Xa{\red{Y̯}}eZ}	& ``intensive middle''\\\hline
	\end{tabular}
	\caption{The requirements of functional heads in the Hebrew verb.\label{table:summary}}
\end{table}



\section{Traditional generative treatments of the system} \label{sec:jjmcc}
Before we get to the meat of the dissertation, I would like to acknowledge some of the earlier work on Semitic morphology in an attempt to illustrate what we do and do not know yet. I focus here on the seminal series of works by McCarthy \citep{jjmcc79,jjmcc81,jjmcc89li,jjmccprince90} in order to bring out the inadequacies of a purely phonological account of {Semitic morphology}. Comparisons with more recent analyses will be presented in the next two chapters, where appropriate.

	\subsection{Tiers}
McCarthy's original contribution lay in dividing the Semitic (Arabic) verb into three ``planes'' or ``tiers'': the CV skeleton (C and V slots), the root (consonants) and the melody (individual vowels). For example, the verb \emph{takattab} `got written' was analyzed as follows, with a default verbal vowel -\emph{a}-.
\ex\label{ex:jjmcc-takattab}\emph{takattab} \citep[392]{jjmcc81}:\\
\xy
<0pt,1.5cm>*\asrnode{C}="C0",
<0.75em,1.5cm>*\asrnode{V}="V0",
<1.5em,1.5cm>*\asrnode{C}="C1",
<2.25em,1.5cm>*\asrnode{V}="V1",
<3em,1.5cm>*\asrnode{C}="C2",
<3.75em,1.5cm>*\asrnode{C}="C3",
<4.5em,1.5cm>*\asrnode{V}="V2",
<5.25em,1.5cm>*\asrnode{C}="C4",
<0pt,0cm>*\asrnode{t}="c0";
<0.75em,3cm>*\asrnode{a}="v0";
<1.5em,0cm>*\asrnode{k}="c1";
<2.25em,3cm>*\asrnode{a}="v1";
<3.5em,0cm>*\asrnode{t}="c2";
<4.5em,3cm>*\asrnode{a}="v2";
<5.25em,0cm>*\asrnode{b}="c4";
"C0"+D;"c0"+U**\dir{-};
"V0"+U;"v0"+D**\dir{-};
"C1"+D;"c1"+U**\dir{-};
"C2"+D;"c2"+U**\dir{-};
"C3"+D;"c2"+U**\dir{-};
"C4"+D;"c4"+U**\dir{-};
"V1"+U;"v1"+D**\dir{-};
"V2"+U;"v2"+D**\dir{-};
\endxy
%	\includegraphics[scale=0.4]{figs/jjmcc81-25}
\xe

By including the vocalism on a separate tier, \citeauthor{jjmcc81}'s theory allowed vowels to be manipulated independently of the roots or the skeleton. The melody \emph{u-a-i} is taken to derive the active participle, for {instance}:
\ex \emph{mutakaatib} \citep[401]{jjmcc81}\\
\xy
<0pt,1.5cm>*\asrnode{C}="C0a",
<0.75em,1.5cm>*\asrnode{V}="V0a",
<1.5em,1.5cm>*\asrnode{C}="C0b",
<2.25em,1.5cm>*\asrnode{V}="V0b",
<3em,1.5cm>*\asrnode{C}="C1",
<3.75em,1.5cm>*\asrnode{V}="V1",
<4.5em,1.5cm>*\asrnode{V}="V2",
<5.25em,1.5cm>*\asrnode{C}="C2",
<6em,1.5cm>*\asrnode{V}="V3",
<6.75em,1.5cm>*\asrnode{C}="C3",
<2.25em,0pt>*\asrnode{u}="u";
<3.5em,0pt>*\asrnode{a}="a";
<4.5em,0pt>*\asrnode{i}="i";
<3.5em,-1.5cm>*\asrnode{$\mu$}="mu";
"V0a"+D;"u"+U**\dir{-};
"V0b"+D;"a"+U**\dir{-};
"V1"+D;"a"+U**\dir{-};
"V2"+D;"a"+U**\dir{-};
"V3"+D;"i"+U**\dir{-};
"u"+D;"mu"+U**\dir{-};
"a"+D;"mu"+U**\dir{-};
"i"+D;"mu"+U**\dir{-};
\endxy
\xe
%	\includegraphics[scale=0.6]{figs/jjmcc81-42}

The beauty of this theory is that it allowed for a separation of three morphological elements on three phonological tiers: the root (identity of the consonants), the template (the form of the CV skeleton) and additional inflectional or derivational information (the identity of the vowels). Important extensions were proposed in \cite{jjmccprince90} to account for denominal forms, specifically plurals and diminutives.

The current work shifts the focus to the nature of the CV skeleton and the melody. \citeauthor{jjmcc81}'s approach did not attempt to model the relationships between the semantics of the different templates -- the alternations in argument structure. Yet as we have seen in \S\ref{sec:tradition}, some templates are related to others in ways that remain to be explicated. \citeauthor{jjmcc81}'s work{,} as well as work inspired by {it,} leaves us in prime position to ask the following interrelated questions:
\pex \textbf{Questions on the nature of Semitic addressed in this study}
	\a What is the syntax behind the CV skeleton?
	\a What is the syntax behind the melody?
	\a What is the relationship between different templates, that is, how are argument structure alternations derived?
\xe
My answers to these questions lead us to make different assumptions than \citeauthor{jjmcc81}. Like him, I believe that the consonantal root lies at the core of the lexicon. Unlike his theory, I do not postulate independent CV skeletons and do not accord the prosody morphemic status. The skeletons will be a by-product of how functional heads are pronounced and regulated by the general phonology of the language. There is no skeleton CVCVCCVC as in~(\ref{ex:jjmcc-takattab}) giving \emph{takattab}, for example: there would be a prefix \emph{ta}-, a number of vowels spelling out Voice, gemination spelling out an additional head, and the organization of these different segments will proceed in a way that satisfies the phonology without making reference to prosodic primitives like skeletons. Furthermore, each morpheme will have an explicit syntax and semantics associated with it. Chapter 2 develops the morphosyntactic system and Chapter 3 returns to the morphophonological side of the morphology, making a number of additional contributions to our understanding of how the syntax and the phonology interact.
	
	\subsection{Related work}
A few more pieces of research that capture generalizations important to this dissertation deserve mention. The seminal work by \cite{berman78} underscored the semi-predictable nature of the templates. \citet[Ch.~3]{berman78} made the point that the combination of root and template is neither fully regular nor completely idiosyncratic. Instead, \citeauthor{berman78} proposed a principle of \emph{lexical redundancy} to regulate the system. According to this theory, each root has a ``basic form'' in some template from which other forms are derived. Yet this theory did not formalize the relations between the templates, arbitrarily selecting one as the ``basic form'' and the others as derived from it, for each root. Nevertheless, \citeauthor{berman78}'s clear description of the regularities and irregularities in the morphology of Hebrew laid the groundwork for later works such as \cite{doron03}, \cite{arad05}, \cite{borer13oup} and the current contribution.

Alongside work that analyzed the syntactic and semantic features of roots and templates, other researchers have focused on the morphophonological properties of the system. The research program developed in a series of works by \cite{batel89,batel94} and \cite{ussishkin00phd,ussishkin05}---credited by \cite{ussishkin00phd} at least in part to \cite{horvath81}---denies the existence of the root as an independent morpheme. Instead, all verbs are derived via phonological manipulation of surface forms from each other, rather than from an underlying root. I refer to this idea as the ``stem-based approach''. We return to it in \S\ref{sec:others-diff}, discussing it more in depth there.

Even before the stem-based approach took form, other Semitists explored the idea of a Semitic system which diverged from the traditional descriptions. \cite{schwarzwald73} doubted the productivity of both the root and the templates, making an early argument for frequency effects in the interpretation of different templates. On that view, it is only the high frequency verbs of the language that show reliable alternations between templates. These verbs lead us as analysts to postulate relationships between templates, though when one looks at less frequent verbs, transparent alternations are less likely to hold. Unlike the stem-based hypothesis, which eschewed roots and relied on the template as a morphological primitive, the proposal in \cite{schwarzwald73} kept the root but relegated the template to morphophonological limbo: salient in the grammar but not operative in the syntax. While this early formulation of a template-less idea is intriguing, it cannot hold up to wug studies in which speakers generate argument structure alternations between templates using nonce words \citep{berman93jcl,moorecantwell13}.

To pick out a few studies on Arabic (as gleaned from the helpful overview in \citealt{ussishkin00phd}), \cite{darden92} offered an analysis of Egyptian Arabic that attempted to do without verbal templates; \cite{mcomber95} developed an infixation-based system similar to that of \cite{jjmcc81} which makes crucial reference to morpheme edges; and \cite{ratcliffe97,ratcliffe98} attempted to improve on \cite{jjmccprince90} by restricting the CV skeleton and treating more phenomena as cases of infixation. But let us return to the current study.





Verbs in Hebrew appear in one of seven verbal templates, a morphological phenomenon emblematic of Semitic languages. Descriptively speaking, templates are morphophonological objects made up of consonant and vowel slots; the two templates in (@) will be notated XaYaZ and niXYaZ, where X, Y and Z stand for the root consonants which combine with the template to create a verb. These template often reflect argument structure alternations \citep{doron03,arad05,kastner16phd,kastner17gjgl}, a situation which can also be seen in (@).
(@) ʃavar - niʃbar ‘broke’, kara - nikra ‘tore’, matax - nimtax ‘stretched’.

In these cases, the non-active version is a detransitivized form of the active version and shares the same root as the active verb. The derived verbs are all intransitive and their bases transitive. The English translations are labile, but Hebrew does not employ zero-derivation in the same way (cf. \cite{borer91,kastner18}).

Similar alternations hold in other templates, for example XiYeZ and hitXaYeZ \citep{kastner17gjgl}.
(@) biʃel - hitbaʃel, nipeax - hitnapeax, bitel - hitbatel

The starting point for our discussion of nominalizations is that many verbs in niXYaZ cannot be characterized as part of a causative alternation. The niXYaZ verbs in (@) either are not an anticausative version of a causative verb in (@), or do not have a causative alternant in XaYaZ to begin with.

(@)     laxam 'fought' - nilxam be- 'fought (with)' 
    axaz 'held' - neexaz be- 'held on to' 
    @@




\ex\label{ex:alternations-heb}Hebrew has \textbf{trivalent} morphological marking\citep{kastner18nllt}:\\
	\begin{tabular}{cll|ll|ll}
	& \multicolumn{2}{P{4.2cm}|}{causative} &	\multicolumn{2}{P{4cm}|}{underspecified}	& \multicolumn{2}{P{4.2cm}}{anticausative}\\\cline{2-7}
	\phantom{Semantics} & \multicolumn{2}{c|}{\thif}	&	\multicolumn{2}{c|}{\tkal}	& \multicolumn{2}{c}{\tnif}\\
	& \emph{heexil}	& `fed' &	\emph{axal}	& `ate'	&	\emph{neexal}	& `was eaten' \\
	& \emph{hextiv}	& `dictated' &	\emph{katav}	& `wrote'	&	\emph{nixtav}	& `was written' \\
	\end{tabular}
\xe

\ex\label{typo-feat}\textbf{The typology of Featural:}\\
\begin{tabular}{c|ll|ll|ll}
	& \multicolumn{2}{P{4cm}|}{\vd}	&  \multicolumn{2}{P{4cm}|}{Voice}	& \multicolumn{2}{P{4cm}}{\vz} \\\hline
%&&&&\\
Semantics	 & 		a.	&	&			b.	&& 	c. & \\
$\lambda$x 	 & 
&\Tree
[.VoiceP 
	[.DP ]
	[.
		[.{\vd} ]
		[.vP ]
	]
]
& 
&\Tree
[.VoiceP 
	[.DP ]
	[.
		[.Voice ]
		[.vP ]
	]
]
&& \phantom{Undefined.}
\\\hline
Semantics	 & 		d.		& &			e.	& &	f. & \\
\zero	 &
& \phantom{Undefined.}
&
&\Tree
	[.VoiceP
		[.Voice ]
		[.vP ]
	]
&
&\Tree
	[.VoiceP
		[.{\vz} ]
		[.vP ]
	]\\
\end{tabular}
\xe


\ex\label{ex:alternations-heb2}Featural analysis of the templates:\\
	\begin{tabular}{cll|ll|ll}
	& \multicolumn{2}{P{4cm}|}{\textbf{\vd}}	&	\multicolumn{2}{P{4cm}|}{\textbf{Voice}}	& \multicolumn{2}{P{4cm}}{\textbf{\vz}}\\
	\phantom{Semantics} & \multicolumn{2}{c|}{causative} &	\multicolumn{2}{c|}{transitive}	& \multicolumn{2}{c}{anticausative}\\\cline{2-7}
	& \multicolumn{2}{c|}{\thif}	&	\multicolumn{2}{c|}{\tkal}	& \multicolumn{2}{c}{\tnif}\\
	& \emph{heexil}	& `fed' &	\emph{axal}	& `ate'	&	\emph{neexal}	& `was eaten' \\
	& \emph{hextiv}	& `dictated' &	\emph{katav}	& `wrote'	&	\emph{nixtav}	& `was written' \\
	\end{tabular}
\xe




\section{Templates as functional heads}
Modern Hebrew makes use of seven distinct morphophonological verbal forms in which a given root may or may not be instantiated. The root \root{ktb} which has a general meaning associated with writing can give rise to the verb \emph{katav} `wrote' in the \tkal~template and \emph{hextiv} `dictated' in the \thif~template. ``Causative'' \emph{hextiv} can then be passivized to yield \emph{huxtav} `was dictated' in the passive template \thuf. We can also get a passive reading of `was written' using the \tnif~template, \emph{nixtav}. So far, the argument structure alternations seem easy enough to pin down, as the first approximation in~(\nextx) shows. We begin with a subset of four verbal templates out of the seven.
\pex\label{ex:naive-ktb} A na\"{\i}ve view of argument structure alternations, based on \root{ktb}.
  \a \tkal: unmarked/transitive.
  \a \tnif: passive of \tkal~(\ref{ex:naive-ktb}a).
  \a \thif: causative of \tkal~(\ref{ex:naive-ktb}a).
  \a \thuf: passive of \thif~(\ref{ex:naive-ktb}c).
\xe

Traditional descriptions of the templates fall along these lines. However, if the characterization in~(\lastx) were correct then this would be a very short dissertation. In reality, neither the templates nor the roots play by such simple rules. A number of counterexamples are presented in~(\nextx).
\pex Counterexamples to the generalizations in~(\lastx).
  \a \tkal: does not exist in a large number of roots (\root{kns}, \root{'lm}, \root{tsmtsm}, \root{tlfn}).
  \a \tnif: \emph{nixnas} `entered', \emph{ne'elam} `disappeared', \emph{nilxam} `fought'; are not derived from a form in \tkal~nor are they passive.
  \a \thif: \emph{hexmits} `grew sour', \emph{he'edim} `reddened'. Change-of-state verbs, not derived from a form in \tkal.
  \a \thuf: (no counterexamples -- robust generalization)
\xe

This brief rundown omits further complications introduced by two additional templates, namely \thit~and \tpie, and by \tpua~which is the passive counterpart of the latter. The seven templates, then, do not provide us with deterministic mappings from phonological form (the template) to syntax (argument structure), except in the case of the two passive templates. One goal of this chapter---and of the dissertation in general---is to argue for an analysis which treats templates not as morphosyntactic atoms (morphemes or features) but as an epiphenomenon of distinct functional heads merging in the structure. Both the regularities in~(\ref{ex:naive-ktb}) and the exceptions in~(\lastx) will need to be covered by our theory of morphosyntax.

Consider next the roots. In a ``clean'' system, placing the same root in different templates would at least result in predictable changes in meaning; the alternations of \root{ktb} mentioned above all maintain the same basic notion of writing, modified in different ways. Yet the alternations, too, are not an open-and-shut case, as can be seen for \root{pkd} in~(\nextx).
\pex\label{ex:naive-pkd}
  \a \tkal: \emph{pakad} `ordered'.
  \a \tnif: \emph{nifkad} `was absent'.
  \a \tpie: \emph{piked} `commanded' (and a passive \tpua~form).
  \a \thif: \emph{hefkid} `deposited' (and a passive \thuf~form).
  \a \thit: \emph{hitpaked} `allied himself', `conscripted'.
\xe
One could find a general semantic field of ``counting'' or ``surveying'' running through the use of this root but the alternations are in no way obvious. The problem is exacerbated when considering nominal forms as well: \emph{pakid} `clerk', \emph{mifkada} `headquarters', \emph{pikadon} `deposit'. Templates, then, do not provide us with deterministic mappings from phonological form (the template) to semantics (interpretation of a root), again with the exception of the passive templates.

We have seen that a template cannot definitively clue us in onto the argument structure or interpretation of the verb. It is equally true that a given argument structure alternation or interpretation cannot be deterministically assigned to a certain template: we have already seen transitive verbs at least in \tkal, \tpie~and \thif. Where does this lead us? This dissertation attempts to balance two angles on the morphological phenomena: an empiricist-Semitist one, in which I ask what the templates are and what they tell us; and a theoretical one, in which I take {the} conceptual issue {of locality domains in argument structure alternations} and search for relevant evidence. But it is worth pausing to consider what is at stake. To the extent that the utterances of different languages are generated by similar grammars, the task of the linguist is to identify those parts of the grammar that cannot be learned from simple exposure to input, as well as how the parts that do get learned are encoded. Semitic languages are interesting because they seem to defy a linear account of structure building, at least in one well-circumscribed domain: prosodically strict morphology \citep{jjmcc81}. Any structure that the grammar generates, and any locality conditions or other constraints that hold in said structure, must produce output that can be linearized according to phonological requirements. Theories of locality and allomorphy are especially sensitive to ordering, be it linear or hierarchical. It follows that a satisfactory account of Semitic morphology is necessary in order to test hypotheses of how morphemes are arranged, how they combine, and how they might be learned.

Before continuing there is an important question regarding whether verbs such as those in~(\ref{ex:naive-ktb}) or in~(\ref{ex:naive-pkd}) do in fact share the same root. For example, it could be argued that (\lastx a,b,c,e) as well as the noun `headquarters' share one root that has to do with military concepts, and that (\lastx d) as well as the nouns `clerk' and `deposit' stem from a homophonous root that has to do with financial concepts. There are a number of reasons to reject this claim. First, there are no ``doublets''; if we were dealing with two roots, call them \root{pkd_1} and \root{pkd_2}, then each should be able to instantiate any of the templates. But \emph{hefkid} can only mean `deposited', never something like `installed into command'. The choice of verb for that root in that template has already been made. Second, experimental {studies} have found roots to behave uniformly across their different meanings (though this conclusion has been challenged{ -- see the overview in \S\ref{proc:meg}}). We will sidestep this issue for the bulk of the discussion, returning to it in \S\ref{acq:prod:root}.

In the remainder of this dissertation I will be forced into the not-unwelcome position of claiming that templates are \emph{emergent}, arising from the combination of roots and functional heads in the syntax. Unlike much traditional work that took both roots and templates to be primitives in the system (i.e.~morphemes), and unlike some recent work which takes templates to be morphemic but denies the existence of the root, our primitives will be the root and a collection of hierarchically arranged syntactic heads.

Each of these heads will be assigned an explicit syntax, semantics and phonology. This chapter is devoted to describing the syntax and semantics of the different pieces, evaluating the model's fit to the data and testing additional predictions that are made. The goal is to identify (a) the syntactic features relevant to each morpheme and (b) which rules of interpretation operate on them. Table~\ref{table:summary-syn} summarizes the combinations of functional material that I will discuss.
%\setcounter{table}{0}
	\begin{table}[ht] \centering \small
		\begin{tabular}{|llll||c|c|l|c|}\hline
			\multicolumn{4}{|c||}{Heads} & Syntax 	& Semantics & Phonology & Section\\\hline\hline
			
			& Voice& &	& (underspecified) 	& (underspecified)	&  \emph{XaYaZ} & \S\ref{syn:templates:tkal} \\\hline
			
			& Voice&\red{\va}&	& (underspecified)	& \red{Action}	 & \emph{Xi{\red\dgs{Y}}eZ}&  \S\ref{syn:templates:tpie}	\\
			
			\olive{Pass} & Voice&\red{\va}&	& \olive{Passive}	& \red{Action}	 & \emph{X\olive{u}{\red{\dgs{Y}}}\olive{a}Z}&  \S\ref{syn:templates:pass}	\\\hline
			
			& \blue{\vd}& &		& \blue{EA}	& (underspecified)	 & \emph{{\blue{he}}-XYiZ} & \S\ref{syn:templates:thif} \\
			
			\olive{Pass} & \blue{\vd}& &		& \olive{Passive}, \blue{EA}	& (underspecified)	 & \emph{{\blue{h}}\olive{u}-XY\olive{a}Z} & \S\ref{syn:templates:pass} \\\hline
			
			& \blue{\vz}& &		& \blue{No EA}	& (underspecified)	 & \multirow{2}{*}{\emph{{\blue{ni}}-XYaZ}} & \S\ref{syn:middle:nonactive} \\
			
			& Voice& &\blue{\pz}	& \blue{EA = Figure} & (underspecified)	 &  & \S\ref{syn:middle:active} \\\hline
						
			& \blue{\vz}&\red{\va}&	& \blue{No EA}	& \red{Action}	 & \multirow{2}{*}{\emph{{\blue{hit}}-Xa{\red{Y̯}}eZ} } &  \S\ref{syn:middle:nonactive} \\

			& Voice&\red{\va}&\blue{\pz}	& \blue{EA = Figure} & \red{Action}	 & & \S\ref{syn:middle:active} \\\hline
		\end{tabular}
		\caption{The requirements of functional heads in the Hebrew verb.\label{table:summary-syn}}
	\end{table}

Explaining briefly, Pass is a passivizing head. Voice is a functional head introducing the external argument of the verb. {\vz} is a variant which does not allow anything in its specifier and {\vd} is a variant requiring a DP in its specifier. \emph{p} introduces the subject of a preposition, and \pz~is a variant which does not allow anything in its specifier. {\va} is an agentive modifier. These elements will be introduced as the discussion progresses; the combinatorial possibilities are addressed in \S\ref{syn:crosslx:combinatorics}.

Table~\ref{table:root-summary-syn} summarizes the requirements of root classes in different configurations.
\begin{table}[ht] \centering \small
\begin{tabular}{|ll|l|l|}\hline
	 Morphology & Section & Verb type & Root type \\\hline\hline
	 \multirow{2}{*}{\thit} & \multirow{2}{*}{\S\S\ref{syn:middle:nonactive:anticaus}, \ref{syn:middle:refl}, \ref{syn:middle:roots}} &  Reflexive & Self-oriented\\
	 & & Inchoative & Other-oriented\\\hline
	 
	 \multirow{2}{*}{\thit} & \multirow{2}{*}{\S\S\ref{syn:middle:recip}, \ref{syn:middle:roots}} & Reciprocal & Naturally reciprocal\\
	 & & Figure reflexive & Naturally disjoint (Other-oriented)\\\hline
	 
	 \multirow{2}{*}{\thif} & \multirow{2}{*}{\S\ref{syn:templates:thif}} & Alternating unaccusative & Change of color\\
	 & & Alternating unergative & Emission\\\hline
	 
	 \multirow{2}{*}{\tpie} & \multirow{2}{*}{\S\ref{syn:templates:tpie}} & Pluractional object & Other-oriented\\
	 && Pluractional event & Self-oriented (activity)\\\hline
	 
	 Passive participle & \S\ref{syn:templates:adjpass} & Resultative adjective & Change of state\\\hline
	 
	 \tkal & \S\ref{syn:templates:tkal} & Underspecified & (all)\\\hline
\end{tabular}
\caption{The requirements of root classes in the Hebrew verb.\label{table:root-summary-syn}}
\end{table}



The discussion in this paper highlights how roots place requirements on the syntactic derivation. In English, for instance, it has been suggested in different ways that there is a difference between the semantics of \root{\gsc{DESTROY}}, \root{\gsc{GROW}} and \root{\gsc{BREAK}} which goes beyond pure meaning. This difference leads to an inability to take complements in nominalized form \citep{chomsky70,marantz97}.
\pex \root{\gsc{DESTROY}}: change of state, externally caused
\a The enemy's destruction of the city.
\a The city's destruction (by the enemy).
\xe

\pex \root{\gsc{GROW}}: change of state, internally caused
\a \ljudge{*} John's growth of tomatoes.
\a The tomatoes' growth (*by John).
\xe

\pex \root{\gsc{BREAK}}: result
\a \ljudge{*} John's break of the glass.
\a \ljudge{*} The glass' break.
\xe
Similar observations have been made more recently for a variety of phenomena in different languages \citep{haspelmath93,unaccusativity95,schaefer08}. The details are less important right now than the intuition that something about the lexical semantics of the root constrains what should otherwise be an identical syntactic derivation. In these cases, the underlying assumption is that the morphosyntax of the verbs \emph{destroy}, \emph{grow} and \emph{break} is identical in that they are all made up of a root and a verbalizer, with no extra syntactic material determining their argument structure.

Nevertheless, argument structure alternations can be conditioned by additional syntactic material. For instance, markers such as German \emph{sich} and Romance \gsc{SE} famously reduce the total arity of the verb, descriptively speaking \citep[e.g.][]{labelle08,schaefer08,cuervo14}. In order to account for the Hebrew facts, I will take the connection highlighted in the previous section---that between argument structure and the template---and cash it out in terms of the syntactic head Voice.

Throughout the paper I assume that morphological structure is built up in the syntax \citep{dm}, with late insertion of phonological material proceeding from the most deeply embedded element outwards \citep{bobaljik00,embick10}. The external argument is introduced by the functional head Voice \citep{kratzer96,pylkkanen08}. Acategorial roots modify one of the ``categorizing'' heads v, n and a \citep{marantz97,arad03,harley14thlia}. To see how roots affect argument structure, we begin with anticausatives.


%%%%%%%
The present study examines the division of labor between syntax, semantics, phonology and the lexicon. Generative approaches to linguistic theorizing have so far resulted in a wealth of knowledge about how an abstract syntax generates structure which is then interpreted by the semantics and by the phonology. We also have a basic vocabulary allowing us to describe how individual lexical items might have their own idiosyncrasies in the syntax (different features), in the semantics (different meanings) and in the phonology (lexical exceptionality). What this dissertation tests is the hypothesis that the syntax feeds both interfaces in the same way.

The core idea has two parts. The first is that there is a universal set of syntactic elements which can be arranged in a hierarchical way. Once the syntax generates a chunk of structure (be it a phase or an entire utterance), it must be interpreted compositionally at the interface with the semantics and linearized in order to be interpreted at the interface with the phonology. In both cases, I argue that the same kind of locality constraints hold on interpretation. The second part has to do with how individual lexical items muddy the waters. The grammar is rigid, but lexical material can influence how functional material is interpreted. The important point here is that lexical material is separated from the syntax proper: its idiosyncrasies only kick in at the two interfaces. As such, lexical material (a root) has no syntactic features, only properties which are directly related either to meaning or to pronunciation.

In support of this claim I approach a notorious empirical landscape: the non-concatenative morphology of Modern Hebrew. In Semitic languages like Hebrew, words are famously made up of various grammatical and lexical elements interleaved in a single, short (often disyllabic or trisyllabic) phonological word. Two syllables might convey a lexical root, tense information, derivational information and agreement information all at once. It is therefore not immediately obvious that hierarchical structure is there to be found. Yet that is exactly the claim put forward here. By demonstrating that the hypothesis outlined above is a valid one for Hebrew, this dissertation entails that it is valid for natural language as a whole. Functional material (in the form of structure) and lexical material (in the form of roots) can be seen to combine in systematic, predictable ways.

Taking the verbal system as the main object of inquiry, the following chapters analyze its syntactic, semantic and phonological properties in order to address a number of Hebrew-specific, Semitic-specific and language-general questions. At its analytical core, this work asks how we might apply the theories that have been developed for other languages to Semitic: how these theories might be tested and how they should be modified to accommodate a broader range of data. I frame this question in terms of the abstract syntactic structure and the way it feeds into semantic and phonological interpretation. The resulting discussion addresses three main questions: the proper description of Hebrew morphology, the proper description of the syntax and its relation to the interfaces with the semantics and the phonology, and ultimately the way such a system is learned by the child. Let us first see what the system looks like.



We will focus on three verbal forms in Hebrew (\emph{templates}): {\tkal}, {\tnif} and {\thif}.

	\subsection{{\tkal} varies in its argument structure (underspecified)}
\pex Transitive:
	\a \begingl
		\gla teo \textbf{axal} et ha-laxmania//
		\glb Theo ate \gsc{ACC} the-bread.roll//
		\glft `Theo are the bread roll.'//
	\endgl
	\a \begingl
		\gla ha-balʃan \textbf{katav} et ha-maamar ha-arox//
		\glb the-linguist wrote \gsc{ACC} the-article the-long//
		\glft `The linguist wrote the long article.'//
	\endgl
\xe

\pex Unergative:
	\a \begingl
		\gla teo \textbf{rakad} ve-rakad ve-rakad//
		\glb Theo danced and-danced and-danced//
		\glft `Theo danced and danced and danced.'//
	\endgl
	\a \begingl
		\gla teo \textbf{halax} kol ha-boker//
		\glb Theo walked all the-morning//
		\glft `Theo walked all morning long.'//
	\endgl
\xe

\pex Unaccusative:
	\a \begingl
		\gla \textbf{nafal} le-teo ha-bakbuk//
		\glb fell to-Theo the-bottle//
		\glft `Theo's bottle fell.'//
	\endgl
	
	\a \begingl
		\gla ha-bakbuk \textbf{kafa} ba-makpi//
		\glb the-bottle froze in.the-freezer//
		\glft `The bottle froze in the freezer.'//
	\endgl
\xe


	\subsection{{\tnif} is non-active}
\pex
	\a \begingl
		\gla \textbf{niʃbar} l-i ha-ʃaon//
		\glb broke to-me the-watch//
		\glft `My watch broke.//
	\endgl
	\a \begingl
		\gla \textbf{neelam} l-i ha-tamsir//
		\glb disappeared to-me the-handout//
		\glft `My handout disappeared.'//
	\endgl
\xe

	\subsection{{\thif} is active}
\pex
	\a \begingl
		\gla kevin \textbf{heexil} et ema (duvdevanim)//
		\glb Kevin fed \gsc{ACC} Emma blueberries.//
		\glft `Kevin fed Emma (blueberries).'//
	\endgl %Emma is a dog
		
	\a \begingl
		\gla ha-zamar \textbf{herkid} et ha-orxim//
		\glb the-singer made.dance \gsc{ACC} the-guests//
		\glft `The singer made the guests dance.'//
	\endgl
\xe

\newpage
	\subsection{That's all we need}
\pex Importantly, we can find triplets:
	\a Causative {\thif}\\
		\begingl
		\gla ha-more \caus{hextiv} (la-talmidim) et reʃimat ha-nosim//
		\glb the-teacher dictated to.the-students \gsc{ACC} list.of the-topics//
		\glft `The teacher dictated the list of topics (to the students).'//
	\endgl
	
	\a Transitive {\tkal}\\
		\begingl
		\gla ha-talmidim \textbf{katv-u} et ha-nosim//
		\glb the-students wrote-\gsc{PL} \gsc{ACC} the-topics//
		\glft `The students wrote the topics down.'//
	\endgl
	
	\a Non-active (mediopassive) {\tnif}\\
		\begingl
		\gla ha-xiburim \anticaus{nixtev-u} (al-jedej ha-talmidim)//
		\glb the-essays were.written-\gsc{PL} by the-students//
		\glft `The essays were written (by the students)'.//
	\endgl
\xe


